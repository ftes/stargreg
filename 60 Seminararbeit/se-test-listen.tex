\section{Die Definition und Anwendung von zwei neuen Listenumgebungen}

\subsection{Das Layout der Standardlistenumgebung von \LaTeX}

Stichpunktlisten werden in \LaTeX{} mit der \verb+itemize+-Umgebung erzeugt. 
Die Stichpunktliste 

\begin{itemize}
\item 1. Stichpunkt der ersten Ebene
\begin{itemize}
\item 1. Stichpunkt der zweiten Ebene
\item 2. Stichpunkt der zweiten Ebene
\begin{itemize}
\item 1. Stichpunkt der dritten Ebene
\item 2. Stichpunkt der dritten Ebene
\begin{itemize}
\item 1. Stichpunkt der vierten Ebene
\item 2. Stichpunkt der vierten Ebene
\end{itemize}
\end{itemize}
\end{itemize}
\item 2. Stichpunkt der ersten Ebene
\item 3. Stichpunkt der ersten Ebene
\end{itemize}

wird durch die folgenden Anweisungen erreicht:

\vspace*{-\baselineskip}

\begin{verbatim}
\begin{itemize}
\item 1. Stichpunkt der ersten Ebene
\begin{itemize}
\item 1. Stichpunkt der zweiten Ebene
\item 2. Stichpunkt der zweiten Ebene
\begin{itemize}
\item 1. Stichpunkt der dritten Ebene
\item 2. Stichpunkt der dritten Ebene
\begin{itemize}
\item 1. Stichpunkt der vierten Ebene
\item 2. Stichpunkt der vierten Ebene
\end{itemize}
\end{itemize}
\end{itemize}
\item 2. Stichpunkt der ersten Ebene
\item 3. Stichpunkt der ersten Ebene
\end{itemize}
\end{verbatim}

\subsection{Die neue Listenumgebung \texttt{seList} f\"ur Stichpunktlisten}

Weder die Einr\"uckung der einzelnen Ebenen noch die gro{\ss}en Abst\"ande zwischen den einzelnen Stichpunkten sind bei der \verb+itemize+-Umgebung 
bez\"uglich des Layouts sonderlich \"uberzeugend. 

Daher wird eine neue \verb+seList+-Umgebung zur Verf\"ugung gestellt. 

\begin{seList}
\item 1. Stichpunkt der ersten Ebene
\begin{seList}
\item 1. Stichpunkt der zweiten Ebene
\item 2. Stichpunkt der zweiten Ebene
\begin{seList}
\item 1. Stichpunkt der dritten Ebene
\item 2. Stichpunkt der dritten Ebene
\begin{seList}
\item 1. Stichpunkt der vierten Ebene
\item 2. Stichpunkt der vierten Ebene
\begin{seList}
\item 1. Stichpunkt der f\"unften Ebene
\item 2. Stichpunkt der f\"unften Ebene
\end{seList}
\end{seList}
\end{seList}
\end{seList}
\item 2. Stichpunkt der ersten Ebene
\item 3. Stichpunkt der ersten Ebene
\end{seList}

Der \LaTeX-Quelltext f\"ur diese Liste ist: 

\vspace*{-\baselineskip}
\begin{verbatim}
\begin{seList}
\item 1. Stichpunkt der ersten Ebene
\begin{seList}
\item 1. Stichpunkt der zweiten Ebene
\item 2. Stichpunkt der zweiten Ebene
\begin{seList}
\item 1. Stichpunkt der dritten Ebene
\item 2. Stichpunkt der dritten Ebene
\begin{seList}
\item 1. Stichpunkt der vierten Ebene
\item 2. Stichpunkt der vierten Ebene
\begin{seList}
\item 1. Stichpunkt der f\"unften Ebene
\item 2. Stichpunkt der f\"unften Ebene
\end{seList}
\end{seList}
\end{seList}
\end{seList}
\item 2. Stichpunkt der ersten Ebene
\item 3. Stichpunkt der ersten Ebene
\end{seList}
\end{verbatim}

\vspace*{-\baselineskip}
Neben der Eigenschaft, im Gegensatz zur \verb+itemize+-Umgebung f\"unf Verschachtelungsebenen angeben zu k\"onnen, ist es m\"oglich,
die Zeilenabst\"ande f\"ur die einzelnen Ebenen zu konfigurieren. 

Mit dem Kommando \newline 
\hspace*{\fill}\verb+\seSetlistbaselineskip{b1}{b2}{b3}{b4}{b5}+\hspace*{\fill}\newline 
kann f\"ur die Verschachtelungsebene $i$ der Grundlinienabstand \texttt{b$_{i}$} festgelegt 
werden. Als Einheit wird der Wert von \verb+\baselineskip+ (Grundlinienabstand des Dokuments) verwendet. Die folgenden Werte sind f\"ur ein Dokument voreingestellt:\newline
\hspace*{\fill}\verb+\seSetlistbaselineskip{1}{0.75}{0.75}{0.75}{0.75}+\hspace*{\fill}\newline\vspace*{-\baselineskip}

Mit dem Kommando \newline
\hspace*{\fill}\verb+\seResetlistbaselineskip{}+\hspace*{\fill}\newline
wird die letzte \"Anderung der Werte r\"uckg\"angig gemacht.

\newpage
\subsection{Die neue Listenumgebung \texttt{seToplist} f\"ur Listen mit einem Label und Aufz\"ahlungslisten}

Die neue Listenumgebung \verb+seToplist+ erlaubt es, jeden Stichpunkt mit einem Label zu versehen.
Die Liste\footnote{Die folgenden Werte sind frei erfunden.} 

\begin{seToplist}{Mercedes Benz:}
\item[Audi:] 400000 Gesamtverk\"aufe
\begin{seToplist}{3er Reihe:}
\item[A4:] 200000 Verk\"aufe
\item[A5:] 50000 Verk\"aufe
\item[A6:] 150000 Verk\"aufe
\end{seToplist}
\item[Mercedes Benz:] 500000 Gesamtverk\"aufe 
\item[BMW:] 650000 Gesamtverk\"aufe 
\begin{seToplist}{3er Reihe:}
\item[1er Reihe:] 100000 Verk\"aufe
\item[3er Reihe:] 300000 Verk\"aufe
\item[5er Reihe:] 250000 Verk\"aufe
\end{seToplist}
\end{seToplist}

wird durch die folgenden \LaTeX-Anweisungen erzeugt:

\vspace*{-\baselineskip}
\begin{verbatim}
\begin{seToplist}{Mercedes Benz:}
\item[Audi:] 400000 Gesamtverk\"aufe
\begin{seToplist}{3er Reihe:}
\item[A4:] 200000 Verk\"aufe
\item[A5:] 50000 Verk\"aufe
\item[A6:] 150000 Verk\"aufe
\end{seToplist}
\item[Mercedes Benz:] 500000 Gesamtverk\"aufe 
\item[BMW:] 650000 Gesamtverk\"aufe 
\begin{seToplist}{3er Reihe:}
\item[1er Reihe:] 100000 Verk\"aufe
\item[3er Reihe:] 300000 Verk\"aufe
\item[5er Reihe:] 250000 Verk\"aufe
\end{seToplist}
\end{seToplist}
\end{verbatim}

\vspace*{-\baselineskip}
Der Parameter \verb+par+ von \verb+\begin{seToplist}{par}+ definiert die Breite des Labels f\"ur die 
zugeh\"orige Liste.

F\"ur die \verb+seToplist+-Umgebung k\"onnen ebenfalls f\"unf Verschachtelungsebenen definiert werden. 
\"Uber die Kommandos \newline
\hspace*{\fill}\verb+\seSettoplistbaselineskip{b1}{b2}{b3}{b4}{b5}+\hspace*{\fill}\newline 
bzw. \newline
\hspace*{\fill}\verb+\seResettoplistbaselineskip{}+\hspace*{\fill}\newline
lassen sich analog zur \verb+seList+-Umgebung die Grundlinienabst\"ande der einzelnen Verschachtelungsebenen 
ver\"andern bzw. zur\"ucksetzen. Die folgenden Werte sind f\"ur ein Dokument voreingestellt:\newline
\hspace*{\fill}\verb+\seSettoplistbaselineskip{1}{0.75}{0.75}{0.75}{0.75}+\hspace*{\fill}\newline\vspace*{-\baselineskip}

Durch eine entsprechende Wahl der Labels k\"onnen Aufz\"ahlungslisten erzeugt werden:

\begin{seToplist}{a)}
\item[a)] Deutsche Automarken
\begin{seToplist}{1)}
\item[1)] Mercedes Benz
\item[2)] Audi 
\item[3)] VW
\item[4)] BMW 
\end{seToplist}
\item[b)] Japanische Automarken
\begin{seToplist}{1)}
\item[1)] Toyota
\item[2)] Honda
\item[3)] Mazda
\end{seToplist}
\end{seToplist}


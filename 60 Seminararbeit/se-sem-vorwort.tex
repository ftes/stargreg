Wer hat nicht schon ein Mal den Wunsch verspürt, Vorstandschef eines Großunternehmens zu sein? 
Als Teilnehmer eines Unternehmensplanspiels, bei dem der Spieler mit  einer realitätsnahen, simulierten Welt interagiert, ist dieser Wunsch gar nicht so weit hergeholt. 
\\
\\
Schon seit Jahrhunderten ist man auf der Suche, neue Lernmethoden für die Weiterbildung zu erforschen und auszubauen. Dabei stehen neben theoretischen Methoden auch immer mehr die praktischen im Vordergrund.  Eine spezielle Methode, um insbesondere komplexe Sachverhalte oder Systeme, wie beispielsweise die Funktionsweise eines Unternehmens, zu veranschaulichen, stellt hierbei das Planspiel dar. Man unterscheidet haptische (z.B. Brettspiele) und computergestützte Planspiele, die mittlerweile flächendeckend und in hoher Zahl am Markt erhältlich sind. Besonders populär ist das Unternehmensplanspiel TOPSIM der Firma \textit{TATA Interactive Systems}, das an ca. 1500 Hochschulen, Akademien und Unternehmen verwendet wird.\seFootcite{Vgl.}{}{TIS:PM} 
\\
\\
Die zunehmende Bedeutung von Planspielen als Lernmethode ist gewiss nicht nur auf deren Anwendung beschränkt. Eine weitere Lernkomponente ergibt sich auf Ebene der Anforderungsanalyse und Entwicklung eines computergestützten Planspiels, was insbesondere für Studierende der Wirtschaftsinformatik große Lernchancen mit sich bringt. Da zu ihrem Bereichsfeld sowohl die betriebswirtschaftlichen als auch software- und programmiertechnischen Anforderungen an Analyse, Entwurf und Implementierung einer derartigen Software gehören, wird diese Aufgabe gerne in den Lehrplan übernommen. 
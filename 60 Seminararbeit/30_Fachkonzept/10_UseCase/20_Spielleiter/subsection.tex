\subsection{Spielleiter}
\label{sec:fachkonzept-usecase-spielleiter}

\autorbeginn{Marcel}

Zentraler Akteur neben dem Spieler ist in dem UseCase Diagramm der Spielleiter. Er ist im Hintergrund tätig und leitet das gesamte Unternehmensplanspiel. Zur aktuellen Implementierungsphase ist es vorgesehen, dass der Spielleiter manuell von einer Person bedient wird. Es ist jedoch vorgesehen, dass die Funktionen des Spielleiters zu einem späteren Zeitpunkt vollständig durch den Computer übernommen werden, sodass für die Aufgaben des Spielleiters keine weitere Person benötigt wird.
 
Zu Beginn des Unternehmensplanspiels richtet der Spielleiter die Unternehmen ein. Er bekommt dazu von alle Mitspielern den Unternehmensnamen mitgeteilt und richtet auf Basis dieser Daten pro Mitspieler ein Unternehmen ein.
 
Neben dieser Aufgabe muss der Spielleiter alle Spieldaten anlegen, die für das Unternehmensplanspiel notwendig sind. Dazu müssen von ihm alle Bauteiltypen, Personaltypen und Raumschifftypen angelegt werden. Nach dem Anlegen dieser Daten legt der Spielleiter die Spielrunden an.
 
Eine weitere zentrale Aufgabe des Spielleiters ist es die Simulation der Spielrunden einzuleiten. Spielrunden können aber erst eingeleitet werden, wenn der Spielleiter überprüft hat, dass alle Spieler die Spielrunden eingecheckt haben. Jeder Spieler muss dazu aller relevanten Tätigkeiten pro Spielrunde ausführen und diese daraufhin bestätigen. Ebenso wird eine neue Spielrunde eingeleitet.  Mit dem Einleiten einer neuen Spielrunde gibt der Spielleiter ebenso Informationen zur neuen Spielrunde aus.
 
Die Unternehmen zu bewerten, liegt ebenfalls in der Hand des Spielleiters. Dies geschieht anhand einer festgelegten Formel, die zu einem späteren Zeitpunkt dieser Seminararbeit erläutert wird.

\autorende{}
%###
\subsection{Spieler}
%###
\label{sec:fachkonzept-spieler}
\autorbeginn{Julia}

Der zweite Akteur in diesem UseCase Diagramm stellt der Spieler dar. Er steuert ein komplettes Unternehmen und übernimmt somit auch alle im Unternehmen anfallenden Aufgaben.
Um das Spiel zu starten übermittelt der Spieler den selbst ausgewählten Unternehmensnamen an den Spielleiter welcher wie oben beschrieben das Spiel einrichtet.

Ist dies geschehen, so hat der Spieler die Möglichkeit jegliche Informationen über das eigene Unternehmen und alle vorhandenen Abteilungen abrufen. Im Planspiel sind folgende Abteilungen integriert: Finanzwesen, Einkauf, Produktion, Verkauf und Personalwesen. Somit kann der Spieler die aktuelle Finanzlage des Unternehmens, die verkauften Raumschiffe, die gekauften Bauteile, die eingestellten Mitarbeiter sowie die eingelagerten Raumschiffe einsehen. Dadurch ist ein optimaler Überblick über das gesamte Unternehmen gewährleistet.

Dem Spieler ist es darauf hin ermöglicht, weiteres Personal einzustellen, falls er dies aus Kapazitäts- oder Qualitätsgründen benötigt. Bereits eingestelltes Personal kann weitergebildet werden wodurch die Fähigkeiten der einzelnen Mitarbeiter gesteigert werden. Sind im Unternehmen zu viele Mitarbeiter angestellt, so hat der Spieler ebenfalls die Möglichkeit, Personal zu entlassen.

Im Einkaufsbereich kann der Spieler verschiedene Bauteile einkaufen, welche zur Produktion von Raumschiffen benötigt werden. Als nächster Schritt können Produktionsaufträge angelegt werden. 

Anschließend kann der Spieler Verkaufsangebote für die produzierten Raumschiffe abgeben. Als letzten Schritt wird die Spielrunde vom Spieler eingecheckt und somit beendet. 

Details zu den einzelnen Vorgänge, wie zum Beispiel das Einkaufen von Bauteilen, die Personalverwaltung oder die Produktion, werden im Laufe der Seminararbeit genauer erläutert.

\autorende{}
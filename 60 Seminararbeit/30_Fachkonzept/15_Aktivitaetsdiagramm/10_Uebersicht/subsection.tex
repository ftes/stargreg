\subsection{Übersicht}
\label{sec:fachkonzept-aktivitaetsdiagramm-uebersicht}

Das Aktivitätsdiagramm auf \vref{img:fachkonzept-aktivitaetsdiagramm-uebersicht} soll den Spielablauf aus Sicht des Spielers verdeutlichen.  

Um das Spiel zu starten muss der Spieler einen Namen für sein Unternehmen festlegen. Dies stellt die erste Aktion dar. Danach analysiert der Spieler die ihm zur Verfügung stehenden Informationen. Ist dies abgeschlossen, so gelangt er zu einem Entscheidungsknoten. Hierbei kann sich der Spieler zwischen folgenden Aktivitäten entscheiden: Personal verwalten, Einkäufe tätigen, Produktionsaufträge anlegen oder Verkaufsangebot abgeben. Diese einzelnen Vorgänge werden im Laufe dieses Kapitels genauer erläutert. Er kann sich aber auch dazu entscheiden, keine Transaktion zu tätigen. 

Diese verschiedenen Aktivitäten werden in einem Entscheidungsknoten zusammengeführt. Hat der Spieler weiteren Informationsbedarf, so gelangt er zur Aktivität “Auswirkungen analysieren” und kann sich die Veränderungen anschauen, die seine Transaktion mit sich geführt hat. Besteht kein Informationsbedarf, kann er diese Aktivität überspringen. 

Möchte der Spieler nun weitere Transaktionen tätigen, so kann er wieder zum Entscheidungsknoten nach oben springen und hat wieder die Wahl zwischen Personal verwalten, Einkäufe tätigen, Produktionsaufträge anlegen oder ein Verkaufsangebot abgeben. Entscheidet sich der Spieler gegen eine weitere Transaktion, so folgt die Aktivität “Runde einchecken”. Anschließend folgt wieder ein Entscheidungsknoten. Existiert noch eine weitere Spielrunde, so gelangt der Spieler wieder zur Aktivität “Informationen analysieren”. War dies die letzte Spielrunde, so wird dem Spieler die Endbewertung angezeigt. Danach ist das Spiel beendet.

\begin{figure}[h]
  \centering
  \fbox{
    \includegraphics[width=0.9\textwidth]{30_Fachkonzept/15_aktivitaetsdiagramm/activity1.pdf}
  }
  \caption{Aktivitätsdiagramm}
  \label{img:fachkonzept-aktivitaetsdiagramm-uebersicht}
\end{figure}
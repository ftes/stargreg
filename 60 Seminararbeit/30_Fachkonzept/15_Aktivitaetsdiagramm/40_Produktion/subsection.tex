\subsection{Produktionsauftrag anlegen}
\label{sec:fachkonzept-aktivitaetsdiagramm-produktion}

Beim Anlegen eines Produktionsauftrages, wie in \vref{img:fachkonzept-aktivitaetsdiagramm-produktion} dargestellt, ist zuerst das zu produzierende Raumschiff auszuwählen. Anschließend legt der Spieler die Anzahl fest. Nun muss geprüft werden, ob zum einen genügend Bauteile für die Produktion vorhanden sind und zum anderen ob das eingestellte Personal die Anzahl an Raumschiffen in einer Periode produzieren kann. Fällt die Prüfung positiv aus, so muss der Spieler seine Auswahl bestätigen. Sind die Kapazitäten jedoch nicht ausreichend, befindet sich der Spieler wieder bei der Aktivität “Raumschifftyp auswählen”.

\begin{figure}[h]
  \centering
    \includegraphics[trim = 0cm 1cm 1cm 1cm, width=0.7\textwidth]{30_Fachkonzept/15_aktivitaetsdiagramm/activity4.pdf}
  \caption{Aktivitätsdiagramm zur Produktion}
  \label{img:fachkonzept-aktivitaetsdiagramm-produktion}
\end{figure}
\subsection{Entscheidungsfindung der Unternehmen}
\label{sub:fachkonzept-implementierung-entscheidungsfindung}

\autorbeginn{Philipp}

Zur Entscheidungsfindung der Unternehmen trägt die Informationsphase bei. Die Methode gebeAnfangsInformationenAus()
der Klasse Unternehmen wird zu Beginn jeder Spielrunde ausgeführt und versorgt den Spieler auf einer Übersichtsseite
mit Informationen bezüglich seines Absatzes, seiner Finanzen, der angefallenen Lagerkosten und des aktuellen Preisniveaus.
Diese Informationen dienen Ihm für weitere Entscheidungen.

Wenn beispielsweise der Preis für ein Bauteil gerade sehr niedrig ist, so könnte er in dieser Spielrunde besonders viele
Bauteile dieses Typs kaufen. Dies bringt zum einen Kosteneinsparungen mit sich, zum anderen steigen die Preise in der nächsten
Spielrunde und die Konkurrenzunternehmen müssen die zuvor günstigen Bauteile zu hohen Preisen kaufen. Desweiteren erscheint
beim Eintreten eines Events, das fest in unserer Storyline vorgegeben ist, eine Nachricht. Wird in dieser Nachricht
beispielsweise ein Nachfragerückgang bei X-Wings vorhergesagt, so sollte der Spieler die Preise für X-Wings senken
um diese absetzen zu können.

Mit Hilfe der Methode gebeEndInformationenAus() kann der Spieler jederzeit einsehen welche Änderungen er im Laufe der
Spielrunde vorgenommen hat und wie sich diese auswirken.
\subsubsection{Bewertung}
\label{subsub:fachkonzept-implementierung-spiellogik-bewertung}

Die Bewertung der Unternehmen ist in den Unternehmensobjekten selbst realisiert. In der Klasse Unternehmen gibt es die Methode getROI(), die den relativen ROI für das Unternehmen gemäß \ref{lis:fachkonzept-implementierung-spiellogik-fehlerkosten} von \ref{sub:spielwelt-logik-bewertung}.

\begin{programm}[ht]
\begin{lstlisting}[breaklines=true]
public double getROI() {
  double rOI = finanzen.getKontostand();
  for (BauteilTyp bauteilTyp : spielWelt.getBauteilMarkt().getTypen()) {
    rOI += lager.getAnzahl(bauteilTyp) * bauteilTyp.getPreis() * 0.5;
  }
  for (RaumschiffTyp raumschiffTyp : spielWelt.getRaumschiffMarkt().getTypen()) {
    rOI += lager.getAnzahl(raumschiffTyp) * raumschiffTyp.getKosten() * 0.75;
  }
  rOI = (rOI - finanzen.getStartKapital()) / finanzen.getStartKapital();
    return rOI;
  }
\end{lstlisting}
\caption{Die Methode berechneFehlerhafteMenge() der Klasse ProduktionsAbteilug\label{lis:fachkonzept-implementierung-spiellogik-fehlerkosten}}
\end{programm}
\subsection{Umsetzung der Spiellogik}
\label{sub:fachkonzept-implementierung-spiellogik}

Die Umsetzung der Spiellogik (siehe \ref{sec:spielwelt-logik} konnte in Java direkt in den jeweiligen Klassen vollzogen werden, da die dahinter liegenden Algorithmen unabhängig von der gewählten Programmiersprache sind. Es mussten allerdings oft die Methoden der eigenen Klasse Util (siehe KAPITEL XXXYYYYZZZ) zu Hilfe gezogen werden, um Beispielsweise eine Gruppierung der Bauteile zu realisieren. Nachfolgend wird auf die jeweiligen Programmteile der entsprechenden Klassen eingegangen.

\subsubsection{Flexibilität der Datenbasis}
\label{subsub:spielwelt-datenbasis-einleitung-fleibilität}

Da die Auswirkungen der Standardwerte auf den gesamten Spielverlauf im Vorhinein eines ersten Spieltests kaum absehbar sind, muss gewährleistet sein, dass die Werte innerhalb der Projektdurchführung angepasst werden können. Auch verworfene oder neue Ideen bringen oft eine Veränderung der Werte mit sich, weshalb auch hier die Notwendigkeit der Flexibiltät gegeben ist.


\subsubsection{Flexibilität der Datenbasis}
\label{subsub:spielwelt-datenbasis-einleitung-fleibilität}

Da die Auswirkungen der Standardwerte auf den gesamten Spielverlauf im Vorhinein eines ersten Spieltests kaum absehbar sind, muss gewährleistet sein, dass die Werte innerhalb der Projektdurchführung angepasst werden können. Auch verworfene oder neue Ideen bringen oft eine Veränderung der Werte mit sich, weshalb auch hier die Notwendigkeit der Flexibiltät gegeben ist.


\subsubsection{Flexibilität der Datenbasis}
\label{subsub:spielwelt-datenbasis-einleitung-fleibilität}

Da die Auswirkungen der Standardwerte auf den gesamten Spielverlauf im Vorhinein eines ersten Spieltests kaum absehbar sind, muss gewährleistet sein, dass die Werte innerhalb der Projektdurchführung angepasst werden können. Auch verworfene oder neue Ideen bringen oft eine Veränderung der Werte mit sich, weshalb auch hier die Notwendigkeit der Flexibiltät gegeben ist.


\subsubsection{Flexibilität der Datenbasis}
\label{subsub:spielwelt-datenbasis-einleitung-fleibilität}

Da die Auswirkungen der Standardwerte auf den gesamten Spielverlauf im Vorhinein eines ersten Spieltests kaum absehbar sind, muss gewährleistet sein, dass die Werte innerhalb der Projektdurchführung angepasst werden können. Auch verworfene oder neue Ideen bringen oft eine Veränderung der Werte mit sich, weshalb auch hier die Notwendigkeit der Flexibiltät gegeben ist.

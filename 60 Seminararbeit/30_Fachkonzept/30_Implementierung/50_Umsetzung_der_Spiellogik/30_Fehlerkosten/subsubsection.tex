\subsubsection{Fehlerkosten}
\label{subsub:fachkonzept-implementierung-spiellogik-fehlerkosten}

In der Klasse ProduktionsAbteilung ist die Berechnung der Fehlerkosten implementiert. Hierfür dient die Methode \textit{berechneFehlerhafteMenge()} (siehe \ref{lis:fachkonzept-implementierung-spiellogik-fehlerkosten}), worin zufällige Zahlen von 0 bis <1 mit der durchschnittlichen Qualität des Personals verglichen werden. Eine durchschnittliche Qualität von 80\% entspricht dabei beispielsweise einem Wert von 0,8. 

\begin{programm}[htbp]
\begin{lstlisting}[breaklines=true]
private int berechneFehlerhafteMenge(int menge) {
  int fehlerhafteMenge = 0;
  for (int i=0; i<menge; i++) {
    if (Math.random() > unternehmen.getPersonal().getDurchschnittlicheQualitaet()) {
	  fehlerhafteMenge++;
	}
  }
  return fehlerhafteMenge;
}
\end{lstlisting}
\caption{\textit{berechneFehlerhafteMenge()} der Klasse ProduktionsAbteilug\label{lis:fachkonzept-implementierung-spiellogik-fehlerkosten}}
\end{programm}

In der Methode \textit{simuliere()} der Klasse ProduktionsAbteilung wird für jeden Produktionsauftrag die entsprechende fehlerhafte Menge mit den dazugehörigen Fehlerkosten verrechnet. 





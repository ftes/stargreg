\section{Einleitung}
\label{sec:junit-einleitung}

\autorbeginn{Fredrik}

\newcommand{\junit}{\textit{jUnit}}

Eine der zentralen Anforderungen an die Implementierung des Unternehmensplanspiels war die Verwendung von \junit Tests. Diese sollen die Funktionalität des Codes verifizieren und für einen schnellen Überblick über die Funktionsweise dies Spiels sorgen. Zusätzlich tragen existierende \junit Tests zu einer effizienteren Programmierfortschritt bei, da anlehnend an die testgetriebene Entwicklung die Validität jeder Veränderung daran geprüft werden kann, ob festgelegte Endzustände nach wie vor erreicht werden.

Für Star Greg wurden einerseits \junit Tests erstellt, die die kritischen Abschnitte der Spiel-Logik, die nicht als trivial außer Acht gelassen werden können, verfizieren, andererseits gibt ein Test des gesamten Spielablaufs die Möglichkeit nachzuvollziehen, wie ein Spielverlauf aussehen kann, und welche Funktionalitäten auf welche Art und Weise verwendet werden. Diese Testfälle wurden in einem seperaten \textit{test}-Paket angelegt und orientieren sich in der Namensgebung an den Klassen, die sie verifizieren sollen. In den folgenden zwei Abschnitten werden diese beiden Felder detaillierter dargestellt.

\autorende{}
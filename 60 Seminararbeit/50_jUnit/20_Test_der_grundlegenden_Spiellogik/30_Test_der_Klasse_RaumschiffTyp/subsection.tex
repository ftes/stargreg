\subsection{Test der Klasse RaumschiffTyp}
\label{sub:junit-spiellogik-raumschifftyp}  

In der Klasse RaumschiffTyp muss getestet werden, ob die Kosten und die Lagerplatzeinheiten eines Raumschiffs richtig berechnet werden, da diese sich aus den Kosten der einzelnen Bauteiltypmengen der Raumschiffe ergeben.
\\
Für die Testmethoden werden verschiedene Bauteiltypen und Raumschifftypen deklariert. Anschließend folgt eine Initialisierung in der \textit{setUpBeforeClass()} Methode für beide Typen nach den Werten der Datenbasis.
\\
Die mengenmäßige Zuteilung der Bauteile erfolgt in den Testmethoden selbst. In der Testmethode \textit{testGetKosten()} folgt nach der Zuteilung ein Vergleich der erwarteten und der tatsächlichen Kosten (siehe \ref{sub:junit-spiellogik-raumschifftyp}).

\begin{programm}[ht]
\begin{lstlisting}[breaklines=true]
public void testGetKosten(){
  [..]
  Assert.assertTrue(corellian_corvette.getKosten() == (rumpf.getPreis()*38 + hitzeschild.getPreis()*16 + triebwerk.getPreis()*6));
}
\end{lstlisting}
\caption{\textit{testGetKosten()} der Klasse RaumschiffTypTest\label{sub:junit-spiellogik-raumschifftyp}}
\end{programm}

In der zweiten Testmethode wird getestet, ob die Raumschifflagerkosten den summierten Lagerkosten der zugehörigen Bauteile entsprechen (siehe \ref{sub:junit-spiellogik-raumschifftyp-1}).
 
\begin{programm}[ht]
\begin{lstlisting}[breaklines=true]
public void testGetLagerplatzEinheiten() {
  [..]
  Assert.assertTrue(millenium_falke.getLagerplatzEinheiten() == (rumpf.getLagerplatzEinheiten()*40 + hitzeschild.getLagerplatzEinheiten()*30 + triebwerk.getLagerplatzEinheiten()*10));
}
\end{lstlisting}
\caption{\textit{testGetLagerkosten()} der Klasse RaumschiffTypTest\label{sub:junit-spiellogik-raumschifftyp-1}}
\end{programm}

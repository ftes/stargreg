\subsection{Test der Klasse BauteilTyp}
\label{sub:junit-spiellogik-bauteiltyp}  

\autorbeginn{Jan}

Es gibt in der Klasse BauteilTyp muss die Methode \textit{berechnePreis()} getestet werden, um sicherzustellen, dass das die richtige Preisberechnung nach einer Spielrunde auch gewährleistet ist.
\\
In der Klasse BauteilTypTest wird hierfür zunächst der Bauteiltyp Rumpfbauteil deklariert, der anschließend in der Methode \textit{setUpBeforeClass()} mit den Werten der Datenbasis initialisiert. In der Testmethode \textit{testBerechnePreis()} wird dann auf dem Rumpfbauteil die zu testende Methode berechnePreise() ausgeführt. D die Methode \textit{berechnePreis()} und ein Vergleich mit dem erwarteten Wert folgt (siehe \ref{sub:junit-spiellogik-bauteiltyp}).

\begin{programm}[htbp]
\begin{lstlisting}[breaklines=true]
public void testBerechnePreis() {
  rumpfbauteil.berechnePreis(.5);
  Assert.assertTrue(Math.round(rumpfbauteil.getPreis()) == 130);
}
\end{lstlisting}
\caption{\textit{testBerechnePreis()} der Klasse BauteilTypTest\label{sub:junit-spiellogik-bauteiltyp}}
\end{programm}

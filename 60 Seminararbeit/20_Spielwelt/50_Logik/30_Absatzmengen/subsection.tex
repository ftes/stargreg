%###
\subsection{Absatzmengen}
%###
\label{sub:spielwelt-logik-absatzmengen}

Nachdem alle Unternehmen die aktuelle Spielrunde eingecheckt haben, wird das Spiel simuliert. In der Simulationsphase
wird unter anderem der Absatz berechnet, der jedem Unternehmen zusteht. Derer günstigste Anbieter soll eine höhere
Anzahl an Raumschiffen absetzen können, als der mit dem höchsten Preis.

Um diese Vorgabe umzusetzen gibt es verschiedene Herangehensweisen. So könnte man einerseits die Angebote der Höhe nach
anordnen und den einzelnen Rängen konstante Prozentsätze zuordnen. Beispielsweise würde man dem günstigsten Anbieter
einen Absatz von 40\% seiner produzierten Raumschiffe zusichern, dem Zweitgünstigsten 25\%, dem Drittgrünstigsten 17,5\%,
dem Viertgünstigsten 12,5\%, und so weiter. Diese Methode bringt jedoch fast nur Nachteile mit sich. So würde der günstigste
Anbieter verglichen mit dem Zweitgünstigsten in jedem Fall 15\% mehr Raumschiffe absetzen können, auch wenn sich der Preis
nur geringfügig unterscheidet. Angenommen man würde dem teuersten Anbieter einen Absatz von 2\% zusichern, dann wär es möglich
den Verkaufspreis viel zu hoch anzusetzen und trotzdem einen Absatz und somit einen hohen Gewinn zu erzielen. Aus diesen
Gründen haben wir uns dafür entschieden, den Absatz anhand einer Potenzfunktion darzustellen:

\begin{equation}
     a_i = \frac{1}{p_i^3}
     \label{alg:spielwelt-logik-absatzmengen-1}
\end{equation}

Da die einzelnen Raumschifftypen verschiedene Nachfrager ansprechen und das Unternehmen verschieden hohe Preise für die
verschiedenen Raumschifftypen festlegen kann, wird dies für jeden der 3 Raumschifftypen abgebildet.

SCHAUBILD

Diese Potenzfunktion zeigt welche Unternehmen wie viele Raumschiffe zu welchem Preis abgesetzen können. Der asymptotische
Verlauf der Kurve sorgt dafür, dass Anbietern mit niedrigen Preisen mehr Absatz zusteht, als Anbietern mit hohen Preisen.

Um für jeden Anbieter den prozentualen Absatz zu bestimmen, wird der individuelle Absatz relativ zum Gesamtabsatz berechnet:

\begin{equation}
     a_{rel. i} = \frac{a_i}{\sum a_i}
     \label{alg:spielwelt-logik-absatzmengen-2}
\end{equation}

Dadurch, dass man durch die Berechnung des relativen Absatzes die verschiedenen Preise der Unternehmen mit einbezieht, hat
diese Variante den Vorteil, dass geringe Preisunterschiede auch nur geringe Abweichungen beim Absatz zulassen. Außerdem hat
ein Unternehmen, das einen unverhältnismäßig hohen Preis festgelegt hat, keine Chance Raumschiffe abzusetzen. Daher spiegelt
diese Variante einen sehr realistischen Absatzmarkt wider.

Die Nachfrage bestimmt sich wie folgt:

% Achtung: Umlaute in Equation gehen nicht, entweder mit \textit{} oder garnicht
\begin{equation}
     N{neu} = N{alt} \cdot (1 - (\frac{p_{guenstigster}}{kosten \cdot 3,5})^4)
     \label{alg:spielwelt-logik-absatzmengen-3}
\end{equation}

Über diese Formel wird unter Berücksichtigung der Bauteilkosten die Nachfrage bestimmt. Auch diese Funktion wurde implementiert
um eine möglichst realistische Situation herzustellen. Würden in der Realität die Preise aller Anbieter eine bestimmte Grenze
überschreiten, so wäre niemand mehr bereit ein Raumschiff zu kaufen. Diese Grenze wurde im Planspiel auf das 3,5-fache der Kosten
gesetzt. Das heißt, dass ein Raumschiff, das mit Kosten von 6000\curr{} produziert wurde, ab einem Preis von 21.000\curr{} keine
Nachfrage mehr erzielt.

Mit Hilfe dieser Funktionen wurde der Absatzalgorithmus implementiert. Er kümmert sich darum, dass der Spieler mit dem niedrigsten
Preis die Anzahl an Raumschiffen bis zu seinem errechneten relativen Absatz verkaufen darf. Nach aufsteigendem Preis wird auf
diese Weise weiter vorgegangen. Hat ein Unternehmen jedoch nicht ausreichend Raumschiffe produziert, kann es also dem ihm eigentlich
zustehenden Absatz nicht nachkommen, so geht die übrige nicht befriedigte Nachfrage auf den nächst günstigeren Anbieter über.

\bigskip

Anhand eines Beispiels soll die Funktionsweise des Absatzalgorithmus verdeutlicht werden:

BEISPIEL inkl. FUNKTIONSGRAPH
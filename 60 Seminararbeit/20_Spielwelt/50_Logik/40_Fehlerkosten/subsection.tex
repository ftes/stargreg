%###
\subsection{Fehlerkosten}
%###
\label{sub:spielwelt-logik-fehlerkosten}

Da es, wie in den vorherigen Kapitel genauer beschrieben, mehrere Personaltypen gibt,
die sich in ihrer Qualität unterscheiden und je nach Qualitätsstufe unterschiedliche
Kosten bei der Produktion verursachen, musste ein Algorithmus definiert werden, der diese
Mehrkosten in jeder Spielrunde ermittelt.

Zur Ermittlung der Mehrkosten werden Reparaturkosten, die für die Reparatur eines
fehlerhaften Raumschiffs anfallen, herangezogen. Das heißt also, wenn ein Unternehmen
ein fehlerhaftes Raumschiff produziert, wird das Konto des Unternehmens mit Mehrkosten,
die für die Reparatur oder Behebung des Fehlers anfallen, belastet. Diese Kosten belaufen sich
auf 50\% der Produktionskosten.

Andere Überlegungen waren beispielsweise der Ausschuss ganzer Raumschiffe oder der Ausfall
einzelner Bauteile. Vergleicht man den Ausschuss bei der Raumschiffproduktion beispielsweise
mit der der Automobilproduktion, so stellt man fest, dass es beim Zusammenbau einzelner
Fertigbauteile nicht zum Ausschuss eines kompletten Endprodukts kommen kann. Daher erschien
uns dieser Ansatz zum einen äußerst unrealistisch und zum anderen würde der Ausschuss eines
ganzen Raumschiffs aufgrund des hohen Produktionspreises einen zu hohen wirtschaftlichen Nachteil
für das jeweilige Unternehmen darstellen. Um das Planspiel nicht zu sehr von Zufallsereignissen
abhängig zu machen wurde daher auf diese Variante verzichtet.

Auch wenn der Ausfall einzelner Bauteile sehr realistisch erscheint und auch finanziell für die
Unternehmen zu verkraften wäre, würden sich bei dieser Variante andere Probleme ergeben. So ist es
einem Unternehmen möglich Fertigbauteile einzukaufen um daraus Raumschiffe zu produzieren. Fällt
nun ein Teil aus, sind möglicherweise nicht genügend Bauteile auf Lager, um das Raumschiff trotz
des Ausfalls eines Bauteils produzieren zu können. In der Realität hätte das Unternehmen die Chance,
Teile kurzfristig dazuzukaufen, wohingegen beim Planspiel das Unternehmen nach dem Einchecken der
Spielrunde keine Möglichkeit hat etwas am Einkauf oder der Produktion für die jeweilige Runde zu ändern.
Aus diesem Grund wurde auch diese Variante als mangelhaft eingestuft und man hat sich auf die am besten
ins Planspiel eingliederbare und im nachfolgenden Abschnitt beschriebene Variante der Reparaturkosten
entschieden.

Aufgrund der Tatsache, dass sich das Personal eines Unternehmens aus allen drei Personaltypen
zusammensetzen kann, ist zunächst die durchschnittliche Personalqualität, des im Unternehmen
eingesetzten Personals zu bestimmen:

\begin{equation}
     \overline{q} = \frac{1}{n} \cdot \sum\limits_{i=1}^n {q_i \cdot n_i}
     \label{alg:spielwelt-logik-fehlerkosten-1}
\end{equation}

Mit dieser Formel wird die Durchschnittsqualität des Personals als gewichtetes arithmetisches Mittel
berechnet. Die daraus erhaltenen Werte liegen in jedem Fall im Bereich 0,69 - 0,99.

Mit Hilfe dieser Information ist man nun in der Lage, die durch das Personal verursachte fehlerhaft
produzierte Menge zu ermitteln. Zur Berechnung wird die in Java vorgefertigte Math.random()-Funktion
der Klasse random verwendet. Diese Funktion generiert für jedes in Produktion gegebene Raumschiff eine
Zufallszahl zwischen 0,0 und 1,0. Sollte die erzeugte Zufallszahl den Wert der Durchschnittsqualität
des Personals überbieten, so wird dieses Raumschiff als fehlerhaft eingestuft. Ist die Qualität des
Personals also höher, so erhöht sich die Wahrscheinlichkeit fehlerfreie Raumschiffe zu produzieren.
Wiederholt man diesen Vorgang für jedes produzierte Raumschiff, so lassen sich auf diese Weise die
anfallenden Mehrkosten errechnen.

\bigskip

An einem vereinfachten Beispiel veranschaulicht, könnte dies wie folgt aussehen:

Das Unternehmen “Weedman Ships” beschäftigt 13 R2D2, 18 Kampfdroiden und 9 Droiden des Typs Droideka.
Mit diesem Personal will das Unternehmen einen X-Wing , zwei Corellian Corvettes und ein Raumschiff
des Typs Millenium Falke produzieren.

Mit diesen Angaben lässt sich zunächst die durchschnittliche Personalqualität des Unternehmens ermitteln:

% Achtung: Nonumber, Beispiele gehören nicht ins Formelverzeichnis
\begin{equation}
     \frac{13 \cdot 0,69 + 18 \cdot 0,84 + 9 \cdot 0,99}{40} \approx 0,78 \nonumber
     \label{alg:spielwelt-logik-fehlerkosten-Beispiel}
\end{equation}

Wie in \ref{tab:spielwelt-logik-fehlerkosten-beispiel} dargestellt, werden nun Zufallszahlen generiert um
festzustellen, ob bei der Produktion Fehler auftraten.

\medskip

\begin{table}[htb]
     \centering
     \begin{tabular}{ | l | l | l | }
          \hline
          Raumschiff & Zufallszahl & Fehlerhaft? \\
          \hline \hline
          X-Wing & 0,3 & Nein \\ \hline
          Corellian Corvette & 0,9 & Ja \\ \hline
          Corellian Corvette & 0,7 & Nein \\ \hline
          Millenium Falke & 0,1 & Nein \\ \hline
     \end{tabular}
     \caption{Beispiel zur Berechnung der Fehlerkosten}
     \label{tab:spielwelt-logik-fehlerkosten-beispiel}
\end{table}

Beim ersten Raumschiff erzeugt die Funktion Math.random() die Zufallszahl 0,3. Da diese die Personalqualität
von 0,78 nicht überbietet, liegt kein Fehler vor. Dagegen weist eine der Corellian Corvettes einen Fehler auf.
Die zweite Corellian Corvette und der Millenium Falke wurden wieder fehlerfrei produziert.
Aufgrund der nur mittelmäßigen Qualität des Personals liegen bei einem von vier Raumschiffen Fehler vor. Hätte das
Unternehmen verstärkt in ihr Personal investiert und in Form von Aufrüstung oder Neueinstellungen die Qualität verbessert,
so hätte dieser Fehler möglicherweise vermieden werden können.

Um in diesem Beispiel den Fehler bei der Corellian Corvette zu beheben, würden bei Produktionskosten von 12000\curr{}
Reparaturkosten von 6000\curr{} anfallen.
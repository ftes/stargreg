%###
\subsection{Einleitung}
%###
\label{sub:spielwelt-logik-einleitung}

\autorbeginn{Fredrik}

Bereits sehr früh in der Planungsphase des Projekts kam der Wunsch auf, das zu erstellende Planspiel
durch den Einsatz flexibler Spielteile möglichst realitätsnah zu gestalten. Hierbei sollten nicht
alle Werte wie z.B. Preise für die Dauer des Spiels fest vorgegeben sein, sondern anhand von
dynamischen Einflüssen zur Spielzeit schwanken. Dies macht das Spiel für den Spieler einerseits
interessanter, da er die Auswirkungen seiner Entscheidungen und derer seiner Mitspieler auf die
Spielwelt beobachten kann. Andererseits erhöht dies auch den Lerneffekt, da Spieler die möglichen
Auswirkungen mit in ihre Entscheidungsfindung einbauen und somit verantwortungsvoller handeln
müssen.

Um das Planspiel in der vorgegebenen Zeit realisieren zu können, war es nötig, einige Bereiche für
diese dynamische Umsetzung auszuwählen und andere außen vor zu lassen. Dabei wurde die Entscheidung
gefällt, sowohl die Bauteilpreise, die Absatzmengen auf dem Raumschiffmarkt und die Fehlerkosten
flexibel zu gestalten. Nicht in die Betrachtung einbezogen wurden beispielsweise die Personalkosten
und die Grundnachfrage auf dem Raumschiffmarkt, die sich stattdessen abhängig von der
Spiel-Geschichte entwickeln, um so die Spannung für die Spieler zu erhöhen.

Die folgenden Abschnitte geben einen Überblick über die Berechnungsvorschriften und Abläufe bei der
flexiblen Berechnung der oben erwähnten Werte, woran sich zudem die Beschreibung der Bewertung der
Unternehmen am Ende des Spiels anschließt.


\subsection{Bewertung}
\label{sub:spielwelt-logik-bewertung}

Eine Bewertung der einzelnen Spieler findet am Ende des Planspiels statt. Sie errechnet sich zu 70\% aus dem
ROI der Unternehmen und zu 30\% aus dem Marktanteil. Der ROI war für uns hierbei die wichtigste Kennzahl um die
erfolgreiche Geschäftsführung und damit den Erfolg der Unternehmen zu messen. Da es für ein Unternehmen allerdings
auch wichtig ist, sich am Markt zu etablieren und ein gutes Image aufzubauen, haben wir uns dafür entschieden den
Marktanteil mit in die Endauswertung einfließen zu lassen.

Der ROI ist eine Gegenüberstellung von Gewinn und eingesetztem Kapital:

\begin{equation}
     ROI = \frac{Gewinn}{eingesetztes Kapital}
     \label{alg:spielwelt-logik-bewertung-ROI}
\end{equation}

Der Gewinn berechnet sich wie folgt:
\begin{equation}
     Gewinn = Endkapital - Startkapital + FE_{im Lager} \cdot 0,75 + Bauteile_{im Lager} \cdot 0,5
     \label{alg:spielwelt-logik-bewertung-Gewinn}
\end{equation}

Um das Ergebnis nicht zu verzerren, werden Raumschiffe und Bauteile im Lager mit berücksichtigt und dem Gewinn hinzuaddiert.
Produzierte Raumschiffe, die sich im Lager befinden, werden zu 75\% und gekaufte Bauteile zu 50\% ihres Wertes mit eingerechnet.
Das eingesetzte Kapitel ist im Planspiel in Form des Startkapitals fest vorgegeben und für jeden Spieler gleich.

Jedoch ist zu Berücksichtigen, dass nur mit positiven ROI-Werten gerechnet werden darf. Aus diesem Grund ist es notwendig
beim Auftreten einer oder mehrerer negativen ROI-Werten alle ROI's in soweit zu erhöhen, dass sich der niedrigste ROI bei
0 befindet. Auf diese Weise lässt sich nun der ROI anteilig berechnen:

\begin{equation}
     ROI_{rel.} = \frac{ROI_{korrigiert}}{\sum ROI_{korrigiert}}
     \label{alg:spielwelt-logik-bewertung-rel. ROI}
\end{equation}

Beim Marktanteil handelt es sich um eine Gegenüberstellung des Umsatzes eines Unternehmen und der Umsätze aller Unternehmen:

\begin{equation}
     Marktanteil = \frac{Umsatz}{\sum Umsatz}
     \label{alg:spielwelt-logik-bewertung-Marktanteil}
\end{equation}

Mit Hilfe des anteiligen ROI's und des Marktanteils können nun Punkte berechnet werden:

\begin{equation}
     Punkte = ROI_{rel.} \cdot 70 + Marktanteil \cdot 30
     \label{alg:spielwelt-logik-bewertung-Punkte}
\end{equation}

Anhand dieser Punkte wird eine Rangfolge der Unternehmen aufgestellt. Das Unternehmen mit den meisten Punkten gewinnt das
Spiel.

\bigskip

Für die Bewertung hätte es noch andere Möglichkeiten gegeben. So wurde in der Gruppe zu Beginn des Projekts diskutiert,
ob man die Endauswertung nicht anhand von Medaillen umsetzt, die die Spieler in jeder Spielrunde “gewinnen” können. Auf
diese Weise hätte es beispielsweise eine Medaille für den größten Marktanteil oder den besten ROI der Runde gegeben. Mit
dieser Variante hätte man die Spieler anhand von Zwischenbewertungen mit zusätzlichen Informationen zur momentanen
Unternehmenslage versorgen können.

\autorende{}
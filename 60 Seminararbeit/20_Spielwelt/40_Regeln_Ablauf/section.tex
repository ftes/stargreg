\section{Spielablauf}
\label{sec:spielwelt-regeln}

Der idealtypische Verlauf von Star Greg lässt in vier Phasen unterteilen (vgl. Abbildung 2). Anhand der Beschreibung eines beispielhaften Rundenverlaufs wird dargestellt, was in den einzelnen Phasen geschieht und was der Spieler entscheiden kann.

\\
\\
BILD!!!
\\
\\

{\Large 1. Informationsphase}
\\
\\
{\large Spieleinführung}
\\
Der Spieler startet Star Greg und bekommt zunächst eine Spieleinführung, in der einerseits die Welt als solche, deren Akteure und schließlich die von ihm einzunehmende Rolle innerhalb dieser Welt beschrieben wird. Aufgabe des Spielers ist es, im Weltall ein Unternehmen zu führen, das Raumschiffe produziert, und neben seinen Spielgegner auf dem Oligopolmarkt zu existieren oder gar die Marktmacht zu erringen. Die Entscheidungsmöglichkeiten, sowie die rundenbasierte Struktur des Spiels, werden in den Spielregeln offengelegt. Die Spieleinführung ist für alle Teilnehmer gleich. So werden mögliche Verständnisfragen geklärt, was wichtig ist, um allen Spielern die optimalen und insbesondere gleichen Voraussetzungen zu gewährleisten. 
\\
\\
{\large Start des Spiels}
\\
Zu Beginn legt der Spieler seinen Unternehmensnamen und das Unternehmenslogo fest, unter den er im Verlauf des Spiels identifiziert wird. Das Startkapital liegt für jeden Teilnehmer bei einer Million Euro. Falls außerordentliche Vorkommnisse vorliegen, wird der Spieler noch bevor er im Spiel aktiv wird, darüber informiert. Ebenso kann er sich an dieser Stelle, sofern er bereits eine Runde gespielt hat, einen Überblick über die Daten und Fakten aus seiner letzten Runde verschaffen. Diese Informationen wurden bewusst auf ein Minimum reduziert, um dem Spieler einerseits ein Gefühl zu geben, wie der Markt auf seine Entscheidungen reagiert, aber andererseits nicht zu viele Informationen über den Erfolg anderer Teilnehmer zu offenbaren:

\begin{itemize}
\item[•] 'Star der letzten Runde': Das Raumschiff, das den höchsten Umsatz erzielt hat
\item[•] Anzahl der verkauften Raumschiffe
\item[•] ROI
\item[•] Marktanteil im Verhältnis zur Branche
\end{itemize}

In der ersten Runde liegen diese Daten allerdings noch nicht vor! 
\\
\\
{\Large 2. Steuerungsphase}
\\
Nun muss sich der Spieler aktiv werden, Handlungspotentiale und Möglichkeiten erkennen und abwägen und schließlich Entscheidungen treffen. Dazu werden von ihm zunächst Bauteile eingekauft. Beim Kauf orientiert er sich an drei Größen: 

\begin{itemize}
\item[•] momentane Bauteilpreise
\item[•] Vorstellung über die Produktionsmenge der einzelnen Raumschifftypen
\item[•] Lagerkosten 
\end{itemize}

Demnach sollten also die Bauteile so eingekauft werden, dass eine gewisse Menge des Raumschifftyps produziert werden kann, der momentan am günstigsten liegt. Zu beachten sei außerdem, dass die im nächsten Schritt angegebene Produktionsmenge, auch gleichzeitig die am Markt angebotene Menge ist, da stets der gesamt Bestand an fertigen Raumschiffen hierbei relevant ist. So fallen in der nächsten Runde im schlimmsten Fall, also wenn kein einziges Raumschiff verkauft werden konnte, Lagerkosten für alle produzierten Raumschiffe an.  
\\
Im letzten Schritt dieser Phase müssen die Preise festgelegt werden. Da der Spieler den Marktpreis nicht kennt, muss er auf Grundlage seiner Herstellkosten und seiner Gewinnerzielungsabsicht einen hierfür geeigneten Preis abwägen. An dieser Stelle wird die als optionale Erweiterung festgelegte und bislang nicht umgesetzte Marktforschung relevant.  Sie soll es ermöglichen, Informationen über den Markt und die Konkurrenz gegen zusätzlichen Aufwand zu erhalten. So könnte sich der Spiele gerade auch die Preisfindung erleichtern. 
\\
\\
{\Large 3. Interaktionsphase}
\\
Nachdem der Spieler in der Steuerungsphase seine Strategien analysiert und durchdacht hat, gibt er in der Interaktionsphase nun seine endgültigen Entscheidungen ab. Er setzt eine Produktionsmenge und einen zugehörigen Preis fest, und gibt über das Einchecken dieser Daten ein verbindliches Angebot am Markt ab. 





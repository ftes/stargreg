\subsection{Steuerungsphase}
\label{sub:spielwelt-regeln-steuerungsphase}


Nun muss der Spieler aktiv werden, Handlungspotentiale und Möglichkeiten erkennen und abwägen und schließlich Entscheidungen treffen. Dazu werden von ihm zunächst Bauteile eingekauft. Beim Kauf orientiert er sich an drei Größen: 

\begin{seList}
\item momentane Bauteilpreise
\item Vorstellung über die Produktionsmenge der einzelnen Raumschifftypen
\item Lagerkosten 
\end{seList}

Demnach sollten also die Bauteile so eingekauft werden, dass eine gewisse Menge des Raumschifftyps produziert werden kann, der momentan am günstigsten liegt. Zu beachten sei außerdem, dass die im nächsten Schritt angegebene Produktionsmenge, auch gleichzeitig die am Markt angebotene Menge ist, da stets der gesamt Bestand an fertigen Raumschiffen hierbei relevant ist. So fallen in der nächsten Runde im schlimmsten Fall, also wenn kein einziges Raumschiff verkauft werden konnte, Lagerkosten für alle produzierten Raumschiffe an.  
\\
\\
Im letzten Schritt dieser Phase müssen die Preise festgelegt werden. Da der Spieler den Marktpreis nicht kennt, muss er auf Grundlage seiner Herstellkosten und seiner Gewinnerzielungsabsicht einen hierfür geeigneten Preis abwägen. An dieser Stelle wird die als optionale Erweiterung festgelegte und bislang nicht umgesetzte Marktforschung relevant.  Sie soll es ermöglichen, Informationen über den Markt und die Konkurrenz gegen zusätzlichen Aufwand zu erhalten. So könnte sich der Spiele gerade auch die Preisfindung erleichtern. 







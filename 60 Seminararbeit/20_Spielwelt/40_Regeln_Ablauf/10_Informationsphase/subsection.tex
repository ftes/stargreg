\subsection{Informationsphase}
\label{sub:spielwelt-regeln-informationsphase}



\textbf{Spieleinführung}
\\
Der Spieler startet Star Greg und bekommt zunächst eine Spieleinführung, in der einerseits die Welt als solche, deren Akteure und schließlich die von ihm einzunehmende Rolle innerhalb dieser Welt beschrieben wird. Aufgabe des Spielers ist es, im Weltall ein Unternehmen zu führen, das Raumschiffe produziert, und neben seinen Spielgegner auf dem Oligopolmarkt zu existieren oder gar die Marktmacht zu erringen. Die Entscheidungsmöglichkeiten, sowie die rundenbasierte Struktur des Spiels, werden in den Spielregeln offengelegt. Die Spieleinführung ist für alle Teilnehmer gleich. So werden mögliche Verständnisfragen geklärt, was wichtig ist, um allen Spielern die optimalen und insbesondere gleichen Voraussetzungen zu gewährleisten. 
\\
\\

\textbf{Start des Spiels}
\\
Beginn legt der Spieler seinen Unternehmensnamen und das Unternehmenslogo fest, unter den er im Verlauf des Spiels identifiziert wird. Das Startkapital liegt für jeden Teilnehmer bei einer Million Euro. Falls außerordentliche Vorkommnisse vorliegen, wird der Spieler noch bevor er im Spiel aktiv wird, darüber informiert. Ebenso kann er sich an dieser Stelle, sofern er bereits eine Runde gespielt hat, einen Überblick über die Daten und Fakten aus seiner letzten Runde verschaffen. Diese Informationen wurden bewusst auf ein Minimum reduziert, um dem Spieler einerseits ein Gefühl zu geben, wie der Markt auf seine Entscheidungen reagiert, aber andererseits nicht zu viele Informationen über den Erfolg anderer Teilnehmer zu offenbaren:

\begin{itemize}
\item[•] 'Star der letzten Runde': Das Raumschiff, das den höchsten Umsatz erzielt hat
\item[•] Anzahl der verkauften Raumschiffe
\item[•] ROI
\item[•] Marktanteil im Verhältnis zur Branche
\end{itemize}

In der ersten Runde liegen diese Daten allerdings noch nicht vor! 

%###
\subsubsection{Zusammensetzung der Raumschiffe}
%###
\label{subsub:spielwelt-datenbasis-raumschiffe-zusammensetzung}

Aus den eingekauften Bauteilen können die Unternehmen Raumschiffe fertigen lassen, wobei jedes Raumschiff dabei eine fest vorgeschriebenen Anzahl von Bauteilen hat. Aufgrund der schwankenden Bauteilpreise schwanken auch die Kosten für die Produktion der Raumschiffe, sofern man keine Bauteile besitzt, die in der letzten Runde eingelagert wurden. In Abbildung \ref{tab:spielwelt-datenbasis-raumschiffe-zusammensetzung} ist die sind die Werte für die einzelnen Typen aufgelistet.

\begin{table}[ht]\small
     \centering
     \begin{tabular}{ | l | c | c | c | c |  }
          \hline
          Raumschifftyp & Rumpfbauteil & Hitzeschild & Triebwerk & Sonderbauteiltyp \\
          \hline \hline
          X-Wing & 18 & \ 6 & \ 4 & Geschütz\\ \hline
          Corellian Corvette & 38 & 16 & \ 6 & Transportkapsel \\ \hline
          Millenium Falke & 40 & 30 & 10 & Forschungsausrüstung \\
          \hline
     \end{tabular}
     \caption{Bauteilmengen der Raumschiffe}
     \label{tab:spielwelt-datenbasis-raumschiffe-zusammensetzung}
\end{table}

Die Preise zur für die Raumschifftypen am Beginn eines Spiels gemessen an den Grundpreisen der Bauteile berechnen sich daher wie folgt:

\begin{itemize}
\item[] X-Wing \qquad\             = 18 * 100,00\curr{} +  6 * 200,00\curr{} +  4 * 500,00\curr{} + 1000,00\curr{} =  6000,00\curr{}
\item[] Corellian Corvette = 38 * 100,00\curr{} + 16 * 200,00\curr{} +  6 * 500,00\curr{} + 2000,00\curr{} = 12000,00\curr{}
\item[] Millenium Falke \quad    = 40 * 100,00\curr{} + 30 * 200,00\curr{} + 10 * 500,00\curr{} + 3000,00\curr{} = 18000,00\curr{}
\end{itemize}

An der Rechnung lässt sich erkennen, dass die jeweiligen Raumschifftypen zu Beginn des Spiels eine Staffelung in ihrem Gesamtpreis aufweisen. Wichtig war es ein Schema zu finden, bei dem jedes der Raumschiffe ein bestimmten Grundbauteiltyp hat, für den der Spieler bei dem jeweiligen Raumschiff am meisten investieren muss. So bezahlt man beim X-Wing anfangs am meisten für Triebwerke (2000,00\curr{}), bei der Corellian Corvette für die Rumpfbauteile (3800,00\curr{}) und beim Millenium Falken für die Hitzeschilde (6000,00\curr{}). Somit ist gewährleistet, dass alle Bauteile preislich gesehen steigen und schwanken können (vgl. Algorithmen). 
\\
\\
Die Lagerkosten für die Raumschiffe ergeben sich aus den Lagerkosten der jeweils verwendeten Bauteile multipliziert mit deren Menge. Da die Lagerplatzpreise/Lagerplatzeinheit am Anfang des Spiels einem Zehntel des jeweilgen Grundbauteils entsprechen, sind die Lagerpreise für die Raumschiffe hier auch noch gestaffelt wie die Raumschiffpreise (siehe \ref{tab:spielwelt-datenbasis-raumschiffe-zusammensetzung-1}). Ein Spieler muss also sehr genau abwägen, ob er seine Raumschiffe wirklich am Raumschiffmarkt verkaufen kann, da die anfallenden Lagerkosten sein Kapital bei Nichtverkauf erheblich schmälern können.

\begin{table}[ht]\small
     \centering
     \begin{tabular}{ | l | c | c | }
          \hline
          Raumschifftyp & Startpreis & Lagerkosten zu Beginn \\
          \hline \hline
          X-Wing &  6000,00\curr{} & \ 600,00\curr{}\\ \hline
          Corellian Corvette & 12000,00\curr{} & 1200,00\curr{}\\ \hline
          Millenium Falke & 18000,00\curr{} & 1800,00\curr{} \\
          \hline
     \end{tabular}
     \caption{Lagerkosten der Raumschifftypen zu Beginn}
     \label{tab:spielwelt-datenbasis-raumschiffe-zusammensetzung-1}
\end{table}

Desweiteren können in der Raumschiffproduktion können durch den Einsatz von qualitativ minderwertigem Personal Zusatzkosten für die Produktion entstehen (siehe 2.3.3 Personal). Diese Zusatzkosten errechnen sich ebenfalls aus den Werten der derzeitigen Bauteilpreise. Dabei wird jedes Bauteil mit dem festen Werten 50\% verrechnet, sodass die Zusatzkosten eines Bauteils meist der Hälfte des Einkaufspreises des Bauteils selbst entspricht (siehe \ref{tab:spielwelt-datenbasis-raumschiffe-zusammensetzung-2}). Auch hier muss der Spieler seine Entscheidungen genau durchdenken, da die Einstellung von minderwertigem Personal zu sehr hohen Zusatzkosten führen kann.

\begin{table}[ht]\small
     \centering
     \begin{tabular}{ | l | c | c | }
          \hline
          Raumschifftyp & Startpreis & Zusatzkosten zu Beginn \\
          \hline \hline
          X-Wing &  6000,00\curr{} & 3000,00\curr{}\\ \hline
          Corellian Corvette & 12000,00\curr{} & 6000,00\curr{}\\ \hline
          Millenium Falke & 18000,00\curr{} & 9000,00\curr{} \\
          \hline
     \end{tabular}
     \caption{Zusatzkosten der Raumschifftypen zu Beginn}
     \label{tab:spielwelt-datenbasis-raumschiffe-zusammensetzung-2}
\end{table}

Anfallende Zusatzkosten sollten von den Spielern tendenziell eher bezahlt als nicht bezahlt werden, weil ansonsten die Gesamtkosten der verwendeten Bauteile und der Personalaufwand für das entsprechende Raumschiff keine Einnahmen generiert haben. Aufgrund dieser Tatsache fiel auch die Entscheidung, dass die Konten in StarGreg von dem jeweiligen Spieler überzogen werden können. So hat ein Spieler, der durch seine Produktion bereits an seiner Kapitalgrenze angelangt ist, trotzdem noch die Möglichkeit, seine reparaturbedürftigen Raumschiffe zum Verkauf bereit zu machen und sie abzusetzen. Der Zinssatz bei Überziehung des Kontos von 25\% ist dabei über die verschiedenen Spielrunden immer der Gleiche.





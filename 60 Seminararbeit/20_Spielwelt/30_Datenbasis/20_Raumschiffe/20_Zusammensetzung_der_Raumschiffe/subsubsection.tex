\subsubsection{Zusammensetzung der Raumschiffe}
\label{subsub:spielwelt-datenbasis-raumschiffe-zusammensetzung}

Aus den eingekauften Bauteilen können die Unternehmen Raumschiffe fertigen lassen, wobei jedes Raumschiff dabei eine fest vorgeschriebenen Anzahl von Bauteilen hat. Aufgrund der schwankenden Bauteilpreise schwanken auch die Kosten für die Produktion der Raumschiffe, sofern man keine Bauteile besitzt, die in der letzten Runde eingelagert wurden. In Abbildung \ref{tab:spielwelt-datenbasis-raumschiffe-zusammensetzung} ist die sind die Werte für die einzelnen Typen aufgelistet.

\begin{table}[ht]
     \centering
     \begin{tabular}{ | l | l | l | l | l   }
          \hline
          Raumschifftyp & Rumpfbauteil & Hitzeschild & Triebwerk & Sonderbauteil \\
          \hline \hline
          X-Wing & 18 & 6 & 4 & Geschütz\curr \\ \hline
          Corellian Corvette & 38 & 16 & 6 & Transportkapsel\curr \\ \hline
          Millenium Falke & 40 & 30 & 10 & Forschungsausrüstung\curr \\
          \hline
     \end{tabular}
     \caption{Preise der Sonderbauteile}
     \label{tab:spielwelt-datenbasis-raumschiffe-zusammensetzung}
\end{table}

Die Preise zur für die Raumschifftypen am Beginn eines Spiels gemessen an den Grundpreisen der Bauteile berechnen sich daher wie folgt:

\begin{itemize}
\item[] X-Wing             = 18 * 100,00\curr{} +  6 * 200,00\curr{} +  4 * 500,00\curr{} + 1000,00\curr{} =  6000,00\curr{}
\item[] Corellian Corvette = 38 * 100,00\curr{} + 16 * 200,00\curr{} +  6 * 500,00\curr{} + 2000,00\curr{} = 12000,00\curr{}
\item[] Millenium Falke    = 40 * 100,00\curr{} + 30 * 200,00\curr{} + 10 * 500,00\curr{} + 3000,00\curr{} = 18000,00\curr{}
\end{itemize}

An der Rechnung lässt sich erkennen, dass die jeweiligen Raumschifftypen zu Beginn des Spiels eine Staffelung in ihrem Gesamtpreis aufweisen. Wichtig war es ein Schema zu finden, bei dem jedes der Raumschiffe ein bestimmten Grundbauteiltyp hat, für den der Spieler bei dem jeweiligen Raumschiff am meisten investieren muss. So bezahlt man beim X-Wing anfangs am meisten für Triebwerke (2000,00\curr{}), bei der Corellian Corvette für die Rumpfbauteile (3800,00\curr{}) und beim Millenium Falken für die Hitzeschilde (6000,00\curr{}).
	





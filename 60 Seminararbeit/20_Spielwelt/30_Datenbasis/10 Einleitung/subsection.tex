\subsection{Einleitung}
\label{sub:spielwelt-datenbasis-einleitung}

\subsubsection{Notwendigkeit der Datenbasis}
\label{subsub:spielwelt-datenbasis-einleitung-notwendigkeit}

Die Datenbasis des Unternehmensplanspiels StarGreg ist die zentrale Anlaufstelle der Projektbeteiligten zur Beschaffung der vordefinierten Standardwerte für die Bauteilpreise, die Lagerkosten und die Peronaltypkosten. Die Motivation zur Erstellung der Datenbasis lag vor allem darin, dass das Bild der Spielwelt transparenter wird und einheitliche Werte für die Implementierung der JUnit Test zur Verfügung stehen. Durch den gemeinsamen Einsatz des Programms git zur verteilten Versionsverwaltung von Datein konnte die Anforderung der Einheitlichkeit erreicht werden. 

\subsubsection{Flexibilität der Datenbasis}
\label{subsub:spielwelt-datenbasis-einleitung-fleibilität}

Da die Auswirkungen der Standardwerte auf den gesamten Spielverlauf im Vorhinein eines ersten Spieltests kaum absehbar sind, muss gewährleistet sein, dass die Werte innerhalb der Projektdurchführung angepasst werden können. Auch verworfene oder neue Ideen bringen oft eine Veränderung der Werte mit sich, weshalb auch hier die Notwendigkeit der Flexibiltät gegeben ist.

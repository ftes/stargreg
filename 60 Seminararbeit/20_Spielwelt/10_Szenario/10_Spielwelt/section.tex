\section{Beschreibung der Spielwelt}
\label{sec:spielwelt-szenario-spielwelt}

Wir befinden uns im Jahre 2271, Sternzeit 7410,2. Automobile oder andere antike Fortbewegungsmittel sind kaum mehr anzutreffen. Das Volk bedient sich nun an Raumschiffen, welche für unterschiedliche Zwecke konzipiert wurden. Zu nennen sind dabei die Forschungs-, Transport- und Militärraumschiffe. Da in der Vergangenheit zunehmend Probleme bei Raumschiffherstellern aufgetreten sind, welche man auf Entscheidungen des höheren Managements zurückführen konnte, ist es unumgänglich dieses mithilfe des rundenbasierten Planspiels Star Greg zu schulen. Die Commander der Raumschiffhersteller finden sich in folgender Situation wieder.

Die wenigen Unternehmen in dieser Galaxis, die Raumschiffe anbieten, sind mit den grundlegensten Abteilungen des 21. Jahrhunderts ausgestattet. Diese umfassen derzeit das Finanzwesen, das Personalwesen, den Einkauf, die Produktion und den Vertrieb. Droiden haben sich seit geraumer Zeit als nützliche und zuverlässige Arbeitnehmer erwiesen und sind daher als Personal in einem solchen Unternehmen unumgänglich. Dank früherer Forschungsarbeiten sind diese in drei verschiedenen Stufen verfügbar wodurch die Personalstruktur allein den Entscheidungen des Commanders unterliegt. Dieser hat zudem die Aufgabe, das Produktionsprogramm der Raumschiffe aufzustellen und die dafür nötigen Bauteile auf dem intergalaktischen Bauteilmarkt zu erwerben. Hierbei ist vorallem die Zusammensetzung der Raumschiffe durch die verschiedenen Bauteile zu beachten. Produzierte Raumschiffe werden dann durch Angabe eines Preises zum Verkauf angeboten. Des weiteren hat der Commander Einblick in die Finanzlage seines Unternehmens wodurch er seine Ein- und Ausgaben bis auf den kleinsten klingonischen Credit überprüfen kann. Hat der Commander seine unternehmerischen Tätigkeiten für diese Runde abgeschlossen, bestätigt er diese und gelangt dadurch die nächste Runde. Hier wird er mit den Verkaufszahlen der vorherigen Runde konfrontiert und muss nun auf ein Neues Entscheidungen treffen und seine unternehmerischen Fähigkeiten auf die Probe stellen. Ist die vorgegebene Anzahl von zehn spannenden und ereignisreichen Runden beendet, so wird der erfolgreichste Commander gekürt und ausgezeichnet.
%
% Ein kleiner Text, um Abk\"urzungen, Symbole und Glossareintr\"age zu testen
%
%
\section{Kommandos f\"ur die Erzeugung von Abk\"urzungen, Symbolen und Glos\-sar\-eint\-r\"a\-gen}

F\"ur Abk\"urzungen, Symbole und Glossareintr\"age wird das Kommando \verb+\gls{par1}+ verwendet.
\texttt{par1} stellt einen Schl\"ussel dar, der die entsprechende Definition identifiziert (vgl. den Inhalt der Datei
\texttt{pa1-abkuerzungen.tex}). 


Eine Abk\"urzung: \gls{usb}; das zweite Auftreten der Abk\"urzung: \gls{usb}. 
Und jetzt kommt ein Symbol: \gls{pi}; das zweite Symbol ist \gls{ND}.

Und auch ein Eintrag im Glossar muss sein: \gls{glos:AD}; das zweite Auftreten des Eintrags ist \gls{glos:AD}.

\newpage
Und auf der n\"achsten Seite: \gls{glos:AD}. Im Glossar ist jeweils angegeben, auf welchen Seiten der 
Begriff verwendet wurde.

\section{Was sind Planspiele?}
\label{sec:einleitung-was}

Planspiele finden ihre Ursprünge schon sehr früh. Laut TOPSIM fanden die ersten Planspiele etwa 1000 v.Chr. in Indien als Kampfspiele statt, in Persien 800 v.Chr. sogar auch im Rahmen von Schachspielen.  Ebenso werden sie verstärkt in den USA schon seit langer Zeit in die militärische Ausbildung integriert, da so Kriegszüge besser geplant und durchgeführt werden können, ohne dabei reale Ressourcen zu verbrauchen. Um 1950 wurde das  erste computergestützte Unternehmensplanspiel durch die American Management Association entwickelt, das zugleich auch ein einfaches Unternehmensplanspiel darstellte. Bis heute schreitet die Entwicklung dieser Planspiele stark voran, sodass sie zunehmend in sehr vielen Bereichen Anwendung finden. So sind sie auch schon an Schulen und Hochschulen wie der Universität Tübingen, fester Bestandteil des Lehrplans. An der Universität Tübingen nehmen Studenten ganz Aktuell im Rahmen eines Seminars an dem Planspiel 'National Model United Nations Conference' (NMUN) teil, was weltweit von Studenten praktiziert wird. Es handelt sich hierbei um die größte Simulation der Arbeit der Vereinten Nationen.
\\ 
Nach Gabler handelt es sich hierbei um eine modellhafte Simulation von Unternehmensprozessen, in der die Spieler jeweils in die Rolle eines Unternehmenschefs agieren. Sehr ursprüngliche Planspiele stammen aus den USA und wurden dort für  militärische Zwecke innerhalb der Ausbildungsmethoden eingesetzt, um beispielsweise Personalentwicklungen verständlicher zu machen. 
\\
Auf der Ebene des Unternehmensplanspiels geht es um das Konkurrieren mehrerer Unternehmen auf einem Oligopolmarkt, die zu Beginn mit gleichen Startbedingungen, d.h. gleicher Betriebsgröße und Finanzstruktur, ausgestattet sind. Im Folgenden müssen die Spieler in der Rolle des Unternehmers innerhalb einer simulierten, modellhaften Realität strategisch und betriebswirtschaftlich Handeln, um ihr Unternehmen auf dem Konkurrenzmarkt aufrecht zu erhalten. Dabei  unterliegt die simulierte Welt verschiedenen dynamischen Konzepten, wie dem Konjunkturverlauf oder dem Zusammenspiel von Angebot und Nachfrage. In der Regel sind Unternehmensplanspiele rundenbasiert aufgebaut, sodass bei jeder neuen Runde Pläne und Teilpläne sorgfältig aufgestellt und danach mehrere Entscheidungen getroffen werden müssen. Die Auswirkungen dieser Entscheidungen werden durch Algorithmen ermittelt. Normalerweise dauert eine Spielrunde bis zu mehrere Stunden, das ganze Unternehmensplanspiel bis zu mehreren Tagen.
\\
Ziel eines Planspiels ist es, den Teilnehmern eine Vorstellung davon zu geben, wie ein Unternehmen als Ganzes funktioniert und wie Entscheidungen zu treffen sind, um im Wettbewerb erfolgreich bestehen zu können. Durch Zwischenbewertungen kann der Spieler sein Verhalten und die daraus resultierenden wirtschaftlichen Konsequenzen und Geschäftsprozesse analysieren und anhand dessen seine strategischen und operativen Entscheidungen optimieren. Wichtig ist hierbei nicht nur, dass der Spieler Situationen erkennt und analysiert, sondern insbesondere auch, dass er die Situation einschätzen kann und lernt, auf die richtig Art und Weise zu reagieren. Das Kriterium der Einschätzung von Entscheidungen ist eine herausfordernde Disziplin, die dann aber die Fähigkeit liefert, zwischen Alternativen richtig wählen zu können.
\\
Der große Vorteil der Unternehmensplanspiele liegt ganz klar in ihrer praktischen Orientierung. Durch Learning-by-doing wird ein interaktiver, handlungsorientierter und zeitgemäßer Lernprozess verwendet, der zu dem Kreativität und Selbstständigkeit zulässt. Durch die Simulation der Spielwelt kann der Teilnehmer - ähnlich wie ein Pilot in einem Flugsimulator - seine Fähigkeiten ohne nennenswertes Risiko ausüben und auf die Probe stellen. Dabei ist der unterhaltsame Aspekt nicht Zweck, sondern "Verstärker" des Lernprozesses. Häufig treten die Spieler gruppenweise gegeneinander an, was auch noch Skills im Bereich Teamwork und sozialem Umfang fördert. 

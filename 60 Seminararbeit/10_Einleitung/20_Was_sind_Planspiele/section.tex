\section{Was sind Planspiele?}
\label{sec:einleitung-was}

Planspiele definieren sich je nach Einsatzfeld unterschiedlich in Bezug auf Ablauf, Struktur und Komplexität. Allumfassend ist ihnen nach dem Planspiel-Forscher und Inhaber der Firma UCS Ulrich Creative Simulations GmbH, Dr. Markus Ulrich, folgendes gemein:
\\
"`\textit{Ein Planspiel versetzt die Teilnehmer in eine fiktive Situation, die ein vereinfachtes Abbild der Realität ist. Während mehrerer Spielrunden machen sich die Teilnehmer mit der Situation vertraut, führen Verhandlungen und fällen konkrete Entscheidungen.}"'\seFootcite{}{}{LO:PBNE} 
\\
Typisch dabei ist, dass ein Konflikt als Bestandteil sozialer Wirklichkeit beschrieben wird, den es zu erörtern und lösen gilt.
\\
\\
Planspiele finden ihre Ursprünge schon sehr früh. Laut TOPSIM fanden die ersten Planspiele etwa 1000 v.Chr. als Kampfspiele in Indien statt, in Persien 800 v.Chr. sogar auch im Rahmen von Schachspielen.  Ebenso werden sie in den USA schon seit langer Zeit in die militärische Ausbildung integriert, da so Kriegszüge besser geplant und durchgeführt werden können, ohne dabei reale Ressourcen zu verbrauchen. Um 1950 wurde das  erste computergestützte Unternehmensplanspiel durch die American Management Association entwickelt, das zugleich auch ein einfaches Unternehmensplanspiel darstellte. Bis heute schreitet die Entwicklung dieser Planspiele stark voran, sodass sie zunehmend in sehr vielen Bereichen Anwendung finden.\seFootcite{Vgl.}{}{TIS:PM} So sind sie auch schon an Schulen und Hochschulen wie der Universität Tübingen, fester Bestandteil des Lehrplans. Hier nehmen Studenten ganz Aktuell im Rahmen eines Seminars an dem Planspiel 'National Model United Nations Conference' (NMUN) teil, was die größte Simulation der Arbeit der Vereinten Nationen darstellt und weltweit von Studenten praktiziert wird. Ziel dieses Planspiel ist es, den Teilnehmern eine Vorstellung von politischen Geschehnissen, Strategien und deren Auswirkungen zu vermitteln.\seFootcite{Vgl.}{}{DPR:ERP}
\\
\\
Eine besondere Stellung unter der Vielfalt von Planspielen nehmen die Unternehmensplanspiele ein, zu denen auch das dieser Arbeit zu Grunde liegende Unternehmensplanspiel Star Greg, gehört. Bei dieser Art des Planspiels handelt es sich nach Gabler um eine modellhafte Simulation von Unternehmensprozessen, in der die Spieler jeweils in der Rolle eines Unternehmenschefs agieren. Das heißt, sie Konkurrieren mit ihren Gegner, den anderen Unternehmen, auf einem Oligopolmarkt und sind zu Beginn des Spiels alle mit gleichen Startbedingungen, d.h. gleicher Betriebsgröße und Finanzstruktur, ausgestattet.\seFootcite{Vgl.}{}{GAB:UP} Die Herausforderung stellt sich dem Spieler insofern, innerhalb einer simulierten, modellhaften Realität strategisch und betriebswirtschaftlich zu handeln, um das Unternehmen auf dem Konkurrenzmarkt aufrecht zu erhalten. Dabei  unterliegt die simulierte Welt verschiedenen dynamischen Konzepten, wie dem Konjunkturverlauf oder dem Zusammenspiel von Angebot und Nachfrage. In der Regel sind Unternehmensplanspiele rundenbasiert aufgebaut, sodass bei jeder neuen Runde Pläne und Teilpläne sorgfältig aufgestellt und anhand derer mehrere Entscheidungen getroffen werden müssen. Die Auswirkungen dieser Entscheidungen werden durch Algorithmen ermittelt. Normalerweise dauert eine Spielrunde bis zu mehrere Stunden, das ganze Unternehmensplanspiel bis zu mehreren Tagen.\seFootcite{Vgl.}{}{DPR:ERP}
\\
\\
Ziel eines Planspiels ist es, den Teilnehmern eine Vorstellung davon zu geben, wie ein Unternehmen als Ganzes funktioniert und wie Entscheidungen zu treffen sind, um im Wettbewerb erfolgreich bestehen zu können. Durch Zwischenbewertungen kann der Spieler sein Verhalten und die daraus resultierenden wirtschaftlichen Konsequenzen und Geschäftsprozesse analysieren und anhand dessen seine strategischen und operativen Entscheidungen optimieren. Wichtig ist hierbei nicht nur, dass der Spieler Situationen erkennt und analysiert, sondern insbesondere auch, dass er die Situation einschätzen kann und lernt, auf die richtig Art und Weise zu reagieren. Das Kriterium der Einschätzung von Entscheidungen ist eine herausfordernde Disziplin, die dann aber die Fähigkeit liefert, zwischen Alternativen richtig wählen zu können.\seFootcite{Vgl.}{S. 12-14}{UR:LOG} 
\\
\\
Der große Vorteil der Unternehmensplanspiele liegt ganz klar in ihrer praktischen Orientierung. Durch Learning-by-doing wird ein interaktiver, handlungsorientierter und zeitgemäßer Lernprozess verwendet, der zu dem Kreativität und Selbstständigkeit zulässt. Durch die Simulation der Spielwelt kann der Teilnehmer - ähnlich wie ein Pilot in einem Flugsimulator - seine Fähigkeiten ohne nennenswertes Risiko ausüben und auf die Probe stellen. Dabei ist der unterhaltsame Aspekt nicht Zweck, sondern "Verstärker" des Lernprozesses. Häufig treten die Spieler gruppenweise gegeneinander an, was auch noch Skills im Bereich Teamwork und sozialem Umfang fördert.\seFootcite{Vgl.}{S. 54-55}{SF:MO} 
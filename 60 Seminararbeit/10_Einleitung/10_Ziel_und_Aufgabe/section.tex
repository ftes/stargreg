\section{Ziel und Aufgabe}
\label{sec:einleitung-ziel}

Die vorliegende Arbeit stellt die Studie einer Gruppe von fünf Studierenden der Studienrichtung der Wirtschaftsinformatik (Schwerpunkt Softwaremethodik) an der DHBW Mannheim vor. Im Rahmen der Vorlesung Systemanalyse wurde über zehn Wochen an der Konzeption des Unternehmensplanspiels 'Star Greg' gearbeitet. Vorgaben zur konkreten Aufgabe der Fallstudie beschränken sich inhaltlich auf ein Unternehmensplanspiel, das wie es der Name schon sagt, betriebswirtschaftliche Geschäftsprozesse abbildet und auf mehrere Spieler ausgelegt ist, sodass ein markttypisches Angebot- und Nachfrageverhalten durch den Oligopolmarkt gegeben ist. Die Wahl der Branche des Unternehmens und die Art der Produkte oder Dienstleistungen, die es anbietet, sowie die Welt darf frei gewählt werden. 
\\
Wie in der Aufgabenstellung gefordert, stellt diese Arbeit nach einer kurzen Einführung in die Thematik der Planspiele die simulierte Spielwelt vor. Im Unternehmensspiel Star Greg wird in Anlehnung an \textit{Star Wars} eine Welt im Universum simuliert, in der ein Fertigungsunternehmen Raumschiffe produziert. Die Komponenten und Akteure dieser Welt, sowie ein Einblick in die Spiellogik und ein typischer Spielablauf, sollen hier die grundlegende Konzeption von Star Greg näher beleuchten.  
\\
Im nächsten Kapitel wird das Fachkonzept bzgl. statischer und dynamischer Modelle umfangreich vorgestellt. Anhand der Erörterung verschiedener Problemstellungen werden wichtige, grundlegende Entwurfsentscheidungen begründet.
\\
Diese theoretischen Grundlagen werden im Folgenden durch eine Auswahl an aussagekräftigen UI Mockups transparenter gemacht. Die verschiedenen Screens sollen v.a. auch eine Vorstellung geben, wie der Spieler am Ende agieren kann. 
\\
Darauf aufbauend  der Weltstatische und dynamische Modelle, einen Prototyp der Benutzungsoberfläche, sowie eine angemessen umfangreiche Suite an JUnit Testfällen als Ergebnis liefern. 
\\
Schließlich wird das Gesamtergebnis auf Implementierungsebene in einer  JUnit-Testfallbeschreibung zusammengefasst. Während der erste ausgewählte Test die Spiellogik testet, veranschaulicht der zweite JUnit-Test, wie Star Greg bei einer Spieleranzahl von drei Teilnehmern, deren vordefinierte Eingaben auf Grundlage unterschiedliche Charaktere ermittelt wurden, verläuft. 



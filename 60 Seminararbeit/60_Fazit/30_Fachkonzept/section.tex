\section{Technische Erweiterungen}
\label{sec:fazit-fachkonzept}

Einen Blick auf die technischen Aspekte des Planspiels lässt weitere Punkte offen, welche noch realisiert werden könnten. 

\textbf{Benutzeroberfläche}

Um das Planspiel benutzerfreundlich zu gestalten, wäre eine intuitive Benutzeroberfläche von Nöten. Diese könnte sich an den bereits vorgestellten Mockups orientieren. Darauf aufbauend könnte sich jeder Spieler selbstständig am System anmelden und zur Startseite gelangen. Würde man zudem die Funktionen des Spieleiters in Java-Code umsetzten, so würde das Spiel diese Aufgaben automatisch übernehmen. Auf diese Weise könnten auch unerfahrene Spieler das Planspiel eigenständig durchführen. 

\textbf{Datenbankanbindung}

In der aktuellen Version des Planspiels werden die entstehenden Daten in Java mithilfe von Hashmaps und Vektoren realisiert. Um die Performance und die Übersichtlichkeit des Planspiels zu erhöhen wäre eine Datenbankanbindung sinnvoll. Die zentrale Speicherung auf die Daten würde zudem einen einfacheren Zugriff auf abgespeicherte Daten ermöglichen. Der Hauptvorteil einer Datenbankanbindung würde in der dauerhaften Speicherung der Daten liegen. Auf diese Weise wären die gespeicherten Informationen auch nach dem Schließen des Planspiels abrufbar.
\section{Projektorganisation}
\label{sec:fazit-projektorganisation}

Durch Brainstorming wurden zu Beginn des Projekts viele Ideen zum Planspiel gesammelt. Auf diese Weise wurde dann das Thema “Raumschiffe” ausgewählt. In weiteren Treffen wurde die Idee weiter ausgebaut und mögliche Funktionen diskutiert. Hierbei ging es vorallem um die Vor- und Nachteile, die eine realisierte Funktion auf das Spiel und den Teilnehmer hätte. Diese wurden im Anschluss in die Kategorien “Pflicht”, “Optional” und “Nicht realisierbar” eingeteilt. Der Schritt war das Erstellen des UseCase- beziehungsweise des Klassendiagramms. 

Um die Produktivität zu steigern wurden nun Arbeitsgruppen gebildet. Eine Arbeitsgruppe hat sich um das Erstellen der Mockups anhand des UseCase-Diagramms gekümmert. Die andere hat das Klassendiagramm  in Java-Code umgesetzt. Um die erstellten Klassen zu Testen wurden jUnit Tests angelegt und implementiert. Somit war es möglich die Korrektheit des Spielsablaufs zu testen. Über diesen Zeitraum fanden ständige Treffen zwischen den Arbeitsgruppen statt, um den Fortschritt im Überblick zu behalten und gegebenenfalls Anpassungen vorzunehmen. Auf diese Weise wurde die Aufnahme von optionalen Funktionen besprochen. 

Während des Projekts sind einige Dateien angefallen, welche für jedes Mitglied von Bedeutung war. Um den Datentransfer zu vereinfachen und einen ständigen Austausch zu gewährleisten wurde das Versionsverwaltungssystem Git genutzt. Auf diese Weise hatte jedes Mitglied die aktuellste Version der Daten zur Verfügung und konnte die neusten Änderungen direkt einsehen. 

Das Verwenden des Textverarbeitungsprogramms Latex in Kombination mit Git vereinfachte das Erstellen der Seminararbeit. So war es jedem Mitglied möglich sein eigenes Kapitel zu verfassen und dieses in einer eigenen Datei zu speichern. Diese Dateien wurden am Ende mithilfe von Latex in ein Dokument zusammengeführt.

Abschließend ist zu sagen, dass sowohl die Planung als auch die Durchführung des Projektes reibungslos von statten ging.
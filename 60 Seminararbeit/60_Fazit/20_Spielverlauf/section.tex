\section{Erweiterte Funktionen im Spielverlauf}
\label{sec:fazit-spielverlauf}

\textbf{Finanzwesen}

Eine Überlegung bezüglich des Finanzwesens war es, anstelle des automatisch eingeräumten Dispositionskredits mit festem Zinssatz dem Spieler eine manuelle Kreditaufnahme anzubieten. Auf diese Weise wäre es dem Spieler möglich vorausschauend den Kreditrahmen zu bestimmen und diesen zu einem angegebenen Zinssatz aufzunehmen. Dieser Zinssatz würde dadurch von der Geldnachfrage abhängig sein und sich von Runde zu Runde verändern. Somit würde dem Spieler mehr Verantwortung und Entscheidungskraft übertragen werden. 

\textbf{Produktion}

Mit der Einführung von zusätzlichen Maschinen, welche der Spieler während einer Spielrunde erwerben könnte, wäre das Planspiel realistischer aufgebaut. Durch diese Maschinen könnte die Produktionskapazität weiter erhöht, neue Funktionen in das Raumschiff eingebaut oder der Ausschuss verringert werden. Das zuvor aufgenommene Darlehen könnte beispielsweise zur Finanzierung einer neuen Maschine verwendet werden. 

Im Spielverlauf könnte die revolutionäre Erfindung des Warp-Antriebs die Nachfrage erheblich beeinträchtigen. Dieser Antrieb ermöglicht den Raumschiffen mit Warp-Geschwindigkeit das Weltall zu durchqueren. Nach Einführung des Warp-Antriebs würde die Nachfrage nach Raumschiffen mit herkömmlichen Antrieben stark zurück gehen, was den Spieler dazu zwingen würde, die alten Modelle schnellstmöglich abzusetzen. Des Weiteren müsste der Spieler seinen Maschinenpark aufrüsten um der Nachfrage nach den modernen Raumschiffen gerecht zu werden.

\textbf{Zusatzaufträge}

Um noch mehr Dynamik in das Spiel zu bringen, könnte man in bestimmten Runden die Nachfrage nach Raumschiffen durch Zusatzaufträge erhöhen. Diese kann der Spieler nach Prüfung der Rentabilität annehmen oder ablehnen. Auf diese Weise könnte der Spieler freie Kapazitäten nutzen, seinen Gewinn steigern sowie das Image verbessern. 

\textbf{Beeinflussung der Abnehmer}

Bemerkt der Spieler einen Rückgang seiner Absatzzahlen so wäre es ihm mit dem so genannten “Macht”-Button möglich, die Entscheidungen der Abnehmer zu seinen Gunsten zu manipulieren. Allerdings würde dann das Risiko bestehen, durch die galaktischen Ordnungshüter für eine Runde auf den Planeten Despayre verbannt zu werden. Während dem Aufenthalt auf Despayre wäre es dem Spieler nicht möglich im Namen seines Unternehmens zu handeln. 

Die Einführung einer Marketingabteilung würde dem Spieler noch mehr Möglichkeiten bieten, seine Absatzzahlen zu beeinflussen. Durch Investitionen in die Marktforschung könnte er beispielsweise durch Befragungen Informationen über zukünftige Nachfrageveränderungen erlangen und könnte somit sein Produktionsprogramm darauf abstimmen. Zudem könnte er durch Werbemaßnahmen seine Beliebtheit erhöhen und somit die Abnehmer legal beeinflussen. Dies könnte zu erheblichen Gewinnsteigerungen führen und würde zudem das Spiel noch dynamischer und realistischer machen.
%
% Einlesen der .sty-Dateien
%
%  se-pa1-input-styles.tex
%
%  Joerg Baumgart 01.08.2011
%
%  Zusammenfassung und Konfiguration wichtiger Styles f\"ur die 
%  Erzeugung von Seminar-, Projekt- und Bachelorarbeiten
%
%
\documentclass[12pt,BCOR=10mm,headinclude=on,footinclude=off,bibliography=totoc]{scrreprt}
\usepackage[T1]{fontenc}
\usepackage[utf8x]{inputenc}
\usepackage[ngerman]{babel} % Deutsche Einstellungen
\usepackage{lmodern}

\usepackage{textcomp}
\usepackage{tikz} % Graphikpaket, das zu pdfLaTeX kompatibel ist
\usetikzlibrary{intersections, arrows, calc, backgrounds, fit, decorations.pathreplacing}
\usepackage{amsmath}
\usepackage{amssymb}
\usepackage{xkeyval} % Definition von Kommandos mit mehreren optionalen Argumenten
\usepackage{listings} % Formatierung von Programmlistings
\usepackage{graphicx} % Einbinden von Graphiken
\usepackage{ifthen}
\usepackage{color}
\usepackage{slashbox} % Diagonalen in Tabellenfeldern
\usepackage{framed} % Erzeugung schwarzer Linien am linken Rand zur Hervorhebung von Textteilen
\usepackage{caption} % Korrektes Setzen einer mehrzeiligen float-Unterschrift bei neu definierten float-Umgebungen
%\usepackage{floatrow}

% Es wird jeweils die sty-Datei importiert und entsprechende Konfigurationseinstellungen werden vorgenommen
%
\usepackage{se-jb-scrpage2} % Formatierung der Kopf- und Fu{\ss}zeilen
\usepackage{se-jb-footmisc}    % Fussnoten besser formatieren

\usepackage{se-jb-glossaries} % Abk\"urzungsverzeichnis, Symbolverzeichnis, Glossar
   
\usepackage{se-jb-floatrow}    % Definition und Konfiguration von float-Umgebungen (figure, table, die neue programm-Umgebung)
% Achtung: se-jb-varioref muss nach se-jb-floatrow importiert werden; 
% andernfalls ist der counter programm f\"ur die labelformat-Anweisung noch nicht definiert   
\usepackage{se-jb-varioref}   % Definition von Querverweisen
\usepackage{se-jb-chngcntr}   % Kapitelweise oder globale Nummerierung von Abbildungen etc.
   
\usepackage{se-jb-listen} % Definition neuer, besser formatierter Listen
\usepackage{se-jb-wa-kommandos} % neue Kommandos f\"ur Seminar-, Projekt- und Bachelorarbeiten


%
% Individuelle Konfiguration des Dokumentes
%
%  Individuelle Konfiguration einer Projektarbeit
%
%
%
%

%
% Literaturverzeichnis
% 
\usepackage{se-jb-jurabib-theisen} % Literaturverzeichnis gem\"ass den Vorgaben von Theisen aufbauen



% Weitere Optionseinstellungen f\"ur das Koma-Script
%
% Zwischen Abs\"atzen einen Abstand von 0.5 \baselineskip erzeugen
\KOMAoption{parskip}{full}
%
% Vergleiche Duden "Gliederung von Nummern, S.111" 
% DIN 5008 anschauen, wenn sie neu ver\"offentlicht wurde
\KOMAoption{numbers}{noendperiod}
%
%



%  Voreinstellungen f\"ur floats
%  Durch die verwendeten Parameter wird die Wahrscheinlichkeit deutlich kleiner, 
%  dass Gleitobjekte (z. B. Abbildungen) ans Ende des Dokumentes verschoben 
%  werden; 
%  Achtung: clearpage erzwingt die Ausgabe von Gleitobjekten
%
\renewcommand{\topfraction}{1}  % Gleitobjekte d\"urfen eine Seite zu 100% belegen 
\renewcommand{\bottomfraction}{1} % Entsprechender Wert f\"ur den unteren Teil der Seite
\renewcommand{\textfraction}{0} % Eine Seite darf auch ohne Fliesstext existieren
%%%\renewcommand{\floatpagefraction}{1} % Bedeutung unklar, daher keine Ver\"anderung des Vorgabewertes 
                                                                        % von 0.5; eventuell bringt ein \"Anderung auf 1 etwas, wenn 
                                                                         % Probleme mit floats auftreten
                                                                         
                                                                         
                                                                         
% Konfiguration von Programm-Listings
% 
% Achtung: hier gibt es nahezu beliebig viele weitere Konfigurationm\"oglichkeiten; vgl. Paketdokumentation
%
\lstset{language=Java,basicstyle=\ttfamily,keywordstyle=\color{blue},captionpos=b,aboveskip=0mm,belowskip=0mm,
          xleftmargin=0em}               
          
%
% Grundkonfiguration der Abs\"ande zwischen den Items der maximal f\"unf Verschachtelungsebenen der 
% neuen Listenumgebungen
%                                                                             
% Initialisierung der Abst\"ande zwischen den items f\"ur seList; Grundeinheit: 0.5\baselineskip; siehe se-jb-listen
\seSetlistbaselineskip{1}{0.75}{0.75}{0.75}{0.75}
% Initialisierung der Abst\"ande zwischen den items f\"ur seToplist; Grundeinheit: 0.5\baselineskip; siehe se-jb-listen
\seSettoplistbaselineskip{1}{0.75}{0.75}{0.75}{0.75}     


%
%  Konfiguration der verschiedenen Verzeichnisse
%
%  abstandEintrag: Wert wird mit \baselineskip multipliziert
%

%
%  Abbildungsverzeichnis
%
\seKonfigurationAbb[
%verzeichnisname=Abbildungsverzeichnis,
labeltextLinks=, % kein Text links;
%labeltextRechts=:,
labelbreite=1cm,
%labeleinzug=1cm,
%abstandEintrag=1,
%newpage=ja,
%pnumwidth=20mm,
%dotsep=1000,
%tocrmarg=4.5cm,
%abstandVerzeichnis=-1mm
]

%
% LIstingverzeichnis
%
\seKonfigurationPrg[
%verzeichnisname=Listing-Verzeichnis,
labeltextLinks=,
%labeltextRechts=:,
labelbreite=1cm,
%labeleinzug=2cm,
%abstandEintrag=1,
%newpage=ja,
%%pnumwidth=20mm,
%dotsep=1000,
%tocrmarg=4.5cm,
%abstandVerzeichnis=-10mm
]

%
% Tabellenverzeichnis
%
\seKonfigurationTab[
%verzeichnisname=Liste der Tabellen,
labeltextLinks=,
%labeltextRechts=:,
labelbreite=1cm,
%labeleinzug=0.5cm,
%abstandEintrag=1,
%newpage=ja,
%pnumwidth=20mm,
%dotsep=1000,
%tocrmarg=4.5cm,
%abstandVerzeichnis=-10mm
]

%
% Abk\"urzungsverzeichnis
%
\seKonfigurationAbk[
%verzeichnisname=Liste der Abk\"urzungen,
%labelbreite=3cm,
%labeleinzug=0.5cm,
%abstandEintrag=1,
%newpage=ja,
%abstandVerzeichnis=-10mm
]

%
% Symbolverzeichnis
% 
\seKonfigurationSym[
%verzeichnisname=Liste der Symbole,
%labelbreite=4cm,
%labeleinzug=3.5cm,
%abstandEintrag=1,
%newpage=ja,
%abstandVerzeichnis=-10mm
]

%
% Glossar
%
\seKonfigurationGlo[
%verzeichnisname=Glossar,
%abstandEintrag=0,
]



% (eventuelle) Neudefinition f\"ur die Unter-/\"Uberschriften von Abbildungen, Tabellen und Listings
%
%
%\renewcommand{\seCaptionNameAbbildung}{Abb.}
%\renewcommand{\seCaptionNameTabelle}{Tab.}
%\renewcommand{\seCaptionNameProgramm}{Prg.}


% % (eventuelle) Neudefinition f\"ur Querverweise innerhalb des Textes
%
%
%
%\renewcommand{\seQuerverweisSeite}{Seite}
%\renewcommand{\seQuerverweisAbbildung}{Abb.}
%\renewcommand{\seQuerverweisTabelle}{Tab.}
%\renewcommand{\seQuerverweisProgramm}{Prg.}
%\renewcommand{\seQuerverweisKapitel}{Kap.}
%\renewcommand{\seQuerverweisGleichung}{Gl.}

% Kommandos, die direkt nach \begin{document} ausgef\"uhrt werden m\"ussen
%
%
%
\AtBeginDocument{%
\renewcommand{\listfigurename}{\seAbbildungenVerzeichnisname}
\renewcommand{\listtablename}{\seTabellenVerzeichnisname}
\renewcommand{\figurename}{\seCaptionNameAbbildung}
\renewcommand{\tablename}{\seCaptionNameTabelle}
\pagenumbering{roman}
}
                                                              
                                                                         

%
% Individuelle Definition von Abk\"urzungen, Symbolen und eventuell Glossareintr\"agen
%
%  J\"org Baumgart
%  Definition einiger Abk\"urzungen
%  
%Befehle f�r Abk\"urzungen
\newacronym{dhbw}{DHBW}{Duale Hochschule Baden-W\"urttemberg}
\newacronym{usb}{USB}{Universal Serial Bus}

%Befehle f�r Symbole
%
% Achtung: ohne sort wird nach Name sortiert
\newglossaryentry{pi}{
name=$\pi$,
description={Die Kreiszahl},
type=symbolslist,
sort=a
}

\newglossaryentry{ND}{
name=$\mbox{\textsl{ND}}$,
description={Nutzungsdauer einer Maschine},
type=symbolslist,%
sort=b
}


% Glossareintr\"age
\newglossaryentry{glos:AD}{
first=Active Directory\textsuperscript{GL},
name=Active Directory,
description={Active Directory ist in einem Windows 2000/Windows
Server 2003-Netzwerk der Verzeichnisdienst, der die zentrale
Organisation und Verwaltung aller Netzwerkressourcen erlaubt. Es
erm\"oglicht den Benutzern \"uber eine einzige zentrale Anmeldung den
Zugriff auf alle Ressourcen und den Administratoren die zentral
organisierte Verwaltung, transparent von der Netzwerktopologie und
den eingesetzten Netzwerkprotokollen. Das daf\"ur ben\"otigte
Betriebssystem ist entweder Windows 2000 Server oder
Windows Server 2003, welches auf dem zentralen
Dom\"anencontroller installiert wird. Dieser h\"alt alle Daten des
Active Directory vor, wie z.\,B. Benutzernamen und
Kennw\"orter.\seFootcite{Vgl.}{S. 200}{Dud09}}
}

\begin{document}

%Insert Image: file, label, caption
\newcommand{\img}[3]{
	\begin{figure}[h]
	\centering
	\fbox{\includegraphics[width=0.9\textwidth]{#1}}
	\caption[#3]
	\label{#2}
	\end{figure}
}

\newcommand{\curr}{\textcolonmonetary}
\newcommand{\autorbeginn}[1]{\textit{Beginn: #1} \def \autortmp {#1}}
\newcommand{\autorende}{\textit{Ende: \autortmp}}

% Erzeugung des Titelblatts
%
%
%
\seTitelblattSeminararbeit[
%hilfslinien=ja,
%dhbwlogoSkalierung=0.5,
%dhbwlogoDeltaX=2.4,
%dhbwlogoDeltaY=-10,
thema=Star Greg\\Das Unternehmensplanspiel,
verfasser={Britta Jochum\\Julia Lakatos\\Philipp Mail\\Jan Schlenker\\Marcel Steinleitner\\Fredrik Teschke},
firma=,
%verfasserin=,
matrikelnummer=,
kurs=WWI\,10\,SWM\,A,
%studiengangsleiterin=,
studiengangsleiter=Prof. Dr.-Ing. Jörg Baumgart,
modul=Umsetzung von Methoden der Wirtschaftsinformatik,
lehrveranstaltung=Fallstudie Systemanalyse,
%dozentin=,
dozent=Gregor Tielsch
]



% Erzeugung der Kurzfassung; Verfasser, Firma und Thema werden automatisch \"ubernommen
%
% Der optionale Parameter kann verwendet werden, um f\"ur das Thema der Arbeit eine 
% andere Formatierung vorzunehmen; das sollte in der Regel nicht erforderlich sein;
% ausserdem besteht die Gefahr inkonsistenter Titel auf dem Titelblatt und in der 
% Kurzfassung
%
%\seKurzfassung{} % dieses Kommando sollte standardm\"assig verwendet werden
\seKurzfassung[Star Greg - Das Unternehmensplanspiel]{}

% Beispiel f\"ur ein Kapitel, dass vor dem Einleitungskapitel kommt, z. B. ein Vorwort oder eine Danksagung
\seKapitelVorEinleitung{Vorwort}

\autorbeginn{Britta}

Wer hat nicht schon ein Mal den Wunsch verspürt, Vorstandschef eines Großunternehmens zu sein? 
Als Teilnehmer eines Unternehmensplanspiels, bei dem der Spieler mit  einer realitätsnahen, simulierten Welt interagiert, ist dieser Wunsch gar nicht so weit hergeholt. 

Schon seit Jahrhunderten ist man auf der Suche, neue Lernmethoden für die Weiterbildung zu erforschen und auszubauen. Dabei stehen neben theoretischen Methoden auch immer mehr die praktischen im Vordergrund.  Eine spezielle Methode, um insbesondere komplexe Sachverhalte oder Systeme, wie beispielsweise die Funktionsweise eines Unternehmens, zu veranschaulichen, stellt hierbei das Planspiel dar. Man unterscheidet haptische (z.B. Brettspiele) und computergestützte Planspiele, die mittlerweile flächendeckend und in hoher Zahl am Markt erhältlich sind. Besonders populär ist das Unternehmensplanspiel TOPSIM der Firma \textit{TATA Interactive Systems}, das an ca. 1500 Hochschulen, Akademien und Unternehmen verwendet wird.\seFootcite{Vgl.}{}{TIS:PM} 

Die zunehmende Bedeutung von Planspielen als Lernmethode ist gewiss nicht nur auf deren Anwendung beschränkt. Eine weitere Lernkomponente ergibt sich auf Ebene der Anforderungsanalyse und Entwicklung eines computergestützten Planspiels, was insbesondere für Studierende der Wirtschaftsinformatik große Lernchancen mit sich bringt. Da zu ihrem Bereichsfeld sowohl die betriebswirtschaftlichen als auch software- und programmiertechnischen Anforderungen an Analyse, Entwurf und Implementierung einer derartigen Software gehören, wird diese Aufgabe gerne in den Lehrplan übernommen.

\autorende{}


% Ausgabe des Inhaltsverzeichnisses
%
%
\seInhaltsverzeichnis[%
einrueckung=ja,
gliederungsebenen=4
]



% Ausgabe der verschiedenen Verzeichnisse
% abk: Abk\"urzungsverzeichnis
% sym: Symbolverzeichnis
% abb: Abbildungsverzeichnis
% tab: Tabellenverzeichnis
% prg: Listingverzeichnis
%
%
% Achtung: Abk\"urzungs- und Symbolverzeichnis werden nur ausgegeben, wenn mindest ein Symbol bzw. 
%                mindestens eine Abk\"urzung in der Arbeit verwendet wurden
%
%
% gliederungsebene:
% -- section: die Verzeichnisse werden einem Kapitel "Verzeichnisse" untergliedert
% -- chapter: die Verzeichnisse sind jeweils eigene Kapitel
% imInhaltsverzeichnis: ja/nein -- Sollen die Verzeichnisse im Inhaltsverzeichnis enthalten sein?
\seVerzeichnisse[gliederungsebene=section,imInhaltsverzeichnis=ja]{abk}{sym}{abb}{tab}{prg}






% Erstes eigentliches Kapitel der Arbeit; typischerweise das Einleitungskapitel;
% hier muss wieder auf die Nummierung mit arabischen Seitenzahlen umgestellt werden
%

 \chapter{Spielwelt}
\label{chp:spielwelt}

\section{Aktivitätsdiagramm}
\label{sec:fachkonzept-aktivitaetsdiagramm}

\autorbeginn{Fredrik, Julia}

Das Aktivitätsdiagramm auf \vref{img:fachkonzept-aktivitaetsdiagramm-uebersicht} soll den Spielablauf aus Sicht des Spielers verdeutlichen.  

Um das Spiel zu starten muss der Spieler einen Namen für sein Unternehmen festlegen. Dies stellt die erste Aktion dar. Danach analysiert der Spieler die ihm zur Verfügung stehenden Informationen. Ist dies abgeschlossen, so gelangt er zu einem Entscheidungsknoten. Hierbei kann sich der Spieler zwischen folgenden Aktivitäten entscheiden: Personal verwalten, Einkäufe tätigen, Produktionsaufträge anlegen oder Verkaufsangebot abgeben. Diese einzelnen Vorgänge werden im Laufe dieses Kapitels genauer erläutert. Er kann sich aber auch dazu entscheiden, keine Transaktion zu tätigen. 

Diese verschiedenen Aktivitäten werden in einem Entscheidungsknoten zusammengeführt. Hat der Spieler weiteren Informationsbedarf, so gelangt er zur Aktivität “Auswirkungen analysieren” und kann sich die Veränderungen anschauen, die seine Transaktion mit sich geführt hat. Besteht kein Informationsbedarf, kann er diese Aktivität überspringen. 

Möchte der Spieler nun weitere Transaktionen tätigen, so kann er wieder zum Entscheidungsknoten nach oben springen und hat wieder die Wahl zwischen Personal verwalten, Einkäufe tätigen, Produktionsaufträge anlegen oder ein Verkaufsangebot abgeben. Entscheidet sich der Spieler gegen eine weitere Transaktion, so folgt die Aktivität “Runde einchecken”. Anschließend folgt wieder ein Entscheidungsknoten. Existiert noch eine weitere Spielrunde, so gelangt der Spieler wieder zur Aktivität “Informationen analysieren”. War dies die letzte Spielrunde, so wird dem Spieler die Endbewertung angezeigt. Danach ist das Spiel beendet.

\begin{figure}[h]
  \centering
  \fbox{
    \includegraphics[width=0.9\textwidth]{30_Fachkonzept/15_aktivitaetsdiagramm/activity1.pdf}
  }
  \caption{Aktivitätsdiagramm}
  \label{img:fachkonzept-aktivitaetsdiagramm-uebersicht}
\end{figure}

\medskip

\textbf{Personal verwalten}

Zur genaueren Betrachtung der Aktion “Personal verwalten” dient folgendes Aktivitätsdiagramm auf \vref{img:fachkonzept-aktivitaetsdiagramm-personal}. Entscheidet sich der Spieler für diese Aktion, so hat er die Möglichkeit Personal einzustellen, aufzurüsten oder zu entlassen. 

Um neues Personal einzustellen, muss der Spieler zuerst den Personaltyp auswählen, welchen er einstellen möchte. Danach folgt die Aktivität “Anzahl festlegen”. Anschließend trifft der Spieler auf einen Entscheidungsknoten. Ist genügend Geld vorhanden, muss der Spieler die Auswahl bestätigen. Reicht das verfügbare Geld jedoch nicht aus, gelangt der Spieler wieder zur Aktivität “einzustellender Typ auswählen” und durchläuft den Prozess noch ein mal. 

Möchte der Spieler sein vorhandenes Personal aufrüsten, muss er zunächst den Personaltyp auswählen. Anschließend gelangt er zur Aktivität “Anzahl festlegen”. Ähnlich wie beim Einstellen von neuem Personal wird auch nun geprüft, ob genügend Geld vorhanden ist. Ist dies der Fall, wird die Auswahl bestätigt. Fehlt Geld, befindet sich der Spieler wieder bei der Aktivität “aufzurüstender Typ auswählen”.

Zum Entlassen von Personal ist ebenfalls der Personaltyp und die Anzahl festzulegen. Anschließend wird die Auswahl Bestätigt und die Transaktion ist abgeschlossen. 

\begin{figure}[h]
  \centering
  \fbox{
    \includegraphics[width=0.9\textwidth]{30_Fachkonzept/15_aktivitaetsdiagramm/activity2.pdf}
  }
  \caption{Aktivitätsdiagramm zur Personalverwaltung}
  \label{img:fachkonzept-aktivitaetsdiagramm-personal}
\end{figure}

\medskip

\textbf{Bauteile einkaufen}

Entscheidet sich der Spieler für das Einkaufen von Bauteilen, muss er zunächst den Bauteiltyp wählen und anschließend die einzukaufende Anzahl festlegen. Im Anschluss daran wird wieder die Liquidität des Spielers geprüft. Ist das Geld ausreichend, muss die Auswahl bestätigt werden. Fällt die Prüfung negativ aus, gelangt der Spieler wieder zur Aktivität “Bauteiltyp auswählen”. Dies wird in \vref{img:fachkonzept-aktivitaetsdiagramm-bauteile} verdeutlicht.

\begin{figure}[h]
  \centering
  \fbox{
    \includegraphics[width=0.4\textwidth]{30_Fachkonzept/15_aktivitaetsdiagramm/activity3.pdf}
  }
  \caption{Aktivitätsdiagramm zum Bauteileinkauf}
  \label{img:fachkonzept-aktivitaetsdiagramm-bauteile}
\end{figure}

\medskip

\textbf{Produktionsauftrag anlegen}

Beim Anlegen eines Produktionsauftrages, wie in \vref{img:fachkonzept-aktivitaetsdiagramm-produktion} dargestellt, ist zuerst das zu produzierende Raumschiff auszuwählen. Anschließend legt der Spieler die Anzahl fest. Nun muss geprüft werden, ob zum einen genügend Bauteile für die Produktion vorhanden sind und zum anderen ob das eingestellte Personal die Anzahl an Raumschiffen in einer Periode produzieren kann. Fällt die Prüfung positiv aus, so muss der Spieler seine Auswahl bestätigen. Sind die Kapazitäten jedoch nicht ausreichend, befindet sich der Spieler wieder bei der Aktivität “Raumschifftyp auswählen”.

\begin{figure}[h]
  \centering
  \fbox{
    \includegraphics[width=0.7\textwidth]{30_Fachkonzept/15_aktivitaetsdiagramm/activity4.pdf}
  }
  \caption{Aktivitätsdiagramm zur Produktion}
  \label{img:fachkonzept-aktivitaetsdiagramm-produktion}
\end{figure}

\medskip

\textbf{Verkaufsangebot abgeben}

Die letzte Transaktion kann im Bereich Verkauf getätigt werden. Dies ist auf \vref{img:fachkonzept-aktivitaetsdiagramm-verkauf} zu sehen. Zuerst wird der zu verkaufende Raumschifftyp ausgewählt. Danach entscheidet sich der Spieler für einen Preis, zu dem er die Raumschiffe verkaufen möchte. War bereits zuvor ein Angebot vorhanden, so wird dieses hiermit zurück genommen und danach bestätigt. War dies das erste Angebot, so gelangt der Spieler direkt zur Aktivität “Bestätigen”.

\begin{figure}[h]
  \centering
  \fbox{
    \includegraphics[width=0.7\textwidth]{30_Fachkonzept/15_aktivitaetsdiagramm/activity5.pdf}
  }
  \caption{Aktivitätsdiagramm zum Verkauf}
  \label{img:fachkonzept-aktivitaetsdiagramm-verkauf}
\end{figure}

\autorende{}
\autorbeginn{Britta}
\section{Aktivitätsdiagramm}
\label{sec:fachkonzept-aktivitaetsdiagramm}

\autorbeginn{Fredrik, Julia}

Das Aktivitätsdiagramm auf \vref{img:fachkonzept-aktivitaetsdiagramm-uebersicht} soll den Spielablauf aus Sicht des Spielers verdeutlichen.  

Um das Spiel zu starten muss der Spieler einen Namen für sein Unternehmen festlegen. Dies stellt die erste Aktion dar. Danach analysiert der Spieler die ihm zur Verfügung stehenden Informationen. Ist dies abgeschlossen, so gelangt er zu einem Entscheidungsknoten. Hierbei kann sich der Spieler zwischen folgenden Aktivitäten entscheiden: Personal verwalten, Einkäufe tätigen, Produktionsaufträge anlegen oder Verkaufsangebot abgeben. Diese einzelnen Vorgänge werden im Laufe dieses Kapitels genauer erläutert. Er kann sich aber auch dazu entscheiden, keine Transaktion zu tätigen. 

Diese verschiedenen Aktivitäten werden in einem Entscheidungsknoten zusammengeführt. Hat der Spieler weiteren Informationsbedarf, so gelangt er zur Aktivität “Auswirkungen analysieren” und kann sich die Veränderungen anschauen, die seine Transaktion mit sich geführt hat. Besteht kein Informationsbedarf, kann er diese Aktivität überspringen. 

Möchte der Spieler nun weitere Transaktionen tätigen, so kann er wieder zum Entscheidungsknoten nach oben springen und hat wieder die Wahl zwischen Personal verwalten, Einkäufe tätigen, Produktionsaufträge anlegen oder ein Verkaufsangebot abgeben. Entscheidet sich der Spieler gegen eine weitere Transaktion, so folgt die Aktivität “Runde einchecken”. Anschließend folgt wieder ein Entscheidungsknoten. Existiert noch eine weitere Spielrunde, so gelangt der Spieler wieder zur Aktivität “Informationen analysieren”. War dies die letzte Spielrunde, so wird dem Spieler die Endbewertung angezeigt. Danach ist das Spiel beendet.

\begin{figure}[h]
  \centering
  \fbox{
    \includegraphics[width=0.9\textwidth]{30_Fachkonzept/15_aktivitaetsdiagramm/activity1.pdf}
  }
  \caption{Aktivitätsdiagramm}
  \label{img:fachkonzept-aktivitaetsdiagramm-uebersicht}
\end{figure}

\medskip

\textbf{Personal verwalten}

Zur genaueren Betrachtung der Aktion “Personal verwalten” dient folgendes Aktivitätsdiagramm auf \vref{img:fachkonzept-aktivitaetsdiagramm-personal}. Entscheidet sich der Spieler für diese Aktion, so hat er die Möglichkeit Personal einzustellen, aufzurüsten oder zu entlassen. 

Um neues Personal einzustellen, muss der Spieler zuerst den Personaltyp auswählen, welchen er einstellen möchte. Danach folgt die Aktivität “Anzahl festlegen”. Anschließend trifft der Spieler auf einen Entscheidungsknoten. Ist genügend Geld vorhanden, muss der Spieler die Auswahl bestätigen. Reicht das verfügbare Geld jedoch nicht aus, gelangt der Spieler wieder zur Aktivität “einzustellender Typ auswählen” und durchläuft den Prozess noch ein mal. 

Möchte der Spieler sein vorhandenes Personal aufrüsten, muss er zunächst den Personaltyp auswählen. Anschließend gelangt er zur Aktivität “Anzahl festlegen”. Ähnlich wie beim Einstellen von neuem Personal wird auch nun geprüft, ob genügend Geld vorhanden ist. Ist dies der Fall, wird die Auswahl bestätigt. Fehlt Geld, befindet sich der Spieler wieder bei der Aktivität “aufzurüstender Typ auswählen”.

Zum Entlassen von Personal ist ebenfalls der Personaltyp und die Anzahl festzulegen. Anschließend wird die Auswahl Bestätigt und die Transaktion ist abgeschlossen. 

\begin{figure}[h]
  \centering
  \fbox{
    \includegraphics[width=0.9\textwidth]{30_Fachkonzept/15_aktivitaetsdiagramm/activity2.pdf}
  }
  \caption{Aktivitätsdiagramm zur Personalverwaltung}
  \label{img:fachkonzept-aktivitaetsdiagramm-personal}
\end{figure}

\medskip

\textbf{Bauteile einkaufen}

Entscheidet sich der Spieler für das Einkaufen von Bauteilen, muss er zunächst den Bauteiltyp wählen und anschließend die einzukaufende Anzahl festlegen. Im Anschluss daran wird wieder die Liquidität des Spielers geprüft. Ist das Geld ausreichend, muss die Auswahl bestätigt werden. Fällt die Prüfung negativ aus, gelangt der Spieler wieder zur Aktivität “Bauteiltyp auswählen”. Dies wird in \vref{img:fachkonzept-aktivitaetsdiagramm-bauteile} verdeutlicht.

\begin{figure}[h]
  \centering
  \fbox{
    \includegraphics[width=0.4\textwidth]{30_Fachkonzept/15_aktivitaetsdiagramm/activity3.pdf}
  }
  \caption{Aktivitätsdiagramm zum Bauteileinkauf}
  \label{img:fachkonzept-aktivitaetsdiagramm-bauteile}
\end{figure}

\medskip

\textbf{Produktionsauftrag anlegen}

Beim Anlegen eines Produktionsauftrages, wie in \vref{img:fachkonzept-aktivitaetsdiagramm-produktion} dargestellt, ist zuerst das zu produzierende Raumschiff auszuwählen. Anschließend legt der Spieler die Anzahl fest. Nun muss geprüft werden, ob zum einen genügend Bauteile für die Produktion vorhanden sind und zum anderen ob das eingestellte Personal die Anzahl an Raumschiffen in einer Periode produzieren kann. Fällt die Prüfung positiv aus, so muss der Spieler seine Auswahl bestätigen. Sind die Kapazitäten jedoch nicht ausreichend, befindet sich der Spieler wieder bei der Aktivität “Raumschifftyp auswählen”.

\begin{figure}[h]
  \centering
  \fbox{
    \includegraphics[width=0.7\textwidth]{30_Fachkonzept/15_aktivitaetsdiagramm/activity4.pdf}
  }
  \caption{Aktivitätsdiagramm zur Produktion}
  \label{img:fachkonzept-aktivitaetsdiagramm-produktion}
\end{figure}

\medskip

\textbf{Verkaufsangebot abgeben}

Die letzte Transaktion kann im Bereich Verkauf getätigt werden. Dies ist auf \vref{img:fachkonzept-aktivitaetsdiagramm-verkauf} zu sehen. Zuerst wird der zu verkaufende Raumschifftyp ausgewählt. Danach entscheidet sich der Spieler für einen Preis, zu dem er die Raumschiffe verkaufen möchte. War bereits zuvor ein Angebot vorhanden, so wird dieses hiermit zurück genommen und danach bestätigt. War dies das erste Angebot, so gelangt der Spieler direkt zur Aktivität “Bestätigen”.

\begin{figure}[h]
  \centering
  \fbox{
    \includegraphics[width=0.7\textwidth]{30_Fachkonzept/15_aktivitaetsdiagramm/activity5.pdf}
  }
  \caption{Aktivitätsdiagramm zum Verkauf}
  \label{img:fachkonzept-aktivitaetsdiagramm-verkauf}
\end{figure}

\autorende{}
\autorende{}
\section{Aktivitätsdiagramm}
\label{sec:fachkonzept-aktivitaetsdiagramm}

\autorbeginn{Fredrik, Julia}

Das Aktivitätsdiagramm auf \vref{img:fachkonzept-aktivitaetsdiagramm-uebersicht} soll den Spielablauf aus Sicht des Spielers verdeutlichen.  

Um das Spiel zu starten muss der Spieler einen Namen für sein Unternehmen festlegen. Dies stellt die erste Aktion dar. Danach analysiert der Spieler die ihm zur Verfügung stehenden Informationen. Ist dies abgeschlossen, so gelangt er zu einem Entscheidungsknoten. Hierbei kann sich der Spieler zwischen folgenden Aktivitäten entscheiden: Personal verwalten, Einkäufe tätigen, Produktionsaufträge anlegen oder Verkaufsangebot abgeben. Diese einzelnen Vorgänge werden im Laufe dieses Kapitels genauer erläutert. Er kann sich aber auch dazu entscheiden, keine Transaktion zu tätigen. 

Diese verschiedenen Aktivitäten werden in einem Entscheidungsknoten zusammengeführt. Hat der Spieler weiteren Informationsbedarf, so gelangt er zur Aktivität “Auswirkungen analysieren” und kann sich die Veränderungen anschauen, die seine Transaktion mit sich geführt hat. Besteht kein Informationsbedarf, kann er diese Aktivität überspringen. 

Möchte der Spieler nun weitere Transaktionen tätigen, so kann er wieder zum Entscheidungsknoten nach oben springen und hat wieder die Wahl zwischen Personal verwalten, Einkäufe tätigen, Produktionsaufträge anlegen oder ein Verkaufsangebot abgeben. Entscheidet sich der Spieler gegen eine weitere Transaktion, so folgt die Aktivität “Runde einchecken”. Anschließend folgt wieder ein Entscheidungsknoten. Existiert noch eine weitere Spielrunde, so gelangt der Spieler wieder zur Aktivität “Informationen analysieren”. War dies die letzte Spielrunde, so wird dem Spieler die Endbewertung angezeigt. Danach ist das Spiel beendet.

\begin{figure}[h]
  \centering
  \fbox{
    \includegraphics[width=0.9\textwidth]{30_Fachkonzept/15_aktivitaetsdiagramm/activity1.pdf}
  }
  \caption{Aktivitätsdiagramm}
  \label{img:fachkonzept-aktivitaetsdiagramm-uebersicht}
\end{figure}

\medskip

\textbf{Personal verwalten}

Zur genaueren Betrachtung der Aktion “Personal verwalten” dient folgendes Aktivitätsdiagramm auf \vref{img:fachkonzept-aktivitaetsdiagramm-personal}. Entscheidet sich der Spieler für diese Aktion, so hat er die Möglichkeit Personal einzustellen, aufzurüsten oder zu entlassen. 

Um neues Personal einzustellen, muss der Spieler zuerst den Personaltyp auswählen, welchen er einstellen möchte. Danach folgt die Aktivität “Anzahl festlegen”. Anschließend trifft der Spieler auf einen Entscheidungsknoten. Ist genügend Geld vorhanden, muss der Spieler die Auswahl bestätigen. Reicht das verfügbare Geld jedoch nicht aus, gelangt der Spieler wieder zur Aktivität “einzustellender Typ auswählen” und durchläuft den Prozess noch ein mal. 

Möchte der Spieler sein vorhandenes Personal aufrüsten, muss er zunächst den Personaltyp auswählen. Anschließend gelangt er zur Aktivität “Anzahl festlegen”. Ähnlich wie beim Einstellen von neuem Personal wird auch nun geprüft, ob genügend Geld vorhanden ist. Ist dies der Fall, wird die Auswahl bestätigt. Fehlt Geld, befindet sich der Spieler wieder bei der Aktivität “aufzurüstender Typ auswählen”.

Zum Entlassen von Personal ist ebenfalls der Personaltyp und die Anzahl festzulegen. Anschließend wird die Auswahl Bestätigt und die Transaktion ist abgeschlossen. 

\begin{figure}[h]
  \centering
  \fbox{
    \includegraphics[width=0.9\textwidth]{30_Fachkonzept/15_aktivitaetsdiagramm/activity2.pdf}
  }
  \caption{Aktivitätsdiagramm zur Personalverwaltung}
  \label{img:fachkonzept-aktivitaetsdiagramm-personal}
\end{figure}

\medskip

\textbf{Bauteile einkaufen}

Entscheidet sich der Spieler für das Einkaufen von Bauteilen, muss er zunächst den Bauteiltyp wählen und anschließend die einzukaufende Anzahl festlegen. Im Anschluss daran wird wieder die Liquidität des Spielers geprüft. Ist das Geld ausreichend, muss die Auswahl bestätigt werden. Fällt die Prüfung negativ aus, gelangt der Spieler wieder zur Aktivität “Bauteiltyp auswählen”. Dies wird in \vref{img:fachkonzept-aktivitaetsdiagramm-bauteile} verdeutlicht.

\begin{figure}[h]
  \centering
  \fbox{
    \includegraphics[width=0.4\textwidth]{30_Fachkonzept/15_aktivitaetsdiagramm/activity3.pdf}
  }
  \caption{Aktivitätsdiagramm zum Bauteileinkauf}
  \label{img:fachkonzept-aktivitaetsdiagramm-bauteile}
\end{figure}

\medskip

\textbf{Produktionsauftrag anlegen}

Beim Anlegen eines Produktionsauftrages, wie in \vref{img:fachkonzept-aktivitaetsdiagramm-produktion} dargestellt, ist zuerst das zu produzierende Raumschiff auszuwählen. Anschließend legt der Spieler die Anzahl fest. Nun muss geprüft werden, ob zum einen genügend Bauteile für die Produktion vorhanden sind und zum anderen ob das eingestellte Personal die Anzahl an Raumschiffen in einer Periode produzieren kann. Fällt die Prüfung positiv aus, so muss der Spieler seine Auswahl bestätigen. Sind die Kapazitäten jedoch nicht ausreichend, befindet sich der Spieler wieder bei der Aktivität “Raumschifftyp auswählen”.

\begin{figure}[h]
  \centering
  \fbox{
    \includegraphics[width=0.7\textwidth]{30_Fachkonzept/15_aktivitaetsdiagramm/activity4.pdf}
  }
  \caption{Aktivitätsdiagramm zur Produktion}
  \label{img:fachkonzept-aktivitaetsdiagramm-produktion}
\end{figure}

\medskip

\textbf{Verkaufsangebot abgeben}

Die letzte Transaktion kann im Bereich Verkauf getätigt werden. Dies ist auf \vref{img:fachkonzept-aktivitaetsdiagramm-verkauf} zu sehen. Zuerst wird der zu verkaufende Raumschifftyp ausgewählt. Danach entscheidet sich der Spieler für einen Preis, zu dem er die Raumschiffe verkaufen möchte. War bereits zuvor ein Angebot vorhanden, so wird dieses hiermit zurück genommen und danach bestätigt. War dies das erste Angebot, so gelangt der Spieler direkt zur Aktivität “Bestätigen”.

\begin{figure}[h]
  \centering
  \fbox{
    \includegraphics[width=0.7\textwidth]{30_Fachkonzept/15_aktivitaetsdiagramm/activity5.pdf}
  }
  \caption{Aktivitätsdiagramm zum Verkauf}
  \label{img:fachkonzept-aktivitaetsdiagramm-verkauf}
\end{figure}

\autorende{}
\autorbeginn{Britta}
\section{Aktivitätsdiagramm}
\label{sec:fachkonzept-aktivitaetsdiagramm}

\autorbeginn{Fredrik, Julia}

Das Aktivitätsdiagramm auf \vref{img:fachkonzept-aktivitaetsdiagramm-uebersicht} soll den Spielablauf aus Sicht des Spielers verdeutlichen.  

Um das Spiel zu starten muss der Spieler einen Namen für sein Unternehmen festlegen. Dies stellt die erste Aktion dar. Danach analysiert der Spieler die ihm zur Verfügung stehenden Informationen. Ist dies abgeschlossen, so gelangt er zu einem Entscheidungsknoten. Hierbei kann sich der Spieler zwischen folgenden Aktivitäten entscheiden: Personal verwalten, Einkäufe tätigen, Produktionsaufträge anlegen oder Verkaufsangebot abgeben. Diese einzelnen Vorgänge werden im Laufe dieses Kapitels genauer erläutert. Er kann sich aber auch dazu entscheiden, keine Transaktion zu tätigen. 

Diese verschiedenen Aktivitäten werden in einem Entscheidungsknoten zusammengeführt. Hat der Spieler weiteren Informationsbedarf, so gelangt er zur Aktivität “Auswirkungen analysieren” und kann sich die Veränderungen anschauen, die seine Transaktion mit sich geführt hat. Besteht kein Informationsbedarf, kann er diese Aktivität überspringen. 

Möchte der Spieler nun weitere Transaktionen tätigen, so kann er wieder zum Entscheidungsknoten nach oben springen und hat wieder die Wahl zwischen Personal verwalten, Einkäufe tätigen, Produktionsaufträge anlegen oder ein Verkaufsangebot abgeben. Entscheidet sich der Spieler gegen eine weitere Transaktion, so folgt die Aktivität “Runde einchecken”. Anschließend folgt wieder ein Entscheidungsknoten. Existiert noch eine weitere Spielrunde, so gelangt der Spieler wieder zur Aktivität “Informationen analysieren”. War dies die letzte Spielrunde, so wird dem Spieler die Endbewertung angezeigt. Danach ist das Spiel beendet.

\begin{figure}[h]
  \centering
  \fbox{
    \includegraphics[width=0.9\textwidth]{30_Fachkonzept/15_aktivitaetsdiagramm/activity1.pdf}
  }
  \caption{Aktivitätsdiagramm}
  \label{img:fachkonzept-aktivitaetsdiagramm-uebersicht}
\end{figure}

\medskip

\textbf{Personal verwalten}

Zur genaueren Betrachtung der Aktion “Personal verwalten” dient folgendes Aktivitätsdiagramm auf \vref{img:fachkonzept-aktivitaetsdiagramm-personal}. Entscheidet sich der Spieler für diese Aktion, so hat er die Möglichkeit Personal einzustellen, aufzurüsten oder zu entlassen. 

Um neues Personal einzustellen, muss der Spieler zuerst den Personaltyp auswählen, welchen er einstellen möchte. Danach folgt die Aktivität “Anzahl festlegen”. Anschließend trifft der Spieler auf einen Entscheidungsknoten. Ist genügend Geld vorhanden, muss der Spieler die Auswahl bestätigen. Reicht das verfügbare Geld jedoch nicht aus, gelangt der Spieler wieder zur Aktivität “einzustellender Typ auswählen” und durchläuft den Prozess noch ein mal. 

Möchte der Spieler sein vorhandenes Personal aufrüsten, muss er zunächst den Personaltyp auswählen. Anschließend gelangt er zur Aktivität “Anzahl festlegen”. Ähnlich wie beim Einstellen von neuem Personal wird auch nun geprüft, ob genügend Geld vorhanden ist. Ist dies der Fall, wird die Auswahl bestätigt. Fehlt Geld, befindet sich der Spieler wieder bei der Aktivität “aufzurüstender Typ auswählen”.

Zum Entlassen von Personal ist ebenfalls der Personaltyp und die Anzahl festzulegen. Anschließend wird die Auswahl Bestätigt und die Transaktion ist abgeschlossen. 

\begin{figure}[h]
  \centering
  \fbox{
    \includegraphics[width=0.9\textwidth]{30_Fachkonzept/15_aktivitaetsdiagramm/activity2.pdf}
  }
  \caption{Aktivitätsdiagramm zur Personalverwaltung}
  \label{img:fachkonzept-aktivitaetsdiagramm-personal}
\end{figure}

\medskip

\textbf{Bauteile einkaufen}

Entscheidet sich der Spieler für das Einkaufen von Bauteilen, muss er zunächst den Bauteiltyp wählen und anschließend die einzukaufende Anzahl festlegen. Im Anschluss daran wird wieder die Liquidität des Spielers geprüft. Ist das Geld ausreichend, muss die Auswahl bestätigt werden. Fällt die Prüfung negativ aus, gelangt der Spieler wieder zur Aktivität “Bauteiltyp auswählen”. Dies wird in \vref{img:fachkonzept-aktivitaetsdiagramm-bauteile} verdeutlicht.

\begin{figure}[h]
  \centering
  \fbox{
    \includegraphics[width=0.4\textwidth]{30_Fachkonzept/15_aktivitaetsdiagramm/activity3.pdf}
  }
  \caption{Aktivitätsdiagramm zum Bauteileinkauf}
  \label{img:fachkonzept-aktivitaetsdiagramm-bauteile}
\end{figure}

\medskip

\textbf{Produktionsauftrag anlegen}

Beim Anlegen eines Produktionsauftrages, wie in \vref{img:fachkonzept-aktivitaetsdiagramm-produktion} dargestellt, ist zuerst das zu produzierende Raumschiff auszuwählen. Anschließend legt der Spieler die Anzahl fest. Nun muss geprüft werden, ob zum einen genügend Bauteile für die Produktion vorhanden sind und zum anderen ob das eingestellte Personal die Anzahl an Raumschiffen in einer Periode produzieren kann. Fällt die Prüfung positiv aus, so muss der Spieler seine Auswahl bestätigen. Sind die Kapazitäten jedoch nicht ausreichend, befindet sich der Spieler wieder bei der Aktivität “Raumschifftyp auswählen”.

\begin{figure}[h]
  \centering
  \fbox{
    \includegraphics[width=0.7\textwidth]{30_Fachkonzept/15_aktivitaetsdiagramm/activity4.pdf}
  }
  \caption{Aktivitätsdiagramm zur Produktion}
  \label{img:fachkonzept-aktivitaetsdiagramm-produktion}
\end{figure}

\medskip

\textbf{Verkaufsangebot abgeben}

Die letzte Transaktion kann im Bereich Verkauf getätigt werden. Dies ist auf \vref{img:fachkonzept-aktivitaetsdiagramm-verkauf} zu sehen. Zuerst wird der zu verkaufende Raumschifftyp ausgewählt. Danach entscheidet sich der Spieler für einen Preis, zu dem er die Raumschiffe verkaufen möchte. War bereits zuvor ein Angebot vorhanden, so wird dieses hiermit zurück genommen und danach bestätigt. War dies das erste Angebot, so gelangt der Spieler direkt zur Aktivität “Bestätigen”.

\begin{figure}[h]
  \centering
  \fbox{
    \includegraphics[width=0.7\textwidth]{30_Fachkonzept/15_aktivitaetsdiagramm/activity5.pdf}
  }
  \caption{Aktivitätsdiagramm zum Verkauf}
  \label{img:fachkonzept-aktivitaetsdiagramm-verkauf}
\end{figure}

\autorende{}
\autorende{}
\section{Aktivitätsdiagramm}
\label{sec:fachkonzept-aktivitaetsdiagramm}

\autorbeginn{Fredrik, Julia}

Das Aktivitätsdiagramm auf \vref{img:fachkonzept-aktivitaetsdiagramm-uebersicht} soll den Spielablauf aus Sicht des Spielers verdeutlichen.  

Um das Spiel zu starten muss der Spieler einen Namen für sein Unternehmen festlegen. Dies stellt die erste Aktion dar. Danach analysiert der Spieler die ihm zur Verfügung stehenden Informationen. Ist dies abgeschlossen, so gelangt er zu einem Entscheidungsknoten. Hierbei kann sich der Spieler zwischen folgenden Aktivitäten entscheiden: Personal verwalten, Einkäufe tätigen, Produktionsaufträge anlegen oder Verkaufsangebot abgeben. Diese einzelnen Vorgänge werden im Laufe dieses Kapitels genauer erläutert. Er kann sich aber auch dazu entscheiden, keine Transaktion zu tätigen. 

Diese verschiedenen Aktivitäten werden in einem Entscheidungsknoten zusammengeführt. Hat der Spieler weiteren Informationsbedarf, so gelangt er zur Aktivität “Auswirkungen analysieren” und kann sich die Veränderungen anschauen, die seine Transaktion mit sich geführt hat. Besteht kein Informationsbedarf, kann er diese Aktivität überspringen. 

Möchte der Spieler nun weitere Transaktionen tätigen, so kann er wieder zum Entscheidungsknoten nach oben springen und hat wieder die Wahl zwischen Personal verwalten, Einkäufe tätigen, Produktionsaufträge anlegen oder ein Verkaufsangebot abgeben. Entscheidet sich der Spieler gegen eine weitere Transaktion, so folgt die Aktivität “Runde einchecken”. Anschließend folgt wieder ein Entscheidungsknoten. Existiert noch eine weitere Spielrunde, so gelangt der Spieler wieder zur Aktivität “Informationen analysieren”. War dies die letzte Spielrunde, so wird dem Spieler die Endbewertung angezeigt. Danach ist das Spiel beendet.

\begin{figure}[h]
  \centering
  \fbox{
    \includegraphics[width=0.9\textwidth]{30_Fachkonzept/15_aktivitaetsdiagramm/activity1.pdf}
  }
  \caption{Aktivitätsdiagramm}
  \label{img:fachkonzept-aktivitaetsdiagramm-uebersicht}
\end{figure}

\medskip

\textbf{Personal verwalten}

Zur genaueren Betrachtung der Aktion “Personal verwalten” dient folgendes Aktivitätsdiagramm auf \vref{img:fachkonzept-aktivitaetsdiagramm-personal}. Entscheidet sich der Spieler für diese Aktion, so hat er die Möglichkeit Personal einzustellen, aufzurüsten oder zu entlassen. 

Um neues Personal einzustellen, muss der Spieler zuerst den Personaltyp auswählen, welchen er einstellen möchte. Danach folgt die Aktivität “Anzahl festlegen”. Anschließend trifft der Spieler auf einen Entscheidungsknoten. Ist genügend Geld vorhanden, muss der Spieler die Auswahl bestätigen. Reicht das verfügbare Geld jedoch nicht aus, gelangt der Spieler wieder zur Aktivität “einzustellender Typ auswählen” und durchläuft den Prozess noch ein mal. 

Möchte der Spieler sein vorhandenes Personal aufrüsten, muss er zunächst den Personaltyp auswählen. Anschließend gelangt er zur Aktivität “Anzahl festlegen”. Ähnlich wie beim Einstellen von neuem Personal wird auch nun geprüft, ob genügend Geld vorhanden ist. Ist dies der Fall, wird die Auswahl bestätigt. Fehlt Geld, befindet sich der Spieler wieder bei der Aktivität “aufzurüstender Typ auswählen”.

Zum Entlassen von Personal ist ebenfalls der Personaltyp und die Anzahl festzulegen. Anschließend wird die Auswahl Bestätigt und die Transaktion ist abgeschlossen. 

\begin{figure}[h]
  \centering
  \fbox{
    \includegraphics[width=0.9\textwidth]{30_Fachkonzept/15_aktivitaetsdiagramm/activity2.pdf}
  }
  \caption{Aktivitätsdiagramm zur Personalverwaltung}
  \label{img:fachkonzept-aktivitaetsdiagramm-personal}
\end{figure}

\medskip

\textbf{Bauteile einkaufen}

Entscheidet sich der Spieler für das Einkaufen von Bauteilen, muss er zunächst den Bauteiltyp wählen und anschließend die einzukaufende Anzahl festlegen. Im Anschluss daran wird wieder die Liquidität des Spielers geprüft. Ist das Geld ausreichend, muss die Auswahl bestätigt werden. Fällt die Prüfung negativ aus, gelangt der Spieler wieder zur Aktivität “Bauteiltyp auswählen”. Dies wird in \vref{img:fachkonzept-aktivitaetsdiagramm-bauteile} verdeutlicht.

\begin{figure}[h]
  \centering
  \fbox{
    \includegraphics[width=0.4\textwidth]{30_Fachkonzept/15_aktivitaetsdiagramm/activity3.pdf}
  }
  \caption{Aktivitätsdiagramm zum Bauteileinkauf}
  \label{img:fachkonzept-aktivitaetsdiagramm-bauteile}
\end{figure}

\medskip

\textbf{Produktionsauftrag anlegen}

Beim Anlegen eines Produktionsauftrages, wie in \vref{img:fachkonzept-aktivitaetsdiagramm-produktion} dargestellt, ist zuerst das zu produzierende Raumschiff auszuwählen. Anschließend legt der Spieler die Anzahl fest. Nun muss geprüft werden, ob zum einen genügend Bauteile für die Produktion vorhanden sind und zum anderen ob das eingestellte Personal die Anzahl an Raumschiffen in einer Periode produzieren kann. Fällt die Prüfung positiv aus, so muss der Spieler seine Auswahl bestätigen. Sind die Kapazitäten jedoch nicht ausreichend, befindet sich der Spieler wieder bei der Aktivität “Raumschifftyp auswählen”.

\begin{figure}[h]
  \centering
  \fbox{
    \includegraphics[width=0.7\textwidth]{30_Fachkonzept/15_aktivitaetsdiagramm/activity4.pdf}
  }
  \caption{Aktivitätsdiagramm zur Produktion}
  \label{img:fachkonzept-aktivitaetsdiagramm-produktion}
\end{figure}

\medskip

\textbf{Verkaufsangebot abgeben}

Die letzte Transaktion kann im Bereich Verkauf getätigt werden. Dies ist auf \vref{img:fachkonzept-aktivitaetsdiagramm-verkauf} zu sehen. Zuerst wird der zu verkaufende Raumschifftyp ausgewählt. Danach entscheidet sich der Spieler für einen Preis, zu dem er die Raumschiffe verkaufen möchte. War bereits zuvor ein Angebot vorhanden, so wird dieses hiermit zurück genommen und danach bestätigt. War dies das erste Angebot, so gelangt der Spieler direkt zur Aktivität “Bestätigen”.

\begin{figure}[h]
  \centering
  \fbox{
    \includegraphics[width=0.7\textwidth]{30_Fachkonzept/15_aktivitaetsdiagramm/activity5.pdf}
  }
  \caption{Aktivitätsdiagramm zum Verkauf}
  \label{img:fachkonzept-aktivitaetsdiagramm-verkauf}
\end{figure}

\autorende{}

\chapter{Spielwelt}
\label{chp:spielwelt}

\section{Aktivitätsdiagramm}
\label{sec:fachkonzept-aktivitaetsdiagramm}

\autorbeginn{Fredrik, Julia}

Das Aktivitätsdiagramm auf \vref{img:fachkonzept-aktivitaetsdiagramm-uebersicht} soll den Spielablauf aus Sicht des Spielers verdeutlichen.  

Um das Spiel zu starten muss der Spieler einen Namen für sein Unternehmen festlegen. Dies stellt die erste Aktion dar. Danach analysiert der Spieler die ihm zur Verfügung stehenden Informationen. Ist dies abgeschlossen, so gelangt er zu einem Entscheidungsknoten. Hierbei kann sich der Spieler zwischen folgenden Aktivitäten entscheiden: Personal verwalten, Einkäufe tätigen, Produktionsaufträge anlegen oder Verkaufsangebot abgeben. Diese einzelnen Vorgänge werden im Laufe dieses Kapitels genauer erläutert. Er kann sich aber auch dazu entscheiden, keine Transaktion zu tätigen. 

Diese verschiedenen Aktivitäten werden in einem Entscheidungsknoten zusammengeführt. Hat der Spieler weiteren Informationsbedarf, so gelangt er zur Aktivität “Auswirkungen analysieren” und kann sich die Veränderungen anschauen, die seine Transaktion mit sich geführt hat. Besteht kein Informationsbedarf, kann er diese Aktivität überspringen. 

Möchte der Spieler nun weitere Transaktionen tätigen, so kann er wieder zum Entscheidungsknoten nach oben springen und hat wieder die Wahl zwischen Personal verwalten, Einkäufe tätigen, Produktionsaufträge anlegen oder ein Verkaufsangebot abgeben. Entscheidet sich der Spieler gegen eine weitere Transaktion, so folgt die Aktivität “Runde einchecken”. Anschließend folgt wieder ein Entscheidungsknoten. Existiert noch eine weitere Spielrunde, so gelangt der Spieler wieder zur Aktivität “Informationen analysieren”. War dies die letzte Spielrunde, so wird dem Spieler die Endbewertung angezeigt. Danach ist das Spiel beendet.

\begin{figure}[h]
  \centering
  \fbox{
    \includegraphics[width=0.9\textwidth]{30_Fachkonzept/15_aktivitaetsdiagramm/activity1.pdf}
  }
  \caption{Aktivitätsdiagramm}
  \label{img:fachkonzept-aktivitaetsdiagramm-uebersicht}
\end{figure}

\medskip

\textbf{Personal verwalten}

Zur genaueren Betrachtung der Aktion “Personal verwalten” dient folgendes Aktivitätsdiagramm auf \vref{img:fachkonzept-aktivitaetsdiagramm-personal}. Entscheidet sich der Spieler für diese Aktion, so hat er die Möglichkeit Personal einzustellen, aufzurüsten oder zu entlassen. 

Um neues Personal einzustellen, muss der Spieler zuerst den Personaltyp auswählen, welchen er einstellen möchte. Danach folgt die Aktivität “Anzahl festlegen”. Anschließend trifft der Spieler auf einen Entscheidungsknoten. Ist genügend Geld vorhanden, muss der Spieler die Auswahl bestätigen. Reicht das verfügbare Geld jedoch nicht aus, gelangt der Spieler wieder zur Aktivität “einzustellender Typ auswählen” und durchläuft den Prozess noch ein mal. 

Möchte der Spieler sein vorhandenes Personal aufrüsten, muss er zunächst den Personaltyp auswählen. Anschließend gelangt er zur Aktivität “Anzahl festlegen”. Ähnlich wie beim Einstellen von neuem Personal wird auch nun geprüft, ob genügend Geld vorhanden ist. Ist dies der Fall, wird die Auswahl bestätigt. Fehlt Geld, befindet sich der Spieler wieder bei der Aktivität “aufzurüstender Typ auswählen”.

Zum Entlassen von Personal ist ebenfalls der Personaltyp und die Anzahl festzulegen. Anschließend wird die Auswahl Bestätigt und die Transaktion ist abgeschlossen. 

\begin{figure}[h]
  \centering
  \fbox{
    \includegraphics[width=0.9\textwidth]{30_Fachkonzept/15_aktivitaetsdiagramm/activity2.pdf}
  }
  \caption{Aktivitätsdiagramm zur Personalverwaltung}
  \label{img:fachkonzept-aktivitaetsdiagramm-personal}
\end{figure}

\medskip

\textbf{Bauteile einkaufen}

Entscheidet sich der Spieler für das Einkaufen von Bauteilen, muss er zunächst den Bauteiltyp wählen und anschließend die einzukaufende Anzahl festlegen. Im Anschluss daran wird wieder die Liquidität des Spielers geprüft. Ist das Geld ausreichend, muss die Auswahl bestätigt werden. Fällt die Prüfung negativ aus, gelangt der Spieler wieder zur Aktivität “Bauteiltyp auswählen”. Dies wird in \vref{img:fachkonzept-aktivitaetsdiagramm-bauteile} verdeutlicht.

\begin{figure}[h]
  \centering
  \fbox{
    \includegraphics[width=0.4\textwidth]{30_Fachkonzept/15_aktivitaetsdiagramm/activity3.pdf}
  }
  \caption{Aktivitätsdiagramm zum Bauteileinkauf}
  \label{img:fachkonzept-aktivitaetsdiagramm-bauteile}
\end{figure}

\medskip

\textbf{Produktionsauftrag anlegen}

Beim Anlegen eines Produktionsauftrages, wie in \vref{img:fachkonzept-aktivitaetsdiagramm-produktion} dargestellt, ist zuerst das zu produzierende Raumschiff auszuwählen. Anschließend legt der Spieler die Anzahl fest. Nun muss geprüft werden, ob zum einen genügend Bauteile für die Produktion vorhanden sind und zum anderen ob das eingestellte Personal die Anzahl an Raumschiffen in einer Periode produzieren kann. Fällt die Prüfung positiv aus, so muss der Spieler seine Auswahl bestätigen. Sind die Kapazitäten jedoch nicht ausreichend, befindet sich der Spieler wieder bei der Aktivität “Raumschifftyp auswählen”.

\begin{figure}[h]
  \centering
  \fbox{
    \includegraphics[width=0.7\textwidth]{30_Fachkonzept/15_aktivitaetsdiagramm/activity4.pdf}
  }
  \caption{Aktivitätsdiagramm zur Produktion}
  \label{img:fachkonzept-aktivitaetsdiagramm-produktion}
\end{figure}

\medskip

\textbf{Verkaufsangebot abgeben}

Die letzte Transaktion kann im Bereich Verkauf getätigt werden. Dies ist auf \vref{img:fachkonzept-aktivitaetsdiagramm-verkauf} zu sehen. Zuerst wird der zu verkaufende Raumschifftyp ausgewählt. Danach entscheidet sich der Spieler für einen Preis, zu dem er die Raumschiffe verkaufen möchte. War bereits zuvor ein Angebot vorhanden, so wird dieses hiermit zurück genommen und danach bestätigt. War dies das erste Angebot, so gelangt der Spieler direkt zur Aktivität “Bestätigen”.

\begin{figure}[h]
  \centering
  \fbox{
    \includegraphics[width=0.7\textwidth]{30_Fachkonzept/15_aktivitaetsdiagramm/activity5.pdf}
  }
  \caption{Aktivitätsdiagramm zum Verkauf}
  \label{img:fachkonzept-aktivitaetsdiagramm-verkauf}
\end{figure}

\autorende{}
\autorbeginn{Britta}
\section{Aktivitätsdiagramm}
\label{sec:fachkonzept-aktivitaetsdiagramm}

\autorbeginn{Fredrik, Julia}

Das Aktivitätsdiagramm auf \vref{img:fachkonzept-aktivitaetsdiagramm-uebersicht} soll den Spielablauf aus Sicht des Spielers verdeutlichen.  

Um das Spiel zu starten muss der Spieler einen Namen für sein Unternehmen festlegen. Dies stellt die erste Aktion dar. Danach analysiert der Spieler die ihm zur Verfügung stehenden Informationen. Ist dies abgeschlossen, so gelangt er zu einem Entscheidungsknoten. Hierbei kann sich der Spieler zwischen folgenden Aktivitäten entscheiden: Personal verwalten, Einkäufe tätigen, Produktionsaufträge anlegen oder Verkaufsangebot abgeben. Diese einzelnen Vorgänge werden im Laufe dieses Kapitels genauer erläutert. Er kann sich aber auch dazu entscheiden, keine Transaktion zu tätigen. 

Diese verschiedenen Aktivitäten werden in einem Entscheidungsknoten zusammengeführt. Hat der Spieler weiteren Informationsbedarf, so gelangt er zur Aktivität “Auswirkungen analysieren” und kann sich die Veränderungen anschauen, die seine Transaktion mit sich geführt hat. Besteht kein Informationsbedarf, kann er diese Aktivität überspringen. 

Möchte der Spieler nun weitere Transaktionen tätigen, so kann er wieder zum Entscheidungsknoten nach oben springen und hat wieder die Wahl zwischen Personal verwalten, Einkäufe tätigen, Produktionsaufträge anlegen oder ein Verkaufsangebot abgeben. Entscheidet sich der Spieler gegen eine weitere Transaktion, so folgt die Aktivität “Runde einchecken”. Anschließend folgt wieder ein Entscheidungsknoten. Existiert noch eine weitere Spielrunde, so gelangt der Spieler wieder zur Aktivität “Informationen analysieren”. War dies die letzte Spielrunde, so wird dem Spieler die Endbewertung angezeigt. Danach ist das Spiel beendet.

\begin{figure}[h]
  \centering
  \fbox{
    \includegraphics[width=0.9\textwidth]{30_Fachkonzept/15_aktivitaetsdiagramm/activity1.pdf}
  }
  \caption{Aktivitätsdiagramm}
  \label{img:fachkonzept-aktivitaetsdiagramm-uebersicht}
\end{figure}

\medskip

\textbf{Personal verwalten}

Zur genaueren Betrachtung der Aktion “Personal verwalten” dient folgendes Aktivitätsdiagramm auf \vref{img:fachkonzept-aktivitaetsdiagramm-personal}. Entscheidet sich der Spieler für diese Aktion, so hat er die Möglichkeit Personal einzustellen, aufzurüsten oder zu entlassen. 

Um neues Personal einzustellen, muss der Spieler zuerst den Personaltyp auswählen, welchen er einstellen möchte. Danach folgt die Aktivität “Anzahl festlegen”. Anschließend trifft der Spieler auf einen Entscheidungsknoten. Ist genügend Geld vorhanden, muss der Spieler die Auswahl bestätigen. Reicht das verfügbare Geld jedoch nicht aus, gelangt der Spieler wieder zur Aktivität “einzustellender Typ auswählen” und durchläuft den Prozess noch ein mal. 

Möchte der Spieler sein vorhandenes Personal aufrüsten, muss er zunächst den Personaltyp auswählen. Anschließend gelangt er zur Aktivität “Anzahl festlegen”. Ähnlich wie beim Einstellen von neuem Personal wird auch nun geprüft, ob genügend Geld vorhanden ist. Ist dies der Fall, wird die Auswahl bestätigt. Fehlt Geld, befindet sich der Spieler wieder bei der Aktivität “aufzurüstender Typ auswählen”.

Zum Entlassen von Personal ist ebenfalls der Personaltyp und die Anzahl festzulegen. Anschließend wird die Auswahl Bestätigt und die Transaktion ist abgeschlossen. 

\begin{figure}[h]
  \centering
  \fbox{
    \includegraphics[width=0.9\textwidth]{30_Fachkonzept/15_aktivitaetsdiagramm/activity2.pdf}
  }
  \caption{Aktivitätsdiagramm zur Personalverwaltung}
  \label{img:fachkonzept-aktivitaetsdiagramm-personal}
\end{figure}

\medskip

\textbf{Bauteile einkaufen}

Entscheidet sich der Spieler für das Einkaufen von Bauteilen, muss er zunächst den Bauteiltyp wählen und anschließend die einzukaufende Anzahl festlegen. Im Anschluss daran wird wieder die Liquidität des Spielers geprüft. Ist das Geld ausreichend, muss die Auswahl bestätigt werden. Fällt die Prüfung negativ aus, gelangt der Spieler wieder zur Aktivität “Bauteiltyp auswählen”. Dies wird in \vref{img:fachkonzept-aktivitaetsdiagramm-bauteile} verdeutlicht.

\begin{figure}[h]
  \centering
  \fbox{
    \includegraphics[width=0.4\textwidth]{30_Fachkonzept/15_aktivitaetsdiagramm/activity3.pdf}
  }
  \caption{Aktivitätsdiagramm zum Bauteileinkauf}
  \label{img:fachkonzept-aktivitaetsdiagramm-bauteile}
\end{figure}

\medskip

\textbf{Produktionsauftrag anlegen}

Beim Anlegen eines Produktionsauftrages, wie in \vref{img:fachkonzept-aktivitaetsdiagramm-produktion} dargestellt, ist zuerst das zu produzierende Raumschiff auszuwählen. Anschließend legt der Spieler die Anzahl fest. Nun muss geprüft werden, ob zum einen genügend Bauteile für die Produktion vorhanden sind und zum anderen ob das eingestellte Personal die Anzahl an Raumschiffen in einer Periode produzieren kann. Fällt die Prüfung positiv aus, so muss der Spieler seine Auswahl bestätigen. Sind die Kapazitäten jedoch nicht ausreichend, befindet sich der Spieler wieder bei der Aktivität “Raumschifftyp auswählen”.

\begin{figure}[h]
  \centering
  \fbox{
    \includegraphics[width=0.7\textwidth]{30_Fachkonzept/15_aktivitaetsdiagramm/activity4.pdf}
  }
  \caption{Aktivitätsdiagramm zur Produktion}
  \label{img:fachkonzept-aktivitaetsdiagramm-produktion}
\end{figure}

\medskip

\textbf{Verkaufsangebot abgeben}

Die letzte Transaktion kann im Bereich Verkauf getätigt werden. Dies ist auf \vref{img:fachkonzept-aktivitaetsdiagramm-verkauf} zu sehen. Zuerst wird der zu verkaufende Raumschifftyp ausgewählt. Danach entscheidet sich der Spieler für einen Preis, zu dem er die Raumschiffe verkaufen möchte. War bereits zuvor ein Angebot vorhanden, so wird dieses hiermit zurück genommen und danach bestätigt. War dies das erste Angebot, so gelangt der Spieler direkt zur Aktivität “Bestätigen”.

\begin{figure}[h]
  \centering
  \fbox{
    \includegraphics[width=0.7\textwidth]{30_Fachkonzept/15_aktivitaetsdiagramm/activity5.pdf}
  }
  \caption{Aktivitätsdiagramm zum Verkauf}
  \label{img:fachkonzept-aktivitaetsdiagramm-verkauf}
\end{figure}

\autorende{}
\autorende{}
\section{Aktivitätsdiagramm}
\label{sec:fachkonzept-aktivitaetsdiagramm}

\autorbeginn{Fredrik, Julia}

Das Aktivitätsdiagramm auf \vref{img:fachkonzept-aktivitaetsdiagramm-uebersicht} soll den Spielablauf aus Sicht des Spielers verdeutlichen.  

Um das Spiel zu starten muss der Spieler einen Namen für sein Unternehmen festlegen. Dies stellt die erste Aktion dar. Danach analysiert der Spieler die ihm zur Verfügung stehenden Informationen. Ist dies abgeschlossen, so gelangt er zu einem Entscheidungsknoten. Hierbei kann sich der Spieler zwischen folgenden Aktivitäten entscheiden: Personal verwalten, Einkäufe tätigen, Produktionsaufträge anlegen oder Verkaufsangebot abgeben. Diese einzelnen Vorgänge werden im Laufe dieses Kapitels genauer erläutert. Er kann sich aber auch dazu entscheiden, keine Transaktion zu tätigen. 

Diese verschiedenen Aktivitäten werden in einem Entscheidungsknoten zusammengeführt. Hat der Spieler weiteren Informationsbedarf, so gelangt er zur Aktivität “Auswirkungen analysieren” und kann sich die Veränderungen anschauen, die seine Transaktion mit sich geführt hat. Besteht kein Informationsbedarf, kann er diese Aktivität überspringen. 

Möchte der Spieler nun weitere Transaktionen tätigen, so kann er wieder zum Entscheidungsknoten nach oben springen und hat wieder die Wahl zwischen Personal verwalten, Einkäufe tätigen, Produktionsaufträge anlegen oder ein Verkaufsangebot abgeben. Entscheidet sich der Spieler gegen eine weitere Transaktion, so folgt die Aktivität “Runde einchecken”. Anschließend folgt wieder ein Entscheidungsknoten. Existiert noch eine weitere Spielrunde, so gelangt der Spieler wieder zur Aktivität “Informationen analysieren”. War dies die letzte Spielrunde, so wird dem Spieler die Endbewertung angezeigt. Danach ist das Spiel beendet.

\begin{figure}[h]
  \centering
  \fbox{
    \includegraphics[width=0.9\textwidth]{30_Fachkonzept/15_aktivitaetsdiagramm/activity1.pdf}
  }
  \caption{Aktivitätsdiagramm}
  \label{img:fachkonzept-aktivitaetsdiagramm-uebersicht}
\end{figure}

\medskip

\textbf{Personal verwalten}

Zur genaueren Betrachtung der Aktion “Personal verwalten” dient folgendes Aktivitätsdiagramm auf \vref{img:fachkonzept-aktivitaetsdiagramm-personal}. Entscheidet sich der Spieler für diese Aktion, so hat er die Möglichkeit Personal einzustellen, aufzurüsten oder zu entlassen. 

Um neues Personal einzustellen, muss der Spieler zuerst den Personaltyp auswählen, welchen er einstellen möchte. Danach folgt die Aktivität “Anzahl festlegen”. Anschließend trifft der Spieler auf einen Entscheidungsknoten. Ist genügend Geld vorhanden, muss der Spieler die Auswahl bestätigen. Reicht das verfügbare Geld jedoch nicht aus, gelangt der Spieler wieder zur Aktivität “einzustellender Typ auswählen” und durchläuft den Prozess noch ein mal. 

Möchte der Spieler sein vorhandenes Personal aufrüsten, muss er zunächst den Personaltyp auswählen. Anschließend gelangt er zur Aktivität “Anzahl festlegen”. Ähnlich wie beim Einstellen von neuem Personal wird auch nun geprüft, ob genügend Geld vorhanden ist. Ist dies der Fall, wird die Auswahl bestätigt. Fehlt Geld, befindet sich der Spieler wieder bei der Aktivität “aufzurüstender Typ auswählen”.

Zum Entlassen von Personal ist ebenfalls der Personaltyp und die Anzahl festzulegen. Anschließend wird die Auswahl Bestätigt und die Transaktion ist abgeschlossen. 

\begin{figure}[h]
  \centering
  \fbox{
    \includegraphics[width=0.9\textwidth]{30_Fachkonzept/15_aktivitaetsdiagramm/activity2.pdf}
  }
  \caption{Aktivitätsdiagramm zur Personalverwaltung}
  \label{img:fachkonzept-aktivitaetsdiagramm-personal}
\end{figure}

\medskip

\textbf{Bauteile einkaufen}

Entscheidet sich der Spieler für das Einkaufen von Bauteilen, muss er zunächst den Bauteiltyp wählen und anschließend die einzukaufende Anzahl festlegen. Im Anschluss daran wird wieder die Liquidität des Spielers geprüft. Ist das Geld ausreichend, muss die Auswahl bestätigt werden. Fällt die Prüfung negativ aus, gelangt der Spieler wieder zur Aktivität “Bauteiltyp auswählen”. Dies wird in \vref{img:fachkonzept-aktivitaetsdiagramm-bauteile} verdeutlicht.

\begin{figure}[h]
  \centering
  \fbox{
    \includegraphics[width=0.4\textwidth]{30_Fachkonzept/15_aktivitaetsdiagramm/activity3.pdf}
  }
  \caption{Aktivitätsdiagramm zum Bauteileinkauf}
  \label{img:fachkonzept-aktivitaetsdiagramm-bauteile}
\end{figure}

\medskip

\textbf{Produktionsauftrag anlegen}

Beim Anlegen eines Produktionsauftrages, wie in \vref{img:fachkonzept-aktivitaetsdiagramm-produktion} dargestellt, ist zuerst das zu produzierende Raumschiff auszuwählen. Anschließend legt der Spieler die Anzahl fest. Nun muss geprüft werden, ob zum einen genügend Bauteile für die Produktion vorhanden sind und zum anderen ob das eingestellte Personal die Anzahl an Raumschiffen in einer Periode produzieren kann. Fällt die Prüfung positiv aus, so muss der Spieler seine Auswahl bestätigen. Sind die Kapazitäten jedoch nicht ausreichend, befindet sich der Spieler wieder bei der Aktivität “Raumschifftyp auswählen”.

\begin{figure}[h]
  \centering
  \fbox{
    \includegraphics[width=0.7\textwidth]{30_Fachkonzept/15_aktivitaetsdiagramm/activity4.pdf}
  }
  \caption{Aktivitätsdiagramm zur Produktion}
  \label{img:fachkonzept-aktivitaetsdiagramm-produktion}
\end{figure}

\medskip

\textbf{Verkaufsangebot abgeben}

Die letzte Transaktion kann im Bereich Verkauf getätigt werden. Dies ist auf \vref{img:fachkonzept-aktivitaetsdiagramm-verkauf} zu sehen. Zuerst wird der zu verkaufende Raumschifftyp ausgewählt. Danach entscheidet sich der Spieler für einen Preis, zu dem er die Raumschiffe verkaufen möchte. War bereits zuvor ein Angebot vorhanden, so wird dieses hiermit zurück genommen und danach bestätigt. War dies das erste Angebot, so gelangt der Spieler direkt zur Aktivität “Bestätigen”.

\begin{figure}[h]
  \centering
  \fbox{
    \includegraphics[width=0.7\textwidth]{30_Fachkonzept/15_aktivitaetsdiagramm/activity5.pdf}
  }
  \caption{Aktivitätsdiagramm zum Verkauf}
  \label{img:fachkonzept-aktivitaetsdiagramm-verkauf}
\end{figure}

\autorende{}
\autorbeginn{Britta}
\section{Aktivitätsdiagramm}
\label{sec:fachkonzept-aktivitaetsdiagramm}

\autorbeginn{Fredrik, Julia}

Das Aktivitätsdiagramm auf \vref{img:fachkonzept-aktivitaetsdiagramm-uebersicht} soll den Spielablauf aus Sicht des Spielers verdeutlichen.  

Um das Spiel zu starten muss der Spieler einen Namen für sein Unternehmen festlegen. Dies stellt die erste Aktion dar. Danach analysiert der Spieler die ihm zur Verfügung stehenden Informationen. Ist dies abgeschlossen, so gelangt er zu einem Entscheidungsknoten. Hierbei kann sich der Spieler zwischen folgenden Aktivitäten entscheiden: Personal verwalten, Einkäufe tätigen, Produktionsaufträge anlegen oder Verkaufsangebot abgeben. Diese einzelnen Vorgänge werden im Laufe dieses Kapitels genauer erläutert. Er kann sich aber auch dazu entscheiden, keine Transaktion zu tätigen. 

Diese verschiedenen Aktivitäten werden in einem Entscheidungsknoten zusammengeführt. Hat der Spieler weiteren Informationsbedarf, so gelangt er zur Aktivität “Auswirkungen analysieren” und kann sich die Veränderungen anschauen, die seine Transaktion mit sich geführt hat. Besteht kein Informationsbedarf, kann er diese Aktivität überspringen. 

Möchte der Spieler nun weitere Transaktionen tätigen, so kann er wieder zum Entscheidungsknoten nach oben springen und hat wieder die Wahl zwischen Personal verwalten, Einkäufe tätigen, Produktionsaufträge anlegen oder ein Verkaufsangebot abgeben. Entscheidet sich der Spieler gegen eine weitere Transaktion, so folgt die Aktivität “Runde einchecken”. Anschließend folgt wieder ein Entscheidungsknoten. Existiert noch eine weitere Spielrunde, so gelangt der Spieler wieder zur Aktivität “Informationen analysieren”. War dies die letzte Spielrunde, so wird dem Spieler die Endbewertung angezeigt. Danach ist das Spiel beendet.

\begin{figure}[h]
  \centering
  \fbox{
    \includegraphics[width=0.9\textwidth]{30_Fachkonzept/15_aktivitaetsdiagramm/activity1.pdf}
  }
  \caption{Aktivitätsdiagramm}
  \label{img:fachkonzept-aktivitaetsdiagramm-uebersicht}
\end{figure}

\medskip

\textbf{Personal verwalten}

Zur genaueren Betrachtung der Aktion “Personal verwalten” dient folgendes Aktivitätsdiagramm auf \vref{img:fachkonzept-aktivitaetsdiagramm-personal}. Entscheidet sich der Spieler für diese Aktion, so hat er die Möglichkeit Personal einzustellen, aufzurüsten oder zu entlassen. 

Um neues Personal einzustellen, muss der Spieler zuerst den Personaltyp auswählen, welchen er einstellen möchte. Danach folgt die Aktivität “Anzahl festlegen”. Anschließend trifft der Spieler auf einen Entscheidungsknoten. Ist genügend Geld vorhanden, muss der Spieler die Auswahl bestätigen. Reicht das verfügbare Geld jedoch nicht aus, gelangt der Spieler wieder zur Aktivität “einzustellender Typ auswählen” und durchläuft den Prozess noch ein mal. 

Möchte der Spieler sein vorhandenes Personal aufrüsten, muss er zunächst den Personaltyp auswählen. Anschließend gelangt er zur Aktivität “Anzahl festlegen”. Ähnlich wie beim Einstellen von neuem Personal wird auch nun geprüft, ob genügend Geld vorhanden ist. Ist dies der Fall, wird die Auswahl bestätigt. Fehlt Geld, befindet sich der Spieler wieder bei der Aktivität “aufzurüstender Typ auswählen”.

Zum Entlassen von Personal ist ebenfalls der Personaltyp und die Anzahl festzulegen. Anschließend wird die Auswahl Bestätigt und die Transaktion ist abgeschlossen. 

\begin{figure}[h]
  \centering
  \fbox{
    \includegraphics[width=0.9\textwidth]{30_Fachkonzept/15_aktivitaetsdiagramm/activity2.pdf}
  }
  \caption{Aktivitätsdiagramm zur Personalverwaltung}
  \label{img:fachkonzept-aktivitaetsdiagramm-personal}
\end{figure}

\medskip

\textbf{Bauteile einkaufen}

Entscheidet sich der Spieler für das Einkaufen von Bauteilen, muss er zunächst den Bauteiltyp wählen und anschließend die einzukaufende Anzahl festlegen. Im Anschluss daran wird wieder die Liquidität des Spielers geprüft. Ist das Geld ausreichend, muss die Auswahl bestätigt werden. Fällt die Prüfung negativ aus, gelangt der Spieler wieder zur Aktivität “Bauteiltyp auswählen”. Dies wird in \vref{img:fachkonzept-aktivitaetsdiagramm-bauteile} verdeutlicht.

\begin{figure}[h]
  \centering
  \fbox{
    \includegraphics[width=0.4\textwidth]{30_Fachkonzept/15_aktivitaetsdiagramm/activity3.pdf}
  }
  \caption{Aktivitätsdiagramm zum Bauteileinkauf}
  \label{img:fachkonzept-aktivitaetsdiagramm-bauteile}
\end{figure}

\medskip

\textbf{Produktionsauftrag anlegen}

Beim Anlegen eines Produktionsauftrages, wie in \vref{img:fachkonzept-aktivitaetsdiagramm-produktion} dargestellt, ist zuerst das zu produzierende Raumschiff auszuwählen. Anschließend legt der Spieler die Anzahl fest. Nun muss geprüft werden, ob zum einen genügend Bauteile für die Produktion vorhanden sind und zum anderen ob das eingestellte Personal die Anzahl an Raumschiffen in einer Periode produzieren kann. Fällt die Prüfung positiv aus, so muss der Spieler seine Auswahl bestätigen. Sind die Kapazitäten jedoch nicht ausreichend, befindet sich der Spieler wieder bei der Aktivität “Raumschifftyp auswählen”.

\begin{figure}[h]
  \centering
  \fbox{
    \includegraphics[width=0.7\textwidth]{30_Fachkonzept/15_aktivitaetsdiagramm/activity4.pdf}
  }
  \caption{Aktivitätsdiagramm zur Produktion}
  \label{img:fachkonzept-aktivitaetsdiagramm-produktion}
\end{figure}

\medskip

\textbf{Verkaufsangebot abgeben}

Die letzte Transaktion kann im Bereich Verkauf getätigt werden. Dies ist auf \vref{img:fachkonzept-aktivitaetsdiagramm-verkauf} zu sehen. Zuerst wird der zu verkaufende Raumschifftyp ausgewählt. Danach entscheidet sich der Spieler für einen Preis, zu dem er die Raumschiffe verkaufen möchte. War bereits zuvor ein Angebot vorhanden, so wird dieses hiermit zurück genommen und danach bestätigt. War dies das erste Angebot, so gelangt der Spieler direkt zur Aktivität “Bestätigen”.

\begin{figure}[h]
  \centering
  \fbox{
    \includegraphics[width=0.7\textwidth]{30_Fachkonzept/15_aktivitaetsdiagramm/activity5.pdf}
  }
  \caption{Aktivitätsdiagramm zum Verkauf}
  \label{img:fachkonzept-aktivitaetsdiagramm-verkauf}
\end{figure}

\autorende{}
\autorende{}
\section{Aktivitätsdiagramm}
\label{sec:fachkonzept-aktivitaetsdiagramm}

\autorbeginn{Fredrik, Julia}

Das Aktivitätsdiagramm auf \vref{img:fachkonzept-aktivitaetsdiagramm-uebersicht} soll den Spielablauf aus Sicht des Spielers verdeutlichen.  

Um das Spiel zu starten muss der Spieler einen Namen für sein Unternehmen festlegen. Dies stellt die erste Aktion dar. Danach analysiert der Spieler die ihm zur Verfügung stehenden Informationen. Ist dies abgeschlossen, so gelangt er zu einem Entscheidungsknoten. Hierbei kann sich der Spieler zwischen folgenden Aktivitäten entscheiden: Personal verwalten, Einkäufe tätigen, Produktionsaufträge anlegen oder Verkaufsangebot abgeben. Diese einzelnen Vorgänge werden im Laufe dieses Kapitels genauer erläutert. Er kann sich aber auch dazu entscheiden, keine Transaktion zu tätigen. 

Diese verschiedenen Aktivitäten werden in einem Entscheidungsknoten zusammengeführt. Hat der Spieler weiteren Informationsbedarf, so gelangt er zur Aktivität “Auswirkungen analysieren” und kann sich die Veränderungen anschauen, die seine Transaktion mit sich geführt hat. Besteht kein Informationsbedarf, kann er diese Aktivität überspringen. 

Möchte der Spieler nun weitere Transaktionen tätigen, so kann er wieder zum Entscheidungsknoten nach oben springen und hat wieder die Wahl zwischen Personal verwalten, Einkäufe tätigen, Produktionsaufträge anlegen oder ein Verkaufsangebot abgeben. Entscheidet sich der Spieler gegen eine weitere Transaktion, so folgt die Aktivität “Runde einchecken”. Anschließend folgt wieder ein Entscheidungsknoten. Existiert noch eine weitere Spielrunde, so gelangt der Spieler wieder zur Aktivität “Informationen analysieren”. War dies die letzte Spielrunde, so wird dem Spieler die Endbewertung angezeigt. Danach ist das Spiel beendet.

\begin{figure}[h]
  \centering
  \fbox{
    \includegraphics[width=0.9\textwidth]{30_Fachkonzept/15_aktivitaetsdiagramm/activity1.pdf}
  }
  \caption{Aktivitätsdiagramm}
  \label{img:fachkonzept-aktivitaetsdiagramm-uebersicht}
\end{figure}

\medskip

\textbf{Personal verwalten}

Zur genaueren Betrachtung der Aktion “Personal verwalten” dient folgendes Aktivitätsdiagramm auf \vref{img:fachkonzept-aktivitaetsdiagramm-personal}. Entscheidet sich der Spieler für diese Aktion, so hat er die Möglichkeit Personal einzustellen, aufzurüsten oder zu entlassen. 

Um neues Personal einzustellen, muss der Spieler zuerst den Personaltyp auswählen, welchen er einstellen möchte. Danach folgt die Aktivität “Anzahl festlegen”. Anschließend trifft der Spieler auf einen Entscheidungsknoten. Ist genügend Geld vorhanden, muss der Spieler die Auswahl bestätigen. Reicht das verfügbare Geld jedoch nicht aus, gelangt der Spieler wieder zur Aktivität “einzustellender Typ auswählen” und durchläuft den Prozess noch ein mal. 

Möchte der Spieler sein vorhandenes Personal aufrüsten, muss er zunächst den Personaltyp auswählen. Anschließend gelangt er zur Aktivität “Anzahl festlegen”. Ähnlich wie beim Einstellen von neuem Personal wird auch nun geprüft, ob genügend Geld vorhanden ist. Ist dies der Fall, wird die Auswahl bestätigt. Fehlt Geld, befindet sich der Spieler wieder bei der Aktivität “aufzurüstender Typ auswählen”.

Zum Entlassen von Personal ist ebenfalls der Personaltyp und die Anzahl festzulegen. Anschließend wird die Auswahl Bestätigt und die Transaktion ist abgeschlossen. 

\begin{figure}[h]
  \centering
  \fbox{
    \includegraphics[width=0.9\textwidth]{30_Fachkonzept/15_aktivitaetsdiagramm/activity2.pdf}
  }
  \caption{Aktivitätsdiagramm zur Personalverwaltung}
  \label{img:fachkonzept-aktivitaetsdiagramm-personal}
\end{figure}

\medskip

\textbf{Bauteile einkaufen}

Entscheidet sich der Spieler für das Einkaufen von Bauteilen, muss er zunächst den Bauteiltyp wählen und anschließend die einzukaufende Anzahl festlegen. Im Anschluss daran wird wieder die Liquidität des Spielers geprüft. Ist das Geld ausreichend, muss die Auswahl bestätigt werden. Fällt die Prüfung negativ aus, gelangt der Spieler wieder zur Aktivität “Bauteiltyp auswählen”. Dies wird in \vref{img:fachkonzept-aktivitaetsdiagramm-bauteile} verdeutlicht.

\begin{figure}[h]
  \centering
  \fbox{
    \includegraphics[width=0.4\textwidth]{30_Fachkonzept/15_aktivitaetsdiagramm/activity3.pdf}
  }
  \caption{Aktivitätsdiagramm zum Bauteileinkauf}
  \label{img:fachkonzept-aktivitaetsdiagramm-bauteile}
\end{figure}

\medskip

\textbf{Produktionsauftrag anlegen}

Beim Anlegen eines Produktionsauftrages, wie in \vref{img:fachkonzept-aktivitaetsdiagramm-produktion} dargestellt, ist zuerst das zu produzierende Raumschiff auszuwählen. Anschließend legt der Spieler die Anzahl fest. Nun muss geprüft werden, ob zum einen genügend Bauteile für die Produktion vorhanden sind und zum anderen ob das eingestellte Personal die Anzahl an Raumschiffen in einer Periode produzieren kann. Fällt die Prüfung positiv aus, so muss der Spieler seine Auswahl bestätigen. Sind die Kapazitäten jedoch nicht ausreichend, befindet sich der Spieler wieder bei der Aktivität “Raumschifftyp auswählen”.

\begin{figure}[h]
  \centering
  \fbox{
    \includegraphics[width=0.7\textwidth]{30_Fachkonzept/15_aktivitaetsdiagramm/activity4.pdf}
  }
  \caption{Aktivitätsdiagramm zur Produktion}
  \label{img:fachkonzept-aktivitaetsdiagramm-produktion}
\end{figure}

\medskip

\textbf{Verkaufsangebot abgeben}

Die letzte Transaktion kann im Bereich Verkauf getätigt werden. Dies ist auf \vref{img:fachkonzept-aktivitaetsdiagramm-verkauf} zu sehen. Zuerst wird der zu verkaufende Raumschifftyp ausgewählt. Danach entscheidet sich der Spieler für einen Preis, zu dem er die Raumschiffe verkaufen möchte. War bereits zuvor ein Angebot vorhanden, so wird dieses hiermit zurück genommen und danach bestätigt. War dies das erste Angebot, so gelangt der Spieler direkt zur Aktivität “Bestätigen”.

\begin{figure}[h]
  \centering
  \fbox{
    \includegraphics[width=0.7\textwidth]{30_Fachkonzept/15_aktivitaetsdiagramm/activity5.pdf}
  }
  \caption{Aktivitätsdiagramm zum Verkauf}
  \label{img:fachkonzept-aktivitaetsdiagramm-verkauf}
\end{figure}

\autorende{}

\chapter{Spielwelt}
\label{chp:spielwelt}

\section{Aktivitätsdiagramm}
\label{sec:fachkonzept-aktivitaetsdiagramm}

\autorbeginn{Fredrik, Julia}

Das Aktivitätsdiagramm auf \vref{img:fachkonzept-aktivitaetsdiagramm-uebersicht} soll den Spielablauf aus Sicht des Spielers verdeutlichen.  

Um das Spiel zu starten muss der Spieler einen Namen für sein Unternehmen festlegen. Dies stellt die erste Aktion dar. Danach analysiert der Spieler die ihm zur Verfügung stehenden Informationen. Ist dies abgeschlossen, so gelangt er zu einem Entscheidungsknoten. Hierbei kann sich der Spieler zwischen folgenden Aktivitäten entscheiden: Personal verwalten, Einkäufe tätigen, Produktionsaufträge anlegen oder Verkaufsangebot abgeben. Diese einzelnen Vorgänge werden im Laufe dieses Kapitels genauer erläutert. Er kann sich aber auch dazu entscheiden, keine Transaktion zu tätigen. 

Diese verschiedenen Aktivitäten werden in einem Entscheidungsknoten zusammengeführt. Hat der Spieler weiteren Informationsbedarf, so gelangt er zur Aktivität “Auswirkungen analysieren” und kann sich die Veränderungen anschauen, die seine Transaktion mit sich geführt hat. Besteht kein Informationsbedarf, kann er diese Aktivität überspringen. 

Möchte der Spieler nun weitere Transaktionen tätigen, so kann er wieder zum Entscheidungsknoten nach oben springen und hat wieder die Wahl zwischen Personal verwalten, Einkäufe tätigen, Produktionsaufträge anlegen oder ein Verkaufsangebot abgeben. Entscheidet sich der Spieler gegen eine weitere Transaktion, so folgt die Aktivität “Runde einchecken”. Anschließend folgt wieder ein Entscheidungsknoten. Existiert noch eine weitere Spielrunde, so gelangt der Spieler wieder zur Aktivität “Informationen analysieren”. War dies die letzte Spielrunde, so wird dem Spieler die Endbewertung angezeigt. Danach ist das Spiel beendet.

\begin{figure}[h]
  \centering
  \fbox{
    \includegraphics[width=0.9\textwidth]{30_Fachkonzept/15_aktivitaetsdiagramm/activity1.pdf}
  }
  \caption{Aktivitätsdiagramm}
  \label{img:fachkonzept-aktivitaetsdiagramm-uebersicht}
\end{figure}

\medskip

\textbf{Personal verwalten}

Zur genaueren Betrachtung der Aktion “Personal verwalten” dient folgendes Aktivitätsdiagramm auf \vref{img:fachkonzept-aktivitaetsdiagramm-personal}. Entscheidet sich der Spieler für diese Aktion, so hat er die Möglichkeit Personal einzustellen, aufzurüsten oder zu entlassen. 

Um neues Personal einzustellen, muss der Spieler zuerst den Personaltyp auswählen, welchen er einstellen möchte. Danach folgt die Aktivität “Anzahl festlegen”. Anschließend trifft der Spieler auf einen Entscheidungsknoten. Ist genügend Geld vorhanden, muss der Spieler die Auswahl bestätigen. Reicht das verfügbare Geld jedoch nicht aus, gelangt der Spieler wieder zur Aktivität “einzustellender Typ auswählen” und durchläuft den Prozess noch ein mal. 

Möchte der Spieler sein vorhandenes Personal aufrüsten, muss er zunächst den Personaltyp auswählen. Anschließend gelangt er zur Aktivität “Anzahl festlegen”. Ähnlich wie beim Einstellen von neuem Personal wird auch nun geprüft, ob genügend Geld vorhanden ist. Ist dies der Fall, wird die Auswahl bestätigt. Fehlt Geld, befindet sich der Spieler wieder bei der Aktivität “aufzurüstender Typ auswählen”.

Zum Entlassen von Personal ist ebenfalls der Personaltyp und die Anzahl festzulegen. Anschließend wird die Auswahl Bestätigt und die Transaktion ist abgeschlossen. 

\begin{figure}[h]
  \centering
  \fbox{
    \includegraphics[width=0.9\textwidth]{30_Fachkonzept/15_aktivitaetsdiagramm/activity2.pdf}
  }
  \caption{Aktivitätsdiagramm zur Personalverwaltung}
  \label{img:fachkonzept-aktivitaetsdiagramm-personal}
\end{figure}

\medskip

\textbf{Bauteile einkaufen}

Entscheidet sich der Spieler für das Einkaufen von Bauteilen, muss er zunächst den Bauteiltyp wählen und anschließend die einzukaufende Anzahl festlegen. Im Anschluss daran wird wieder die Liquidität des Spielers geprüft. Ist das Geld ausreichend, muss die Auswahl bestätigt werden. Fällt die Prüfung negativ aus, gelangt der Spieler wieder zur Aktivität “Bauteiltyp auswählen”. Dies wird in \vref{img:fachkonzept-aktivitaetsdiagramm-bauteile} verdeutlicht.

\begin{figure}[h]
  \centering
  \fbox{
    \includegraphics[width=0.4\textwidth]{30_Fachkonzept/15_aktivitaetsdiagramm/activity3.pdf}
  }
  \caption{Aktivitätsdiagramm zum Bauteileinkauf}
  \label{img:fachkonzept-aktivitaetsdiagramm-bauteile}
\end{figure}

\medskip

\textbf{Produktionsauftrag anlegen}

Beim Anlegen eines Produktionsauftrages, wie in \vref{img:fachkonzept-aktivitaetsdiagramm-produktion} dargestellt, ist zuerst das zu produzierende Raumschiff auszuwählen. Anschließend legt der Spieler die Anzahl fest. Nun muss geprüft werden, ob zum einen genügend Bauteile für die Produktion vorhanden sind und zum anderen ob das eingestellte Personal die Anzahl an Raumschiffen in einer Periode produzieren kann. Fällt die Prüfung positiv aus, so muss der Spieler seine Auswahl bestätigen. Sind die Kapazitäten jedoch nicht ausreichend, befindet sich der Spieler wieder bei der Aktivität “Raumschifftyp auswählen”.

\begin{figure}[h]
  \centering
  \fbox{
    \includegraphics[width=0.7\textwidth]{30_Fachkonzept/15_aktivitaetsdiagramm/activity4.pdf}
  }
  \caption{Aktivitätsdiagramm zur Produktion}
  \label{img:fachkonzept-aktivitaetsdiagramm-produktion}
\end{figure}

\medskip

\textbf{Verkaufsangebot abgeben}

Die letzte Transaktion kann im Bereich Verkauf getätigt werden. Dies ist auf \vref{img:fachkonzept-aktivitaetsdiagramm-verkauf} zu sehen. Zuerst wird der zu verkaufende Raumschifftyp ausgewählt. Danach entscheidet sich der Spieler für einen Preis, zu dem er die Raumschiffe verkaufen möchte. War bereits zuvor ein Angebot vorhanden, so wird dieses hiermit zurück genommen und danach bestätigt. War dies das erste Angebot, so gelangt der Spieler direkt zur Aktivität “Bestätigen”.

\begin{figure}[h]
  \centering
  \fbox{
    \includegraphics[width=0.7\textwidth]{30_Fachkonzept/15_aktivitaetsdiagramm/activity5.pdf}
  }
  \caption{Aktivitätsdiagramm zum Verkauf}
  \label{img:fachkonzept-aktivitaetsdiagramm-verkauf}
\end{figure}

\autorende{}
\autorbeginn{Britta}
\section{Aktivitätsdiagramm}
\label{sec:fachkonzept-aktivitaetsdiagramm}

\autorbeginn{Fredrik, Julia}

Das Aktivitätsdiagramm auf \vref{img:fachkonzept-aktivitaetsdiagramm-uebersicht} soll den Spielablauf aus Sicht des Spielers verdeutlichen.  

Um das Spiel zu starten muss der Spieler einen Namen für sein Unternehmen festlegen. Dies stellt die erste Aktion dar. Danach analysiert der Spieler die ihm zur Verfügung stehenden Informationen. Ist dies abgeschlossen, so gelangt er zu einem Entscheidungsknoten. Hierbei kann sich der Spieler zwischen folgenden Aktivitäten entscheiden: Personal verwalten, Einkäufe tätigen, Produktionsaufträge anlegen oder Verkaufsangebot abgeben. Diese einzelnen Vorgänge werden im Laufe dieses Kapitels genauer erläutert. Er kann sich aber auch dazu entscheiden, keine Transaktion zu tätigen. 

Diese verschiedenen Aktivitäten werden in einem Entscheidungsknoten zusammengeführt. Hat der Spieler weiteren Informationsbedarf, so gelangt er zur Aktivität “Auswirkungen analysieren” und kann sich die Veränderungen anschauen, die seine Transaktion mit sich geführt hat. Besteht kein Informationsbedarf, kann er diese Aktivität überspringen. 

Möchte der Spieler nun weitere Transaktionen tätigen, so kann er wieder zum Entscheidungsknoten nach oben springen und hat wieder die Wahl zwischen Personal verwalten, Einkäufe tätigen, Produktionsaufträge anlegen oder ein Verkaufsangebot abgeben. Entscheidet sich der Spieler gegen eine weitere Transaktion, so folgt die Aktivität “Runde einchecken”. Anschließend folgt wieder ein Entscheidungsknoten. Existiert noch eine weitere Spielrunde, so gelangt der Spieler wieder zur Aktivität “Informationen analysieren”. War dies die letzte Spielrunde, so wird dem Spieler die Endbewertung angezeigt. Danach ist das Spiel beendet.

\begin{figure}[h]
  \centering
  \fbox{
    \includegraphics[width=0.9\textwidth]{30_Fachkonzept/15_aktivitaetsdiagramm/activity1.pdf}
  }
  \caption{Aktivitätsdiagramm}
  \label{img:fachkonzept-aktivitaetsdiagramm-uebersicht}
\end{figure}

\medskip

\textbf{Personal verwalten}

Zur genaueren Betrachtung der Aktion “Personal verwalten” dient folgendes Aktivitätsdiagramm auf \vref{img:fachkonzept-aktivitaetsdiagramm-personal}. Entscheidet sich der Spieler für diese Aktion, so hat er die Möglichkeit Personal einzustellen, aufzurüsten oder zu entlassen. 

Um neues Personal einzustellen, muss der Spieler zuerst den Personaltyp auswählen, welchen er einstellen möchte. Danach folgt die Aktivität “Anzahl festlegen”. Anschließend trifft der Spieler auf einen Entscheidungsknoten. Ist genügend Geld vorhanden, muss der Spieler die Auswahl bestätigen. Reicht das verfügbare Geld jedoch nicht aus, gelangt der Spieler wieder zur Aktivität “einzustellender Typ auswählen” und durchläuft den Prozess noch ein mal. 

Möchte der Spieler sein vorhandenes Personal aufrüsten, muss er zunächst den Personaltyp auswählen. Anschließend gelangt er zur Aktivität “Anzahl festlegen”. Ähnlich wie beim Einstellen von neuem Personal wird auch nun geprüft, ob genügend Geld vorhanden ist. Ist dies der Fall, wird die Auswahl bestätigt. Fehlt Geld, befindet sich der Spieler wieder bei der Aktivität “aufzurüstender Typ auswählen”.

Zum Entlassen von Personal ist ebenfalls der Personaltyp und die Anzahl festzulegen. Anschließend wird die Auswahl Bestätigt und die Transaktion ist abgeschlossen. 

\begin{figure}[h]
  \centering
  \fbox{
    \includegraphics[width=0.9\textwidth]{30_Fachkonzept/15_aktivitaetsdiagramm/activity2.pdf}
  }
  \caption{Aktivitätsdiagramm zur Personalverwaltung}
  \label{img:fachkonzept-aktivitaetsdiagramm-personal}
\end{figure}

\medskip

\textbf{Bauteile einkaufen}

Entscheidet sich der Spieler für das Einkaufen von Bauteilen, muss er zunächst den Bauteiltyp wählen und anschließend die einzukaufende Anzahl festlegen. Im Anschluss daran wird wieder die Liquidität des Spielers geprüft. Ist das Geld ausreichend, muss die Auswahl bestätigt werden. Fällt die Prüfung negativ aus, gelangt der Spieler wieder zur Aktivität “Bauteiltyp auswählen”. Dies wird in \vref{img:fachkonzept-aktivitaetsdiagramm-bauteile} verdeutlicht.

\begin{figure}[h]
  \centering
  \fbox{
    \includegraphics[width=0.4\textwidth]{30_Fachkonzept/15_aktivitaetsdiagramm/activity3.pdf}
  }
  \caption{Aktivitätsdiagramm zum Bauteileinkauf}
  \label{img:fachkonzept-aktivitaetsdiagramm-bauteile}
\end{figure}

\medskip

\textbf{Produktionsauftrag anlegen}

Beim Anlegen eines Produktionsauftrages, wie in \vref{img:fachkonzept-aktivitaetsdiagramm-produktion} dargestellt, ist zuerst das zu produzierende Raumschiff auszuwählen. Anschließend legt der Spieler die Anzahl fest. Nun muss geprüft werden, ob zum einen genügend Bauteile für die Produktion vorhanden sind und zum anderen ob das eingestellte Personal die Anzahl an Raumschiffen in einer Periode produzieren kann. Fällt die Prüfung positiv aus, so muss der Spieler seine Auswahl bestätigen. Sind die Kapazitäten jedoch nicht ausreichend, befindet sich der Spieler wieder bei der Aktivität “Raumschifftyp auswählen”.

\begin{figure}[h]
  \centering
  \fbox{
    \includegraphics[width=0.7\textwidth]{30_Fachkonzept/15_aktivitaetsdiagramm/activity4.pdf}
  }
  \caption{Aktivitätsdiagramm zur Produktion}
  \label{img:fachkonzept-aktivitaetsdiagramm-produktion}
\end{figure}

\medskip

\textbf{Verkaufsangebot abgeben}

Die letzte Transaktion kann im Bereich Verkauf getätigt werden. Dies ist auf \vref{img:fachkonzept-aktivitaetsdiagramm-verkauf} zu sehen. Zuerst wird der zu verkaufende Raumschifftyp ausgewählt. Danach entscheidet sich der Spieler für einen Preis, zu dem er die Raumschiffe verkaufen möchte. War bereits zuvor ein Angebot vorhanden, so wird dieses hiermit zurück genommen und danach bestätigt. War dies das erste Angebot, so gelangt der Spieler direkt zur Aktivität “Bestätigen”.

\begin{figure}[h]
  \centering
  \fbox{
    \includegraphics[width=0.7\textwidth]{30_Fachkonzept/15_aktivitaetsdiagramm/activity5.pdf}
  }
  \caption{Aktivitätsdiagramm zum Verkauf}
  \label{img:fachkonzept-aktivitaetsdiagramm-verkauf}
\end{figure}

\autorende{}
\autorende{}
\section{Aktivitätsdiagramm}
\label{sec:fachkonzept-aktivitaetsdiagramm}

\autorbeginn{Fredrik, Julia}

Das Aktivitätsdiagramm auf \vref{img:fachkonzept-aktivitaetsdiagramm-uebersicht} soll den Spielablauf aus Sicht des Spielers verdeutlichen.  

Um das Spiel zu starten muss der Spieler einen Namen für sein Unternehmen festlegen. Dies stellt die erste Aktion dar. Danach analysiert der Spieler die ihm zur Verfügung stehenden Informationen. Ist dies abgeschlossen, so gelangt er zu einem Entscheidungsknoten. Hierbei kann sich der Spieler zwischen folgenden Aktivitäten entscheiden: Personal verwalten, Einkäufe tätigen, Produktionsaufträge anlegen oder Verkaufsangebot abgeben. Diese einzelnen Vorgänge werden im Laufe dieses Kapitels genauer erläutert. Er kann sich aber auch dazu entscheiden, keine Transaktion zu tätigen. 

Diese verschiedenen Aktivitäten werden in einem Entscheidungsknoten zusammengeführt. Hat der Spieler weiteren Informationsbedarf, so gelangt er zur Aktivität “Auswirkungen analysieren” und kann sich die Veränderungen anschauen, die seine Transaktion mit sich geführt hat. Besteht kein Informationsbedarf, kann er diese Aktivität überspringen. 

Möchte der Spieler nun weitere Transaktionen tätigen, so kann er wieder zum Entscheidungsknoten nach oben springen und hat wieder die Wahl zwischen Personal verwalten, Einkäufe tätigen, Produktionsaufträge anlegen oder ein Verkaufsangebot abgeben. Entscheidet sich der Spieler gegen eine weitere Transaktion, so folgt die Aktivität “Runde einchecken”. Anschließend folgt wieder ein Entscheidungsknoten. Existiert noch eine weitere Spielrunde, so gelangt der Spieler wieder zur Aktivität “Informationen analysieren”. War dies die letzte Spielrunde, so wird dem Spieler die Endbewertung angezeigt. Danach ist das Spiel beendet.

\begin{figure}[h]
  \centering
  \fbox{
    \includegraphics[width=0.9\textwidth]{30_Fachkonzept/15_aktivitaetsdiagramm/activity1.pdf}
  }
  \caption{Aktivitätsdiagramm}
  \label{img:fachkonzept-aktivitaetsdiagramm-uebersicht}
\end{figure}

\medskip

\textbf{Personal verwalten}

Zur genaueren Betrachtung der Aktion “Personal verwalten” dient folgendes Aktivitätsdiagramm auf \vref{img:fachkonzept-aktivitaetsdiagramm-personal}. Entscheidet sich der Spieler für diese Aktion, so hat er die Möglichkeit Personal einzustellen, aufzurüsten oder zu entlassen. 

Um neues Personal einzustellen, muss der Spieler zuerst den Personaltyp auswählen, welchen er einstellen möchte. Danach folgt die Aktivität “Anzahl festlegen”. Anschließend trifft der Spieler auf einen Entscheidungsknoten. Ist genügend Geld vorhanden, muss der Spieler die Auswahl bestätigen. Reicht das verfügbare Geld jedoch nicht aus, gelangt der Spieler wieder zur Aktivität “einzustellender Typ auswählen” und durchläuft den Prozess noch ein mal. 

Möchte der Spieler sein vorhandenes Personal aufrüsten, muss er zunächst den Personaltyp auswählen. Anschließend gelangt er zur Aktivität “Anzahl festlegen”. Ähnlich wie beim Einstellen von neuem Personal wird auch nun geprüft, ob genügend Geld vorhanden ist. Ist dies der Fall, wird die Auswahl bestätigt. Fehlt Geld, befindet sich der Spieler wieder bei der Aktivität “aufzurüstender Typ auswählen”.

Zum Entlassen von Personal ist ebenfalls der Personaltyp und die Anzahl festzulegen. Anschließend wird die Auswahl Bestätigt und die Transaktion ist abgeschlossen. 

\begin{figure}[h]
  \centering
  \fbox{
    \includegraphics[width=0.9\textwidth]{30_Fachkonzept/15_aktivitaetsdiagramm/activity2.pdf}
  }
  \caption{Aktivitätsdiagramm zur Personalverwaltung}
  \label{img:fachkonzept-aktivitaetsdiagramm-personal}
\end{figure}

\medskip

\textbf{Bauteile einkaufen}

Entscheidet sich der Spieler für das Einkaufen von Bauteilen, muss er zunächst den Bauteiltyp wählen und anschließend die einzukaufende Anzahl festlegen. Im Anschluss daran wird wieder die Liquidität des Spielers geprüft. Ist das Geld ausreichend, muss die Auswahl bestätigt werden. Fällt die Prüfung negativ aus, gelangt der Spieler wieder zur Aktivität “Bauteiltyp auswählen”. Dies wird in \vref{img:fachkonzept-aktivitaetsdiagramm-bauteile} verdeutlicht.

\begin{figure}[h]
  \centering
  \fbox{
    \includegraphics[width=0.4\textwidth]{30_Fachkonzept/15_aktivitaetsdiagramm/activity3.pdf}
  }
  \caption{Aktivitätsdiagramm zum Bauteileinkauf}
  \label{img:fachkonzept-aktivitaetsdiagramm-bauteile}
\end{figure}

\medskip

\textbf{Produktionsauftrag anlegen}

Beim Anlegen eines Produktionsauftrages, wie in \vref{img:fachkonzept-aktivitaetsdiagramm-produktion} dargestellt, ist zuerst das zu produzierende Raumschiff auszuwählen. Anschließend legt der Spieler die Anzahl fest. Nun muss geprüft werden, ob zum einen genügend Bauteile für die Produktion vorhanden sind und zum anderen ob das eingestellte Personal die Anzahl an Raumschiffen in einer Periode produzieren kann. Fällt die Prüfung positiv aus, so muss der Spieler seine Auswahl bestätigen. Sind die Kapazitäten jedoch nicht ausreichend, befindet sich der Spieler wieder bei der Aktivität “Raumschifftyp auswählen”.

\begin{figure}[h]
  \centering
  \fbox{
    \includegraphics[width=0.7\textwidth]{30_Fachkonzept/15_aktivitaetsdiagramm/activity4.pdf}
  }
  \caption{Aktivitätsdiagramm zur Produktion}
  \label{img:fachkonzept-aktivitaetsdiagramm-produktion}
\end{figure}

\medskip

\textbf{Verkaufsangebot abgeben}

Die letzte Transaktion kann im Bereich Verkauf getätigt werden. Dies ist auf \vref{img:fachkonzept-aktivitaetsdiagramm-verkauf} zu sehen. Zuerst wird der zu verkaufende Raumschifftyp ausgewählt. Danach entscheidet sich der Spieler für einen Preis, zu dem er die Raumschiffe verkaufen möchte. War bereits zuvor ein Angebot vorhanden, so wird dieses hiermit zurück genommen und danach bestätigt. War dies das erste Angebot, so gelangt der Spieler direkt zur Aktivität “Bestätigen”.

\begin{figure}[h]
  \centering
  \fbox{
    \includegraphics[width=0.7\textwidth]{30_Fachkonzept/15_aktivitaetsdiagramm/activity5.pdf}
  }
  \caption{Aktivitätsdiagramm zum Verkauf}
  \label{img:fachkonzept-aktivitaetsdiagramm-verkauf}
\end{figure}

\autorende{}
\autorbeginn{Britta}
\section{Aktivitätsdiagramm}
\label{sec:fachkonzept-aktivitaetsdiagramm}

\autorbeginn{Fredrik, Julia}

Das Aktivitätsdiagramm auf \vref{img:fachkonzept-aktivitaetsdiagramm-uebersicht} soll den Spielablauf aus Sicht des Spielers verdeutlichen.  

Um das Spiel zu starten muss der Spieler einen Namen für sein Unternehmen festlegen. Dies stellt die erste Aktion dar. Danach analysiert der Spieler die ihm zur Verfügung stehenden Informationen. Ist dies abgeschlossen, so gelangt er zu einem Entscheidungsknoten. Hierbei kann sich der Spieler zwischen folgenden Aktivitäten entscheiden: Personal verwalten, Einkäufe tätigen, Produktionsaufträge anlegen oder Verkaufsangebot abgeben. Diese einzelnen Vorgänge werden im Laufe dieses Kapitels genauer erläutert. Er kann sich aber auch dazu entscheiden, keine Transaktion zu tätigen. 

Diese verschiedenen Aktivitäten werden in einem Entscheidungsknoten zusammengeführt. Hat der Spieler weiteren Informationsbedarf, so gelangt er zur Aktivität “Auswirkungen analysieren” und kann sich die Veränderungen anschauen, die seine Transaktion mit sich geführt hat. Besteht kein Informationsbedarf, kann er diese Aktivität überspringen. 

Möchte der Spieler nun weitere Transaktionen tätigen, so kann er wieder zum Entscheidungsknoten nach oben springen und hat wieder die Wahl zwischen Personal verwalten, Einkäufe tätigen, Produktionsaufträge anlegen oder ein Verkaufsangebot abgeben. Entscheidet sich der Spieler gegen eine weitere Transaktion, so folgt die Aktivität “Runde einchecken”. Anschließend folgt wieder ein Entscheidungsknoten. Existiert noch eine weitere Spielrunde, so gelangt der Spieler wieder zur Aktivität “Informationen analysieren”. War dies die letzte Spielrunde, so wird dem Spieler die Endbewertung angezeigt. Danach ist das Spiel beendet.

\begin{figure}[h]
  \centering
  \fbox{
    \includegraphics[width=0.9\textwidth]{30_Fachkonzept/15_aktivitaetsdiagramm/activity1.pdf}
  }
  \caption{Aktivitätsdiagramm}
  \label{img:fachkonzept-aktivitaetsdiagramm-uebersicht}
\end{figure}

\medskip

\textbf{Personal verwalten}

Zur genaueren Betrachtung der Aktion “Personal verwalten” dient folgendes Aktivitätsdiagramm auf \vref{img:fachkonzept-aktivitaetsdiagramm-personal}. Entscheidet sich der Spieler für diese Aktion, so hat er die Möglichkeit Personal einzustellen, aufzurüsten oder zu entlassen. 

Um neues Personal einzustellen, muss der Spieler zuerst den Personaltyp auswählen, welchen er einstellen möchte. Danach folgt die Aktivität “Anzahl festlegen”. Anschließend trifft der Spieler auf einen Entscheidungsknoten. Ist genügend Geld vorhanden, muss der Spieler die Auswahl bestätigen. Reicht das verfügbare Geld jedoch nicht aus, gelangt der Spieler wieder zur Aktivität “einzustellender Typ auswählen” und durchläuft den Prozess noch ein mal. 

Möchte der Spieler sein vorhandenes Personal aufrüsten, muss er zunächst den Personaltyp auswählen. Anschließend gelangt er zur Aktivität “Anzahl festlegen”. Ähnlich wie beim Einstellen von neuem Personal wird auch nun geprüft, ob genügend Geld vorhanden ist. Ist dies der Fall, wird die Auswahl bestätigt. Fehlt Geld, befindet sich der Spieler wieder bei der Aktivität “aufzurüstender Typ auswählen”.

Zum Entlassen von Personal ist ebenfalls der Personaltyp und die Anzahl festzulegen. Anschließend wird die Auswahl Bestätigt und die Transaktion ist abgeschlossen. 

\begin{figure}[h]
  \centering
  \fbox{
    \includegraphics[width=0.9\textwidth]{30_Fachkonzept/15_aktivitaetsdiagramm/activity2.pdf}
  }
  \caption{Aktivitätsdiagramm zur Personalverwaltung}
  \label{img:fachkonzept-aktivitaetsdiagramm-personal}
\end{figure}

\medskip

\textbf{Bauteile einkaufen}

Entscheidet sich der Spieler für das Einkaufen von Bauteilen, muss er zunächst den Bauteiltyp wählen und anschließend die einzukaufende Anzahl festlegen. Im Anschluss daran wird wieder die Liquidität des Spielers geprüft. Ist das Geld ausreichend, muss die Auswahl bestätigt werden. Fällt die Prüfung negativ aus, gelangt der Spieler wieder zur Aktivität “Bauteiltyp auswählen”. Dies wird in \vref{img:fachkonzept-aktivitaetsdiagramm-bauteile} verdeutlicht.

\begin{figure}[h]
  \centering
  \fbox{
    \includegraphics[width=0.4\textwidth]{30_Fachkonzept/15_aktivitaetsdiagramm/activity3.pdf}
  }
  \caption{Aktivitätsdiagramm zum Bauteileinkauf}
  \label{img:fachkonzept-aktivitaetsdiagramm-bauteile}
\end{figure}

\medskip

\textbf{Produktionsauftrag anlegen}

Beim Anlegen eines Produktionsauftrages, wie in \vref{img:fachkonzept-aktivitaetsdiagramm-produktion} dargestellt, ist zuerst das zu produzierende Raumschiff auszuwählen. Anschließend legt der Spieler die Anzahl fest. Nun muss geprüft werden, ob zum einen genügend Bauteile für die Produktion vorhanden sind und zum anderen ob das eingestellte Personal die Anzahl an Raumschiffen in einer Periode produzieren kann. Fällt die Prüfung positiv aus, so muss der Spieler seine Auswahl bestätigen. Sind die Kapazitäten jedoch nicht ausreichend, befindet sich der Spieler wieder bei der Aktivität “Raumschifftyp auswählen”.

\begin{figure}[h]
  \centering
  \fbox{
    \includegraphics[width=0.7\textwidth]{30_Fachkonzept/15_aktivitaetsdiagramm/activity4.pdf}
  }
  \caption{Aktivitätsdiagramm zur Produktion}
  \label{img:fachkonzept-aktivitaetsdiagramm-produktion}
\end{figure}

\medskip

\textbf{Verkaufsangebot abgeben}

Die letzte Transaktion kann im Bereich Verkauf getätigt werden. Dies ist auf \vref{img:fachkonzept-aktivitaetsdiagramm-verkauf} zu sehen. Zuerst wird der zu verkaufende Raumschifftyp ausgewählt. Danach entscheidet sich der Spieler für einen Preis, zu dem er die Raumschiffe verkaufen möchte. War bereits zuvor ein Angebot vorhanden, so wird dieses hiermit zurück genommen und danach bestätigt. War dies das erste Angebot, so gelangt der Spieler direkt zur Aktivität “Bestätigen”.

\begin{figure}[h]
  \centering
  \fbox{
    \includegraphics[width=0.7\textwidth]{30_Fachkonzept/15_aktivitaetsdiagramm/activity5.pdf}
  }
  \caption{Aktivitätsdiagramm zum Verkauf}
  \label{img:fachkonzept-aktivitaetsdiagramm-verkauf}
\end{figure}

\autorende{}
\autorende{}
\section{Aktivitätsdiagramm}
\label{sec:fachkonzept-aktivitaetsdiagramm}

\autorbeginn{Fredrik, Julia}

Das Aktivitätsdiagramm auf \vref{img:fachkonzept-aktivitaetsdiagramm-uebersicht} soll den Spielablauf aus Sicht des Spielers verdeutlichen.  

Um das Spiel zu starten muss der Spieler einen Namen für sein Unternehmen festlegen. Dies stellt die erste Aktion dar. Danach analysiert der Spieler die ihm zur Verfügung stehenden Informationen. Ist dies abgeschlossen, so gelangt er zu einem Entscheidungsknoten. Hierbei kann sich der Spieler zwischen folgenden Aktivitäten entscheiden: Personal verwalten, Einkäufe tätigen, Produktionsaufträge anlegen oder Verkaufsangebot abgeben. Diese einzelnen Vorgänge werden im Laufe dieses Kapitels genauer erläutert. Er kann sich aber auch dazu entscheiden, keine Transaktion zu tätigen. 

Diese verschiedenen Aktivitäten werden in einem Entscheidungsknoten zusammengeführt. Hat der Spieler weiteren Informationsbedarf, so gelangt er zur Aktivität “Auswirkungen analysieren” und kann sich die Veränderungen anschauen, die seine Transaktion mit sich geführt hat. Besteht kein Informationsbedarf, kann er diese Aktivität überspringen. 

Möchte der Spieler nun weitere Transaktionen tätigen, so kann er wieder zum Entscheidungsknoten nach oben springen und hat wieder die Wahl zwischen Personal verwalten, Einkäufe tätigen, Produktionsaufträge anlegen oder ein Verkaufsangebot abgeben. Entscheidet sich der Spieler gegen eine weitere Transaktion, so folgt die Aktivität “Runde einchecken”. Anschließend folgt wieder ein Entscheidungsknoten. Existiert noch eine weitere Spielrunde, so gelangt der Spieler wieder zur Aktivität “Informationen analysieren”. War dies die letzte Spielrunde, so wird dem Spieler die Endbewertung angezeigt. Danach ist das Spiel beendet.

\begin{figure}[h]
  \centering
  \fbox{
    \includegraphics[width=0.9\textwidth]{30_Fachkonzept/15_aktivitaetsdiagramm/activity1.pdf}
  }
  \caption{Aktivitätsdiagramm}
  \label{img:fachkonzept-aktivitaetsdiagramm-uebersicht}
\end{figure}

\medskip

\textbf{Personal verwalten}

Zur genaueren Betrachtung der Aktion “Personal verwalten” dient folgendes Aktivitätsdiagramm auf \vref{img:fachkonzept-aktivitaetsdiagramm-personal}. Entscheidet sich der Spieler für diese Aktion, so hat er die Möglichkeit Personal einzustellen, aufzurüsten oder zu entlassen. 

Um neues Personal einzustellen, muss der Spieler zuerst den Personaltyp auswählen, welchen er einstellen möchte. Danach folgt die Aktivität “Anzahl festlegen”. Anschließend trifft der Spieler auf einen Entscheidungsknoten. Ist genügend Geld vorhanden, muss der Spieler die Auswahl bestätigen. Reicht das verfügbare Geld jedoch nicht aus, gelangt der Spieler wieder zur Aktivität “einzustellender Typ auswählen” und durchläuft den Prozess noch ein mal. 

Möchte der Spieler sein vorhandenes Personal aufrüsten, muss er zunächst den Personaltyp auswählen. Anschließend gelangt er zur Aktivität “Anzahl festlegen”. Ähnlich wie beim Einstellen von neuem Personal wird auch nun geprüft, ob genügend Geld vorhanden ist. Ist dies der Fall, wird die Auswahl bestätigt. Fehlt Geld, befindet sich der Spieler wieder bei der Aktivität “aufzurüstender Typ auswählen”.

Zum Entlassen von Personal ist ebenfalls der Personaltyp und die Anzahl festzulegen. Anschließend wird die Auswahl Bestätigt und die Transaktion ist abgeschlossen. 

\begin{figure}[h]
  \centering
  \fbox{
    \includegraphics[width=0.9\textwidth]{30_Fachkonzept/15_aktivitaetsdiagramm/activity2.pdf}
  }
  \caption{Aktivitätsdiagramm zur Personalverwaltung}
  \label{img:fachkonzept-aktivitaetsdiagramm-personal}
\end{figure}

\medskip

\textbf{Bauteile einkaufen}

Entscheidet sich der Spieler für das Einkaufen von Bauteilen, muss er zunächst den Bauteiltyp wählen und anschließend die einzukaufende Anzahl festlegen. Im Anschluss daran wird wieder die Liquidität des Spielers geprüft. Ist das Geld ausreichend, muss die Auswahl bestätigt werden. Fällt die Prüfung negativ aus, gelangt der Spieler wieder zur Aktivität “Bauteiltyp auswählen”. Dies wird in \vref{img:fachkonzept-aktivitaetsdiagramm-bauteile} verdeutlicht.

\begin{figure}[h]
  \centering
  \fbox{
    \includegraphics[width=0.4\textwidth]{30_Fachkonzept/15_aktivitaetsdiagramm/activity3.pdf}
  }
  \caption{Aktivitätsdiagramm zum Bauteileinkauf}
  \label{img:fachkonzept-aktivitaetsdiagramm-bauteile}
\end{figure}

\medskip

\textbf{Produktionsauftrag anlegen}

Beim Anlegen eines Produktionsauftrages, wie in \vref{img:fachkonzept-aktivitaetsdiagramm-produktion} dargestellt, ist zuerst das zu produzierende Raumschiff auszuwählen. Anschließend legt der Spieler die Anzahl fest. Nun muss geprüft werden, ob zum einen genügend Bauteile für die Produktion vorhanden sind und zum anderen ob das eingestellte Personal die Anzahl an Raumschiffen in einer Periode produzieren kann. Fällt die Prüfung positiv aus, so muss der Spieler seine Auswahl bestätigen. Sind die Kapazitäten jedoch nicht ausreichend, befindet sich der Spieler wieder bei der Aktivität “Raumschifftyp auswählen”.

\begin{figure}[h]
  \centering
  \fbox{
    \includegraphics[width=0.7\textwidth]{30_Fachkonzept/15_aktivitaetsdiagramm/activity4.pdf}
  }
  \caption{Aktivitätsdiagramm zur Produktion}
  \label{img:fachkonzept-aktivitaetsdiagramm-produktion}
\end{figure}

\medskip

\textbf{Verkaufsangebot abgeben}

Die letzte Transaktion kann im Bereich Verkauf getätigt werden. Dies ist auf \vref{img:fachkonzept-aktivitaetsdiagramm-verkauf} zu sehen. Zuerst wird der zu verkaufende Raumschifftyp ausgewählt. Danach entscheidet sich der Spieler für einen Preis, zu dem er die Raumschiffe verkaufen möchte. War bereits zuvor ein Angebot vorhanden, so wird dieses hiermit zurück genommen und danach bestätigt. War dies das erste Angebot, so gelangt der Spieler direkt zur Aktivität “Bestätigen”.

\begin{figure}[h]
  \centering
  \fbox{
    \includegraphics[width=0.7\textwidth]{30_Fachkonzept/15_aktivitaetsdiagramm/activity5.pdf}
  }
  \caption{Aktivitätsdiagramm zum Verkauf}
  \label{img:fachkonzept-aktivitaetsdiagramm-verkauf}
\end{figure}

\autorende{}

\chapter{Spielwelt}
\label{chp:spielwelt}

\section{Aktivitätsdiagramm}
\label{sec:fachkonzept-aktivitaetsdiagramm}

\autorbeginn{Fredrik, Julia}

Das Aktivitätsdiagramm auf \vref{img:fachkonzept-aktivitaetsdiagramm-uebersicht} soll den Spielablauf aus Sicht des Spielers verdeutlichen.  

Um das Spiel zu starten muss der Spieler einen Namen für sein Unternehmen festlegen. Dies stellt die erste Aktion dar. Danach analysiert der Spieler die ihm zur Verfügung stehenden Informationen. Ist dies abgeschlossen, so gelangt er zu einem Entscheidungsknoten. Hierbei kann sich der Spieler zwischen folgenden Aktivitäten entscheiden: Personal verwalten, Einkäufe tätigen, Produktionsaufträge anlegen oder Verkaufsangebot abgeben. Diese einzelnen Vorgänge werden im Laufe dieses Kapitels genauer erläutert. Er kann sich aber auch dazu entscheiden, keine Transaktion zu tätigen. 

Diese verschiedenen Aktivitäten werden in einem Entscheidungsknoten zusammengeführt. Hat der Spieler weiteren Informationsbedarf, so gelangt er zur Aktivität “Auswirkungen analysieren” und kann sich die Veränderungen anschauen, die seine Transaktion mit sich geführt hat. Besteht kein Informationsbedarf, kann er diese Aktivität überspringen. 

Möchte der Spieler nun weitere Transaktionen tätigen, so kann er wieder zum Entscheidungsknoten nach oben springen und hat wieder die Wahl zwischen Personal verwalten, Einkäufe tätigen, Produktionsaufträge anlegen oder ein Verkaufsangebot abgeben. Entscheidet sich der Spieler gegen eine weitere Transaktion, so folgt die Aktivität “Runde einchecken”. Anschließend folgt wieder ein Entscheidungsknoten. Existiert noch eine weitere Spielrunde, so gelangt der Spieler wieder zur Aktivität “Informationen analysieren”. War dies die letzte Spielrunde, so wird dem Spieler die Endbewertung angezeigt. Danach ist das Spiel beendet.

\begin{figure}[h]
  \centering
  \fbox{
    \includegraphics[width=0.9\textwidth]{30_Fachkonzept/15_aktivitaetsdiagramm/activity1.pdf}
  }
  \caption{Aktivitätsdiagramm}
  \label{img:fachkonzept-aktivitaetsdiagramm-uebersicht}
\end{figure}

\medskip

\textbf{Personal verwalten}

Zur genaueren Betrachtung der Aktion “Personal verwalten” dient folgendes Aktivitätsdiagramm auf \vref{img:fachkonzept-aktivitaetsdiagramm-personal}. Entscheidet sich der Spieler für diese Aktion, so hat er die Möglichkeit Personal einzustellen, aufzurüsten oder zu entlassen. 

Um neues Personal einzustellen, muss der Spieler zuerst den Personaltyp auswählen, welchen er einstellen möchte. Danach folgt die Aktivität “Anzahl festlegen”. Anschließend trifft der Spieler auf einen Entscheidungsknoten. Ist genügend Geld vorhanden, muss der Spieler die Auswahl bestätigen. Reicht das verfügbare Geld jedoch nicht aus, gelangt der Spieler wieder zur Aktivität “einzustellender Typ auswählen” und durchläuft den Prozess noch ein mal. 

Möchte der Spieler sein vorhandenes Personal aufrüsten, muss er zunächst den Personaltyp auswählen. Anschließend gelangt er zur Aktivität “Anzahl festlegen”. Ähnlich wie beim Einstellen von neuem Personal wird auch nun geprüft, ob genügend Geld vorhanden ist. Ist dies der Fall, wird die Auswahl bestätigt. Fehlt Geld, befindet sich der Spieler wieder bei der Aktivität “aufzurüstender Typ auswählen”.

Zum Entlassen von Personal ist ebenfalls der Personaltyp und die Anzahl festzulegen. Anschließend wird die Auswahl Bestätigt und die Transaktion ist abgeschlossen. 

\begin{figure}[h]
  \centering
  \fbox{
    \includegraphics[width=0.9\textwidth]{30_Fachkonzept/15_aktivitaetsdiagramm/activity2.pdf}
  }
  \caption{Aktivitätsdiagramm zur Personalverwaltung}
  \label{img:fachkonzept-aktivitaetsdiagramm-personal}
\end{figure}

\medskip

\textbf{Bauteile einkaufen}

Entscheidet sich der Spieler für das Einkaufen von Bauteilen, muss er zunächst den Bauteiltyp wählen und anschließend die einzukaufende Anzahl festlegen. Im Anschluss daran wird wieder die Liquidität des Spielers geprüft. Ist das Geld ausreichend, muss die Auswahl bestätigt werden. Fällt die Prüfung negativ aus, gelangt der Spieler wieder zur Aktivität “Bauteiltyp auswählen”. Dies wird in \vref{img:fachkonzept-aktivitaetsdiagramm-bauteile} verdeutlicht.

\begin{figure}[h]
  \centering
  \fbox{
    \includegraphics[width=0.4\textwidth]{30_Fachkonzept/15_aktivitaetsdiagramm/activity3.pdf}
  }
  \caption{Aktivitätsdiagramm zum Bauteileinkauf}
  \label{img:fachkonzept-aktivitaetsdiagramm-bauteile}
\end{figure}

\medskip

\textbf{Produktionsauftrag anlegen}

Beim Anlegen eines Produktionsauftrages, wie in \vref{img:fachkonzept-aktivitaetsdiagramm-produktion} dargestellt, ist zuerst das zu produzierende Raumschiff auszuwählen. Anschließend legt der Spieler die Anzahl fest. Nun muss geprüft werden, ob zum einen genügend Bauteile für die Produktion vorhanden sind und zum anderen ob das eingestellte Personal die Anzahl an Raumschiffen in einer Periode produzieren kann. Fällt die Prüfung positiv aus, so muss der Spieler seine Auswahl bestätigen. Sind die Kapazitäten jedoch nicht ausreichend, befindet sich der Spieler wieder bei der Aktivität “Raumschifftyp auswählen”.

\begin{figure}[h]
  \centering
  \fbox{
    \includegraphics[width=0.7\textwidth]{30_Fachkonzept/15_aktivitaetsdiagramm/activity4.pdf}
  }
  \caption{Aktivitätsdiagramm zur Produktion}
  \label{img:fachkonzept-aktivitaetsdiagramm-produktion}
\end{figure}

\medskip

\textbf{Verkaufsangebot abgeben}

Die letzte Transaktion kann im Bereich Verkauf getätigt werden. Dies ist auf \vref{img:fachkonzept-aktivitaetsdiagramm-verkauf} zu sehen. Zuerst wird der zu verkaufende Raumschifftyp ausgewählt. Danach entscheidet sich der Spieler für einen Preis, zu dem er die Raumschiffe verkaufen möchte. War bereits zuvor ein Angebot vorhanden, so wird dieses hiermit zurück genommen und danach bestätigt. War dies das erste Angebot, so gelangt der Spieler direkt zur Aktivität “Bestätigen”.

\begin{figure}[h]
  \centering
  \fbox{
    \includegraphics[width=0.7\textwidth]{30_Fachkonzept/15_aktivitaetsdiagramm/activity5.pdf}
  }
  \caption{Aktivitätsdiagramm zum Verkauf}
  \label{img:fachkonzept-aktivitaetsdiagramm-verkauf}
\end{figure}

\autorende{}
\autorbeginn{Britta}
\section{Aktivitätsdiagramm}
\label{sec:fachkonzept-aktivitaetsdiagramm}

\autorbeginn{Fredrik, Julia}

Das Aktivitätsdiagramm auf \vref{img:fachkonzept-aktivitaetsdiagramm-uebersicht} soll den Spielablauf aus Sicht des Spielers verdeutlichen.  

Um das Spiel zu starten muss der Spieler einen Namen für sein Unternehmen festlegen. Dies stellt die erste Aktion dar. Danach analysiert der Spieler die ihm zur Verfügung stehenden Informationen. Ist dies abgeschlossen, so gelangt er zu einem Entscheidungsknoten. Hierbei kann sich der Spieler zwischen folgenden Aktivitäten entscheiden: Personal verwalten, Einkäufe tätigen, Produktionsaufträge anlegen oder Verkaufsangebot abgeben. Diese einzelnen Vorgänge werden im Laufe dieses Kapitels genauer erläutert. Er kann sich aber auch dazu entscheiden, keine Transaktion zu tätigen. 

Diese verschiedenen Aktivitäten werden in einem Entscheidungsknoten zusammengeführt. Hat der Spieler weiteren Informationsbedarf, so gelangt er zur Aktivität “Auswirkungen analysieren” und kann sich die Veränderungen anschauen, die seine Transaktion mit sich geführt hat. Besteht kein Informationsbedarf, kann er diese Aktivität überspringen. 

Möchte der Spieler nun weitere Transaktionen tätigen, so kann er wieder zum Entscheidungsknoten nach oben springen und hat wieder die Wahl zwischen Personal verwalten, Einkäufe tätigen, Produktionsaufträge anlegen oder ein Verkaufsangebot abgeben. Entscheidet sich der Spieler gegen eine weitere Transaktion, so folgt die Aktivität “Runde einchecken”. Anschließend folgt wieder ein Entscheidungsknoten. Existiert noch eine weitere Spielrunde, so gelangt der Spieler wieder zur Aktivität “Informationen analysieren”. War dies die letzte Spielrunde, so wird dem Spieler die Endbewertung angezeigt. Danach ist das Spiel beendet.

\begin{figure}[h]
  \centering
  \fbox{
    \includegraphics[width=0.9\textwidth]{30_Fachkonzept/15_aktivitaetsdiagramm/activity1.pdf}
  }
  \caption{Aktivitätsdiagramm}
  \label{img:fachkonzept-aktivitaetsdiagramm-uebersicht}
\end{figure}

\medskip

\textbf{Personal verwalten}

Zur genaueren Betrachtung der Aktion “Personal verwalten” dient folgendes Aktivitätsdiagramm auf \vref{img:fachkonzept-aktivitaetsdiagramm-personal}. Entscheidet sich der Spieler für diese Aktion, so hat er die Möglichkeit Personal einzustellen, aufzurüsten oder zu entlassen. 

Um neues Personal einzustellen, muss der Spieler zuerst den Personaltyp auswählen, welchen er einstellen möchte. Danach folgt die Aktivität “Anzahl festlegen”. Anschließend trifft der Spieler auf einen Entscheidungsknoten. Ist genügend Geld vorhanden, muss der Spieler die Auswahl bestätigen. Reicht das verfügbare Geld jedoch nicht aus, gelangt der Spieler wieder zur Aktivität “einzustellender Typ auswählen” und durchläuft den Prozess noch ein mal. 

Möchte der Spieler sein vorhandenes Personal aufrüsten, muss er zunächst den Personaltyp auswählen. Anschließend gelangt er zur Aktivität “Anzahl festlegen”. Ähnlich wie beim Einstellen von neuem Personal wird auch nun geprüft, ob genügend Geld vorhanden ist. Ist dies der Fall, wird die Auswahl bestätigt. Fehlt Geld, befindet sich der Spieler wieder bei der Aktivität “aufzurüstender Typ auswählen”.

Zum Entlassen von Personal ist ebenfalls der Personaltyp und die Anzahl festzulegen. Anschließend wird die Auswahl Bestätigt und die Transaktion ist abgeschlossen. 

\begin{figure}[h]
  \centering
  \fbox{
    \includegraphics[width=0.9\textwidth]{30_Fachkonzept/15_aktivitaetsdiagramm/activity2.pdf}
  }
  \caption{Aktivitätsdiagramm zur Personalverwaltung}
  \label{img:fachkonzept-aktivitaetsdiagramm-personal}
\end{figure}

\medskip

\textbf{Bauteile einkaufen}

Entscheidet sich der Spieler für das Einkaufen von Bauteilen, muss er zunächst den Bauteiltyp wählen und anschließend die einzukaufende Anzahl festlegen. Im Anschluss daran wird wieder die Liquidität des Spielers geprüft. Ist das Geld ausreichend, muss die Auswahl bestätigt werden. Fällt die Prüfung negativ aus, gelangt der Spieler wieder zur Aktivität “Bauteiltyp auswählen”. Dies wird in \vref{img:fachkonzept-aktivitaetsdiagramm-bauteile} verdeutlicht.

\begin{figure}[h]
  \centering
  \fbox{
    \includegraphics[width=0.4\textwidth]{30_Fachkonzept/15_aktivitaetsdiagramm/activity3.pdf}
  }
  \caption{Aktivitätsdiagramm zum Bauteileinkauf}
  \label{img:fachkonzept-aktivitaetsdiagramm-bauteile}
\end{figure}

\medskip

\textbf{Produktionsauftrag anlegen}

Beim Anlegen eines Produktionsauftrages, wie in \vref{img:fachkonzept-aktivitaetsdiagramm-produktion} dargestellt, ist zuerst das zu produzierende Raumschiff auszuwählen. Anschließend legt der Spieler die Anzahl fest. Nun muss geprüft werden, ob zum einen genügend Bauteile für die Produktion vorhanden sind und zum anderen ob das eingestellte Personal die Anzahl an Raumschiffen in einer Periode produzieren kann. Fällt die Prüfung positiv aus, so muss der Spieler seine Auswahl bestätigen. Sind die Kapazitäten jedoch nicht ausreichend, befindet sich der Spieler wieder bei der Aktivität “Raumschifftyp auswählen”.

\begin{figure}[h]
  \centering
  \fbox{
    \includegraphics[width=0.7\textwidth]{30_Fachkonzept/15_aktivitaetsdiagramm/activity4.pdf}
  }
  \caption{Aktivitätsdiagramm zur Produktion}
  \label{img:fachkonzept-aktivitaetsdiagramm-produktion}
\end{figure}

\medskip

\textbf{Verkaufsangebot abgeben}

Die letzte Transaktion kann im Bereich Verkauf getätigt werden. Dies ist auf \vref{img:fachkonzept-aktivitaetsdiagramm-verkauf} zu sehen. Zuerst wird der zu verkaufende Raumschifftyp ausgewählt. Danach entscheidet sich der Spieler für einen Preis, zu dem er die Raumschiffe verkaufen möchte. War bereits zuvor ein Angebot vorhanden, so wird dieses hiermit zurück genommen und danach bestätigt. War dies das erste Angebot, so gelangt der Spieler direkt zur Aktivität “Bestätigen”.

\begin{figure}[h]
  \centering
  \fbox{
    \includegraphics[width=0.7\textwidth]{30_Fachkonzept/15_aktivitaetsdiagramm/activity5.pdf}
  }
  \caption{Aktivitätsdiagramm zum Verkauf}
  \label{img:fachkonzept-aktivitaetsdiagramm-verkauf}
\end{figure}

\autorende{}
\autorende{}
\section{Aktivitätsdiagramm}
\label{sec:fachkonzept-aktivitaetsdiagramm}

\autorbeginn{Fredrik, Julia}

Das Aktivitätsdiagramm auf \vref{img:fachkonzept-aktivitaetsdiagramm-uebersicht} soll den Spielablauf aus Sicht des Spielers verdeutlichen.  

Um das Spiel zu starten muss der Spieler einen Namen für sein Unternehmen festlegen. Dies stellt die erste Aktion dar. Danach analysiert der Spieler die ihm zur Verfügung stehenden Informationen. Ist dies abgeschlossen, so gelangt er zu einem Entscheidungsknoten. Hierbei kann sich der Spieler zwischen folgenden Aktivitäten entscheiden: Personal verwalten, Einkäufe tätigen, Produktionsaufträge anlegen oder Verkaufsangebot abgeben. Diese einzelnen Vorgänge werden im Laufe dieses Kapitels genauer erläutert. Er kann sich aber auch dazu entscheiden, keine Transaktion zu tätigen. 

Diese verschiedenen Aktivitäten werden in einem Entscheidungsknoten zusammengeführt. Hat der Spieler weiteren Informationsbedarf, so gelangt er zur Aktivität “Auswirkungen analysieren” und kann sich die Veränderungen anschauen, die seine Transaktion mit sich geführt hat. Besteht kein Informationsbedarf, kann er diese Aktivität überspringen. 

Möchte der Spieler nun weitere Transaktionen tätigen, so kann er wieder zum Entscheidungsknoten nach oben springen und hat wieder die Wahl zwischen Personal verwalten, Einkäufe tätigen, Produktionsaufträge anlegen oder ein Verkaufsangebot abgeben. Entscheidet sich der Spieler gegen eine weitere Transaktion, so folgt die Aktivität “Runde einchecken”. Anschließend folgt wieder ein Entscheidungsknoten. Existiert noch eine weitere Spielrunde, so gelangt der Spieler wieder zur Aktivität “Informationen analysieren”. War dies die letzte Spielrunde, so wird dem Spieler die Endbewertung angezeigt. Danach ist das Spiel beendet.

\begin{figure}[h]
  \centering
  \fbox{
    \includegraphics[width=0.9\textwidth]{30_Fachkonzept/15_aktivitaetsdiagramm/activity1.pdf}
  }
  \caption{Aktivitätsdiagramm}
  \label{img:fachkonzept-aktivitaetsdiagramm-uebersicht}
\end{figure}

\medskip

\textbf{Personal verwalten}

Zur genaueren Betrachtung der Aktion “Personal verwalten” dient folgendes Aktivitätsdiagramm auf \vref{img:fachkonzept-aktivitaetsdiagramm-personal}. Entscheidet sich der Spieler für diese Aktion, so hat er die Möglichkeit Personal einzustellen, aufzurüsten oder zu entlassen. 

Um neues Personal einzustellen, muss der Spieler zuerst den Personaltyp auswählen, welchen er einstellen möchte. Danach folgt die Aktivität “Anzahl festlegen”. Anschließend trifft der Spieler auf einen Entscheidungsknoten. Ist genügend Geld vorhanden, muss der Spieler die Auswahl bestätigen. Reicht das verfügbare Geld jedoch nicht aus, gelangt der Spieler wieder zur Aktivität “einzustellender Typ auswählen” und durchläuft den Prozess noch ein mal. 

Möchte der Spieler sein vorhandenes Personal aufrüsten, muss er zunächst den Personaltyp auswählen. Anschließend gelangt er zur Aktivität “Anzahl festlegen”. Ähnlich wie beim Einstellen von neuem Personal wird auch nun geprüft, ob genügend Geld vorhanden ist. Ist dies der Fall, wird die Auswahl bestätigt. Fehlt Geld, befindet sich der Spieler wieder bei der Aktivität “aufzurüstender Typ auswählen”.

Zum Entlassen von Personal ist ebenfalls der Personaltyp und die Anzahl festzulegen. Anschließend wird die Auswahl Bestätigt und die Transaktion ist abgeschlossen. 

\begin{figure}[h]
  \centering
  \fbox{
    \includegraphics[width=0.9\textwidth]{30_Fachkonzept/15_aktivitaetsdiagramm/activity2.pdf}
  }
  \caption{Aktivitätsdiagramm zur Personalverwaltung}
  \label{img:fachkonzept-aktivitaetsdiagramm-personal}
\end{figure}

\medskip

\textbf{Bauteile einkaufen}

Entscheidet sich der Spieler für das Einkaufen von Bauteilen, muss er zunächst den Bauteiltyp wählen und anschließend die einzukaufende Anzahl festlegen. Im Anschluss daran wird wieder die Liquidität des Spielers geprüft. Ist das Geld ausreichend, muss die Auswahl bestätigt werden. Fällt die Prüfung negativ aus, gelangt der Spieler wieder zur Aktivität “Bauteiltyp auswählen”. Dies wird in \vref{img:fachkonzept-aktivitaetsdiagramm-bauteile} verdeutlicht.

\begin{figure}[h]
  \centering
  \fbox{
    \includegraphics[width=0.4\textwidth]{30_Fachkonzept/15_aktivitaetsdiagramm/activity3.pdf}
  }
  \caption{Aktivitätsdiagramm zum Bauteileinkauf}
  \label{img:fachkonzept-aktivitaetsdiagramm-bauteile}
\end{figure}

\medskip

\textbf{Produktionsauftrag anlegen}

Beim Anlegen eines Produktionsauftrages, wie in \vref{img:fachkonzept-aktivitaetsdiagramm-produktion} dargestellt, ist zuerst das zu produzierende Raumschiff auszuwählen. Anschließend legt der Spieler die Anzahl fest. Nun muss geprüft werden, ob zum einen genügend Bauteile für die Produktion vorhanden sind und zum anderen ob das eingestellte Personal die Anzahl an Raumschiffen in einer Periode produzieren kann. Fällt die Prüfung positiv aus, so muss der Spieler seine Auswahl bestätigen. Sind die Kapazitäten jedoch nicht ausreichend, befindet sich der Spieler wieder bei der Aktivität “Raumschifftyp auswählen”.

\begin{figure}[h]
  \centering
  \fbox{
    \includegraphics[width=0.7\textwidth]{30_Fachkonzept/15_aktivitaetsdiagramm/activity4.pdf}
  }
  \caption{Aktivitätsdiagramm zur Produktion}
  \label{img:fachkonzept-aktivitaetsdiagramm-produktion}
\end{figure}

\medskip

\textbf{Verkaufsangebot abgeben}

Die letzte Transaktion kann im Bereich Verkauf getätigt werden. Dies ist auf \vref{img:fachkonzept-aktivitaetsdiagramm-verkauf} zu sehen. Zuerst wird der zu verkaufende Raumschifftyp ausgewählt. Danach entscheidet sich der Spieler für einen Preis, zu dem er die Raumschiffe verkaufen möchte. War bereits zuvor ein Angebot vorhanden, so wird dieses hiermit zurück genommen und danach bestätigt. War dies das erste Angebot, so gelangt der Spieler direkt zur Aktivität “Bestätigen”.

\begin{figure}[h]
  \centering
  \fbox{
    \includegraphics[width=0.7\textwidth]{30_Fachkonzept/15_aktivitaetsdiagramm/activity5.pdf}
  }
  \caption{Aktivitätsdiagramm zum Verkauf}
  \label{img:fachkonzept-aktivitaetsdiagramm-verkauf}
\end{figure}

\autorende{}
\autorbeginn{Britta}
\section{Aktivitätsdiagramm}
\label{sec:fachkonzept-aktivitaetsdiagramm}

\autorbeginn{Fredrik, Julia}

Das Aktivitätsdiagramm auf \vref{img:fachkonzept-aktivitaetsdiagramm-uebersicht} soll den Spielablauf aus Sicht des Spielers verdeutlichen.  

Um das Spiel zu starten muss der Spieler einen Namen für sein Unternehmen festlegen. Dies stellt die erste Aktion dar. Danach analysiert der Spieler die ihm zur Verfügung stehenden Informationen. Ist dies abgeschlossen, so gelangt er zu einem Entscheidungsknoten. Hierbei kann sich der Spieler zwischen folgenden Aktivitäten entscheiden: Personal verwalten, Einkäufe tätigen, Produktionsaufträge anlegen oder Verkaufsangebot abgeben. Diese einzelnen Vorgänge werden im Laufe dieses Kapitels genauer erläutert. Er kann sich aber auch dazu entscheiden, keine Transaktion zu tätigen. 

Diese verschiedenen Aktivitäten werden in einem Entscheidungsknoten zusammengeführt. Hat der Spieler weiteren Informationsbedarf, so gelangt er zur Aktivität “Auswirkungen analysieren” und kann sich die Veränderungen anschauen, die seine Transaktion mit sich geführt hat. Besteht kein Informationsbedarf, kann er diese Aktivität überspringen. 

Möchte der Spieler nun weitere Transaktionen tätigen, so kann er wieder zum Entscheidungsknoten nach oben springen und hat wieder die Wahl zwischen Personal verwalten, Einkäufe tätigen, Produktionsaufträge anlegen oder ein Verkaufsangebot abgeben. Entscheidet sich der Spieler gegen eine weitere Transaktion, so folgt die Aktivität “Runde einchecken”. Anschließend folgt wieder ein Entscheidungsknoten. Existiert noch eine weitere Spielrunde, so gelangt der Spieler wieder zur Aktivität “Informationen analysieren”. War dies die letzte Spielrunde, so wird dem Spieler die Endbewertung angezeigt. Danach ist das Spiel beendet.

\begin{figure}[h]
  \centering
  \fbox{
    \includegraphics[width=0.9\textwidth]{30_Fachkonzept/15_aktivitaetsdiagramm/activity1.pdf}
  }
  \caption{Aktivitätsdiagramm}
  \label{img:fachkonzept-aktivitaetsdiagramm-uebersicht}
\end{figure}

\medskip

\textbf{Personal verwalten}

Zur genaueren Betrachtung der Aktion “Personal verwalten” dient folgendes Aktivitätsdiagramm auf \vref{img:fachkonzept-aktivitaetsdiagramm-personal}. Entscheidet sich der Spieler für diese Aktion, so hat er die Möglichkeit Personal einzustellen, aufzurüsten oder zu entlassen. 

Um neues Personal einzustellen, muss der Spieler zuerst den Personaltyp auswählen, welchen er einstellen möchte. Danach folgt die Aktivität “Anzahl festlegen”. Anschließend trifft der Spieler auf einen Entscheidungsknoten. Ist genügend Geld vorhanden, muss der Spieler die Auswahl bestätigen. Reicht das verfügbare Geld jedoch nicht aus, gelangt der Spieler wieder zur Aktivität “einzustellender Typ auswählen” und durchläuft den Prozess noch ein mal. 

Möchte der Spieler sein vorhandenes Personal aufrüsten, muss er zunächst den Personaltyp auswählen. Anschließend gelangt er zur Aktivität “Anzahl festlegen”. Ähnlich wie beim Einstellen von neuem Personal wird auch nun geprüft, ob genügend Geld vorhanden ist. Ist dies der Fall, wird die Auswahl bestätigt. Fehlt Geld, befindet sich der Spieler wieder bei der Aktivität “aufzurüstender Typ auswählen”.

Zum Entlassen von Personal ist ebenfalls der Personaltyp und die Anzahl festzulegen. Anschließend wird die Auswahl Bestätigt und die Transaktion ist abgeschlossen. 

\begin{figure}[h]
  \centering
  \fbox{
    \includegraphics[width=0.9\textwidth]{30_Fachkonzept/15_aktivitaetsdiagramm/activity2.pdf}
  }
  \caption{Aktivitätsdiagramm zur Personalverwaltung}
  \label{img:fachkonzept-aktivitaetsdiagramm-personal}
\end{figure}

\medskip

\textbf{Bauteile einkaufen}

Entscheidet sich der Spieler für das Einkaufen von Bauteilen, muss er zunächst den Bauteiltyp wählen und anschließend die einzukaufende Anzahl festlegen. Im Anschluss daran wird wieder die Liquidität des Spielers geprüft. Ist das Geld ausreichend, muss die Auswahl bestätigt werden. Fällt die Prüfung negativ aus, gelangt der Spieler wieder zur Aktivität “Bauteiltyp auswählen”. Dies wird in \vref{img:fachkonzept-aktivitaetsdiagramm-bauteile} verdeutlicht.

\begin{figure}[h]
  \centering
  \fbox{
    \includegraphics[width=0.4\textwidth]{30_Fachkonzept/15_aktivitaetsdiagramm/activity3.pdf}
  }
  \caption{Aktivitätsdiagramm zum Bauteileinkauf}
  \label{img:fachkonzept-aktivitaetsdiagramm-bauteile}
\end{figure}

\medskip

\textbf{Produktionsauftrag anlegen}

Beim Anlegen eines Produktionsauftrages, wie in \vref{img:fachkonzept-aktivitaetsdiagramm-produktion} dargestellt, ist zuerst das zu produzierende Raumschiff auszuwählen. Anschließend legt der Spieler die Anzahl fest. Nun muss geprüft werden, ob zum einen genügend Bauteile für die Produktion vorhanden sind und zum anderen ob das eingestellte Personal die Anzahl an Raumschiffen in einer Periode produzieren kann. Fällt die Prüfung positiv aus, so muss der Spieler seine Auswahl bestätigen. Sind die Kapazitäten jedoch nicht ausreichend, befindet sich der Spieler wieder bei der Aktivität “Raumschifftyp auswählen”.

\begin{figure}[h]
  \centering
  \fbox{
    \includegraphics[width=0.7\textwidth]{30_Fachkonzept/15_aktivitaetsdiagramm/activity4.pdf}
  }
  \caption{Aktivitätsdiagramm zur Produktion}
  \label{img:fachkonzept-aktivitaetsdiagramm-produktion}
\end{figure}

\medskip

\textbf{Verkaufsangebot abgeben}

Die letzte Transaktion kann im Bereich Verkauf getätigt werden. Dies ist auf \vref{img:fachkonzept-aktivitaetsdiagramm-verkauf} zu sehen. Zuerst wird der zu verkaufende Raumschifftyp ausgewählt. Danach entscheidet sich der Spieler für einen Preis, zu dem er die Raumschiffe verkaufen möchte. War bereits zuvor ein Angebot vorhanden, so wird dieses hiermit zurück genommen und danach bestätigt. War dies das erste Angebot, so gelangt der Spieler direkt zur Aktivität “Bestätigen”.

\begin{figure}[h]
  \centering
  \fbox{
    \includegraphics[width=0.7\textwidth]{30_Fachkonzept/15_aktivitaetsdiagramm/activity5.pdf}
  }
  \caption{Aktivitätsdiagramm zum Verkauf}
  \label{img:fachkonzept-aktivitaetsdiagramm-verkauf}
\end{figure}

\autorende{}
\autorende{}
\section{Aktivitätsdiagramm}
\label{sec:fachkonzept-aktivitaetsdiagramm}

\autorbeginn{Fredrik, Julia}

Das Aktivitätsdiagramm auf \vref{img:fachkonzept-aktivitaetsdiagramm-uebersicht} soll den Spielablauf aus Sicht des Spielers verdeutlichen.  

Um das Spiel zu starten muss der Spieler einen Namen für sein Unternehmen festlegen. Dies stellt die erste Aktion dar. Danach analysiert der Spieler die ihm zur Verfügung stehenden Informationen. Ist dies abgeschlossen, so gelangt er zu einem Entscheidungsknoten. Hierbei kann sich der Spieler zwischen folgenden Aktivitäten entscheiden: Personal verwalten, Einkäufe tätigen, Produktionsaufträge anlegen oder Verkaufsangebot abgeben. Diese einzelnen Vorgänge werden im Laufe dieses Kapitels genauer erläutert. Er kann sich aber auch dazu entscheiden, keine Transaktion zu tätigen. 

Diese verschiedenen Aktivitäten werden in einem Entscheidungsknoten zusammengeführt. Hat der Spieler weiteren Informationsbedarf, so gelangt er zur Aktivität “Auswirkungen analysieren” und kann sich die Veränderungen anschauen, die seine Transaktion mit sich geführt hat. Besteht kein Informationsbedarf, kann er diese Aktivität überspringen. 

Möchte der Spieler nun weitere Transaktionen tätigen, so kann er wieder zum Entscheidungsknoten nach oben springen und hat wieder die Wahl zwischen Personal verwalten, Einkäufe tätigen, Produktionsaufträge anlegen oder ein Verkaufsangebot abgeben. Entscheidet sich der Spieler gegen eine weitere Transaktion, so folgt die Aktivität “Runde einchecken”. Anschließend folgt wieder ein Entscheidungsknoten. Existiert noch eine weitere Spielrunde, so gelangt der Spieler wieder zur Aktivität “Informationen analysieren”. War dies die letzte Spielrunde, so wird dem Spieler die Endbewertung angezeigt. Danach ist das Spiel beendet.

\begin{figure}[h]
  \centering
  \fbox{
    \includegraphics[width=0.9\textwidth]{30_Fachkonzept/15_aktivitaetsdiagramm/activity1.pdf}
  }
  \caption{Aktivitätsdiagramm}
  \label{img:fachkonzept-aktivitaetsdiagramm-uebersicht}
\end{figure}

\medskip

\textbf{Personal verwalten}

Zur genaueren Betrachtung der Aktion “Personal verwalten” dient folgendes Aktivitätsdiagramm auf \vref{img:fachkonzept-aktivitaetsdiagramm-personal}. Entscheidet sich der Spieler für diese Aktion, so hat er die Möglichkeit Personal einzustellen, aufzurüsten oder zu entlassen. 

Um neues Personal einzustellen, muss der Spieler zuerst den Personaltyp auswählen, welchen er einstellen möchte. Danach folgt die Aktivität “Anzahl festlegen”. Anschließend trifft der Spieler auf einen Entscheidungsknoten. Ist genügend Geld vorhanden, muss der Spieler die Auswahl bestätigen. Reicht das verfügbare Geld jedoch nicht aus, gelangt der Spieler wieder zur Aktivität “einzustellender Typ auswählen” und durchläuft den Prozess noch ein mal. 

Möchte der Spieler sein vorhandenes Personal aufrüsten, muss er zunächst den Personaltyp auswählen. Anschließend gelangt er zur Aktivität “Anzahl festlegen”. Ähnlich wie beim Einstellen von neuem Personal wird auch nun geprüft, ob genügend Geld vorhanden ist. Ist dies der Fall, wird die Auswahl bestätigt. Fehlt Geld, befindet sich der Spieler wieder bei der Aktivität “aufzurüstender Typ auswählen”.

Zum Entlassen von Personal ist ebenfalls der Personaltyp und die Anzahl festzulegen. Anschließend wird die Auswahl Bestätigt und die Transaktion ist abgeschlossen. 

\begin{figure}[h]
  \centering
  \fbox{
    \includegraphics[width=0.9\textwidth]{30_Fachkonzept/15_aktivitaetsdiagramm/activity2.pdf}
  }
  \caption{Aktivitätsdiagramm zur Personalverwaltung}
  \label{img:fachkonzept-aktivitaetsdiagramm-personal}
\end{figure}

\medskip

\textbf{Bauteile einkaufen}

Entscheidet sich der Spieler für das Einkaufen von Bauteilen, muss er zunächst den Bauteiltyp wählen und anschließend die einzukaufende Anzahl festlegen. Im Anschluss daran wird wieder die Liquidität des Spielers geprüft. Ist das Geld ausreichend, muss die Auswahl bestätigt werden. Fällt die Prüfung negativ aus, gelangt der Spieler wieder zur Aktivität “Bauteiltyp auswählen”. Dies wird in \vref{img:fachkonzept-aktivitaetsdiagramm-bauteile} verdeutlicht.

\begin{figure}[h]
  \centering
  \fbox{
    \includegraphics[width=0.4\textwidth]{30_Fachkonzept/15_aktivitaetsdiagramm/activity3.pdf}
  }
  \caption{Aktivitätsdiagramm zum Bauteileinkauf}
  \label{img:fachkonzept-aktivitaetsdiagramm-bauteile}
\end{figure}

\medskip

\textbf{Produktionsauftrag anlegen}

Beim Anlegen eines Produktionsauftrages, wie in \vref{img:fachkonzept-aktivitaetsdiagramm-produktion} dargestellt, ist zuerst das zu produzierende Raumschiff auszuwählen. Anschließend legt der Spieler die Anzahl fest. Nun muss geprüft werden, ob zum einen genügend Bauteile für die Produktion vorhanden sind und zum anderen ob das eingestellte Personal die Anzahl an Raumschiffen in einer Periode produzieren kann. Fällt die Prüfung positiv aus, so muss der Spieler seine Auswahl bestätigen. Sind die Kapazitäten jedoch nicht ausreichend, befindet sich der Spieler wieder bei der Aktivität “Raumschifftyp auswählen”.

\begin{figure}[h]
  \centering
  \fbox{
    \includegraphics[width=0.7\textwidth]{30_Fachkonzept/15_aktivitaetsdiagramm/activity4.pdf}
  }
  \caption{Aktivitätsdiagramm zur Produktion}
  \label{img:fachkonzept-aktivitaetsdiagramm-produktion}
\end{figure}

\medskip

\textbf{Verkaufsangebot abgeben}

Die letzte Transaktion kann im Bereich Verkauf getätigt werden. Dies ist auf \vref{img:fachkonzept-aktivitaetsdiagramm-verkauf} zu sehen. Zuerst wird der zu verkaufende Raumschifftyp ausgewählt. Danach entscheidet sich der Spieler für einen Preis, zu dem er die Raumschiffe verkaufen möchte. War bereits zuvor ein Angebot vorhanden, so wird dieses hiermit zurück genommen und danach bestätigt. War dies das erste Angebot, so gelangt der Spieler direkt zur Aktivität “Bestätigen”.

\begin{figure}[h]
  \centering
  \fbox{
    \includegraphics[width=0.7\textwidth]{30_Fachkonzept/15_aktivitaetsdiagramm/activity5.pdf}
  }
  \caption{Aktivitätsdiagramm zum Verkauf}
  \label{img:fachkonzept-aktivitaetsdiagramm-verkauf}
\end{figure}

\autorende{}

\chapter{Spielwelt}
\label{chp:spielwelt}

\section{Aktivitätsdiagramm}
\label{sec:fachkonzept-aktivitaetsdiagramm}

\autorbeginn{Fredrik, Julia}

Das Aktivitätsdiagramm auf \vref{img:fachkonzept-aktivitaetsdiagramm-uebersicht} soll den Spielablauf aus Sicht des Spielers verdeutlichen.  

Um das Spiel zu starten muss der Spieler einen Namen für sein Unternehmen festlegen. Dies stellt die erste Aktion dar. Danach analysiert der Spieler die ihm zur Verfügung stehenden Informationen. Ist dies abgeschlossen, so gelangt er zu einem Entscheidungsknoten. Hierbei kann sich der Spieler zwischen folgenden Aktivitäten entscheiden: Personal verwalten, Einkäufe tätigen, Produktionsaufträge anlegen oder Verkaufsangebot abgeben. Diese einzelnen Vorgänge werden im Laufe dieses Kapitels genauer erläutert. Er kann sich aber auch dazu entscheiden, keine Transaktion zu tätigen. 

Diese verschiedenen Aktivitäten werden in einem Entscheidungsknoten zusammengeführt. Hat der Spieler weiteren Informationsbedarf, so gelangt er zur Aktivität “Auswirkungen analysieren” und kann sich die Veränderungen anschauen, die seine Transaktion mit sich geführt hat. Besteht kein Informationsbedarf, kann er diese Aktivität überspringen. 

Möchte der Spieler nun weitere Transaktionen tätigen, so kann er wieder zum Entscheidungsknoten nach oben springen und hat wieder die Wahl zwischen Personal verwalten, Einkäufe tätigen, Produktionsaufträge anlegen oder ein Verkaufsangebot abgeben. Entscheidet sich der Spieler gegen eine weitere Transaktion, so folgt die Aktivität “Runde einchecken”. Anschließend folgt wieder ein Entscheidungsknoten. Existiert noch eine weitere Spielrunde, so gelangt der Spieler wieder zur Aktivität “Informationen analysieren”. War dies die letzte Spielrunde, so wird dem Spieler die Endbewertung angezeigt. Danach ist das Spiel beendet.

\begin{figure}[h]
  \centering
  \fbox{
    \includegraphics[width=0.9\textwidth]{30_Fachkonzept/15_aktivitaetsdiagramm/activity1.pdf}
  }
  \caption{Aktivitätsdiagramm}
  \label{img:fachkonzept-aktivitaetsdiagramm-uebersicht}
\end{figure}

\medskip

\textbf{Personal verwalten}

Zur genaueren Betrachtung der Aktion “Personal verwalten” dient folgendes Aktivitätsdiagramm auf \vref{img:fachkonzept-aktivitaetsdiagramm-personal}. Entscheidet sich der Spieler für diese Aktion, so hat er die Möglichkeit Personal einzustellen, aufzurüsten oder zu entlassen. 

Um neues Personal einzustellen, muss der Spieler zuerst den Personaltyp auswählen, welchen er einstellen möchte. Danach folgt die Aktivität “Anzahl festlegen”. Anschließend trifft der Spieler auf einen Entscheidungsknoten. Ist genügend Geld vorhanden, muss der Spieler die Auswahl bestätigen. Reicht das verfügbare Geld jedoch nicht aus, gelangt der Spieler wieder zur Aktivität “einzustellender Typ auswählen” und durchläuft den Prozess noch ein mal. 

Möchte der Spieler sein vorhandenes Personal aufrüsten, muss er zunächst den Personaltyp auswählen. Anschließend gelangt er zur Aktivität “Anzahl festlegen”. Ähnlich wie beim Einstellen von neuem Personal wird auch nun geprüft, ob genügend Geld vorhanden ist. Ist dies der Fall, wird die Auswahl bestätigt. Fehlt Geld, befindet sich der Spieler wieder bei der Aktivität “aufzurüstender Typ auswählen”.

Zum Entlassen von Personal ist ebenfalls der Personaltyp und die Anzahl festzulegen. Anschließend wird die Auswahl Bestätigt und die Transaktion ist abgeschlossen. 

\begin{figure}[h]
  \centering
  \fbox{
    \includegraphics[width=0.9\textwidth]{30_Fachkonzept/15_aktivitaetsdiagramm/activity2.pdf}
  }
  \caption{Aktivitätsdiagramm zur Personalverwaltung}
  \label{img:fachkonzept-aktivitaetsdiagramm-personal}
\end{figure}

\medskip

\textbf{Bauteile einkaufen}

Entscheidet sich der Spieler für das Einkaufen von Bauteilen, muss er zunächst den Bauteiltyp wählen und anschließend die einzukaufende Anzahl festlegen. Im Anschluss daran wird wieder die Liquidität des Spielers geprüft. Ist das Geld ausreichend, muss die Auswahl bestätigt werden. Fällt die Prüfung negativ aus, gelangt der Spieler wieder zur Aktivität “Bauteiltyp auswählen”. Dies wird in \vref{img:fachkonzept-aktivitaetsdiagramm-bauteile} verdeutlicht.

\begin{figure}[h]
  \centering
  \fbox{
    \includegraphics[width=0.4\textwidth]{30_Fachkonzept/15_aktivitaetsdiagramm/activity3.pdf}
  }
  \caption{Aktivitätsdiagramm zum Bauteileinkauf}
  \label{img:fachkonzept-aktivitaetsdiagramm-bauteile}
\end{figure}

\medskip

\textbf{Produktionsauftrag anlegen}

Beim Anlegen eines Produktionsauftrages, wie in \vref{img:fachkonzept-aktivitaetsdiagramm-produktion} dargestellt, ist zuerst das zu produzierende Raumschiff auszuwählen. Anschließend legt der Spieler die Anzahl fest. Nun muss geprüft werden, ob zum einen genügend Bauteile für die Produktion vorhanden sind und zum anderen ob das eingestellte Personal die Anzahl an Raumschiffen in einer Periode produzieren kann. Fällt die Prüfung positiv aus, so muss der Spieler seine Auswahl bestätigen. Sind die Kapazitäten jedoch nicht ausreichend, befindet sich der Spieler wieder bei der Aktivität “Raumschifftyp auswählen”.

\begin{figure}[h]
  \centering
  \fbox{
    \includegraphics[width=0.7\textwidth]{30_Fachkonzept/15_aktivitaetsdiagramm/activity4.pdf}
  }
  \caption{Aktivitätsdiagramm zur Produktion}
  \label{img:fachkonzept-aktivitaetsdiagramm-produktion}
\end{figure}

\medskip

\textbf{Verkaufsangebot abgeben}

Die letzte Transaktion kann im Bereich Verkauf getätigt werden. Dies ist auf \vref{img:fachkonzept-aktivitaetsdiagramm-verkauf} zu sehen. Zuerst wird der zu verkaufende Raumschifftyp ausgewählt. Danach entscheidet sich der Spieler für einen Preis, zu dem er die Raumschiffe verkaufen möchte. War bereits zuvor ein Angebot vorhanden, so wird dieses hiermit zurück genommen und danach bestätigt. War dies das erste Angebot, so gelangt der Spieler direkt zur Aktivität “Bestätigen”.

\begin{figure}[h]
  \centering
  \fbox{
    \includegraphics[width=0.7\textwidth]{30_Fachkonzept/15_aktivitaetsdiagramm/activity5.pdf}
  }
  \caption{Aktivitätsdiagramm zum Verkauf}
  \label{img:fachkonzept-aktivitaetsdiagramm-verkauf}
\end{figure}

\autorende{}
\autorbeginn{Britta}
\section{Aktivitätsdiagramm}
\label{sec:fachkonzept-aktivitaetsdiagramm}

\autorbeginn{Fredrik, Julia}

Das Aktivitätsdiagramm auf \vref{img:fachkonzept-aktivitaetsdiagramm-uebersicht} soll den Spielablauf aus Sicht des Spielers verdeutlichen.  

Um das Spiel zu starten muss der Spieler einen Namen für sein Unternehmen festlegen. Dies stellt die erste Aktion dar. Danach analysiert der Spieler die ihm zur Verfügung stehenden Informationen. Ist dies abgeschlossen, so gelangt er zu einem Entscheidungsknoten. Hierbei kann sich der Spieler zwischen folgenden Aktivitäten entscheiden: Personal verwalten, Einkäufe tätigen, Produktionsaufträge anlegen oder Verkaufsangebot abgeben. Diese einzelnen Vorgänge werden im Laufe dieses Kapitels genauer erläutert. Er kann sich aber auch dazu entscheiden, keine Transaktion zu tätigen. 

Diese verschiedenen Aktivitäten werden in einem Entscheidungsknoten zusammengeführt. Hat der Spieler weiteren Informationsbedarf, so gelangt er zur Aktivität “Auswirkungen analysieren” und kann sich die Veränderungen anschauen, die seine Transaktion mit sich geführt hat. Besteht kein Informationsbedarf, kann er diese Aktivität überspringen. 

Möchte der Spieler nun weitere Transaktionen tätigen, so kann er wieder zum Entscheidungsknoten nach oben springen und hat wieder die Wahl zwischen Personal verwalten, Einkäufe tätigen, Produktionsaufträge anlegen oder ein Verkaufsangebot abgeben. Entscheidet sich der Spieler gegen eine weitere Transaktion, so folgt die Aktivität “Runde einchecken”. Anschließend folgt wieder ein Entscheidungsknoten. Existiert noch eine weitere Spielrunde, so gelangt der Spieler wieder zur Aktivität “Informationen analysieren”. War dies die letzte Spielrunde, so wird dem Spieler die Endbewertung angezeigt. Danach ist das Spiel beendet.

\begin{figure}[h]
  \centering
  \fbox{
    \includegraphics[width=0.9\textwidth]{30_Fachkonzept/15_aktivitaetsdiagramm/activity1.pdf}
  }
  \caption{Aktivitätsdiagramm}
  \label{img:fachkonzept-aktivitaetsdiagramm-uebersicht}
\end{figure}

\medskip

\textbf{Personal verwalten}

Zur genaueren Betrachtung der Aktion “Personal verwalten” dient folgendes Aktivitätsdiagramm auf \vref{img:fachkonzept-aktivitaetsdiagramm-personal}. Entscheidet sich der Spieler für diese Aktion, so hat er die Möglichkeit Personal einzustellen, aufzurüsten oder zu entlassen. 

Um neues Personal einzustellen, muss der Spieler zuerst den Personaltyp auswählen, welchen er einstellen möchte. Danach folgt die Aktivität “Anzahl festlegen”. Anschließend trifft der Spieler auf einen Entscheidungsknoten. Ist genügend Geld vorhanden, muss der Spieler die Auswahl bestätigen. Reicht das verfügbare Geld jedoch nicht aus, gelangt der Spieler wieder zur Aktivität “einzustellender Typ auswählen” und durchläuft den Prozess noch ein mal. 

Möchte der Spieler sein vorhandenes Personal aufrüsten, muss er zunächst den Personaltyp auswählen. Anschließend gelangt er zur Aktivität “Anzahl festlegen”. Ähnlich wie beim Einstellen von neuem Personal wird auch nun geprüft, ob genügend Geld vorhanden ist. Ist dies der Fall, wird die Auswahl bestätigt. Fehlt Geld, befindet sich der Spieler wieder bei der Aktivität “aufzurüstender Typ auswählen”.

Zum Entlassen von Personal ist ebenfalls der Personaltyp und die Anzahl festzulegen. Anschließend wird die Auswahl Bestätigt und die Transaktion ist abgeschlossen. 

\begin{figure}[h]
  \centering
  \fbox{
    \includegraphics[width=0.9\textwidth]{30_Fachkonzept/15_aktivitaetsdiagramm/activity2.pdf}
  }
  \caption{Aktivitätsdiagramm zur Personalverwaltung}
  \label{img:fachkonzept-aktivitaetsdiagramm-personal}
\end{figure}

\medskip

\textbf{Bauteile einkaufen}

Entscheidet sich der Spieler für das Einkaufen von Bauteilen, muss er zunächst den Bauteiltyp wählen und anschließend die einzukaufende Anzahl festlegen. Im Anschluss daran wird wieder die Liquidität des Spielers geprüft. Ist das Geld ausreichend, muss die Auswahl bestätigt werden. Fällt die Prüfung negativ aus, gelangt der Spieler wieder zur Aktivität “Bauteiltyp auswählen”. Dies wird in \vref{img:fachkonzept-aktivitaetsdiagramm-bauteile} verdeutlicht.

\begin{figure}[h]
  \centering
  \fbox{
    \includegraphics[width=0.4\textwidth]{30_Fachkonzept/15_aktivitaetsdiagramm/activity3.pdf}
  }
  \caption{Aktivitätsdiagramm zum Bauteileinkauf}
  \label{img:fachkonzept-aktivitaetsdiagramm-bauteile}
\end{figure}

\medskip

\textbf{Produktionsauftrag anlegen}

Beim Anlegen eines Produktionsauftrages, wie in \vref{img:fachkonzept-aktivitaetsdiagramm-produktion} dargestellt, ist zuerst das zu produzierende Raumschiff auszuwählen. Anschließend legt der Spieler die Anzahl fest. Nun muss geprüft werden, ob zum einen genügend Bauteile für die Produktion vorhanden sind und zum anderen ob das eingestellte Personal die Anzahl an Raumschiffen in einer Periode produzieren kann. Fällt die Prüfung positiv aus, so muss der Spieler seine Auswahl bestätigen. Sind die Kapazitäten jedoch nicht ausreichend, befindet sich der Spieler wieder bei der Aktivität “Raumschifftyp auswählen”.

\begin{figure}[h]
  \centering
  \fbox{
    \includegraphics[width=0.7\textwidth]{30_Fachkonzept/15_aktivitaetsdiagramm/activity4.pdf}
  }
  \caption{Aktivitätsdiagramm zur Produktion}
  \label{img:fachkonzept-aktivitaetsdiagramm-produktion}
\end{figure}

\medskip

\textbf{Verkaufsangebot abgeben}

Die letzte Transaktion kann im Bereich Verkauf getätigt werden. Dies ist auf \vref{img:fachkonzept-aktivitaetsdiagramm-verkauf} zu sehen. Zuerst wird der zu verkaufende Raumschifftyp ausgewählt. Danach entscheidet sich der Spieler für einen Preis, zu dem er die Raumschiffe verkaufen möchte. War bereits zuvor ein Angebot vorhanden, so wird dieses hiermit zurück genommen und danach bestätigt. War dies das erste Angebot, so gelangt der Spieler direkt zur Aktivität “Bestätigen”.

\begin{figure}[h]
  \centering
  \fbox{
    \includegraphics[width=0.7\textwidth]{30_Fachkonzept/15_aktivitaetsdiagramm/activity5.pdf}
  }
  \caption{Aktivitätsdiagramm zum Verkauf}
  \label{img:fachkonzept-aktivitaetsdiagramm-verkauf}
\end{figure}

\autorende{}
\autorende{}
\section{Aktivitätsdiagramm}
\label{sec:fachkonzept-aktivitaetsdiagramm}

\autorbeginn{Fredrik, Julia}

Das Aktivitätsdiagramm auf \vref{img:fachkonzept-aktivitaetsdiagramm-uebersicht} soll den Spielablauf aus Sicht des Spielers verdeutlichen.  

Um das Spiel zu starten muss der Spieler einen Namen für sein Unternehmen festlegen. Dies stellt die erste Aktion dar. Danach analysiert der Spieler die ihm zur Verfügung stehenden Informationen. Ist dies abgeschlossen, so gelangt er zu einem Entscheidungsknoten. Hierbei kann sich der Spieler zwischen folgenden Aktivitäten entscheiden: Personal verwalten, Einkäufe tätigen, Produktionsaufträge anlegen oder Verkaufsangebot abgeben. Diese einzelnen Vorgänge werden im Laufe dieses Kapitels genauer erläutert. Er kann sich aber auch dazu entscheiden, keine Transaktion zu tätigen. 

Diese verschiedenen Aktivitäten werden in einem Entscheidungsknoten zusammengeführt. Hat der Spieler weiteren Informationsbedarf, so gelangt er zur Aktivität “Auswirkungen analysieren” und kann sich die Veränderungen anschauen, die seine Transaktion mit sich geführt hat. Besteht kein Informationsbedarf, kann er diese Aktivität überspringen. 

Möchte der Spieler nun weitere Transaktionen tätigen, so kann er wieder zum Entscheidungsknoten nach oben springen und hat wieder die Wahl zwischen Personal verwalten, Einkäufe tätigen, Produktionsaufträge anlegen oder ein Verkaufsangebot abgeben. Entscheidet sich der Spieler gegen eine weitere Transaktion, so folgt die Aktivität “Runde einchecken”. Anschließend folgt wieder ein Entscheidungsknoten. Existiert noch eine weitere Spielrunde, so gelangt der Spieler wieder zur Aktivität “Informationen analysieren”. War dies die letzte Spielrunde, so wird dem Spieler die Endbewertung angezeigt. Danach ist das Spiel beendet.

\begin{figure}[h]
  \centering
  \fbox{
    \includegraphics[width=0.9\textwidth]{30_Fachkonzept/15_aktivitaetsdiagramm/activity1.pdf}
  }
  \caption{Aktivitätsdiagramm}
  \label{img:fachkonzept-aktivitaetsdiagramm-uebersicht}
\end{figure}

\medskip

\textbf{Personal verwalten}

Zur genaueren Betrachtung der Aktion “Personal verwalten” dient folgendes Aktivitätsdiagramm auf \vref{img:fachkonzept-aktivitaetsdiagramm-personal}. Entscheidet sich der Spieler für diese Aktion, so hat er die Möglichkeit Personal einzustellen, aufzurüsten oder zu entlassen. 

Um neues Personal einzustellen, muss der Spieler zuerst den Personaltyp auswählen, welchen er einstellen möchte. Danach folgt die Aktivität “Anzahl festlegen”. Anschließend trifft der Spieler auf einen Entscheidungsknoten. Ist genügend Geld vorhanden, muss der Spieler die Auswahl bestätigen. Reicht das verfügbare Geld jedoch nicht aus, gelangt der Spieler wieder zur Aktivität “einzustellender Typ auswählen” und durchläuft den Prozess noch ein mal. 

Möchte der Spieler sein vorhandenes Personal aufrüsten, muss er zunächst den Personaltyp auswählen. Anschließend gelangt er zur Aktivität “Anzahl festlegen”. Ähnlich wie beim Einstellen von neuem Personal wird auch nun geprüft, ob genügend Geld vorhanden ist. Ist dies der Fall, wird die Auswahl bestätigt. Fehlt Geld, befindet sich der Spieler wieder bei der Aktivität “aufzurüstender Typ auswählen”.

Zum Entlassen von Personal ist ebenfalls der Personaltyp und die Anzahl festzulegen. Anschließend wird die Auswahl Bestätigt und die Transaktion ist abgeschlossen. 

\begin{figure}[h]
  \centering
  \fbox{
    \includegraphics[width=0.9\textwidth]{30_Fachkonzept/15_aktivitaetsdiagramm/activity2.pdf}
  }
  \caption{Aktivitätsdiagramm zur Personalverwaltung}
  \label{img:fachkonzept-aktivitaetsdiagramm-personal}
\end{figure}

\medskip

\textbf{Bauteile einkaufen}

Entscheidet sich der Spieler für das Einkaufen von Bauteilen, muss er zunächst den Bauteiltyp wählen und anschließend die einzukaufende Anzahl festlegen. Im Anschluss daran wird wieder die Liquidität des Spielers geprüft. Ist das Geld ausreichend, muss die Auswahl bestätigt werden. Fällt die Prüfung negativ aus, gelangt der Spieler wieder zur Aktivität “Bauteiltyp auswählen”. Dies wird in \vref{img:fachkonzept-aktivitaetsdiagramm-bauteile} verdeutlicht.

\begin{figure}[h]
  \centering
  \fbox{
    \includegraphics[width=0.4\textwidth]{30_Fachkonzept/15_aktivitaetsdiagramm/activity3.pdf}
  }
  \caption{Aktivitätsdiagramm zum Bauteileinkauf}
  \label{img:fachkonzept-aktivitaetsdiagramm-bauteile}
\end{figure}

\medskip

\textbf{Produktionsauftrag anlegen}

Beim Anlegen eines Produktionsauftrages, wie in \vref{img:fachkonzept-aktivitaetsdiagramm-produktion} dargestellt, ist zuerst das zu produzierende Raumschiff auszuwählen. Anschließend legt der Spieler die Anzahl fest. Nun muss geprüft werden, ob zum einen genügend Bauteile für die Produktion vorhanden sind und zum anderen ob das eingestellte Personal die Anzahl an Raumschiffen in einer Periode produzieren kann. Fällt die Prüfung positiv aus, so muss der Spieler seine Auswahl bestätigen. Sind die Kapazitäten jedoch nicht ausreichend, befindet sich der Spieler wieder bei der Aktivität “Raumschifftyp auswählen”.

\begin{figure}[h]
  \centering
  \fbox{
    \includegraphics[width=0.7\textwidth]{30_Fachkonzept/15_aktivitaetsdiagramm/activity4.pdf}
  }
  \caption{Aktivitätsdiagramm zur Produktion}
  \label{img:fachkonzept-aktivitaetsdiagramm-produktion}
\end{figure}

\medskip

\textbf{Verkaufsangebot abgeben}

Die letzte Transaktion kann im Bereich Verkauf getätigt werden. Dies ist auf \vref{img:fachkonzept-aktivitaetsdiagramm-verkauf} zu sehen. Zuerst wird der zu verkaufende Raumschifftyp ausgewählt. Danach entscheidet sich der Spieler für einen Preis, zu dem er die Raumschiffe verkaufen möchte. War bereits zuvor ein Angebot vorhanden, so wird dieses hiermit zurück genommen und danach bestätigt. War dies das erste Angebot, so gelangt der Spieler direkt zur Aktivität “Bestätigen”.

\begin{figure}[h]
  \centering
  \fbox{
    \includegraphics[width=0.7\textwidth]{30_Fachkonzept/15_aktivitaetsdiagramm/activity5.pdf}
  }
  \caption{Aktivitätsdiagramm zum Verkauf}
  \label{img:fachkonzept-aktivitaetsdiagramm-verkauf}
\end{figure}

\autorende{}
\autorbeginn{Britta}
\section{Aktivitätsdiagramm}
\label{sec:fachkonzept-aktivitaetsdiagramm}

\autorbeginn{Fredrik, Julia}

Das Aktivitätsdiagramm auf \vref{img:fachkonzept-aktivitaetsdiagramm-uebersicht} soll den Spielablauf aus Sicht des Spielers verdeutlichen.  

Um das Spiel zu starten muss der Spieler einen Namen für sein Unternehmen festlegen. Dies stellt die erste Aktion dar. Danach analysiert der Spieler die ihm zur Verfügung stehenden Informationen. Ist dies abgeschlossen, so gelangt er zu einem Entscheidungsknoten. Hierbei kann sich der Spieler zwischen folgenden Aktivitäten entscheiden: Personal verwalten, Einkäufe tätigen, Produktionsaufträge anlegen oder Verkaufsangebot abgeben. Diese einzelnen Vorgänge werden im Laufe dieses Kapitels genauer erläutert. Er kann sich aber auch dazu entscheiden, keine Transaktion zu tätigen. 

Diese verschiedenen Aktivitäten werden in einem Entscheidungsknoten zusammengeführt. Hat der Spieler weiteren Informationsbedarf, so gelangt er zur Aktivität “Auswirkungen analysieren” und kann sich die Veränderungen anschauen, die seine Transaktion mit sich geführt hat. Besteht kein Informationsbedarf, kann er diese Aktivität überspringen. 

Möchte der Spieler nun weitere Transaktionen tätigen, so kann er wieder zum Entscheidungsknoten nach oben springen und hat wieder die Wahl zwischen Personal verwalten, Einkäufe tätigen, Produktionsaufträge anlegen oder ein Verkaufsangebot abgeben. Entscheidet sich der Spieler gegen eine weitere Transaktion, so folgt die Aktivität “Runde einchecken”. Anschließend folgt wieder ein Entscheidungsknoten. Existiert noch eine weitere Spielrunde, so gelangt der Spieler wieder zur Aktivität “Informationen analysieren”. War dies die letzte Spielrunde, so wird dem Spieler die Endbewertung angezeigt. Danach ist das Spiel beendet.

\begin{figure}[h]
  \centering
  \fbox{
    \includegraphics[width=0.9\textwidth]{30_Fachkonzept/15_aktivitaetsdiagramm/activity1.pdf}
  }
  \caption{Aktivitätsdiagramm}
  \label{img:fachkonzept-aktivitaetsdiagramm-uebersicht}
\end{figure}

\medskip

\textbf{Personal verwalten}

Zur genaueren Betrachtung der Aktion “Personal verwalten” dient folgendes Aktivitätsdiagramm auf \vref{img:fachkonzept-aktivitaetsdiagramm-personal}. Entscheidet sich der Spieler für diese Aktion, so hat er die Möglichkeit Personal einzustellen, aufzurüsten oder zu entlassen. 

Um neues Personal einzustellen, muss der Spieler zuerst den Personaltyp auswählen, welchen er einstellen möchte. Danach folgt die Aktivität “Anzahl festlegen”. Anschließend trifft der Spieler auf einen Entscheidungsknoten. Ist genügend Geld vorhanden, muss der Spieler die Auswahl bestätigen. Reicht das verfügbare Geld jedoch nicht aus, gelangt der Spieler wieder zur Aktivität “einzustellender Typ auswählen” und durchläuft den Prozess noch ein mal. 

Möchte der Spieler sein vorhandenes Personal aufrüsten, muss er zunächst den Personaltyp auswählen. Anschließend gelangt er zur Aktivität “Anzahl festlegen”. Ähnlich wie beim Einstellen von neuem Personal wird auch nun geprüft, ob genügend Geld vorhanden ist. Ist dies der Fall, wird die Auswahl bestätigt. Fehlt Geld, befindet sich der Spieler wieder bei der Aktivität “aufzurüstender Typ auswählen”.

Zum Entlassen von Personal ist ebenfalls der Personaltyp und die Anzahl festzulegen. Anschließend wird die Auswahl Bestätigt und die Transaktion ist abgeschlossen. 

\begin{figure}[h]
  \centering
  \fbox{
    \includegraphics[width=0.9\textwidth]{30_Fachkonzept/15_aktivitaetsdiagramm/activity2.pdf}
  }
  \caption{Aktivitätsdiagramm zur Personalverwaltung}
  \label{img:fachkonzept-aktivitaetsdiagramm-personal}
\end{figure}

\medskip

\textbf{Bauteile einkaufen}

Entscheidet sich der Spieler für das Einkaufen von Bauteilen, muss er zunächst den Bauteiltyp wählen und anschließend die einzukaufende Anzahl festlegen. Im Anschluss daran wird wieder die Liquidität des Spielers geprüft. Ist das Geld ausreichend, muss die Auswahl bestätigt werden. Fällt die Prüfung negativ aus, gelangt der Spieler wieder zur Aktivität “Bauteiltyp auswählen”. Dies wird in \vref{img:fachkonzept-aktivitaetsdiagramm-bauteile} verdeutlicht.

\begin{figure}[h]
  \centering
  \fbox{
    \includegraphics[width=0.4\textwidth]{30_Fachkonzept/15_aktivitaetsdiagramm/activity3.pdf}
  }
  \caption{Aktivitätsdiagramm zum Bauteileinkauf}
  \label{img:fachkonzept-aktivitaetsdiagramm-bauteile}
\end{figure}

\medskip

\textbf{Produktionsauftrag anlegen}

Beim Anlegen eines Produktionsauftrages, wie in \vref{img:fachkonzept-aktivitaetsdiagramm-produktion} dargestellt, ist zuerst das zu produzierende Raumschiff auszuwählen. Anschließend legt der Spieler die Anzahl fest. Nun muss geprüft werden, ob zum einen genügend Bauteile für die Produktion vorhanden sind und zum anderen ob das eingestellte Personal die Anzahl an Raumschiffen in einer Periode produzieren kann. Fällt die Prüfung positiv aus, so muss der Spieler seine Auswahl bestätigen. Sind die Kapazitäten jedoch nicht ausreichend, befindet sich der Spieler wieder bei der Aktivität “Raumschifftyp auswählen”.

\begin{figure}[h]
  \centering
  \fbox{
    \includegraphics[width=0.7\textwidth]{30_Fachkonzept/15_aktivitaetsdiagramm/activity4.pdf}
  }
  \caption{Aktivitätsdiagramm zur Produktion}
  \label{img:fachkonzept-aktivitaetsdiagramm-produktion}
\end{figure}

\medskip

\textbf{Verkaufsangebot abgeben}

Die letzte Transaktion kann im Bereich Verkauf getätigt werden. Dies ist auf \vref{img:fachkonzept-aktivitaetsdiagramm-verkauf} zu sehen. Zuerst wird der zu verkaufende Raumschifftyp ausgewählt. Danach entscheidet sich der Spieler für einen Preis, zu dem er die Raumschiffe verkaufen möchte. War bereits zuvor ein Angebot vorhanden, so wird dieses hiermit zurück genommen und danach bestätigt. War dies das erste Angebot, so gelangt der Spieler direkt zur Aktivität “Bestätigen”.

\begin{figure}[h]
  \centering
  \fbox{
    \includegraphics[width=0.7\textwidth]{30_Fachkonzept/15_aktivitaetsdiagramm/activity5.pdf}
  }
  \caption{Aktivitätsdiagramm zum Verkauf}
  \label{img:fachkonzept-aktivitaetsdiagramm-verkauf}
\end{figure}

\autorende{}
\autorende{}
\section{Aktivitätsdiagramm}
\label{sec:fachkonzept-aktivitaetsdiagramm}

\autorbeginn{Fredrik, Julia}

Das Aktivitätsdiagramm auf \vref{img:fachkonzept-aktivitaetsdiagramm-uebersicht} soll den Spielablauf aus Sicht des Spielers verdeutlichen.  

Um das Spiel zu starten muss der Spieler einen Namen für sein Unternehmen festlegen. Dies stellt die erste Aktion dar. Danach analysiert der Spieler die ihm zur Verfügung stehenden Informationen. Ist dies abgeschlossen, so gelangt er zu einem Entscheidungsknoten. Hierbei kann sich der Spieler zwischen folgenden Aktivitäten entscheiden: Personal verwalten, Einkäufe tätigen, Produktionsaufträge anlegen oder Verkaufsangebot abgeben. Diese einzelnen Vorgänge werden im Laufe dieses Kapitels genauer erläutert. Er kann sich aber auch dazu entscheiden, keine Transaktion zu tätigen. 

Diese verschiedenen Aktivitäten werden in einem Entscheidungsknoten zusammengeführt. Hat der Spieler weiteren Informationsbedarf, so gelangt er zur Aktivität “Auswirkungen analysieren” und kann sich die Veränderungen anschauen, die seine Transaktion mit sich geführt hat. Besteht kein Informationsbedarf, kann er diese Aktivität überspringen. 

Möchte der Spieler nun weitere Transaktionen tätigen, so kann er wieder zum Entscheidungsknoten nach oben springen und hat wieder die Wahl zwischen Personal verwalten, Einkäufe tätigen, Produktionsaufträge anlegen oder ein Verkaufsangebot abgeben. Entscheidet sich der Spieler gegen eine weitere Transaktion, so folgt die Aktivität “Runde einchecken”. Anschließend folgt wieder ein Entscheidungsknoten. Existiert noch eine weitere Spielrunde, so gelangt der Spieler wieder zur Aktivität “Informationen analysieren”. War dies die letzte Spielrunde, so wird dem Spieler die Endbewertung angezeigt. Danach ist das Spiel beendet.

\begin{figure}[h]
  \centering
  \fbox{
    \includegraphics[width=0.9\textwidth]{30_Fachkonzept/15_aktivitaetsdiagramm/activity1.pdf}
  }
  \caption{Aktivitätsdiagramm}
  \label{img:fachkonzept-aktivitaetsdiagramm-uebersicht}
\end{figure}

\medskip

\textbf{Personal verwalten}

Zur genaueren Betrachtung der Aktion “Personal verwalten” dient folgendes Aktivitätsdiagramm auf \vref{img:fachkonzept-aktivitaetsdiagramm-personal}. Entscheidet sich der Spieler für diese Aktion, so hat er die Möglichkeit Personal einzustellen, aufzurüsten oder zu entlassen. 

Um neues Personal einzustellen, muss der Spieler zuerst den Personaltyp auswählen, welchen er einstellen möchte. Danach folgt die Aktivität “Anzahl festlegen”. Anschließend trifft der Spieler auf einen Entscheidungsknoten. Ist genügend Geld vorhanden, muss der Spieler die Auswahl bestätigen. Reicht das verfügbare Geld jedoch nicht aus, gelangt der Spieler wieder zur Aktivität “einzustellender Typ auswählen” und durchläuft den Prozess noch ein mal. 

Möchte der Spieler sein vorhandenes Personal aufrüsten, muss er zunächst den Personaltyp auswählen. Anschließend gelangt er zur Aktivität “Anzahl festlegen”. Ähnlich wie beim Einstellen von neuem Personal wird auch nun geprüft, ob genügend Geld vorhanden ist. Ist dies der Fall, wird die Auswahl bestätigt. Fehlt Geld, befindet sich der Spieler wieder bei der Aktivität “aufzurüstender Typ auswählen”.

Zum Entlassen von Personal ist ebenfalls der Personaltyp und die Anzahl festzulegen. Anschließend wird die Auswahl Bestätigt und die Transaktion ist abgeschlossen. 

\begin{figure}[h]
  \centering
  \fbox{
    \includegraphics[width=0.9\textwidth]{30_Fachkonzept/15_aktivitaetsdiagramm/activity2.pdf}
  }
  \caption{Aktivitätsdiagramm zur Personalverwaltung}
  \label{img:fachkonzept-aktivitaetsdiagramm-personal}
\end{figure}

\medskip

\textbf{Bauteile einkaufen}

Entscheidet sich der Spieler für das Einkaufen von Bauteilen, muss er zunächst den Bauteiltyp wählen und anschließend die einzukaufende Anzahl festlegen. Im Anschluss daran wird wieder die Liquidität des Spielers geprüft. Ist das Geld ausreichend, muss die Auswahl bestätigt werden. Fällt die Prüfung negativ aus, gelangt der Spieler wieder zur Aktivität “Bauteiltyp auswählen”. Dies wird in \vref{img:fachkonzept-aktivitaetsdiagramm-bauteile} verdeutlicht.

\begin{figure}[h]
  \centering
  \fbox{
    \includegraphics[width=0.4\textwidth]{30_Fachkonzept/15_aktivitaetsdiagramm/activity3.pdf}
  }
  \caption{Aktivitätsdiagramm zum Bauteileinkauf}
  \label{img:fachkonzept-aktivitaetsdiagramm-bauteile}
\end{figure}

\medskip

\textbf{Produktionsauftrag anlegen}

Beim Anlegen eines Produktionsauftrages, wie in \vref{img:fachkonzept-aktivitaetsdiagramm-produktion} dargestellt, ist zuerst das zu produzierende Raumschiff auszuwählen. Anschließend legt der Spieler die Anzahl fest. Nun muss geprüft werden, ob zum einen genügend Bauteile für die Produktion vorhanden sind und zum anderen ob das eingestellte Personal die Anzahl an Raumschiffen in einer Periode produzieren kann. Fällt die Prüfung positiv aus, so muss der Spieler seine Auswahl bestätigen. Sind die Kapazitäten jedoch nicht ausreichend, befindet sich der Spieler wieder bei der Aktivität “Raumschifftyp auswählen”.

\begin{figure}[h]
  \centering
  \fbox{
    \includegraphics[width=0.7\textwidth]{30_Fachkonzept/15_aktivitaetsdiagramm/activity4.pdf}
  }
  \caption{Aktivitätsdiagramm zur Produktion}
  \label{img:fachkonzept-aktivitaetsdiagramm-produktion}
\end{figure}

\medskip

\textbf{Verkaufsangebot abgeben}

Die letzte Transaktion kann im Bereich Verkauf getätigt werden. Dies ist auf \vref{img:fachkonzept-aktivitaetsdiagramm-verkauf} zu sehen. Zuerst wird der zu verkaufende Raumschifftyp ausgewählt. Danach entscheidet sich der Spieler für einen Preis, zu dem er die Raumschiffe verkaufen möchte. War bereits zuvor ein Angebot vorhanden, so wird dieses hiermit zurück genommen und danach bestätigt. War dies das erste Angebot, so gelangt der Spieler direkt zur Aktivität “Bestätigen”.

\begin{figure}[h]
  \centering
  \fbox{
    \includegraphics[width=0.7\textwidth]{30_Fachkonzept/15_aktivitaetsdiagramm/activity5.pdf}
  }
  \caption{Aktivitätsdiagramm zum Verkauf}
  \label{img:fachkonzept-aktivitaetsdiagramm-verkauf}
\end{figure}

\autorende{}

\chapter{Spielwelt}
\label{chp:spielwelt}

\section{Aktivitätsdiagramm}
\label{sec:fachkonzept-aktivitaetsdiagramm}

\autorbeginn{Fredrik, Julia}

Das Aktivitätsdiagramm auf \vref{img:fachkonzept-aktivitaetsdiagramm-uebersicht} soll den Spielablauf aus Sicht des Spielers verdeutlichen.  

Um das Spiel zu starten muss der Spieler einen Namen für sein Unternehmen festlegen. Dies stellt die erste Aktion dar. Danach analysiert der Spieler die ihm zur Verfügung stehenden Informationen. Ist dies abgeschlossen, so gelangt er zu einem Entscheidungsknoten. Hierbei kann sich der Spieler zwischen folgenden Aktivitäten entscheiden: Personal verwalten, Einkäufe tätigen, Produktionsaufträge anlegen oder Verkaufsangebot abgeben. Diese einzelnen Vorgänge werden im Laufe dieses Kapitels genauer erläutert. Er kann sich aber auch dazu entscheiden, keine Transaktion zu tätigen. 

Diese verschiedenen Aktivitäten werden in einem Entscheidungsknoten zusammengeführt. Hat der Spieler weiteren Informationsbedarf, so gelangt er zur Aktivität “Auswirkungen analysieren” und kann sich die Veränderungen anschauen, die seine Transaktion mit sich geführt hat. Besteht kein Informationsbedarf, kann er diese Aktivität überspringen. 

Möchte der Spieler nun weitere Transaktionen tätigen, so kann er wieder zum Entscheidungsknoten nach oben springen und hat wieder die Wahl zwischen Personal verwalten, Einkäufe tätigen, Produktionsaufträge anlegen oder ein Verkaufsangebot abgeben. Entscheidet sich der Spieler gegen eine weitere Transaktion, so folgt die Aktivität “Runde einchecken”. Anschließend folgt wieder ein Entscheidungsknoten. Existiert noch eine weitere Spielrunde, so gelangt der Spieler wieder zur Aktivität “Informationen analysieren”. War dies die letzte Spielrunde, so wird dem Spieler die Endbewertung angezeigt. Danach ist das Spiel beendet.

\begin{figure}[h]
  \centering
  \fbox{
    \includegraphics[width=0.9\textwidth]{30_Fachkonzept/15_aktivitaetsdiagramm/activity1.pdf}
  }
  \caption{Aktivitätsdiagramm}
  \label{img:fachkonzept-aktivitaetsdiagramm-uebersicht}
\end{figure}

\medskip

\textbf{Personal verwalten}

Zur genaueren Betrachtung der Aktion “Personal verwalten” dient folgendes Aktivitätsdiagramm auf \vref{img:fachkonzept-aktivitaetsdiagramm-personal}. Entscheidet sich der Spieler für diese Aktion, so hat er die Möglichkeit Personal einzustellen, aufzurüsten oder zu entlassen. 

Um neues Personal einzustellen, muss der Spieler zuerst den Personaltyp auswählen, welchen er einstellen möchte. Danach folgt die Aktivität “Anzahl festlegen”. Anschließend trifft der Spieler auf einen Entscheidungsknoten. Ist genügend Geld vorhanden, muss der Spieler die Auswahl bestätigen. Reicht das verfügbare Geld jedoch nicht aus, gelangt der Spieler wieder zur Aktivität “einzustellender Typ auswählen” und durchläuft den Prozess noch ein mal. 

Möchte der Spieler sein vorhandenes Personal aufrüsten, muss er zunächst den Personaltyp auswählen. Anschließend gelangt er zur Aktivität “Anzahl festlegen”. Ähnlich wie beim Einstellen von neuem Personal wird auch nun geprüft, ob genügend Geld vorhanden ist. Ist dies der Fall, wird die Auswahl bestätigt. Fehlt Geld, befindet sich der Spieler wieder bei der Aktivität “aufzurüstender Typ auswählen”.

Zum Entlassen von Personal ist ebenfalls der Personaltyp und die Anzahl festzulegen. Anschließend wird die Auswahl Bestätigt und die Transaktion ist abgeschlossen. 

\begin{figure}[h]
  \centering
  \fbox{
    \includegraphics[width=0.9\textwidth]{30_Fachkonzept/15_aktivitaetsdiagramm/activity2.pdf}
  }
  \caption{Aktivitätsdiagramm zur Personalverwaltung}
  \label{img:fachkonzept-aktivitaetsdiagramm-personal}
\end{figure}

\medskip

\textbf{Bauteile einkaufen}

Entscheidet sich der Spieler für das Einkaufen von Bauteilen, muss er zunächst den Bauteiltyp wählen und anschließend die einzukaufende Anzahl festlegen. Im Anschluss daran wird wieder die Liquidität des Spielers geprüft. Ist das Geld ausreichend, muss die Auswahl bestätigt werden. Fällt die Prüfung negativ aus, gelangt der Spieler wieder zur Aktivität “Bauteiltyp auswählen”. Dies wird in \vref{img:fachkonzept-aktivitaetsdiagramm-bauteile} verdeutlicht.

\begin{figure}[h]
  \centering
  \fbox{
    \includegraphics[width=0.4\textwidth]{30_Fachkonzept/15_aktivitaetsdiagramm/activity3.pdf}
  }
  \caption{Aktivitätsdiagramm zum Bauteileinkauf}
  \label{img:fachkonzept-aktivitaetsdiagramm-bauteile}
\end{figure}

\medskip

\textbf{Produktionsauftrag anlegen}

Beim Anlegen eines Produktionsauftrages, wie in \vref{img:fachkonzept-aktivitaetsdiagramm-produktion} dargestellt, ist zuerst das zu produzierende Raumschiff auszuwählen. Anschließend legt der Spieler die Anzahl fest. Nun muss geprüft werden, ob zum einen genügend Bauteile für die Produktion vorhanden sind und zum anderen ob das eingestellte Personal die Anzahl an Raumschiffen in einer Periode produzieren kann. Fällt die Prüfung positiv aus, so muss der Spieler seine Auswahl bestätigen. Sind die Kapazitäten jedoch nicht ausreichend, befindet sich der Spieler wieder bei der Aktivität “Raumschifftyp auswählen”.

\begin{figure}[h]
  \centering
  \fbox{
    \includegraphics[width=0.7\textwidth]{30_Fachkonzept/15_aktivitaetsdiagramm/activity4.pdf}
  }
  \caption{Aktivitätsdiagramm zur Produktion}
  \label{img:fachkonzept-aktivitaetsdiagramm-produktion}
\end{figure}

\medskip

\textbf{Verkaufsangebot abgeben}

Die letzte Transaktion kann im Bereich Verkauf getätigt werden. Dies ist auf \vref{img:fachkonzept-aktivitaetsdiagramm-verkauf} zu sehen. Zuerst wird der zu verkaufende Raumschifftyp ausgewählt. Danach entscheidet sich der Spieler für einen Preis, zu dem er die Raumschiffe verkaufen möchte. War bereits zuvor ein Angebot vorhanden, so wird dieses hiermit zurück genommen und danach bestätigt. War dies das erste Angebot, so gelangt der Spieler direkt zur Aktivität “Bestätigen”.

\begin{figure}[h]
  \centering
  \fbox{
    \includegraphics[width=0.7\textwidth]{30_Fachkonzept/15_aktivitaetsdiagramm/activity5.pdf}
  }
  \caption{Aktivitätsdiagramm zum Verkauf}
  \label{img:fachkonzept-aktivitaetsdiagramm-verkauf}
\end{figure}

\autorende{}
\autorbeginn{Britta}
\section{Aktivitätsdiagramm}
\label{sec:fachkonzept-aktivitaetsdiagramm}

\autorbeginn{Fredrik, Julia}

Das Aktivitätsdiagramm auf \vref{img:fachkonzept-aktivitaetsdiagramm-uebersicht} soll den Spielablauf aus Sicht des Spielers verdeutlichen.  

Um das Spiel zu starten muss der Spieler einen Namen für sein Unternehmen festlegen. Dies stellt die erste Aktion dar. Danach analysiert der Spieler die ihm zur Verfügung stehenden Informationen. Ist dies abgeschlossen, so gelangt er zu einem Entscheidungsknoten. Hierbei kann sich der Spieler zwischen folgenden Aktivitäten entscheiden: Personal verwalten, Einkäufe tätigen, Produktionsaufträge anlegen oder Verkaufsangebot abgeben. Diese einzelnen Vorgänge werden im Laufe dieses Kapitels genauer erläutert. Er kann sich aber auch dazu entscheiden, keine Transaktion zu tätigen. 

Diese verschiedenen Aktivitäten werden in einem Entscheidungsknoten zusammengeführt. Hat der Spieler weiteren Informationsbedarf, so gelangt er zur Aktivität “Auswirkungen analysieren” und kann sich die Veränderungen anschauen, die seine Transaktion mit sich geführt hat. Besteht kein Informationsbedarf, kann er diese Aktivität überspringen. 

Möchte der Spieler nun weitere Transaktionen tätigen, so kann er wieder zum Entscheidungsknoten nach oben springen und hat wieder die Wahl zwischen Personal verwalten, Einkäufe tätigen, Produktionsaufträge anlegen oder ein Verkaufsangebot abgeben. Entscheidet sich der Spieler gegen eine weitere Transaktion, so folgt die Aktivität “Runde einchecken”. Anschließend folgt wieder ein Entscheidungsknoten. Existiert noch eine weitere Spielrunde, so gelangt der Spieler wieder zur Aktivität “Informationen analysieren”. War dies die letzte Spielrunde, so wird dem Spieler die Endbewertung angezeigt. Danach ist das Spiel beendet.

\begin{figure}[h]
  \centering
  \fbox{
    \includegraphics[width=0.9\textwidth]{30_Fachkonzept/15_aktivitaetsdiagramm/activity1.pdf}
  }
  \caption{Aktivitätsdiagramm}
  \label{img:fachkonzept-aktivitaetsdiagramm-uebersicht}
\end{figure}

\medskip

\textbf{Personal verwalten}

Zur genaueren Betrachtung der Aktion “Personal verwalten” dient folgendes Aktivitätsdiagramm auf \vref{img:fachkonzept-aktivitaetsdiagramm-personal}. Entscheidet sich der Spieler für diese Aktion, so hat er die Möglichkeit Personal einzustellen, aufzurüsten oder zu entlassen. 

Um neues Personal einzustellen, muss der Spieler zuerst den Personaltyp auswählen, welchen er einstellen möchte. Danach folgt die Aktivität “Anzahl festlegen”. Anschließend trifft der Spieler auf einen Entscheidungsknoten. Ist genügend Geld vorhanden, muss der Spieler die Auswahl bestätigen. Reicht das verfügbare Geld jedoch nicht aus, gelangt der Spieler wieder zur Aktivität “einzustellender Typ auswählen” und durchläuft den Prozess noch ein mal. 

Möchte der Spieler sein vorhandenes Personal aufrüsten, muss er zunächst den Personaltyp auswählen. Anschließend gelangt er zur Aktivität “Anzahl festlegen”. Ähnlich wie beim Einstellen von neuem Personal wird auch nun geprüft, ob genügend Geld vorhanden ist. Ist dies der Fall, wird die Auswahl bestätigt. Fehlt Geld, befindet sich der Spieler wieder bei der Aktivität “aufzurüstender Typ auswählen”.

Zum Entlassen von Personal ist ebenfalls der Personaltyp und die Anzahl festzulegen. Anschließend wird die Auswahl Bestätigt und die Transaktion ist abgeschlossen. 

\begin{figure}[h]
  \centering
  \fbox{
    \includegraphics[width=0.9\textwidth]{30_Fachkonzept/15_aktivitaetsdiagramm/activity2.pdf}
  }
  \caption{Aktivitätsdiagramm zur Personalverwaltung}
  \label{img:fachkonzept-aktivitaetsdiagramm-personal}
\end{figure}

\medskip

\textbf{Bauteile einkaufen}

Entscheidet sich der Spieler für das Einkaufen von Bauteilen, muss er zunächst den Bauteiltyp wählen und anschließend die einzukaufende Anzahl festlegen. Im Anschluss daran wird wieder die Liquidität des Spielers geprüft. Ist das Geld ausreichend, muss die Auswahl bestätigt werden. Fällt die Prüfung negativ aus, gelangt der Spieler wieder zur Aktivität “Bauteiltyp auswählen”. Dies wird in \vref{img:fachkonzept-aktivitaetsdiagramm-bauteile} verdeutlicht.

\begin{figure}[h]
  \centering
  \fbox{
    \includegraphics[width=0.4\textwidth]{30_Fachkonzept/15_aktivitaetsdiagramm/activity3.pdf}
  }
  \caption{Aktivitätsdiagramm zum Bauteileinkauf}
  \label{img:fachkonzept-aktivitaetsdiagramm-bauteile}
\end{figure}

\medskip

\textbf{Produktionsauftrag anlegen}

Beim Anlegen eines Produktionsauftrages, wie in \vref{img:fachkonzept-aktivitaetsdiagramm-produktion} dargestellt, ist zuerst das zu produzierende Raumschiff auszuwählen. Anschließend legt der Spieler die Anzahl fest. Nun muss geprüft werden, ob zum einen genügend Bauteile für die Produktion vorhanden sind und zum anderen ob das eingestellte Personal die Anzahl an Raumschiffen in einer Periode produzieren kann. Fällt die Prüfung positiv aus, so muss der Spieler seine Auswahl bestätigen. Sind die Kapazitäten jedoch nicht ausreichend, befindet sich der Spieler wieder bei der Aktivität “Raumschifftyp auswählen”.

\begin{figure}[h]
  \centering
  \fbox{
    \includegraphics[width=0.7\textwidth]{30_Fachkonzept/15_aktivitaetsdiagramm/activity4.pdf}
  }
  \caption{Aktivitätsdiagramm zur Produktion}
  \label{img:fachkonzept-aktivitaetsdiagramm-produktion}
\end{figure}

\medskip

\textbf{Verkaufsangebot abgeben}

Die letzte Transaktion kann im Bereich Verkauf getätigt werden. Dies ist auf \vref{img:fachkonzept-aktivitaetsdiagramm-verkauf} zu sehen. Zuerst wird der zu verkaufende Raumschifftyp ausgewählt. Danach entscheidet sich der Spieler für einen Preis, zu dem er die Raumschiffe verkaufen möchte. War bereits zuvor ein Angebot vorhanden, so wird dieses hiermit zurück genommen und danach bestätigt. War dies das erste Angebot, so gelangt der Spieler direkt zur Aktivität “Bestätigen”.

\begin{figure}[h]
  \centering
  \fbox{
    \includegraphics[width=0.7\textwidth]{30_Fachkonzept/15_aktivitaetsdiagramm/activity5.pdf}
  }
  \caption{Aktivitätsdiagramm zum Verkauf}
  \label{img:fachkonzept-aktivitaetsdiagramm-verkauf}
\end{figure}

\autorende{}
\autorende{}
\section{Aktivitätsdiagramm}
\label{sec:fachkonzept-aktivitaetsdiagramm}

\autorbeginn{Fredrik, Julia}

Das Aktivitätsdiagramm auf \vref{img:fachkonzept-aktivitaetsdiagramm-uebersicht} soll den Spielablauf aus Sicht des Spielers verdeutlichen.  

Um das Spiel zu starten muss der Spieler einen Namen für sein Unternehmen festlegen. Dies stellt die erste Aktion dar. Danach analysiert der Spieler die ihm zur Verfügung stehenden Informationen. Ist dies abgeschlossen, so gelangt er zu einem Entscheidungsknoten. Hierbei kann sich der Spieler zwischen folgenden Aktivitäten entscheiden: Personal verwalten, Einkäufe tätigen, Produktionsaufträge anlegen oder Verkaufsangebot abgeben. Diese einzelnen Vorgänge werden im Laufe dieses Kapitels genauer erläutert. Er kann sich aber auch dazu entscheiden, keine Transaktion zu tätigen. 

Diese verschiedenen Aktivitäten werden in einem Entscheidungsknoten zusammengeführt. Hat der Spieler weiteren Informationsbedarf, so gelangt er zur Aktivität “Auswirkungen analysieren” und kann sich die Veränderungen anschauen, die seine Transaktion mit sich geführt hat. Besteht kein Informationsbedarf, kann er diese Aktivität überspringen. 

Möchte der Spieler nun weitere Transaktionen tätigen, so kann er wieder zum Entscheidungsknoten nach oben springen und hat wieder die Wahl zwischen Personal verwalten, Einkäufe tätigen, Produktionsaufträge anlegen oder ein Verkaufsangebot abgeben. Entscheidet sich der Spieler gegen eine weitere Transaktion, so folgt die Aktivität “Runde einchecken”. Anschließend folgt wieder ein Entscheidungsknoten. Existiert noch eine weitere Spielrunde, so gelangt der Spieler wieder zur Aktivität “Informationen analysieren”. War dies die letzte Spielrunde, so wird dem Spieler die Endbewertung angezeigt. Danach ist das Spiel beendet.

\begin{figure}[h]
  \centering
  \fbox{
    \includegraphics[width=0.9\textwidth]{30_Fachkonzept/15_aktivitaetsdiagramm/activity1.pdf}
  }
  \caption{Aktivitätsdiagramm}
  \label{img:fachkonzept-aktivitaetsdiagramm-uebersicht}
\end{figure}

\medskip

\textbf{Personal verwalten}

Zur genaueren Betrachtung der Aktion “Personal verwalten” dient folgendes Aktivitätsdiagramm auf \vref{img:fachkonzept-aktivitaetsdiagramm-personal}. Entscheidet sich der Spieler für diese Aktion, so hat er die Möglichkeit Personal einzustellen, aufzurüsten oder zu entlassen. 

Um neues Personal einzustellen, muss der Spieler zuerst den Personaltyp auswählen, welchen er einstellen möchte. Danach folgt die Aktivität “Anzahl festlegen”. Anschließend trifft der Spieler auf einen Entscheidungsknoten. Ist genügend Geld vorhanden, muss der Spieler die Auswahl bestätigen. Reicht das verfügbare Geld jedoch nicht aus, gelangt der Spieler wieder zur Aktivität “einzustellender Typ auswählen” und durchläuft den Prozess noch ein mal. 

Möchte der Spieler sein vorhandenes Personal aufrüsten, muss er zunächst den Personaltyp auswählen. Anschließend gelangt er zur Aktivität “Anzahl festlegen”. Ähnlich wie beim Einstellen von neuem Personal wird auch nun geprüft, ob genügend Geld vorhanden ist. Ist dies der Fall, wird die Auswahl bestätigt. Fehlt Geld, befindet sich der Spieler wieder bei der Aktivität “aufzurüstender Typ auswählen”.

Zum Entlassen von Personal ist ebenfalls der Personaltyp und die Anzahl festzulegen. Anschließend wird die Auswahl Bestätigt und die Transaktion ist abgeschlossen. 

\begin{figure}[h]
  \centering
  \fbox{
    \includegraphics[width=0.9\textwidth]{30_Fachkonzept/15_aktivitaetsdiagramm/activity2.pdf}
  }
  \caption{Aktivitätsdiagramm zur Personalverwaltung}
  \label{img:fachkonzept-aktivitaetsdiagramm-personal}
\end{figure}

\medskip

\textbf{Bauteile einkaufen}

Entscheidet sich der Spieler für das Einkaufen von Bauteilen, muss er zunächst den Bauteiltyp wählen und anschließend die einzukaufende Anzahl festlegen. Im Anschluss daran wird wieder die Liquidität des Spielers geprüft. Ist das Geld ausreichend, muss die Auswahl bestätigt werden. Fällt die Prüfung negativ aus, gelangt der Spieler wieder zur Aktivität “Bauteiltyp auswählen”. Dies wird in \vref{img:fachkonzept-aktivitaetsdiagramm-bauteile} verdeutlicht.

\begin{figure}[h]
  \centering
  \fbox{
    \includegraphics[width=0.4\textwidth]{30_Fachkonzept/15_aktivitaetsdiagramm/activity3.pdf}
  }
  \caption{Aktivitätsdiagramm zum Bauteileinkauf}
  \label{img:fachkonzept-aktivitaetsdiagramm-bauteile}
\end{figure}

\medskip

\textbf{Produktionsauftrag anlegen}

Beim Anlegen eines Produktionsauftrages, wie in \vref{img:fachkonzept-aktivitaetsdiagramm-produktion} dargestellt, ist zuerst das zu produzierende Raumschiff auszuwählen. Anschließend legt der Spieler die Anzahl fest. Nun muss geprüft werden, ob zum einen genügend Bauteile für die Produktion vorhanden sind und zum anderen ob das eingestellte Personal die Anzahl an Raumschiffen in einer Periode produzieren kann. Fällt die Prüfung positiv aus, so muss der Spieler seine Auswahl bestätigen. Sind die Kapazitäten jedoch nicht ausreichend, befindet sich der Spieler wieder bei der Aktivität “Raumschifftyp auswählen”.

\begin{figure}[h]
  \centering
  \fbox{
    \includegraphics[width=0.7\textwidth]{30_Fachkonzept/15_aktivitaetsdiagramm/activity4.pdf}
  }
  \caption{Aktivitätsdiagramm zur Produktion}
  \label{img:fachkonzept-aktivitaetsdiagramm-produktion}
\end{figure}

\medskip

\textbf{Verkaufsangebot abgeben}

Die letzte Transaktion kann im Bereich Verkauf getätigt werden. Dies ist auf \vref{img:fachkonzept-aktivitaetsdiagramm-verkauf} zu sehen. Zuerst wird der zu verkaufende Raumschifftyp ausgewählt. Danach entscheidet sich der Spieler für einen Preis, zu dem er die Raumschiffe verkaufen möchte. War bereits zuvor ein Angebot vorhanden, so wird dieses hiermit zurück genommen und danach bestätigt. War dies das erste Angebot, so gelangt der Spieler direkt zur Aktivität “Bestätigen”.

\begin{figure}[h]
  \centering
  \fbox{
    \includegraphics[width=0.7\textwidth]{30_Fachkonzept/15_aktivitaetsdiagramm/activity5.pdf}
  }
  \caption{Aktivitätsdiagramm zum Verkauf}
  \label{img:fachkonzept-aktivitaetsdiagramm-verkauf}
\end{figure}

\autorende{}
\autorbeginn{Britta}
\section{Aktivitätsdiagramm}
\label{sec:fachkonzept-aktivitaetsdiagramm}

\autorbeginn{Fredrik, Julia}

Das Aktivitätsdiagramm auf \vref{img:fachkonzept-aktivitaetsdiagramm-uebersicht} soll den Spielablauf aus Sicht des Spielers verdeutlichen.  

Um das Spiel zu starten muss der Spieler einen Namen für sein Unternehmen festlegen. Dies stellt die erste Aktion dar. Danach analysiert der Spieler die ihm zur Verfügung stehenden Informationen. Ist dies abgeschlossen, so gelangt er zu einem Entscheidungsknoten. Hierbei kann sich der Spieler zwischen folgenden Aktivitäten entscheiden: Personal verwalten, Einkäufe tätigen, Produktionsaufträge anlegen oder Verkaufsangebot abgeben. Diese einzelnen Vorgänge werden im Laufe dieses Kapitels genauer erläutert. Er kann sich aber auch dazu entscheiden, keine Transaktion zu tätigen. 

Diese verschiedenen Aktivitäten werden in einem Entscheidungsknoten zusammengeführt. Hat der Spieler weiteren Informationsbedarf, so gelangt er zur Aktivität “Auswirkungen analysieren” und kann sich die Veränderungen anschauen, die seine Transaktion mit sich geführt hat. Besteht kein Informationsbedarf, kann er diese Aktivität überspringen. 

Möchte der Spieler nun weitere Transaktionen tätigen, so kann er wieder zum Entscheidungsknoten nach oben springen und hat wieder die Wahl zwischen Personal verwalten, Einkäufe tätigen, Produktionsaufträge anlegen oder ein Verkaufsangebot abgeben. Entscheidet sich der Spieler gegen eine weitere Transaktion, so folgt die Aktivität “Runde einchecken”. Anschließend folgt wieder ein Entscheidungsknoten. Existiert noch eine weitere Spielrunde, so gelangt der Spieler wieder zur Aktivität “Informationen analysieren”. War dies die letzte Spielrunde, so wird dem Spieler die Endbewertung angezeigt. Danach ist das Spiel beendet.

\begin{figure}[h]
  \centering
  \fbox{
    \includegraphics[width=0.9\textwidth]{30_Fachkonzept/15_aktivitaetsdiagramm/activity1.pdf}
  }
  \caption{Aktivitätsdiagramm}
  \label{img:fachkonzept-aktivitaetsdiagramm-uebersicht}
\end{figure}

\medskip

\textbf{Personal verwalten}

Zur genaueren Betrachtung der Aktion “Personal verwalten” dient folgendes Aktivitätsdiagramm auf \vref{img:fachkonzept-aktivitaetsdiagramm-personal}. Entscheidet sich der Spieler für diese Aktion, so hat er die Möglichkeit Personal einzustellen, aufzurüsten oder zu entlassen. 

Um neues Personal einzustellen, muss der Spieler zuerst den Personaltyp auswählen, welchen er einstellen möchte. Danach folgt die Aktivität “Anzahl festlegen”. Anschließend trifft der Spieler auf einen Entscheidungsknoten. Ist genügend Geld vorhanden, muss der Spieler die Auswahl bestätigen. Reicht das verfügbare Geld jedoch nicht aus, gelangt der Spieler wieder zur Aktivität “einzustellender Typ auswählen” und durchläuft den Prozess noch ein mal. 

Möchte der Spieler sein vorhandenes Personal aufrüsten, muss er zunächst den Personaltyp auswählen. Anschließend gelangt er zur Aktivität “Anzahl festlegen”. Ähnlich wie beim Einstellen von neuem Personal wird auch nun geprüft, ob genügend Geld vorhanden ist. Ist dies der Fall, wird die Auswahl bestätigt. Fehlt Geld, befindet sich der Spieler wieder bei der Aktivität “aufzurüstender Typ auswählen”.

Zum Entlassen von Personal ist ebenfalls der Personaltyp und die Anzahl festzulegen. Anschließend wird die Auswahl Bestätigt und die Transaktion ist abgeschlossen. 

\begin{figure}[h]
  \centering
  \fbox{
    \includegraphics[width=0.9\textwidth]{30_Fachkonzept/15_aktivitaetsdiagramm/activity2.pdf}
  }
  \caption{Aktivitätsdiagramm zur Personalverwaltung}
  \label{img:fachkonzept-aktivitaetsdiagramm-personal}
\end{figure}

\medskip

\textbf{Bauteile einkaufen}

Entscheidet sich der Spieler für das Einkaufen von Bauteilen, muss er zunächst den Bauteiltyp wählen und anschließend die einzukaufende Anzahl festlegen. Im Anschluss daran wird wieder die Liquidität des Spielers geprüft. Ist das Geld ausreichend, muss die Auswahl bestätigt werden. Fällt die Prüfung negativ aus, gelangt der Spieler wieder zur Aktivität “Bauteiltyp auswählen”. Dies wird in \vref{img:fachkonzept-aktivitaetsdiagramm-bauteile} verdeutlicht.

\begin{figure}[h]
  \centering
  \fbox{
    \includegraphics[width=0.4\textwidth]{30_Fachkonzept/15_aktivitaetsdiagramm/activity3.pdf}
  }
  \caption{Aktivitätsdiagramm zum Bauteileinkauf}
  \label{img:fachkonzept-aktivitaetsdiagramm-bauteile}
\end{figure}

\medskip

\textbf{Produktionsauftrag anlegen}

Beim Anlegen eines Produktionsauftrages, wie in \vref{img:fachkonzept-aktivitaetsdiagramm-produktion} dargestellt, ist zuerst das zu produzierende Raumschiff auszuwählen. Anschließend legt der Spieler die Anzahl fest. Nun muss geprüft werden, ob zum einen genügend Bauteile für die Produktion vorhanden sind und zum anderen ob das eingestellte Personal die Anzahl an Raumschiffen in einer Periode produzieren kann. Fällt die Prüfung positiv aus, so muss der Spieler seine Auswahl bestätigen. Sind die Kapazitäten jedoch nicht ausreichend, befindet sich der Spieler wieder bei der Aktivität “Raumschifftyp auswählen”.

\begin{figure}[h]
  \centering
  \fbox{
    \includegraphics[width=0.7\textwidth]{30_Fachkonzept/15_aktivitaetsdiagramm/activity4.pdf}
  }
  \caption{Aktivitätsdiagramm zur Produktion}
  \label{img:fachkonzept-aktivitaetsdiagramm-produktion}
\end{figure}

\medskip

\textbf{Verkaufsangebot abgeben}

Die letzte Transaktion kann im Bereich Verkauf getätigt werden. Dies ist auf \vref{img:fachkonzept-aktivitaetsdiagramm-verkauf} zu sehen. Zuerst wird der zu verkaufende Raumschifftyp ausgewählt. Danach entscheidet sich der Spieler für einen Preis, zu dem er die Raumschiffe verkaufen möchte. War bereits zuvor ein Angebot vorhanden, so wird dieses hiermit zurück genommen und danach bestätigt. War dies das erste Angebot, so gelangt der Spieler direkt zur Aktivität “Bestätigen”.

\begin{figure}[h]
  \centering
  \fbox{
    \includegraphics[width=0.7\textwidth]{30_Fachkonzept/15_aktivitaetsdiagramm/activity5.pdf}
  }
  \caption{Aktivitätsdiagramm zum Verkauf}
  \label{img:fachkonzept-aktivitaetsdiagramm-verkauf}
\end{figure}

\autorende{}
\autorende{}
\section{Aktivitätsdiagramm}
\label{sec:fachkonzept-aktivitaetsdiagramm}

\autorbeginn{Fredrik, Julia}

Das Aktivitätsdiagramm auf \vref{img:fachkonzept-aktivitaetsdiagramm-uebersicht} soll den Spielablauf aus Sicht des Spielers verdeutlichen.  

Um das Spiel zu starten muss der Spieler einen Namen für sein Unternehmen festlegen. Dies stellt die erste Aktion dar. Danach analysiert der Spieler die ihm zur Verfügung stehenden Informationen. Ist dies abgeschlossen, so gelangt er zu einem Entscheidungsknoten. Hierbei kann sich der Spieler zwischen folgenden Aktivitäten entscheiden: Personal verwalten, Einkäufe tätigen, Produktionsaufträge anlegen oder Verkaufsangebot abgeben. Diese einzelnen Vorgänge werden im Laufe dieses Kapitels genauer erläutert. Er kann sich aber auch dazu entscheiden, keine Transaktion zu tätigen. 

Diese verschiedenen Aktivitäten werden in einem Entscheidungsknoten zusammengeführt. Hat der Spieler weiteren Informationsbedarf, so gelangt er zur Aktivität “Auswirkungen analysieren” und kann sich die Veränderungen anschauen, die seine Transaktion mit sich geführt hat. Besteht kein Informationsbedarf, kann er diese Aktivität überspringen. 

Möchte der Spieler nun weitere Transaktionen tätigen, so kann er wieder zum Entscheidungsknoten nach oben springen und hat wieder die Wahl zwischen Personal verwalten, Einkäufe tätigen, Produktionsaufträge anlegen oder ein Verkaufsangebot abgeben. Entscheidet sich der Spieler gegen eine weitere Transaktion, so folgt die Aktivität “Runde einchecken”. Anschließend folgt wieder ein Entscheidungsknoten. Existiert noch eine weitere Spielrunde, so gelangt der Spieler wieder zur Aktivität “Informationen analysieren”. War dies die letzte Spielrunde, so wird dem Spieler die Endbewertung angezeigt. Danach ist das Spiel beendet.

\begin{figure}[h]
  \centering
  \fbox{
    \includegraphics[width=0.9\textwidth]{30_Fachkonzept/15_aktivitaetsdiagramm/activity1.pdf}
  }
  \caption{Aktivitätsdiagramm}
  \label{img:fachkonzept-aktivitaetsdiagramm-uebersicht}
\end{figure}

\medskip

\textbf{Personal verwalten}

Zur genaueren Betrachtung der Aktion “Personal verwalten” dient folgendes Aktivitätsdiagramm auf \vref{img:fachkonzept-aktivitaetsdiagramm-personal}. Entscheidet sich der Spieler für diese Aktion, so hat er die Möglichkeit Personal einzustellen, aufzurüsten oder zu entlassen. 

Um neues Personal einzustellen, muss der Spieler zuerst den Personaltyp auswählen, welchen er einstellen möchte. Danach folgt die Aktivität “Anzahl festlegen”. Anschließend trifft der Spieler auf einen Entscheidungsknoten. Ist genügend Geld vorhanden, muss der Spieler die Auswahl bestätigen. Reicht das verfügbare Geld jedoch nicht aus, gelangt der Spieler wieder zur Aktivität “einzustellender Typ auswählen” und durchläuft den Prozess noch ein mal. 

Möchte der Spieler sein vorhandenes Personal aufrüsten, muss er zunächst den Personaltyp auswählen. Anschließend gelangt er zur Aktivität “Anzahl festlegen”. Ähnlich wie beim Einstellen von neuem Personal wird auch nun geprüft, ob genügend Geld vorhanden ist. Ist dies der Fall, wird die Auswahl bestätigt. Fehlt Geld, befindet sich der Spieler wieder bei der Aktivität “aufzurüstender Typ auswählen”.

Zum Entlassen von Personal ist ebenfalls der Personaltyp und die Anzahl festzulegen. Anschließend wird die Auswahl Bestätigt und die Transaktion ist abgeschlossen. 

\begin{figure}[h]
  \centering
  \fbox{
    \includegraphics[width=0.9\textwidth]{30_Fachkonzept/15_aktivitaetsdiagramm/activity2.pdf}
  }
  \caption{Aktivitätsdiagramm zur Personalverwaltung}
  \label{img:fachkonzept-aktivitaetsdiagramm-personal}
\end{figure}

\medskip

\textbf{Bauteile einkaufen}

Entscheidet sich der Spieler für das Einkaufen von Bauteilen, muss er zunächst den Bauteiltyp wählen und anschließend die einzukaufende Anzahl festlegen. Im Anschluss daran wird wieder die Liquidität des Spielers geprüft. Ist das Geld ausreichend, muss die Auswahl bestätigt werden. Fällt die Prüfung negativ aus, gelangt der Spieler wieder zur Aktivität “Bauteiltyp auswählen”. Dies wird in \vref{img:fachkonzept-aktivitaetsdiagramm-bauteile} verdeutlicht.

\begin{figure}[h]
  \centering
  \fbox{
    \includegraphics[width=0.4\textwidth]{30_Fachkonzept/15_aktivitaetsdiagramm/activity3.pdf}
  }
  \caption{Aktivitätsdiagramm zum Bauteileinkauf}
  \label{img:fachkonzept-aktivitaetsdiagramm-bauteile}
\end{figure}

\medskip

\textbf{Produktionsauftrag anlegen}

Beim Anlegen eines Produktionsauftrages, wie in \vref{img:fachkonzept-aktivitaetsdiagramm-produktion} dargestellt, ist zuerst das zu produzierende Raumschiff auszuwählen. Anschließend legt der Spieler die Anzahl fest. Nun muss geprüft werden, ob zum einen genügend Bauteile für die Produktion vorhanden sind und zum anderen ob das eingestellte Personal die Anzahl an Raumschiffen in einer Periode produzieren kann. Fällt die Prüfung positiv aus, so muss der Spieler seine Auswahl bestätigen. Sind die Kapazitäten jedoch nicht ausreichend, befindet sich der Spieler wieder bei der Aktivität “Raumschifftyp auswählen”.

\begin{figure}[h]
  \centering
  \fbox{
    \includegraphics[width=0.7\textwidth]{30_Fachkonzept/15_aktivitaetsdiagramm/activity4.pdf}
  }
  \caption{Aktivitätsdiagramm zur Produktion}
  \label{img:fachkonzept-aktivitaetsdiagramm-produktion}
\end{figure}

\medskip

\textbf{Verkaufsangebot abgeben}

Die letzte Transaktion kann im Bereich Verkauf getätigt werden. Dies ist auf \vref{img:fachkonzept-aktivitaetsdiagramm-verkauf} zu sehen. Zuerst wird der zu verkaufende Raumschifftyp ausgewählt. Danach entscheidet sich der Spieler für einen Preis, zu dem er die Raumschiffe verkaufen möchte. War bereits zuvor ein Angebot vorhanden, so wird dieses hiermit zurück genommen und danach bestätigt. War dies das erste Angebot, so gelangt der Spieler direkt zur Aktivität “Bestätigen”.

\begin{figure}[h]
  \centering
  \fbox{
    \includegraphics[width=0.7\textwidth]{30_Fachkonzept/15_aktivitaetsdiagramm/activity5.pdf}
  }
  \caption{Aktivitätsdiagramm zum Verkauf}
  \label{img:fachkonzept-aktivitaetsdiagramm-verkauf}
\end{figure}

\autorende{}


% Mit markright kann eine verk\"urzte Version der \"Uberschrift f\"ur den Seitenkopf generiert werden
%
%
%\markright{Formaler Aufbau}




% Anhang der Arbeit
% 
%
\seAppendix{}
\chapter{Einige wichtige \LaTeX{}-Kommandos}



%  Testdatei f\"ur die Erzeugung von Literaturreferenzen, die den Regeln von Rene Theisen 
%  (Wissenschaftliches Arbeiten, 2009) folgen
%
%
%
\section{Kommandos f\"ur die Erzeugung von Literaturverweisen}

Das Kommando \verb+\seCite{par1}{par2}{par3}+ erzeugt einen Literaturverweis im Text. 

\begin{seToplist}{\texttt{par1}:}
\item[\texttt{par1}:] Der erste Parameter  definiert einen optionalen Text, der vor dem eigentlichen Literaturverweis ausgegeben 
                               wird, typischerweise Vgl. oder vgl.
\item[\texttt{par2}:] Der zweite Parameter  wird verwendet, um (z.\,B.) zus\"atzliche Seitenangaben f\"ur den Literaturverweis 
                              vorzunehmen.
\item[\texttt{par2}:] Der dritte Parameter ist der entsprechende Schl\"ussel in der .bib-Datei, in der die Literaturquellen 
                              beschrieben sind (vgl. \texttt{wa.bib}).                                                                                       
\end{seToplist}

Als Beispiel f\"ur die Verwendung des \verb+\seCite+-Befehls dient folgendes Zitat: \glqq{}Die \textbf{Funktion} eines 
Anhangs einer wissenschaftlichen Arbeit wird sehr h\"aufig \textbf{missdeutet}, der Anhang selbst nicht selten \textbf{mi{\ss}braucht}.\grqq{} 
(\seCite{}{S. 170}{The:WA}).

Bei der von Theisen vorgeschlagenen Zitierweise erfolgt die Angabe der Literaturverweise in der Regel innerhalb einer Fu{\ss}note. 
Hierf\"ur kann das Kommando \verb+\seFootcite+ verwendet werden, das dieselben Parameter wie \verb+\seCite+ besitzt. 

Als Beispiel f\"ur ein indirektes Zitat l\"asst sich die Aussage von Theisen anf\"uhren, dass Hauptinhalte eines (berechtigten) Anhangs erg\"anzende 
Materialien und Dokumente sind, die weitere themenbezogene Informationen liefern k\"onnen.\seFootcite{Vgl.}{S. 171}{The:WA}

Weder das \verb+\seFootcite+- noch das \verb+\footnote+-Kommande k\"onnen bei Gleitobjekten (Verwendung der \verb+figure+-, \verb+table+- oder 
\verb+programm+-Umgebung) verwendet werden. Ein kleiner Workaround, um \LaTeX{} doch dazu zu bringen, Fu{\ss}noten bei Gleitobjekten 
zu akzeptieren, ist in \vref{gleitobjekte} zu finden.

%
% Ein kleiner Text, um Abk\"urzungen, Symbole und Glossareintr\"age zu testen
%
%
\section{Kommandos f\"ur die Erzeugung von Abk\"urzungen, Symbolen und Glos\-sar\-eint\-r\"a\-gen}

F\"ur Abk\"urzungen, Symbole und Glossareintr\"age wird das Kommando \verb+\gls{par1}+ verwendet.
\texttt{par1} stellt einen Schl\"ussel dar, der die entsprechende Definition identifiziert (vgl. den Inhalt der Datei
\texttt{pa1-abkuerzungen.tex}). 


Eine Abk\"urzung: \gls{usb}; das zweite Auftreten der Abk\"urzung: \gls{usb}. 
Und jetzt kommt ein Symbol: \gls{pi}; das zweite Symbol ist \gls{ND}.

Und auch ein Eintrag im Glossar muss sein: \gls{glos:AD}; das zweite Auftreten des Eintrags ist \gls{glos:AD}.

\newpage
Und auf der n\"achsten Seite: \gls{glos:AD}. Im Glossar ist jeweils angegeben, auf welchen Seiten der 
Begriff verwendet wurde.


\section{Abbildungen, Tabellen und Programmlistings\label{gleitobjekte}}

Ein Rechteck besitzt die in \vref{abb1} dargestellte Struktur.

\begin{figure}[htbp]
\centering
\setlength{\unitlength}{1mm}
\begin{picture}(100,30)
\put(0,0){\framebox(100,30){Ich bin kein Quadrat!}}
\end{picture}
\caption[Die Darstellung eines Rechtecks]{Die Darstellung eines Rechtecks\label{abb1}\footnotemark}
\end{figure}
\footnotetext{\seCite{Vgl.}{S. 400}{The:WA}. Achtung: Dieser Literaturverweis ist  rein fiktiver Natur, 
die Seite 400 existiert in \seCite{}{}{The:WA} nicht!}\label{fussnote}

Der optionale Parameter im folgenden \verb+\caption+-Kommando

\vspace*{-\baselineskip}
\begin{verbatim}
\caption[Die Darstellung eines Rechtecks]%
{Die Darstellung eines Rechtecks\label{abb1}\footnotemark}
\end{verbatim}
\vspace*{-\baselineskip}

definiert den Eintrag f\"ur das Abbildungsverzeichnis. Dort sollte die Fu{\ss}notennummer nicht auftauchen.
Nutzt man den optionalen Parameter nicht, ist es notwendig,  vor \verb+\footnotemark+ noch ein \verb+\protect+ 
einzuf\"ugen, da \LaTeX{} andernfalls die \"Ubersetzung mit einer Fehlermeldung abbricht. 

Eine Notentabelle kann wie in \vref{noten} dargestellt aussehen.

\begin{table}[htbp]%
\centering%
\begin{tabular}{| c | c |}
\hline
Matrikelnummer & Note \\
\hline
\hline
1234567 & 2,7 \\
\hline
2323456 & 3,5 \\
\hline
9865783 & 1,0 \\
\hline
\end{tabular} 
\caption{Ergebnisse der Klausur Programmierung I\label{noten}}
\end{table}


Eines der wichtigsten Java-Programme \"uberhaupt ist in \vref{hello} zu sehen.

\begin{programm}[htbp]
\begin{lstlisting}
public class HelloDHBW {
  public static void main ( String[] args ) {
    System.out.println ( "Hello DHBW" );
  } // main
} // HelloDHBW
\end{lstlisting}
\caption{Die Klasse \texttt{HelloDHBW}\label{hello}}
\end{programm}



\section{Die Definition und Anwendung von zwei neuen Listenumgebungen}

\subsection{Das Layout der Standardlistenumgebung von \LaTeX}

Stichpunktlisten werden in \LaTeX{} mit der \verb+itemize+-Umgebung erzeugt. 
Die Stichpunktliste 

\begin{itemize}
\item 1. Stichpunkt der ersten Ebene
\begin{itemize}
\item 1. Stichpunkt der zweiten Ebene
\item 2. Stichpunkt der zweiten Ebene
\begin{itemize}
\item 1. Stichpunkt der dritten Ebene
\item 2. Stichpunkt der dritten Ebene
\begin{itemize}
\item 1. Stichpunkt der vierten Ebene
\item 2. Stichpunkt der vierten Ebene
\end{itemize}
\end{itemize}
\end{itemize}
\item 2. Stichpunkt der ersten Ebene
\item 3. Stichpunkt der ersten Ebene
\end{itemize}

wird durch die folgenden Anweisungen erreicht:

\vspace*{-\baselineskip}

\begin{verbatim}
\begin{itemize}
\item 1. Stichpunkt der ersten Ebene
\begin{itemize}
\item 1. Stichpunkt der zweiten Ebene
\item 2. Stichpunkt der zweiten Ebene
\begin{itemize}
\item 1. Stichpunkt der dritten Ebene
\item 2. Stichpunkt der dritten Ebene
\begin{itemize}
\item 1. Stichpunkt der vierten Ebene
\item 2. Stichpunkt der vierten Ebene
\end{itemize}
\end{itemize}
\end{itemize}
\item 2. Stichpunkt der ersten Ebene
\item 3. Stichpunkt der ersten Ebene
\end{itemize}
\end{verbatim}

\subsection{Die neue Listenumgebung \texttt{seList} f\"ur Stichpunktlisten}

Weder die Einr\"uckung der einzelnen Ebenen noch die gro{\ss}en Abst\"ande zwischen den einzelnen Stichpunkten sind bei der \verb+itemize+-Umgebung 
bez\"uglich des Layouts sonderlich \"uberzeugend. 

Daher wird eine neue \verb+seList+-Umgebung zur Verf\"ugung gestellt. 

\begin{seList}
\item 1. Stichpunkt der ersten Ebene
\begin{seList}
\item 1. Stichpunkt der zweiten Ebene
\item 2. Stichpunkt der zweiten Ebene
\begin{seList}
\item 1. Stichpunkt der dritten Ebene
\item 2. Stichpunkt der dritten Ebene
\begin{seList}
\item 1. Stichpunkt der vierten Ebene
\item 2. Stichpunkt der vierten Ebene
\begin{seList}
\item 1. Stichpunkt der f\"unften Ebene
\item 2. Stichpunkt der f\"unften Ebene
\end{seList}
\end{seList}
\end{seList}
\end{seList}
\item 2. Stichpunkt der ersten Ebene
\item 3. Stichpunkt der ersten Ebene
\end{seList}

Der \LaTeX-Quelltext f\"ur diese Liste ist: 

\vspace*{-\baselineskip}
\begin{verbatim}
\begin{seList}
\item 1. Stichpunkt der ersten Ebene
\begin{seList}
\item 1. Stichpunkt der zweiten Ebene
\item 2. Stichpunkt der zweiten Ebene
\begin{seList}
\item 1. Stichpunkt der dritten Ebene
\item 2. Stichpunkt der dritten Ebene
\begin{seList}
\item 1. Stichpunkt der vierten Ebene
\item 2. Stichpunkt der vierten Ebene
\begin{seList}
\item 1. Stichpunkt der f\"unften Ebene
\item 2. Stichpunkt der f\"unften Ebene
\end{seList}
\end{seList}
\end{seList}
\end{seList}
\item 2. Stichpunkt der ersten Ebene
\item 3. Stichpunkt der ersten Ebene
\end{seList}
\end{verbatim}

\vspace*{-\baselineskip}
Neben der Eigenschaft, im Gegensatz zur \verb+itemize+-Umgebung f\"unf Verschachtelungsebenen angeben zu k\"onnen, ist es m\"oglich,
die Zeilenabst\"ande f\"ur die einzelnen Ebenen zu konfigurieren. 

Mit dem Kommando \newline 
\hspace*{\fill}\verb+\seSetlistbaselineskip{b1}{b2}{b3}{b4}{b5}+\hspace*{\fill}\newline 
kann f\"ur die Verschachtelungsebene $i$ der Grundlinienabstand \texttt{b$_{i}$} festgelegt 
werden. Als Einheit wird der Wert von \verb+\baselineskip+ (Grundlinienabstand des Dokuments) verwendet. Die folgenden Werte sind f\"ur ein Dokument voreingestellt:\newline
\hspace*{\fill}\verb+\seSetlistbaselineskip{1}{0.75}{0.75}{0.75}{0.75}+\hspace*{\fill}\newline\vspace*{-\baselineskip}

Mit dem Kommando \newline
\hspace*{\fill}\verb+\seResetlistbaselineskip{}+\hspace*{\fill}\newline
wird die letzte \"Anderung der Werte r\"uckg\"angig gemacht.

\newpage
\subsection{Die neue Listenumgebung \texttt{seToplist} f\"ur Listen mit einem Label und Aufz\"ahlungslisten}

Die neue Listenumgebung \verb+seToplist+ erlaubt es, jeden Stichpunkt mit einem Label zu versehen.
Die Liste\footnote{Die folgenden Werte sind frei erfunden.} 

\begin{seToplist}{Mercedes Benz:}
\item[Audi:] 400000 Gesamtverk\"aufe
\begin{seToplist}{3er Reihe:}
\item[A4:] 200000 Verk\"aufe
\item[A5:] 50000 Verk\"aufe
\item[A6:] 150000 Verk\"aufe
\end{seToplist}
\item[Mercedes Benz:] 500000 Gesamtverk\"aufe 
\item[BMW:] 650000 Gesamtverk\"aufe 
\begin{seToplist}{3er Reihe:}
\item[1er Reihe:] 100000 Verk\"aufe
\item[3er Reihe:] 300000 Verk\"aufe
\item[5er Reihe:] 250000 Verk\"aufe
\end{seToplist}
\end{seToplist}

wird durch die folgenden \LaTeX-Anweisungen erzeugt:

\vspace*{-\baselineskip}
\begin{verbatim}
\begin{seToplist}{Mercedes Benz:}
\item[Audi:] 400000 Gesamtverk\"aufe
\begin{seToplist}{3er Reihe:}
\item[A4:] 200000 Verk\"aufe
\item[A5:] 50000 Verk\"aufe
\item[A6:] 150000 Verk\"aufe
\end{seToplist}
\item[Mercedes Benz:] 500000 Gesamtverk\"aufe 
\item[BMW:] 650000 Gesamtverk\"aufe 
\begin{seToplist}{3er Reihe:}
\item[1er Reihe:] 100000 Verk\"aufe
\item[3er Reihe:] 300000 Verk\"aufe
\item[5er Reihe:] 250000 Verk\"aufe
\end{seToplist}
\end{seToplist}
\end{verbatim}

\vspace*{-\baselineskip}
Der Parameter \verb+par+ von \verb+\begin{seToplist}{par}+ definiert die Breite des Labels f\"ur die 
zugeh\"orige Liste.

F\"ur die \verb+seToplist+-Umgebung k\"onnen ebenfalls f\"unf Verschachtelungsebenen definiert werden. 
\"Uber die Kommandos \newline
\hspace*{\fill}\verb+\seSettoplistbaselineskip{b1}{b2}{b3}{b4}{b5}+\hspace*{\fill}\newline 
bzw. \newline
\hspace*{\fill}\verb+\seResettoplistbaselineskip{}+\hspace*{\fill}\newline
lassen sich analog zur \verb+seList+-Umgebung die Grundlinienabst\"ande der einzelnen Verschachtelungsebenen 
ver\"andern bzw. zur\"ucksetzen. Die folgenden Werte sind f\"ur ein Dokument voreingestellt:\newline
\hspace*{\fill}\verb+\seSettoplistbaselineskip{1}{0.75}{0.75}{0.75}{0.75}+\hspace*{\fill}\newline\vspace*{-\baselineskip}

Durch eine entsprechende Wahl der Labels k\"onnen Aufz\"ahlungslisten erzeugt werden:

\begin{seToplist}{a)}
\item[a)] Deutsche Automarken
\begin{seToplist}{1)}
\item[1)] Mercedes Benz
\item[2)] Audi 
\item[3)] VW
\item[4)] BMW 
\end{seToplist}
\item[b)] Japanische Automarken
\begin{seToplist}{1)}
\item[1)] Toyota
\item[2)] Honda
\item[3)] Mazda
\end{seToplist}
\end{seToplist}



\chapter{Hinweise zur Installation und \"Ubersetzung}

\section{Verwendung von TeXShop (Apple-Welt)}

Unter den ausgelieferten Dateien befinden sich zwei \textbf{engine}-Dateien: 

\begin{seList}
\item \verb+dhbw-projektarbeit.engine+
\item \verb+dhbw-projektarbeit-remove-all.engine+ (l\"oscht alle erzeugten \textsl{Hilfsdateien})
\end{seList}

Mit jeder dieser beiden Dateien kann man die Vorlage \texttt{\seVorlage.tex} 
\"ubersetzen. Alle Verzeichnisse (insbesondere Abk\"urzungs- und Symbolverzeichnis) 
sowie das Glossar werden (hoffentlich) korrekt erstellt.  

In den engine-Dateien ist beschrieben, an welcher Stelle sie im Mac OS X Dateisystem 
installiert werden m\"ussen, damit man sie direkt von TeXShop aus aufrufen kann. 

\section{Verwendung von MiKTeX (Windows-Welt)}

F\"ur die \"Ubersetzung wird eine batch-Datei \verb+make-projektarbeit.bat+ zur Verf\"ugung 
gestellt, mit der man in der Windows-\textsl{Eingabeaufforderung} (cmd) die Vorlage \"ubersetzen kann. Der Aufruf lautet:
\texttt{make-projektarbeit.bat \seVorlage}
%\verb+make-projektarbeit.bat se-pa2-vorlage+

Da MiKTeX eine andere Version von \verb+jurabib+ verwendet, mit der sich die 
Vorlage nicht korrekt \"ubersetzen l\"asst, werden die beiden Dateien 

\begin{seList}
\item \verb+jurabib.sty+ und 
\item \verb+jurabib.bst+
\end{seList}

aus der TeX Live Version von Mac OS X mitgeliefert. Damit sollte die 
\"Ubersetzung problemlos funktionieren. 


 

%Dann hoffe ich mal, dass sich mit den Vorlagen etwas anfangen l�sst. Sie sind (absichtlich) in 
%
%einer Version 0.9, da ich an den zugeh�rigen sty-Dateien weitere Erg�nzungen vornehmen werde,
%
%um f�r zuk�nftige Arbeiten neue komfortable Kommandos zur Verf�gung zu stellen. 

%
%  Erzeugung eines Glossars
%
% Achtung: Das Glossar wird nur ausgegeben, wenn mindestens ein Eintrag in der Arbeit 
%                definiert wurde
%
%
\newpage
\sePrintGlossary{}


%
% Literaturverzeichnisses
%
%\newpage
\sePrintBibliography{}

%  Erzeugung von Eintr\"agen im Literaturverzeichnis
%
%  Achtung: in einer Projektarbeit darf da \nocite-Kommando nicht verwendet werden,
%                 da es einen Eintrag im Literaturverzeichnis erzeugt, ohne dass eine 
%                 entsprechende Literaturreferenz im Text der Arbeit angegeben wird
%
%
%
\nocite{DHBW:SG}
\nocite{KM:KS}
\nocite{Dud06}
\nocite{Dud09}
\nocite{Bri:WA}
\nocite{RP:WA}
\nocite{Sch:WAS}
\nocite{BSS:WA}
\nocite{Kor:WA}
\nocite{MK:GWA}
\nocite{ADG:WA}
\nocite{The:WA}
\nocite{BA:WA}
\nocite{Dij:CRT}
\nocite{BC:Cur}
\nocite{Par:ECP}
\nocite{Bro:SBE}
\nocite{GI:ADI}
\nocite{GI:AZI}
\nocite{Den:CD}
\nocite{LMS:Icb}
\nocite{Fre:SIF}




%
% Festlegung des grundlegenden Formatierungsstils des Literaturverzeichnis
%
\bibliographystyle{jurabib}

% Eigentliche Ausgabe der in der Arbeit verwendeten Quellen
%
%
% Angabe der bib-Dateien, in denen die Quellen beschrieben sind;
% die Angabe geht davon aus, dass eine wa.bib-Datei in demselben 
% Verzeichnis liegt, wie se-pa1-vorlage.tex
%
\seBibliography{wa}


%
% Erzeugung der ehrenw\"ortlichen Erkl\"arung
%
% Der optionale Parameter kann verwendet werden, um f\"ur das Thema der Arbeit eine 
% andere Formatierung vorzunehmen; das sollte in der Regel nicht erforderlich sein;
% ausserdem besteht die Gefahr inkonsistenter Titel auf dem Titelblatt und in der 
% ehrenw\"ortlichen Erkl\"arung
%
%\seEhrenwoertlicheErklaerung{} % dieses Kommando sollte standardm\"assig verwendet werden
\seEhrenwoertlicheErklaerung[Star Greg\\Das Unternehmensplanspiel]{}


\end{document}












%
% Einlesen der .sty-Dateien
%
%  se-pa1-input-styles.tex
%
%  Joerg Baumgart 01.08.2011
%
%  Zusammenfassung und Konfiguration wichtiger Styles f\"ur die 
%  Erzeugung von Seminar-, Projekt- und Bachelorarbeiten
%
%
\documentclass[12pt,BCOR=10mm,headinclude=on,footinclude=off,bibliography=totoc]{scrreprt}
\usepackage[T1]{fontenc}
\usepackage[utf8x]{inputenc}
\usepackage[ngerman]{babel} % Deutsche Einstellungen
\usepackage{lmodern}

\usepackage{textcomp}
\usepackage{tikz} % Graphikpaket, das zu pdfLaTeX kompatibel ist
%\usetikzlibrary{intersections, arrows, calc, backgrounds, fit, decorations.pathreplacing}
\usepackage{amsmath}
\usepackage{amssymb}
\usepackage{xkeyval} % Definition von Kommandos mit mehreren optionalen Argumenten
\usepackage{listings} % Formatierung von Programmlistings
\usepackage{graphicx} % Einbinden von Graphiken
\usepackage{ifthen}
\usepackage{color}
\usepackage{slashbox} % Diagonalen in Tabellenfeldern
\usepackage{framed} % Erzeugung schwarzer Linien am linken Rand zur Hervorhebung von Textteilen
\usepackage{caption} % Korrektes Setzen einer mehrzeiligen float-Unterschrift bei neu definierten float-Umgebungen
\usepackage{floatrow}

% Es wird jeweils die sty-Datei importiert und entsprechende Konfigurationseinstellungen werden vorgenommen
%
\usepackage{se-jb-scrpage2} % Formatierung der Kopf- und Fu{\ss}zeilen
\usepackage{se-jb-footmisc}    % Fussnoten besser formatieren

\usepackage{se-jb-glossaries} % Abk\"urzungsverzeichnis, Symbolverzeichnis, Glossar
   
\usepackage{se-jb-floatrow}    % Definition und Konfiguration von float-Umgebungen (figure, table, die neue programm-Umgebung)
% Achtung: se-jb-varioref muss nach se-jb-floatrow importiert werden; 
% andernfalls ist der counter programm f\"ur die labelformat-Anweisung noch nicht definiert   
\usepackage{se-jb-varioref}   % Definition von Querverweisen
\usepackage{se-jb-chngcntr}   % Kapitelweise oder globale Nummerierung von Abbildungen etc.
   
\usepackage{se-jb-listen} % Definition neuer, besser formatierter Listen
\usepackage{se-jb-wa-kommandos} % neue Kommandos f\"ur Seminar-, Projekt- und Bachelorarbeiten


%
% Individuelle Konfiguration des Dokumentes
%
%  Individuelle Konfiguration einer Projektarbeit
%
%
%
%

%
% Literaturverzeichnis
% 
\usepackage{se-jb-jurabib-theisen} % Literaturverzeichnis gem\"ass den Vorgaben von Theisen aufbauen



% Weitere Optionseinstellungen f\"ur das Koma-Script
%
% Zwischen Abs\"atzen einen Abstand von 0.5 \baselineskip erzeugen
\KOMAoption{parskip}{full}
%
% Vergleiche Duden "Gliederung von Nummern, S.111" 
% DIN 5008 anschauen, wenn sie neu ver\"offentlicht wurde
\KOMAoption{numbers}{noendperiod}
%
%



%  Voreinstellungen f\"ur floats
%  Durch die verwendeten Parameter wird die Wahrscheinlichkeit deutlich kleiner, 
%  dass Gleitobjekte (z. B. Abbildungen) ans Ende des Dokumentes verschoben 
%  werden; 
%  Achtung: clearpage erzwingt die Ausgabe von Gleitobjekten
%
\renewcommand{\topfraction}{1}  % Gleitobjekte d\"urfen eine Seite zu 100% belegen 
\renewcommand{\bottomfraction}{1} % Entsprechender Wert f\"ur den unteren Teil der Seite
\renewcommand{\textfraction}{0} % Eine Seite darf auch ohne Fliesstext existieren
%%%\renewcommand{\floatpagefraction}{1} % Bedeutung unklar, daher keine Ver\"anderung des Vorgabewertes 
                                                                        % von 0.5; eventuell bringt ein \"Anderung auf 1 etwas, wenn 
                                                                         % Probleme mit floats auftreten
                                                                         
                                                                         
                                                                         
% Konfiguration von Programm-Listings
% 
% Achtung: hier gibt es nahezu beliebig viele weitere Konfigurationm\"oglichkeiten; vgl. Paketdokumentation
%
\lstset{language=Java,basicstyle=\ttfamily,keywordstyle=\color{blue},captionpos=b,aboveskip=0mm,belowskip=0mm,
          xleftmargin=0em}               
          
%
% Grundkonfiguration der Abs\"ande zwischen den Items der maximal f\"unf Verschachtelungsebenen der 
% neuen Listenumgebungen
%                                                                             
% Initialisierung der Abst\"ande zwischen den items f\"ur seList; Grundeinheit: 0.5\baselineskip; siehe se-jb-listen
\seSetlistbaselineskip{1}{0.75}{0.75}{0.75}{0.75}
% Initialisierung der Abst\"ande zwischen den items f\"ur seToplist; Grundeinheit: 0.5\baselineskip; siehe se-jb-listen
\seSettoplistbaselineskip{1}{0.75}{0.75}{0.75}{0.75}     


%
%  Konfiguration der verschiedenen Verzeichnisse
%
%  abstandEintrag: Wert wird mit \baselineskip multipliziert
%

%
%  Abbildungsverzeichnis
%
\seKonfigurationAbb[
%verzeichnisname=Abbildungsverzeichnis,
labeltextLinks=, % kein Text links;
%labeltextRechts=:,
labelbreite=1cm,
%labeleinzug=1cm,
%abstandEintrag=1,
%newpage=ja,
%pnumwidth=20mm,
%dotsep=1000,
%tocrmarg=4.5cm,
%abstandVerzeichnis=-1mm
]

%
% LIstingverzeichnis
%
\seKonfigurationPrg[
%verzeichnisname=Listing-Verzeichnis,
labeltextLinks=,
%labeltextRechts=:,
labelbreite=1cm,
%labeleinzug=2cm,
%abstandEintrag=1,
%newpage=ja,
%%pnumwidth=20mm,
%dotsep=1000,
%tocrmarg=4.5cm,
%abstandVerzeichnis=-10mm
]

%
% Tabellenverzeichnis
%
\seKonfigurationTab[
%verzeichnisname=Liste der Tabellen,
labeltextLinks=,
%labeltextRechts=:,
labelbreite=1cm,
%labeleinzug=0.5cm,
%abstandEintrag=1,
%newpage=ja,
%pnumwidth=20mm,
%dotsep=1000,
%tocrmarg=4.5cm,
%abstandVerzeichnis=-10mm
]

%
% Abk\"urzungsverzeichnis
%
\seKonfigurationAbk[
%verzeichnisname=Liste der Abk\"urzungen,
%labelbreite=3cm,
%labeleinzug=0.5cm,
%abstandEintrag=1,
%newpage=ja,
%abstandVerzeichnis=-10mm
]

%
% Symbolverzeichnis
% 
\seKonfigurationSym[
%verzeichnisname=Liste der Symbole,
%labelbreite=4cm,
%labeleinzug=3.5cm,
%abstandEintrag=1,
%newpage=ja,
%abstandVerzeichnis=-10mm
]

%
% Glossar
%
\seKonfigurationGlo[
%verzeichnisname=Glossar,
%abstandEintrag=0,
]



% (eventuelle) Neudefinition f\"ur die Unter-/\"Uberschriften von Abbildungen, Tabellen und Listings
%
%
%\renewcommand{\seCaptionNameAbbildung}{Abb.}
%\renewcommand{\seCaptionNameTabelle}{Tab.}
%\renewcommand{\seCaptionNameProgramm}{Prg.}


% % (eventuelle) Neudefinition f\"ur Querverweise innerhalb des Textes
%
%
%
%\renewcommand{\seQuerverweisSeite}{Seite}
%\renewcommand{\seQuerverweisAbbildung}{Abb.}
%\renewcommand{\seQuerverweisTabelle}{Tab.}
%\renewcommand{\seQuerverweisProgramm}{Prg.}
%\renewcommand{\seQuerverweisKapitel}{Kap.}
%\renewcommand{\seQuerverweisGleichung}{Gl.}

% Kommandos, die direkt nach \begin{document} ausgef\"uhrt werden m\"ussen
%
%
%
\AtBeginDocument{%
\renewcommand{\listfigurename}{\seAbbildungenVerzeichnisname}
\renewcommand{\listtablename}{\seTabellenVerzeichnisname}
\renewcommand{\figurename}{\seCaptionNameAbbildung}
\renewcommand{\tablename}{\seCaptionNameTabelle}
\pagenumbering{roman}
}
                                                              
                                                                         

%
% Individuelle Definition von Abk\"urzungen, Symbolen und eventuell Glossareintr\"agen
%
%  J\"org Baumgart
%  Definition einiger Abk\"urzungen
%  
%Befehle für Abk\"urzungen
\newacronym{dhbw}{DHBW}{Duale Hochschule Baden-W\"urttemberg}
\newacronym{usb}{USB}{Universal Serial Bus}

%Befehle für Symbole
%
% Achtung: ohne sort wird nach Name sortiert
\newglossaryentry{pi}{
name=$\pi$,
description={Die Kreiszahl},
type=symbolslist,
sort=a
}

\newglossaryentry{ND}{
name=$\mbox{\textsl{ND}}$,
description={Nutzungsdauer einer Maschine},
type=symbolslist,%
sort=b
}

%\newglossaryentry{Git}{
%name=$\mbox{\textsl{Git}}$,
%description={Git Versionenverwaltungssystem},
%type=symbolslist,%
%sort=g
%}



% Glossareintr\"age
\newglossaryentry{glos:AD}{
first=Active Directory\textsuperscript{GL},
name=Active Directory,
description={Active Directory ist in einem Windows 2000/Windows
Server 2003-Netzwerk der Verzeichnisdienst, der die zentrale
Organisation und Verwaltung aller Netzwerkressourcen erlaubt. Es
erm\"oglicht den Benutzern \"uber eine einzige zentrale Anmeldung den
Zugriff auf alle Ressourcen und den Administratoren die zentral
organisierte Verwaltung, transparent von der Netzwerktopologie und
den eingesetzten Netzwerkprotokollen. Das daf\"ur ben\"otigte
Betriebssystem ist entweder Windows 2000 Server oder
Windows Server 2003, welches auf dem zentralen
Dom\"anencontroller installiert wird. Dieser h\"alt alle Daten des
Active Directory vor, wie z.\,B. Benutzernamen und
Kennw\"orter.\seFootcite{Vgl.}{S. 200}{Dud09}}
}

\begin{document}

%Insert Image: file, label, caption
\newcommand{\img}[3]{
	\begin{figure}[h]
	\centering
	\fbox{\includegraphics[width=0.9\textwidth]{#1}}
	\caption[#3]
	\label{#2}
	\end{figure}
}

\newcommand{\curr}{\textcolonmonetary}
\newcommand{\autorbeginn}[1]{\textit{Beginn: #1} \def \autortmp {#1}}
\newcommand{\autorende}{\textit{Ende: \autortmp}}

% Erzeugung des Titelblatts
%
%
%
\seTitelblattSeminararbeit[
%hilfslinien=ja,
%dhbwlogoSkalierung=0.5,
%dhbwlogoDeltaX=2.4,
%dhbwlogoDeltaY=-10,
thema=Star Greg\\Das Unternehmensplanspiel,
verfasser={Britta Jochum\\Julia Lakatos\\Philipp Mail\\Jan Schlenker\\Marcel Steinleitner\\Fredrik Teschke},
firma=,
%verfasserin=,
matrikelnummer=,
kurs=WWI\,10\,SWM\,A,
%studiengangsleiterin=,
studiengangsleiter=Prof. Dr.-Ing. Jörg Baumgart,
modul=Umsetzung von Methoden der Wirtschaftsinformatik,
lehrveranstaltung=Fallstudie Systemanalyse,
%dozentin=,
dozent=Gregor Tielsch
]



% Erzeugung der Kurzfassung; Verfasser, Firma und Thema werden automatisch \"ubernommen
%
% Der optionale Parameter kann verwendet werden, um f\"ur das Thema der Arbeit eine 
% andere Formatierung vorzunehmen; das sollte in der Regel nicht erforderlich sein;
% ausserdem besteht die Gefahr inkonsistenter Titel auf dem Titelblatt und in der 
% Kurzfassung
%
%\seKurzfassung{} % dieses Kommando sollte standardm\"assig verwendet werden
\seKurzfassung[Star Greg - Das Unternehmensplanspiel]{}

% Beispiel f\"ur ein Kapitel, dass vor dem Einleitungskapitel kommt, z. B. ein Vorwort oder eine Danksagung


\seKapitelVorEinleitung{Vorwort}
\autorbeginn{Britta}
\\
\\
Wer hat nicht schon ein Mal den Wunsch verspürt, Vorstandschef eines Großunternehmens zu sein? 
Als Teilnehmer eines Unternehmensplanspiels, bei dem der Spieler mit  einer realitätsnahen, simulierten Welt interagiert, ist dieser Wunsch gar nicht so weit hergeholt. 
\\
Schon seit Jahrhunderten ist man auf der Suche, neue Lernmethoden für die Weiterbildung zu erforschen und auszubauen. Dabei stehen neben theoretischen Methoden auch immer mehr die praktischen im Vordergrund.  Eine spezielle Methode, um insbesondere komplexe Sachverhalte oder Systeme, wie beispielsweise die Funktionsweise eines Unternehmens, zu veranschaulichen, stellt hierbei das Planspiel dar. Man unterscheidet haptische (z.B. Brettspiele) und computergestützte Planspiele, die mittlerweile flächendeckend und in hoher Zahl am Markt erhältlich sind. Besonders populär ist das Unternehmensplanspiel TOPSIM der Firma \textit{TATA Interactive Systems}, das an ca. 1500 Hochschulen, Akademien und Unternehmen verwendet wird. 
\\
Die zunehmende Bedeutung von Planspielen als Lernmethode ist gewiss nicht nur auf deren Anwendung beschränkt. Eine weitere Lernkomponente ergibt sich auf Ebene der Anforderungsanalyse und Entwicklung eines computergestützten Planspiels, was insbesondere für Studierende der Wirtschaftsinformatik große Lernchancen mit sich bringt. Da zu ihrem Bereichsfeld sowohl die betriebswirtschaftlichen als auch software- und programmiertechnischen Anforderungen an Analyse, Entwurf und Implementierung einer derartigen Software gehören, wird diese Aufgabe gerne in den Lehrplan übernommen. 
\\
\\
\autorende{}


% Ausgabe des Inhaltsverzeichnisses
%
%
\seInhaltsverzeichnis[%
einrueckung=ja,
gliederungsebenen=3
]



% Ausgabe der verschiedenen Verzeichnisse
% abk: Abk\"urzungsverzeichnis
% sym: Symbolverzeichnis
% abb: Abbildungsverzeichnis
% tab: Tabellenverzeichnis
% prg: Listingverzeichnis
%
%
% Achtung: Abk\"urzungs- und Symbolverzeichnis werden nur ausgegeben, wenn mindest ein Symbol bzw. 
%                mindestens eine Abk\"urzung in der Arbeit verwendet wurden
%
%
% gliederungsebene:
% -- section: die Verzeichnisse werden einem Kapitel "Verzeichnisse" untergliedert
% -- chapter: die Verzeichnisse sind jeweils eigene Kapitel
% imInhaltsverzeichnis: ja/nein -- Sollen die Verzeichnisse im Inhaltsverzeichnis enthalten sein?
\seVerzeichnisse[gliederungsebene=section,imInhaltsverzeichnis=ja]{abk}{sym}{abb}{tab}{prg}






% Erstes eigentliches Kapitel der Arbeit; typischerweise das Einleitungskapitel;
% hier muss wieder auf die Nummierung mit arabischen Seitenzahlen umgestellt werden
%

 \chapter{Einleitung}
\pagenumbering{arabic}
\label{chp:einleitung}

\section{Finanzen}
\autorbeginn{Marcel}
\label{sec:ui-bank}

Eine zentrale Informationsquelle des Spielers zeigt die Benutzeroberfläche „Finanzen". Hier erhält der Spieler eine Übersicht über alle im Unternehmen durchgeführten Geldflüsse. Die Benutzeroberfläche ist in drei zentrale Elemente untergliedert. Es handelt sich um die Einnahmen und Ausgaben pro Spielrunde sowie einen kurzen Überblick über die gesamte Geldsituation in dem Unternehmen.
 
Die Einnahmen pro Spielrunde werden den Ausgaben pro Spielrunde gegenübergestellt und befinden sich im linken Teil der Benutzeroberfläche. Das Unternehmensplanspiel Star Greg ist darauf ausgerichtet, ausschließlich Raumschiffe zu produzieren. Deshalb besteht der Umsatz des Unternehmens lediglich aus den Preisen des jeweiligen Raumschiffs multipliziert mit der verkauften Anzahl. Daraus folgt, dass es insgesamt drei Einnahmequellen gibt. Es handelt sich um den Umsatz des X Wings, der Correllian Corvette und den Umsatz des Millenium Falken. Diese drei verschiedenen Umsätze werden in einer einfachen Tabelle dargestellt und aufsummiert. Die Gesamteinnahmen der Spielrunde werden dem Spieler in einem grün hinterlegten Textfeld unterhalb der drei Einnahmequellen angezeigt.
 
Neben den Einnahmen gibt es Ausgaben pro Spielrunde. Diese werden im rechten Teil der Benutzeroberfläche dargestellt. Die Ausgaben fallen in drei Bereichen des Unternehmens an. Es handelt sich um die Produktion, das Lager und das Personal. Die Ausgaben der Produktion setzen sich aus fünf Posten zusammen. Es sind Gesamtkosten für alle gekauften Triebwerke, Hitzeschilder, Rumpfbauteile und Sonderbauteile. Zusätzlich werden zu den Produktionskosten auch die durch den Ausschuss verursachten  Zusatzkosten hinzugerechnet. Unterhalb der Produktionskosten werden die Personalkosten aufgelistet. Diese setzen sich aus den Werbungskosten, laufenden Kosten (Wartungskosten), und den Kosten der Umbaumaßnahmen zusammen. Das dritte Element der Ausgaben pro Spielrunde sind die gesamten Lagerkosten. Sie setzen sich aus den belegten Lagerplatzeinheiten multipliziert mit den Kosten pro Lagerplatzeinheit zusammen.
 
Die gesamten Ausgaben pro Spielrunde werden in einem rot hinterlegten Textfeld dargestellt. Durch das grün hinterlegte Textfeld der Einnahmen pro Spielrunde und das rot hinterlegte Textfeld der Ausgaben pro Spielrunde soll der Spieler schon bei kurzer Betrachtung des Bildschirms einen Überblick über die Geldflüsse im Unternehmen bekommen.
  
Im unteren Teil der Benutzeroberfläche befindet sich ein Überblick über die gesamte Finanzsituation im Unternehmen. In der ersten Zeile der Tabelle wird der Kontostand zu beginn der Spielrunde angezeigt. Darunter befinden sich die Ein- und Ausgaben der Spielrunde sowie der Kontostand am Ende der Spielrunde(s. \ref{img:ui-bank}).

\begin{figure}[htb]
  \centering
  \fbox{
    \includegraphics[width=0.9\textwidth]{40_UI/30_Bank/Bank.jpg}
  }
  \caption{Finanzen}
  \label{img:ui-bank}
\end{figure}


\autorende{}
\section{Finanzen}
\autorbeginn{Marcel}
\label{sec:ui-bank}

Eine zentrale Informationsquelle des Spielers zeigt die Benutzeroberfläche „Finanzen". Hier erhält der Spieler eine Übersicht über alle im Unternehmen durchgeführten Geldflüsse. Die Benutzeroberfläche ist in drei zentrale Elemente untergliedert. Es handelt sich um die Einnahmen und Ausgaben pro Spielrunde sowie einen kurzen Überblick über die gesamte Geldsituation in dem Unternehmen.
 
Die Einnahmen pro Spielrunde werden den Ausgaben pro Spielrunde gegenübergestellt und befinden sich im linken Teil der Benutzeroberfläche. Das Unternehmensplanspiel Star Greg ist darauf ausgerichtet, ausschließlich Raumschiffe zu produzieren. Deshalb besteht der Umsatz des Unternehmens lediglich aus den Preisen des jeweiligen Raumschiffs multipliziert mit der verkauften Anzahl. Daraus folgt, dass es insgesamt drei Einnahmequellen gibt. Es handelt sich um den Umsatz des X Wings, der Correllian Corvette und den Umsatz des Millenium Falken. Diese drei verschiedenen Umsätze werden in einer einfachen Tabelle dargestellt und aufsummiert. Die Gesamteinnahmen der Spielrunde werden dem Spieler in einem grün hinterlegten Textfeld unterhalb der drei Einnahmequellen angezeigt.
 
Neben den Einnahmen gibt es Ausgaben pro Spielrunde. Diese werden im rechten Teil der Benutzeroberfläche dargestellt. Die Ausgaben fallen in drei Bereichen des Unternehmens an. Es handelt sich um die Produktion, das Lager und das Personal. Die Ausgaben der Produktion setzen sich aus fünf Posten zusammen. Es sind Gesamtkosten für alle gekauften Triebwerke, Hitzeschilder, Rumpfbauteile und Sonderbauteile. Zusätzlich werden zu den Produktionskosten auch die durch den Ausschuss verursachten  Zusatzkosten hinzugerechnet. Unterhalb der Produktionskosten werden die Personalkosten aufgelistet. Diese setzen sich aus den Werbungskosten, laufenden Kosten (Wartungskosten), und den Kosten der Umbaumaßnahmen zusammen. Das dritte Element der Ausgaben pro Spielrunde sind die gesamten Lagerkosten. Sie setzen sich aus den belegten Lagerplatzeinheiten multipliziert mit den Kosten pro Lagerplatzeinheit zusammen.
 
Die gesamten Ausgaben pro Spielrunde werden in einem rot hinterlegten Textfeld dargestellt. Durch das grün hinterlegte Textfeld der Einnahmen pro Spielrunde und das rot hinterlegte Textfeld der Ausgaben pro Spielrunde soll der Spieler schon bei kurzer Betrachtung des Bildschirms einen Überblick über die Geldflüsse im Unternehmen bekommen.
  
Im unteren Teil der Benutzeroberfläche befindet sich ein Überblick über die gesamte Finanzsituation im Unternehmen. In der ersten Zeile der Tabelle wird der Kontostand zu beginn der Spielrunde angezeigt. Darunter befinden sich die Ein- und Ausgaben der Spielrunde sowie der Kontostand am Ende der Spielrunde(s. \ref{img:ui-bank}).

\begin{figure}[htb]
  \centering
  \fbox{
    \includegraphics[width=0.9\textwidth]{40_UI/30_Bank/Bank.jpg}
  }
  \caption{Finanzen}
  \label{img:ui-bank}
\end{figure}


\autorende{}




\chapter{Einleitung}
\pagenumbering{arabic}
\label{chp:einleitung}

\section{Finanzen}
\autorbeginn{Marcel}
\label{sec:ui-bank}

Eine zentrale Informationsquelle des Spielers zeigt die Benutzeroberfläche „Finanzen". Hier erhält der Spieler eine Übersicht über alle im Unternehmen durchgeführten Geldflüsse. Die Benutzeroberfläche ist in drei zentrale Elemente untergliedert. Es handelt sich um die Einnahmen und Ausgaben pro Spielrunde sowie einen kurzen Überblick über die gesamte Geldsituation in dem Unternehmen.
 
Die Einnahmen pro Spielrunde werden den Ausgaben pro Spielrunde gegenübergestellt und befinden sich im linken Teil der Benutzeroberfläche. Das Unternehmensplanspiel Star Greg ist darauf ausgerichtet, ausschließlich Raumschiffe zu produzieren. Deshalb besteht der Umsatz des Unternehmens lediglich aus den Preisen des jeweiligen Raumschiffs multipliziert mit der verkauften Anzahl. Daraus folgt, dass es insgesamt drei Einnahmequellen gibt. Es handelt sich um den Umsatz des X Wings, der Correllian Corvette und den Umsatz des Millenium Falken. Diese drei verschiedenen Umsätze werden in einer einfachen Tabelle dargestellt und aufsummiert. Die Gesamteinnahmen der Spielrunde werden dem Spieler in einem grün hinterlegten Textfeld unterhalb der drei Einnahmequellen angezeigt.
 
Neben den Einnahmen gibt es Ausgaben pro Spielrunde. Diese werden im rechten Teil der Benutzeroberfläche dargestellt. Die Ausgaben fallen in drei Bereichen des Unternehmens an. Es handelt sich um die Produktion, das Lager und das Personal. Die Ausgaben der Produktion setzen sich aus fünf Posten zusammen. Es sind Gesamtkosten für alle gekauften Triebwerke, Hitzeschilder, Rumpfbauteile und Sonderbauteile. Zusätzlich werden zu den Produktionskosten auch die durch den Ausschuss verursachten  Zusatzkosten hinzugerechnet. Unterhalb der Produktionskosten werden die Personalkosten aufgelistet. Diese setzen sich aus den Werbungskosten, laufenden Kosten (Wartungskosten), und den Kosten der Umbaumaßnahmen zusammen. Das dritte Element der Ausgaben pro Spielrunde sind die gesamten Lagerkosten. Sie setzen sich aus den belegten Lagerplatzeinheiten multipliziert mit den Kosten pro Lagerplatzeinheit zusammen.
 
Die gesamten Ausgaben pro Spielrunde werden in einem rot hinterlegten Textfeld dargestellt. Durch das grün hinterlegte Textfeld der Einnahmen pro Spielrunde und das rot hinterlegte Textfeld der Ausgaben pro Spielrunde soll der Spieler schon bei kurzer Betrachtung des Bildschirms einen Überblick über die Geldflüsse im Unternehmen bekommen.
  
Im unteren Teil der Benutzeroberfläche befindet sich ein Überblick über die gesamte Finanzsituation im Unternehmen. In der ersten Zeile der Tabelle wird der Kontostand zu beginn der Spielrunde angezeigt. Darunter befinden sich die Ein- und Ausgaben der Spielrunde sowie der Kontostand am Ende der Spielrunde(s. \ref{img:ui-bank}).

\begin{figure}[htb]
  \centering
  \fbox{
    \includegraphics[width=0.9\textwidth]{40_UI/30_Bank/Bank.jpg}
  }
  \caption{Finanzen}
  \label{img:ui-bank}
\end{figure}


\autorende{}
\section{Finanzen}
\autorbeginn{Marcel}
\label{sec:ui-bank}

Eine zentrale Informationsquelle des Spielers zeigt die Benutzeroberfläche „Finanzen". Hier erhält der Spieler eine Übersicht über alle im Unternehmen durchgeführten Geldflüsse. Die Benutzeroberfläche ist in drei zentrale Elemente untergliedert. Es handelt sich um die Einnahmen und Ausgaben pro Spielrunde sowie einen kurzen Überblick über die gesamte Geldsituation in dem Unternehmen.
 
Die Einnahmen pro Spielrunde werden den Ausgaben pro Spielrunde gegenübergestellt und befinden sich im linken Teil der Benutzeroberfläche. Das Unternehmensplanspiel Star Greg ist darauf ausgerichtet, ausschließlich Raumschiffe zu produzieren. Deshalb besteht der Umsatz des Unternehmens lediglich aus den Preisen des jeweiligen Raumschiffs multipliziert mit der verkauften Anzahl. Daraus folgt, dass es insgesamt drei Einnahmequellen gibt. Es handelt sich um den Umsatz des X Wings, der Correllian Corvette und den Umsatz des Millenium Falken. Diese drei verschiedenen Umsätze werden in einer einfachen Tabelle dargestellt und aufsummiert. Die Gesamteinnahmen der Spielrunde werden dem Spieler in einem grün hinterlegten Textfeld unterhalb der drei Einnahmequellen angezeigt.
 
Neben den Einnahmen gibt es Ausgaben pro Spielrunde. Diese werden im rechten Teil der Benutzeroberfläche dargestellt. Die Ausgaben fallen in drei Bereichen des Unternehmens an. Es handelt sich um die Produktion, das Lager und das Personal. Die Ausgaben der Produktion setzen sich aus fünf Posten zusammen. Es sind Gesamtkosten für alle gekauften Triebwerke, Hitzeschilder, Rumpfbauteile und Sonderbauteile. Zusätzlich werden zu den Produktionskosten auch die durch den Ausschuss verursachten  Zusatzkosten hinzugerechnet. Unterhalb der Produktionskosten werden die Personalkosten aufgelistet. Diese setzen sich aus den Werbungskosten, laufenden Kosten (Wartungskosten), und den Kosten der Umbaumaßnahmen zusammen. Das dritte Element der Ausgaben pro Spielrunde sind die gesamten Lagerkosten. Sie setzen sich aus den belegten Lagerplatzeinheiten multipliziert mit den Kosten pro Lagerplatzeinheit zusammen.
 
Die gesamten Ausgaben pro Spielrunde werden in einem rot hinterlegten Textfeld dargestellt. Durch das grün hinterlegte Textfeld der Einnahmen pro Spielrunde und das rot hinterlegte Textfeld der Ausgaben pro Spielrunde soll der Spieler schon bei kurzer Betrachtung des Bildschirms einen Überblick über die Geldflüsse im Unternehmen bekommen.
  
Im unteren Teil der Benutzeroberfläche befindet sich ein Überblick über die gesamte Finanzsituation im Unternehmen. In der ersten Zeile der Tabelle wird der Kontostand zu beginn der Spielrunde angezeigt. Darunter befinden sich die Ein- und Ausgaben der Spielrunde sowie der Kontostand am Ende der Spielrunde(s. \ref{img:ui-bank}).

\begin{figure}[htb]
  \centering
  \fbox{
    \includegraphics[width=0.9\textwidth]{40_UI/30_Bank/Bank.jpg}
  }
  \caption{Finanzen}
  \label{img:ui-bank}
\end{figure}


\autorende{}




\chapter{Einleitung}
\pagenumbering{arabic}
\label{chp:einleitung}

\section{Finanzen}
\autorbeginn{Marcel}
\label{sec:ui-bank}

Eine zentrale Informationsquelle des Spielers zeigt die Benutzeroberfläche „Finanzen". Hier erhält der Spieler eine Übersicht über alle im Unternehmen durchgeführten Geldflüsse. Die Benutzeroberfläche ist in drei zentrale Elemente untergliedert. Es handelt sich um die Einnahmen und Ausgaben pro Spielrunde sowie einen kurzen Überblick über die gesamte Geldsituation in dem Unternehmen.
 
Die Einnahmen pro Spielrunde werden den Ausgaben pro Spielrunde gegenübergestellt und befinden sich im linken Teil der Benutzeroberfläche. Das Unternehmensplanspiel Star Greg ist darauf ausgerichtet, ausschließlich Raumschiffe zu produzieren. Deshalb besteht der Umsatz des Unternehmens lediglich aus den Preisen des jeweiligen Raumschiffs multipliziert mit der verkauften Anzahl. Daraus folgt, dass es insgesamt drei Einnahmequellen gibt. Es handelt sich um den Umsatz des X Wings, der Correllian Corvette und den Umsatz des Millenium Falken. Diese drei verschiedenen Umsätze werden in einer einfachen Tabelle dargestellt und aufsummiert. Die Gesamteinnahmen der Spielrunde werden dem Spieler in einem grün hinterlegten Textfeld unterhalb der drei Einnahmequellen angezeigt.
 
Neben den Einnahmen gibt es Ausgaben pro Spielrunde. Diese werden im rechten Teil der Benutzeroberfläche dargestellt. Die Ausgaben fallen in drei Bereichen des Unternehmens an. Es handelt sich um die Produktion, das Lager und das Personal. Die Ausgaben der Produktion setzen sich aus fünf Posten zusammen. Es sind Gesamtkosten für alle gekauften Triebwerke, Hitzeschilder, Rumpfbauteile und Sonderbauteile. Zusätzlich werden zu den Produktionskosten auch die durch den Ausschuss verursachten  Zusatzkosten hinzugerechnet. Unterhalb der Produktionskosten werden die Personalkosten aufgelistet. Diese setzen sich aus den Werbungskosten, laufenden Kosten (Wartungskosten), und den Kosten der Umbaumaßnahmen zusammen. Das dritte Element der Ausgaben pro Spielrunde sind die gesamten Lagerkosten. Sie setzen sich aus den belegten Lagerplatzeinheiten multipliziert mit den Kosten pro Lagerplatzeinheit zusammen.
 
Die gesamten Ausgaben pro Spielrunde werden in einem rot hinterlegten Textfeld dargestellt. Durch das grün hinterlegte Textfeld der Einnahmen pro Spielrunde und das rot hinterlegte Textfeld der Ausgaben pro Spielrunde soll der Spieler schon bei kurzer Betrachtung des Bildschirms einen Überblick über die Geldflüsse im Unternehmen bekommen.
  
Im unteren Teil der Benutzeroberfläche befindet sich ein Überblick über die gesamte Finanzsituation im Unternehmen. In der ersten Zeile der Tabelle wird der Kontostand zu beginn der Spielrunde angezeigt. Darunter befinden sich die Ein- und Ausgaben der Spielrunde sowie der Kontostand am Ende der Spielrunde(s. \ref{img:ui-bank}).

\begin{figure}[htb]
  \centering
  \fbox{
    \includegraphics[width=0.9\textwidth]{40_UI/30_Bank/Bank.jpg}
  }
  \caption{Finanzen}
  \label{img:ui-bank}
\end{figure}


\autorende{}
\section{Finanzen}
\autorbeginn{Marcel}
\label{sec:ui-bank}

Eine zentrale Informationsquelle des Spielers zeigt die Benutzeroberfläche „Finanzen". Hier erhält der Spieler eine Übersicht über alle im Unternehmen durchgeführten Geldflüsse. Die Benutzeroberfläche ist in drei zentrale Elemente untergliedert. Es handelt sich um die Einnahmen und Ausgaben pro Spielrunde sowie einen kurzen Überblick über die gesamte Geldsituation in dem Unternehmen.
 
Die Einnahmen pro Spielrunde werden den Ausgaben pro Spielrunde gegenübergestellt und befinden sich im linken Teil der Benutzeroberfläche. Das Unternehmensplanspiel Star Greg ist darauf ausgerichtet, ausschließlich Raumschiffe zu produzieren. Deshalb besteht der Umsatz des Unternehmens lediglich aus den Preisen des jeweiligen Raumschiffs multipliziert mit der verkauften Anzahl. Daraus folgt, dass es insgesamt drei Einnahmequellen gibt. Es handelt sich um den Umsatz des X Wings, der Correllian Corvette und den Umsatz des Millenium Falken. Diese drei verschiedenen Umsätze werden in einer einfachen Tabelle dargestellt und aufsummiert. Die Gesamteinnahmen der Spielrunde werden dem Spieler in einem grün hinterlegten Textfeld unterhalb der drei Einnahmequellen angezeigt.
 
Neben den Einnahmen gibt es Ausgaben pro Spielrunde. Diese werden im rechten Teil der Benutzeroberfläche dargestellt. Die Ausgaben fallen in drei Bereichen des Unternehmens an. Es handelt sich um die Produktion, das Lager und das Personal. Die Ausgaben der Produktion setzen sich aus fünf Posten zusammen. Es sind Gesamtkosten für alle gekauften Triebwerke, Hitzeschilder, Rumpfbauteile und Sonderbauteile. Zusätzlich werden zu den Produktionskosten auch die durch den Ausschuss verursachten  Zusatzkosten hinzugerechnet. Unterhalb der Produktionskosten werden die Personalkosten aufgelistet. Diese setzen sich aus den Werbungskosten, laufenden Kosten (Wartungskosten), und den Kosten der Umbaumaßnahmen zusammen. Das dritte Element der Ausgaben pro Spielrunde sind die gesamten Lagerkosten. Sie setzen sich aus den belegten Lagerplatzeinheiten multipliziert mit den Kosten pro Lagerplatzeinheit zusammen.
 
Die gesamten Ausgaben pro Spielrunde werden in einem rot hinterlegten Textfeld dargestellt. Durch das grün hinterlegte Textfeld der Einnahmen pro Spielrunde und das rot hinterlegte Textfeld der Ausgaben pro Spielrunde soll der Spieler schon bei kurzer Betrachtung des Bildschirms einen Überblick über die Geldflüsse im Unternehmen bekommen.
  
Im unteren Teil der Benutzeroberfläche befindet sich ein Überblick über die gesamte Finanzsituation im Unternehmen. In der ersten Zeile der Tabelle wird der Kontostand zu beginn der Spielrunde angezeigt. Darunter befinden sich die Ein- und Ausgaben der Spielrunde sowie der Kontostand am Ende der Spielrunde(s. \ref{img:ui-bank}).

\begin{figure}[htb]
  \centering
  \fbox{
    \includegraphics[width=0.9\textwidth]{40_UI/30_Bank/Bank.jpg}
  }
  \caption{Finanzen}
  \label{img:ui-bank}
\end{figure}


\autorende{}




\chapter{Einleitung}
\pagenumbering{arabic}
\label{chp:einleitung}

\section{Finanzen}
\autorbeginn{Marcel}
\label{sec:ui-bank}

Eine zentrale Informationsquelle des Spielers zeigt die Benutzeroberfläche „Finanzen". Hier erhält der Spieler eine Übersicht über alle im Unternehmen durchgeführten Geldflüsse. Die Benutzeroberfläche ist in drei zentrale Elemente untergliedert. Es handelt sich um die Einnahmen und Ausgaben pro Spielrunde sowie einen kurzen Überblick über die gesamte Geldsituation in dem Unternehmen.
 
Die Einnahmen pro Spielrunde werden den Ausgaben pro Spielrunde gegenübergestellt und befinden sich im linken Teil der Benutzeroberfläche. Das Unternehmensplanspiel Star Greg ist darauf ausgerichtet, ausschließlich Raumschiffe zu produzieren. Deshalb besteht der Umsatz des Unternehmens lediglich aus den Preisen des jeweiligen Raumschiffs multipliziert mit der verkauften Anzahl. Daraus folgt, dass es insgesamt drei Einnahmequellen gibt. Es handelt sich um den Umsatz des X Wings, der Correllian Corvette und den Umsatz des Millenium Falken. Diese drei verschiedenen Umsätze werden in einer einfachen Tabelle dargestellt und aufsummiert. Die Gesamteinnahmen der Spielrunde werden dem Spieler in einem grün hinterlegten Textfeld unterhalb der drei Einnahmequellen angezeigt.
 
Neben den Einnahmen gibt es Ausgaben pro Spielrunde. Diese werden im rechten Teil der Benutzeroberfläche dargestellt. Die Ausgaben fallen in drei Bereichen des Unternehmens an. Es handelt sich um die Produktion, das Lager und das Personal. Die Ausgaben der Produktion setzen sich aus fünf Posten zusammen. Es sind Gesamtkosten für alle gekauften Triebwerke, Hitzeschilder, Rumpfbauteile und Sonderbauteile. Zusätzlich werden zu den Produktionskosten auch die durch den Ausschuss verursachten  Zusatzkosten hinzugerechnet. Unterhalb der Produktionskosten werden die Personalkosten aufgelistet. Diese setzen sich aus den Werbungskosten, laufenden Kosten (Wartungskosten), und den Kosten der Umbaumaßnahmen zusammen. Das dritte Element der Ausgaben pro Spielrunde sind die gesamten Lagerkosten. Sie setzen sich aus den belegten Lagerplatzeinheiten multipliziert mit den Kosten pro Lagerplatzeinheit zusammen.
 
Die gesamten Ausgaben pro Spielrunde werden in einem rot hinterlegten Textfeld dargestellt. Durch das grün hinterlegte Textfeld der Einnahmen pro Spielrunde und das rot hinterlegte Textfeld der Ausgaben pro Spielrunde soll der Spieler schon bei kurzer Betrachtung des Bildschirms einen Überblick über die Geldflüsse im Unternehmen bekommen.
  
Im unteren Teil der Benutzeroberfläche befindet sich ein Überblick über die gesamte Finanzsituation im Unternehmen. In der ersten Zeile der Tabelle wird der Kontostand zu beginn der Spielrunde angezeigt. Darunter befinden sich die Ein- und Ausgaben der Spielrunde sowie der Kontostand am Ende der Spielrunde(s. \ref{img:ui-bank}).

\begin{figure}[htb]
  \centering
  \fbox{
    \includegraphics[width=0.9\textwidth]{40_UI/30_Bank/Bank.jpg}
  }
  \caption{Finanzen}
  \label{img:ui-bank}
\end{figure}


\autorende{}
\section{Finanzen}
\autorbeginn{Marcel}
\label{sec:ui-bank}

Eine zentrale Informationsquelle des Spielers zeigt die Benutzeroberfläche „Finanzen". Hier erhält der Spieler eine Übersicht über alle im Unternehmen durchgeführten Geldflüsse. Die Benutzeroberfläche ist in drei zentrale Elemente untergliedert. Es handelt sich um die Einnahmen und Ausgaben pro Spielrunde sowie einen kurzen Überblick über die gesamte Geldsituation in dem Unternehmen.
 
Die Einnahmen pro Spielrunde werden den Ausgaben pro Spielrunde gegenübergestellt und befinden sich im linken Teil der Benutzeroberfläche. Das Unternehmensplanspiel Star Greg ist darauf ausgerichtet, ausschließlich Raumschiffe zu produzieren. Deshalb besteht der Umsatz des Unternehmens lediglich aus den Preisen des jeweiligen Raumschiffs multipliziert mit der verkauften Anzahl. Daraus folgt, dass es insgesamt drei Einnahmequellen gibt. Es handelt sich um den Umsatz des X Wings, der Correllian Corvette und den Umsatz des Millenium Falken. Diese drei verschiedenen Umsätze werden in einer einfachen Tabelle dargestellt und aufsummiert. Die Gesamteinnahmen der Spielrunde werden dem Spieler in einem grün hinterlegten Textfeld unterhalb der drei Einnahmequellen angezeigt.
 
Neben den Einnahmen gibt es Ausgaben pro Spielrunde. Diese werden im rechten Teil der Benutzeroberfläche dargestellt. Die Ausgaben fallen in drei Bereichen des Unternehmens an. Es handelt sich um die Produktion, das Lager und das Personal. Die Ausgaben der Produktion setzen sich aus fünf Posten zusammen. Es sind Gesamtkosten für alle gekauften Triebwerke, Hitzeschilder, Rumpfbauteile und Sonderbauteile. Zusätzlich werden zu den Produktionskosten auch die durch den Ausschuss verursachten  Zusatzkosten hinzugerechnet. Unterhalb der Produktionskosten werden die Personalkosten aufgelistet. Diese setzen sich aus den Werbungskosten, laufenden Kosten (Wartungskosten), und den Kosten der Umbaumaßnahmen zusammen. Das dritte Element der Ausgaben pro Spielrunde sind die gesamten Lagerkosten. Sie setzen sich aus den belegten Lagerplatzeinheiten multipliziert mit den Kosten pro Lagerplatzeinheit zusammen.
 
Die gesamten Ausgaben pro Spielrunde werden in einem rot hinterlegten Textfeld dargestellt. Durch das grün hinterlegte Textfeld der Einnahmen pro Spielrunde und das rot hinterlegte Textfeld der Ausgaben pro Spielrunde soll der Spieler schon bei kurzer Betrachtung des Bildschirms einen Überblick über die Geldflüsse im Unternehmen bekommen.
  
Im unteren Teil der Benutzeroberfläche befindet sich ein Überblick über die gesamte Finanzsituation im Unternehmen. In der ersten Zeile der Tabelle wird der Kontostand zu beginn der Spielrunde angezeigt. Darunter befinden sich die Ein- und Ausgaben der Spielrunde sowie der Kontostand am Ende der Spielrunde(s. \ref{img:ui-bank}).

\begin{figure}[htb]
  \centering
  \fbox{
    \includegraphics[width=0.9\textwidth]{40_UI/30_Bank/Bank.jpg}
  }
  \caption{Finanzen}
  \label{img:ui-bank}
\end{figure}


\autorende{}




\chapter{Einleitung}
\pagenumbering{arabic}
\label{chp:einleitung}

\section{Finanzen}
\autorbeginn{Marcel}
\label{sec:ui-bank}

Eine zentrale Informationsquelle des Spielers zeigt die Benutzeroberfläche „Finanzen". Hier erhält der Spieler eine Übersicht über alle im Unternehmen durchgeführten Geldflüsse. Die Benutzeroberfläche ist in drei zentrale Elemente untergliedert. Es handelt sich um die Einnahmen und Ausgaben pro Spielrunde sowie einen kurzen Überblick über die gesamte Geldsituation in dem Unternehmen.
 
Die Einnahmen pro Spielrunde werden den Ausgaben pro Spielrunde gegenübergestellt und befinden sich im linken Teil der Benutzeroberfläche. Das Unternehmensplanspiel Star Greg ist darauf ausgerichtet, ausschließlich Raumschiffe zu produzieren. Deshalb besteht der Umsatz des Unternehmens lediglich aus den Preisen des jeweiligen Raumschiffs multipliziert mit der verkauften Anzahl. Daraus folgt, dass es insgesamt drei Einnahmequellen gibt. Es handelt sich um den Umsatz des X Wings, der Correllian Corvette und den Umsatz des Millenium Falken. Diese drei verschiedenen Umsätze werden in einer einfachen Tabelle dargestellt und aufsummiert. Die Gesamteinnahmen der Spielrunde werden dem Spieler in einem grün hinterlegten Textfeld unterhalb der drei Einnahmequellen angezeigt.
 
Neben den Einnahmen gibt es Ausgaben pro Spielrunde. Diese werden im rechten Teil der Benutzeroberfläche dargestellt. Die Ausgaben fallen in drei Bereichen des Unternehmens an. Es handelt sich um die Produktion, das Lager und das Personal. Die Ausgaben der Produktion setzen sich aus fünf Posten zusammen. Es sind Gesamtkosten für alle gekauften Triebwerke, Hitzeschilder, Rumpfbauteile und Sonderbauteile. Zusätzlich werden zu den Produktionskosten auch die durch den Ausschuss verursachten  Zusatzkosten hinzugerechnet. Unterhalb der Produktionskosten werden die Personalkosten aufgelistet. Diese setzen sich aus den Werbungskosten, laufenden Kosten (Wartungskosten), und den Kosten der Umbaumaßnahmen zusammen. Das dritte Element der Ausgaben pro Spielrunde sind die gesamten Lagerkosten. Sie setzen sich aus den belegten Lagerplatzeinheiten multipliziert mit den Kosten pro Lagerplatzeinheit zusammen.
 
Die gesamten Ausgaben pro Spielrunde werden in einem rot hinterlegten Textfeld dargestellt. Durch das grün hinterlegte Textfeld der Einnahmen pro Spielrunde und das rot hinterlegte Textfeld der Ausgaben pro Spielrunde soll der Spieler schon bei kurzer Betrachtung des Bildschirms einen Überblick über die Geldflüsse im Unternehmen bekommen.
  
Im unteren Teil der Benutzeroberfläche befindet sich ein Überblick über die gesamte Finanzsituation im Unternehmen. In der ersten Zeile der Tabelle wird der Kontostand zu beginn der Spielrunde angezeigt. Darunter befinden sich die Ein- und Ausgaben der Spielrunde sowie der Kontostand am Ende der Spielrunde(s. \ref{img:ui-bank}).

\begin{figure}[htb]
  \centering
  \fbox{
    \includegraphics[width=0.9\textwidth]{40_UI/30_Bank/Bank.jpg}
  }
  \caption{Finanzen}
  \label{img:ui-bank}
\end{figure}


\autorende{}
\section{Finanzen}
\autorbeginn{Marcel}
\label{sec:ui-bank}

Eine zentrale Informationsquelle des Spielers zeigt die Benutzeroberfläche „Finanzen". Hier erhält der Spieler eine Übersicht über alle im Unternehmen durchgeführten Geldflüsse. Die Benutzeroberfläche ist in drei zentrale Elemente untergliedert. Es handelt sich um die Einnahmen und Ausgaben pro Spielrunde sowie einen kurzen Überblick über die gesamte Geldsituation in dem Unternehmen.
 
Die Einnahmen pro Spielrunde werden den Ausgaben pro Spielrunde gegenübergestellt und befinden sich im linken Teil der Benutzeroberfläche. Das Unternehmensplanspiel Star Greg ist darauf ausgerichtet, ausschließlich Raumschiffe zu produzieren. Deshalb besteht der Umsatz des Unternehmens lediglich aus den Preisen des jeweiligen Raumschiffs multipliziert mit der verkauften Anzahl. Daraus folgt, dass es insgesamt drei Einnahmequellen gibt. Es handelt sich um den Umsatz des X Wings, der Correllian Corvette und den Umsatz des Millenium Falken. Diese drei verschiedenen Umsätze werden in einer einfachen Tabelle dargestellt und aufsummiert. Die Gesamteinnahmen der Spielrunde werden dem Spieler in einem grün hinterlegten Textfeld unterhalb der drei Einnahmequellen angezeigt.
 
Neben den Einnahmen gibt es Ausgaben pro Spielrunde. Diese werden im rechten Teil der Benutzeroberfläche dargestellt. Die Ausgaben fallen in drei Bereichen des Unternehmens an. Es handelt sich um die Produktion, das Lager und das Personal. Die Ausgaben der Produktion setzen sich aus fünf Posten zusammen. Es sind Gesamtkosten für alle gekauften Triebwerke, Hitzeschilder, Rumpfbauteile und Sonderbauteile. Zusätzlich werden zu den Produktionskosten auch die durch den Ausschuss verursachten  Zusatzkosten hinzugerechnet. Unterhalb der Produktionskosten werden die Personalkosten aufgelistet. Diese setzen sich aus den Werbungskosten, laufenden Kosten (Wartungskosten), und den Kosten der Umbaumaßnahmen zusammen. Das dritte Element der Ausgaben pro Spielrunde sind die gesamten Lagerkosten. Sie setzen sich aus den belegten Lagerplatzeinheiten multipliziert mit den Kosten pro Lagerplatzeinheit zusammen.
 
Die gesamten Ausgaben pro Spielrunde werden in einem rot hinterlegten Textfeld dargestellt. Durch das grün hinterlegte Textfeld der Einnahmen pro Spielrunde und das rot hinterlegte Textfeld der Ausgaben pro Spielrunde soll der Spieler schon bei kurzer Betrachtung des Bildschirms einen Überblick über die Geldflüsse im Unternehmen bekommen.
  
Im unteren Teil der Benutzeroberfläche befindet sich ein Überblick über die gesamte Finanzsituation im Unternehmen. In der ersten Zeile der Tabelle wird der Kontostand zu beginn der Spielrunde angezeigt. Darunter befinden sich die Ein- und Ausgaben der Spielrunde sowie der Kontostand am Ende der Spielrunde(s. \ref{img:ui-bank}).

\begin{figure}[htb]
  \centering
  \fbox{
    \includegraphics[width=0.9\textwidth]{40_UI/30_Bank/Bank.jpg}
  }
  \caption{Finanzen}
  \label{img:ui-bank}
\end{figure}


\autorende{}




\chapter{Einleitung}
\pagenumbering{arabic}
\label{chp:einleitung}

\section{Finanzen}
\autorbeginn{Marcel}
\label{sec:ui-bank}

Eine zentrale Informationsquelle des Spielers zeigt die Benutzeroberfläche „Finanzen". Hier erhält der Spieler eine Übersicht über alle im Unternehmen durchgeführten Geldflüsse. Die Benutzeroberfläche ist in drei zentrale Elemente untergliedert. Es handelt sich um die Einnahmen und Ausgaben pro Spielrunde sowie einen kurzen Überblick über die gesamte Geldsituation in dem Unternehmen.
 
Die Einnahmen pro Spielrunde werden den Ausgaben pro Spielrunde gegenübergestellt und befinden sich im linken Teil der Benutzeroberfläche. Das Unternehmensplanspiel Star Greg ist darauf ausgerichtet, ausschließlich Raumschiffe zu produzieren. Deshalb besteht der Umsatz des Unternehmens lediglich aus den Preisen des jeweiligen Raumschiffs multipliziert mit der verkauften Anzahl. Daraus folgt, dass es insgesamt drei Einnahmequellen gibt. Es handelt sich um den Umsatz des X Wings, der Correllian Corvette und den Umsatz des Millenium Falken. Diese drei verschiedenen Umsätze werden in einer einfachen Tabelle dargestellt und aufsummiert. Die Gesamteinnahmen der Spielrunde werden dem Spieler in einem grün hinterlegten Textfeld unterhalb der drei Einnahmequellen angezeigt.
 
Neben den Einnahmen gibt es Ausgaben pro Spielrunde. Diese werden im rechten Teil der Benutzeroberfläche dargestellt. Die Ausgaben fallen in drei Bereichen des Unternehmens an. Es handelt sich um die Produktion, das Lager und das Personal. Die Ausgaben der Produktion setzen sich aus fünf Posten zusammen. Es sind Gesamtkosten für alle gekauften Triebwerke, Hitzeschilder, Rumpfbauteile und Sonderbauteile. Zusätzlich werden zu den Produktionskosten auch die durch den Ausschuss verursachten  Zusatzkosten hinzugerechnet. Unterhalb der Produktionskosten werden die Personalkosten aufgelistet. Diese setzen sich aus den Werbungskosten, laufenden Kosten (Wartungskosten), und den Kosten der Umbaumaßnahmen zusammen. Das dritte Element der Ausgaben pro Spielrunde sind die gesamten Lagerkosten. Sie setzen sich aus den belegten Lagerplatzeinheiten multipliziert mit den Kosten pro Lagerplatzeinheit zusammen.
 
Die gesamten Ausgaben pro Spielrunde werden in einem rot hinterlegten Textfeld dargestellt. Durch das grün hinterlegte Textfeld der Einnahmen pro Spielrunde und das rot hinterlegte Textfeld der Ausgaben pro Spielrunde soll der Spieler schon bei kurzer Betrachtung des Bildschirms einen Überblick über die Geldflüsse im Unternehmen bekommen.
  
Im unteren Teil der Benutzeroberfläche befindet sich ein Überblick über die gesamte Finanzsituation im Unternehmen. In der ersten Zeile der Tabelle wird der Kontostand zu beginn der Spielrunde angezeigt. Darunter befinden sich die Ein- und Ausgaben der Spielrunde sowie der Kontostand am Ende der Spielrunde(s. \ref{img:ui-bank}).

\begin{figure}[htb]
  \centering
  \fbox{
    \includegraphics[width=0.9\textwidth]{40_UI/30_Bank/Bank.jpg}
  }
  \caption{Finanzen}
  \label{img:ui-bank}
\end{figure}


\autorende{}
\section{Finanzen}
\autorbeginn{Marcel}
\label{sec:ui-bank}

Eine zentrale Informationsquelle des Spielers zeigt die Benutzeroberfläche „Finanzen". Hier erhält der Spieler eine Übersicht über alle im Unternehmen durchgeführten Geldflüsse. Die Benutzeroberfläche ist in drei zentrale Elemente untergliedert. Es handelt sich um die Einnahmen und Ausgaben pro Spielrunde sowie einen kurzen Überblick über die gesamte Geldsituation in dem Unternehmen.
 
Die Einnahmen pro Spielrunde werden den Ausgaben pro Spielrunde gegenübergestellt und befinden sich im linken Teil der Benutzeroberfläche. Das Unternehmensplanspiel Star Greg ist darauf ausgerichtet, ausschließlich Raumschiffe zu produzieren. Deshalb besteht der Umsatz des Unternehmens lediglich aus den Preisen des jeweiligen Raumschiffs multipliziert mit der verkauften Anzahl. Daraus folgt, dass es insgesamt drei Einnahmequellen gibt. Es handelt sich um den Umsatz des X Wings, der Correllian Corvette und den Umsatz des Millenium Falken. Diese drei verschiedenen Umsätze werden in einer einfachen Tabelle dargestellt und aufsummiert. Die Gesamteinnahmen der Spielrunde werden dem Spieler in einem grün hinterlegten Textfeld unterhalb der drei Einnahmequellen angezeigt.
 
Neben den Einnahmen gibt es Ausgaben pro Spielrunde. Diese werden im rechten Teil der Benutzeroberfläche dargestellt. Die Ausgaben fallen in drei Bereichen des Unternehmens an. Es handelt sich um die Produktion, das Lager und das Personal. Die Ausgaben der Produktion setzen sich aus fünf Posten zusammen. Es sind Gesamtkosten für alle gekauften Triebwerke, Hitzeschilder, Rumpfbauteile und Sonderbauteile. Zusätzlich werden zu den Produktionskosten auch die durch den Ausschuss verursachten  Zusatzkosten hinzugerechnet. Unterhalb der Produktionskosten werden die Personalkosten aufgelistet. Diese setzen sich aus den Werbungskosten, laufenden Kosten (Wartungskosten), und den Kosten der Umbaumaßnahmen zusammen. Das dritte Element der Ausgaben pro Spielrunde sind die gesamten Lagerkosten. Sie setzen sich aus den belegten Lagerplatzeinheiten multipliziert mit den Kosten pro Lagerplatzeinheit zusammen.
 
Die gesamten Ausgaben pro Spielrunde werden in einem rot hinterlegten Textfeld dargestellt. Durch das grün hinterlegte Textfeld der Einnahmen pro Spielrunde und das rot hinterlegte Textfeld der Ausgaben pro Spielrunde soll der Spieler schon bei kurzer Betrachtung des Bildschirms einen Überblick über die Geldflüsse im Unternehmen bekommen.
  
Im unteren Teil der Benutzeroberfläche befindet sich ein Überblick über die gesamte Finanzsituation im Unternehmen. In der ersten Zeile der Tabelle wird der Kontostand zu beginn der Spielrunde angezeigt. Darunter befinden sich die Ein- und Ausgaben der Spielrunde sowie der Kontostand am Ende der Spielrunde(s. \ref{img:ui-bank}).

\begin{figure}[htb]
  \centering
  \fbox{
    \includegraphics[width=0.9\textwidth]{40_UI/30_Bank/Bank.jpg}
  }
  \caption{Finanzen}
  \label{img:ui-bank}
\end{figure}


\autorende{}





% Mit markright kann eine verk\"urzte Version der \"Uberschrift f\"ur den Seitenkopf generiert werden
%
%
%\markright{Formaler Aufbau}




% Anhang der Arbeit
% 
%
% \seAppendix{}
% \chapter{Einige wichtige \LaTeX{}-Kommandos}
% 
% 
% 
% \input{se-test-zitieren}
% 
% %
% Ein kleiner Text, um Abk\"urzungen, Symbole und Glossareintr\"age zu testen
%
%
\section{Kommandos f\"ur die Erzeugung von Abk\"urzungen, Symbolen und Glos\-sar\-eint\-r\"a\-gen}

F\"ur Abk\"urzungen, Symbole und Glossareintr\"age wird das Kommando \verb+\gls{par1}+ verwendet.
\texttt{par1} stellt einen Schl\"ussel dar, der die entsprechende Definition identifiziert (vgl. den Inhalt der Datei
\texttt{pa1-abkuerzungen.tex}). 


Eine Abk\"urzung: \gls{usb}; das zweite Auftreten der Abk\"urzung: \gls{usb}. 
Und jetzt kommt ein Symbol: \gls{pi}; das zweite Symbol ist \gls{ND}.

Und auch ein Eintrag im Glossar muss sein: \gls{glos:AD}; das zweite Auftreten des Eintrags ist \gls{glos:AD}.

\newpage
Und auf der n\"achsten Seite: \gls{glos:AD}. Im Glossar ist jeweils angegeben, auf welchen Seiten der 
Begriff verwendet wurde.

% 
% \input{se-test-floats}
% 
% \input{se-test-listen}
% 
% \chapter{Hinweise zur Installation und \"Ubersetzung}
% 
% \section{Verwendung von TeXShop (Apple-Welt)}

Unter den ausgelieferten Dateien befinden sich zwei \textbf{engine}-Dateien: 

\begin{seList}
\item \verb+dhbw-projektarbeit.engine+
\item \verb+dhbw-projektarbeit-remove-all.engine+ (l\"oscht alle erzeugten \textsl{Hilfsdateien})
\end{seList}

Mit jeder dieser beiden Dateien kann man die Vorlage \texttt{\seVorlage.tex} 
\"ubersetzen. Alle Verzeichnisse (insbesondere Abk\"urzungs- und Symbolverzeichnis) 
sowie das Glossar werden (hoffentlich) korrekt erstellt.  

In den engine-Dateien ist beschrieben, an welcher Stelle sie im Mac OS X Dateisystem 
installiert werden m\"ussen, damit man sie direkt von TeXShop aus aufrufen kann. 

\section{Verwendung von MiKTeX (Windows-Welt)}

F\"ur die \"Ubersetzung wird eine batch-Datei \verb+make-projektarbeit.bat+ zur Verf\"ugung 
gestellt, mit der man in der Windows-\textsl{Eingabeaufforderung} (cmd) die Vorlage \"ubersetzen kann. Der Aufruf lautet:
\texttt{make-projektarbeit.bat \seVorlage}
%\verb+make-projektarbeit.bat se-pa2-vorlage+

Da MiKTeX eine andere Version von \verb+jurabib+ verwendet, mit der sich die 
Vorlage nicht korrekt \"ubersetzen l\"asst, werden die beiden Dateien 

\begin{seList}
\item \verb+jurabib.sty+ und 
\item \verb+jurabib.bst+
\end{seList}

aus der TeX Live Version von Mac OS X mitgeliefert. Damit sollte die 
\"Ubersetzung problemlos funktionieren. 


 

%Dann hoffe ich mal, dass sich mit den Vorlagen etwas anfangen l�sst. Sie sind (absichtlich) in 
%
%einer Version 0.9, da ich an den zugeh�rigen sty-Dateien weitere Erg�nzungen vornehmen werde,
%
%um f�r zuk�nftige Arbeiten neue komfortable Kommandos zur Verf�gung zu stellen. 

%
%  Erzeugung eines Glossars
%
% Achtung: Das Glossar wird nur ausgegeben, wenn mindestens ein Eintrag in der Arbeit 
%                definiert wurde
%
%
% \newpage
% \sePrintGlossary{}


%
% Literaturverzeichnisses
%
\newpage
 \sePrintBibliography{}

 \input{se-test-literaturverzeichnis}


%
% Festlegung des grundlegenden Formatierungsstils des Literaturverzeichnis
%
 \bibliographystyle{jurabib}

% Eigentliche Ausgabe der in der Arbeit verwendeten Quellen
%
%
% Angabe der bib-Dateien, in denen die Quellen beschrieben sind;
% die Angabe geht davon aus, dass eine wa.bib-Datei in demselben 
% Verzeichnis liegt, wie se-pa1-vorlage.tex
%
 \seBibliography{wa}


%
% Erzeugung der ehrenw\"ortlichen Erkl\"arung
%
% Der optionale Parameter kann verwendet werden, um f\"ur das Thema der Arbeit eine 
% andere Formatierung vorzunehmen; das sollte in der Regel nicht erforderlich sein;
% ausserdem besteht die Gefahr inkonsistenter Titel auf dem Titelblatt und in der 
% ehrenw\"ortlichen Erkl\"arung
%
%\seEhrenwoertlicheErklaerung{} % dieses Kommando sollte standardm\"assig verwendet werden
% \seEhrenwoertlicheErklaerung[Star Greg\\Das Unternehmensplanspiel]{}


\end{document}












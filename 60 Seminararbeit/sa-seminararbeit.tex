%
% Einlesen der .sty-Dateien
%
%  se-pa1-input-styles.tex
%
%  Joerg Baumgart 01.08.2011
%
%  Zusammenfassung und Konfiguration wichtiger Styles f\"ur die 
%  Erzeugung von Seminar-, Projekt- und Bachelorarbeiten
%
%
\documentclass[12pt,BCOR=10mm,headinclude=on,footinclude=off,bibliography=totoc]{scrreprt}
\usepackage[T1]{fontenc}
\usepackage[utf8x]{inputenc}
\usepackage[ngerman]{babel} % Deutsche Einstellungen
\usepackage{lmodern}

\usepackage{tikz} % Graphikpaket, das zu pdfLaTeX kompatibel ist
\usepackage{xkeyval} % Definition von Kommandos mit mehreren optionalen Argumenten
\usepackage{listings} % Formatierung von Programmlistings
\usepackage{graphicx} % Einbinden von Graphiken
\usepackage{ifthen}
\usepackage{color}
\usepackage{slashbox} % Diagonalen in Tabellenfeldern
\usepackage{framed} % Erzeugung schwarzer Linien am linken Rand zur Hervorhebung von Textteilen
\usepackage{caption} % Korrektes Setzen einer mehrzeiligen float-Unterschrift bei neu definierten float-Umgebungen
\usepackage{floatrow}

% Es wird jeweils die sty-Datei importiert und entsprechende Konfigurationseinstellungen werden vorgenommen

\usepackage{se-jb-scrpage2} % Formatierung der Kopf- und Fu{\ss}zeilen
\usepackage{se-jb-footmisc}    % Fussnoten besser formatieren

\usepackage{se-jb-glossaries} % Abk\"urzungsverzeichnis, Symbolverzeichnis, Glossar
   
\usepackage{se-jb-floatrow}    % Definition und Konfiguration von float-Umgebungen (figure, table, die neue programm-Umgebung)
% Achtung: se-jb-varioref muss nach se-jb-floatrow importiert werden; 
% andernfalls ist der counter programm f\"ur die labelformat-Anweisung noch nicht definiert   
\usepackage{se-jb-varioref}   % Definition von Querverweisen
\usepackage{se-jb-chngcntr}   % Kapitelweise oder globale Nummerierung von Abbildungen etc.
   
\usepackage{se-jb-listen} % Definition neuer, besser formatierter Listen
%\usepackage{pdf-kommandos} 
\usepackage{se-jb-kommandos-v01} % neue Kommandos f\"ur Seminar-, Projekt- und Bachelorarbeiten


%
% Individuelle Konfiguration des Dokumentes
%
%  Individuelle Konfiguration einer Projektarbeit
%
%
%
%

%
% Literaturverzeichnis
% 
\usepackage{se-jb-jurabib-theisen} % Literaturverzeichnis gem\"ass den Vorgaben von Theisen aufbauen



% Weitere Optionseinstellungen f\"ur das Koma-Script
%
% Zwischen Abs\"atzen einen Abstand von 0.5 \baselineskip erzeugen
\KOMAoption{parskip}{full}
%
% Vergleiche Duden "Gliederung von Nummern, S.111" 
% DIN 5008 anschauen, wenn sie neu ver\"offentlicht wurde
\KOMAoption{numbers}{noendperiod}
%
%



%  Voreinstellungen f\"ur floats
%  Durch die verwendeten Parameter wird die Wahrscheinlichkeit deutlich kleiner, 
%  dass Gleitobjekte (z. B. Abbildungen) ans Ende des Dokumentes verschoben 
%  werden; 
%  Achtung: clearpage erzwingt die Ausgabe von Gleitobjekten
%
\renewcommand{\topfraction}{1}  % Gleitobjekte d\"urfen eine Seite zu 100% belegen 
\renewcommand{\bottomfraction}{1} % Entsprechender Wert f\"ur den unteren Teil der Seite
\renewcommand{\textfraction}{0} % Eine Seite darf auch ohne Fliesstext existieren
%%%\renewcommand{\floatpagefraction}{1} % Bedeutung unklar, daher keine Ver\"anderung des Vorgabewertes 
                                                                        % von 0.5; eventuell bringt ein \"Anderung auf 1 etwas, wenn 
                                                                         % Probleme mit floats auftreten
                                                                         
                                                                         
                                                                         
% Konfiguration von Programm-Listings
% 
% Achtung: hier gibt es nahezu beliebig viele weitere Konfigurationm\"oglichkeiten; vgl. Paketdokumentation
%
\lstset{language=Java,basicstyle=\ttfamily,keywordstyle=\color{blue},captionpos=b,aboveskip=0mm,belowskip=0mm,
          xleftmargin=0em}               
          
%
% Grundkonfiguration der Abs\"ande zwischen den Items der maximal f\"unf Verschachtelungsebenen der 
% neuen Listenumgebungen
%                                                                             
% Initialisierung der Abst\"ande zwischen den items f\"ur seList; Grundeinheit: 0.5\baselineskip; siehe se-jb-listen
\seSetlistbaselineskip{1}{0.75}{0.75}{0.75}{0.75}
% Initialisierung der Abst\"ande zwischen den items f\"ur seToplist; Grundeinheit: 0.5\baselineskip; siehe se-jb-listen
\seSettoplistbaselineskip{1}{0.75}{0.75}{0.75}{0.75}     


%
%  Konfiguration der verschiedenen Verzeichnisse
%
%
%
\seKonfigurationAbb[
%verzeichnisname=Abbildungsverzeichnis,
labeltextLinks=, % kein Text links;
%labeltextRechts=:,
labelbreite=1cm,
%labeleinzug=1cm,
%abstandEintrag=1,
%newpage=ja,
%pnumwidth=20mm,
%dotsep=1000,
%tocrmarg=4.5cm,
%abstandVerzeichnis=-1mm
]

\seKonfigurationPrg[
%verzeichnisname=Listing-Verzeichnis,
labeltextLinks=,
%labeltextRechts=:,
labelbreite=1cm,
%labeleinzug=2cm,
%abstandEintrag=1,
%newpage=ja,
%%pnumwidth=20mm,
%dotsep=1000,
%tocrmarg=4.5cm,
%abstandVerzeichnis=-10mm
]

\seKonfigurationTab[
%verzeichnisname=Liste der Tabellen,
labeltextLinks=,
%labeltextRechts=:,
labelbreite=1cm,
%labeleinzug=0.5cm,
%abstandEintrag=1,
%newpage=ja,
%pnumwidth=20mm,
%dotsep=1000,
%tocrmarg=4.5cm,
%abstandVerzeichnis=-10mm
]

\seKonfigurationAbk[
%verzeichnisname=Liste der Abk\"urzungen,
%labelbreite=3cm,
%labeleinzug=0.5cm,
%abstandEintrag=1,
%newpage=ja,
%abstandVerzeichnis=-10mm
]

\seKonfigurationSym[
%verzeichnisname=Liste der Symbole,
%labelbreite=4cm,
%labeleinzug=3.5cm,
%abstandEintrag=1,
%newpage=ja,
%abstandVerzeichnis=-10mm
]


% (eventuelle) Neudefinition f\"ur die Unter-/\"Uberschriften von Abbildungen, Tabellen und Listings
%
%
%\renewcommand{\seCaptionNameAbbildung}{Abb.}
%\renewcommand{\seCaptionNameTabelle}{Tab.}
%\renewcommand{\seCaptionNameProgramm}{Prg.}


% % (eventuelle) Neudefinition f\"ur Querverweise innerhalb des Textes
%
%
%
%\renewcommand{\seQuerverweisSeite}{Seite}
%\renewcommand{\seQuerverweisAbbildung}{Abb.}
%\renewcommand{\seQuerverweisTabelle}{Tab.}
%\renewcommand{\seQuerverweisProgramm}{Prg.}
%\renewcommand{\seQuerverweisKapitel}{Kap.}
%\renewcommand{\seQuerverweisGleichung}{Gl.}

% Kommandos, die direkt nach \begin{document} ausgef\"uhrt werden m\"ussen
%
%
%
\AtBeginDocument{%
\renewcommand{\listfigurename}{\seAbbildungenVerzeichnisname}
\renewcommand{\listtablename}{\seTabellenVerzeichnisname}
\renewcommand{\figurename}{\seCaptionNameAbbildung}
\renewcommand{\tablename}{\seCaptionNameTabelle}
\pagenumbering{roman}
}
                                                              
                                                                         

%
% Definition von Abk\"urzungen, Symbolen und eventuell Glossareintr\"agen
%
%  J\"org Baumgart
%  Definition einiger Abk\"urzungen
%  
%Befehle f�r Abk\"urzungen
\newacronym{dhbw}{DHBW}{Duale Hochschule Baden-W\"urttemberg}
\newacronym{usb}{USB}{Universal Serial Bus}

%Befehle f�r Symbole
%
% Achtung: ohne sort wird nach Name sortiert
\newglossaryentry{pi}{
name=$\pi$,
description={Die Kreiszahl},
type=symbolslist,
sort=b
}

\newglossaryentry{ND}{
name=$\mbox{\textsl{ND}}$,
description={Nutzungsdauer einer Maschine},
type=symbolslist,%
sort=a
}


% Glossareintr\"age
\newglossaryentry{glos:AD}{
first=Active Directory\textsuperscript{GL},
name=Active Directory,
description={Active Directory ist in einem Windows 2000/Windows
Server 2003-Netzwerk der Verzeichnisdienst, der die zentrale
Organisation und Verwaltung aller Netzwerkressourcen erlaubt. Es
erm\"oglicht den Benutzern \"uber eine einzige zentrale Anmeldung den
Zugriff auf alle Ressourcen und den Administratoren die zentral
organisierte Verwaltung, transparent von der Netzwerktopologie und
den eingesetzten Netzwerkprotokollen. Das daf\"ur ben\"otigte
Betriebssystem ist entweder Windows 2000 Server oder
Windows Server 2003, welches auf dem zentralen
Dom\"anencontroller installiert wird. Dieser h\"alt alle Daten des
Active Directory vor, wie z.\,B. Benutzernamen und
Kennw\"orter.\seFootcite{Vgl.}{S. 200}{Dud09}}
} 

\begin{document}

% Erzeugung des Titelblatts
%
%
%
\seTitelblattSeminararbeit[
%hilfslinien=ja,
%dhbwlogoSkalierung=0.5,
%dhbwlogoDeltaX=2.4,
%dhbwlogoDeltaY=-10,
firmenlogo=firmenlogo,
firmenlogoSkalierung=0.5,
firmenlogoDeltaX=0,
firmenlogoDeltaY=0,
thema=Star Greg - Das Unternehmensplanspiel,
verfasser=Marcel Steinleitner Fredrik Teschke,
%verfasserin=,
kurs=WWI\,10\,SWM\,A,
%studiengangsleiterin=,
studiengangsleiter=Prof. Dr.-Ing. Jörg Baumgart,
wissenschaftlicheBetreuerinName=Dr. Melanie Mustermann,
wissenschaftlicheBetreuerinEmail=melanie.mustermann@musterfirma.de,
wissenschaftlicheBetreuerinTelefon=0621/999999,
bearbeitungszeitraumVon=5. September 2011,
bearbeitungszeitraumBis=14. November 2011,
sperrvermerk=nein
]



% Erzeugung der Kurzfassung; Verfasser, Firma und Thema werden automatisch \"ubernommen
% \seKurzfassung


% Beispiel f\"ur ein Kapitel, dass vor dem Einleitungskapitel kommt, z. B. ein Vorwort oder eine Danksagung
% \seKapitelVorEinleitung{Vorwort}



% Ausgabe der verschiedenen Verzeichnisse
% abk: Abk\"urzungsverzeichnis
% sym: Symbolverzeichnis
% abb: Abbildungsverzeichnis
% tab: Tabellenverzeichnis
% prg: Listingverzeichnis
%
%
% Achtung: Abk\"urzungs- und Symbolverzeichnis werden nur ausgegeben, wenn mindest ein Symbol bzw. 
%                mindestens eine Abk\"urzung in der Arbeit verwendet wurden
%
%
% gliederungsebene:
% -- section: die Verzeichnisse werden einem Kapitel "Verzeichnisse" untergliedert
% -- chapter: die Verzeichnisse sind jeweils eigene Kapitel
% imInhaltsverzeichnis: ja/nein -- Sollen die Verzeichnisse im Inhaltsverzeichnis enthalten sein?
\seVerzeichnisse[gliederungsebene=section,imInhaltsverzeichnis=ja]{abk}{sym}{abb}{tab}{prg}


% Ausgabe des Inhaltsverzeichnisses
%
%
\seInhaltsverzeichnis[%
einrueckung=ja,
gliederungsebenen=4
]




\chapter{Einleitung}\pagenumbering{arabic}
\chapter{UI Mockups}


\chapter{Die Spielwelt}
% Mit markright kann eine verk\"urzte Version der \"Uberschrift f\"ur den Seitenkopf generiert werden
%
%
%\markright{Formaler Aufbau}
\chapter{UI Mockups}


\chapter{Fachkonzept}
\chapter{UI Mockups}


\chapter{UI Mockups}
\chapter{UI Mockups}


\chapter{Implementierung}
\chapter{UI Mockups}


\chapter{jUnit-Tests}
\chapter{UI Mockups}


\chapter{Resümee, Fazit und Ausblick}
\chapter{UI Mockups}




% Anhang der Arbeit
% 
%
% \seAppendix{}
% \chapter{Einige wichtige \LaTeX{}-Kommandos}



%%  Testdatei f\"ur die Erzeugung von Literaturreferenzen, die den Regeln von Rene Theisen 
%  (Wissenschaftliches Arbeiten, 2009) folgen
%
%
%
\section{Kommandos f\"ur die Erzeugung von Literaturverweisen}

Das Kommando \verb+\seCite{par1}{par2}{par3}+ erzeugt einen Literaturverweis im Text. 

\begin{seToplist}{\texttt{par1}:}
\item[\texttt{par1}:] Der erste Parameter  definiert einen optionalen Text, der vor dem eigentlichen Literaturverweis ausgegeben 
                               wird, typischerweise Vgl. oder vgl.
\item[\texttt{par2}:] Der zweite Parameter  wird verwendet, um (z.\,B.) zus\"atzliche Seitenangaben f\"ur den Literaturverweis 
                              vorzunehmen.
\item[\texttt{par2}:] Der dritte Parameter ist der entsprechende Schl\"ussel in der .bib-Datei, in der die Literaturquellen 
                              beschrieben sind (vgl. \texttt{wa.bib}).                                                                                       
\end{seToplist}

Als Beispiel f\"ur die Verwendung des \verb+\seCite+-Befehls dient folgendes Zitat: \glqq{}Die \textbf{Funktion} eines 
Anhangs einer wissenschaftlichen Arbeit wird sehr h\"aufig \textbf{missdeutet}, der Anhang selbst nicht selten \textbf{mi{\ss}braucht}.\grqq{} 
(\seCite{}{S. 170}{The:WA}).

Bei der von Theisen vorgeschlagenen Zitierweise erfolgt die Angabe der Literaturverweise in der Regel innerhalb einer Fu{\ss}note. 
Hierf\"ur kann das Kommando \verb+\seFootcite+ verwendet werden, das dieselben Parameter wie \verb+\seCite+ besitzt. 

Als Beispiel f\"ur ein indirektes Zitat l\"asst sich die Aussage von Theisen anf\"uhren, dass Hauptinhalte eines (berechtigten) Anhangs erg\"anzende 
Materialien und Dokumente sind, die weitere themenbezogene Informationen liefern k\"onnen.\seFootcite{Vgl.}{S. 171}{The:WA}

Weder das \verb+\seFootcite+- noch das \verb+\footnote+-Kommande k\"onnen bei Gleitobjekten (Verwendung der \verb+figure+-, \verb+table+- oder 
\verb+programm+-Umgebung) verwendet werden. Ein kleiner Workaround, um \LaTeX{} doch dazu zu bringen, Fu{\ss}noten bei Gleitobjekten 
zu akzeptieren, ist in \vref{gleitobjekte} zu finden.



%
%  Erzeugung eines Glossars
%
% Achtung: Das Glossar wird nur ausgegeben, wenn mindestens ein Eintrag in der Arbeit 
%                definiert wurde
%
%
\newpage
\sePrintGlossary{}


%
% Literaturverzeichnisses
%
%\newpage
% \sePrintBibliography{}

%%  Erzeugung von Eintr\"agen im Literaturverzeichnis
%
%  Achtung: in einer Projektarbeit darf da \nocite-Kommando nicht verwendet werden,
%                 da es einen Eintrag im Literaturverzeichnis erzeugt, ohne dass eine 
%                 entsprechende Literaturreferenz im Text der Arbeit angegeben wird
%
%
%
\nocite{DHBW:SG}
\nocite{KM:KS}
\nocite{Dud06}
\nocite{Dud09}
\nocite{Bri:WA}
\nocite{RP:WA}
\nocite{Sch:WAS}
\nocite{BSS:WA}
\nocite{Kor:WA}
\nocite{MK:GWA}
\nocite{ADG:WA}
\nocite{The:WA}
\nocite{BA:WA}
\nocite{Dij:CRT}
\nocite{BC:Cur}
\nocite{Par:ECP}
\nocite{Bro:SBE}
\nocite{GI:ADI}
\nocite{GI:AZI}
\nocite{Den:CD}
\nocite{LMS:Icb}
\nocite{Fre:SIF}




%
% Festlegung des grundlegenden Formatierungsstils des Literaturverzeichnis
%
% \bibliographystyle{jurabib}

% Eigentliche Ausgabe der in der Arbeit verwendeten Quellen
%
%
% Angabe der bib-Dateien, in denen die Quellen beschrieben sind;
% die Angabe geht davon aus, dass eine wa.bib-Datei in demselben 
% Verzeichnis liegt, wie se-pa1-vorlage.tex
%
% \seBibliography{wa}


%
% Erzeugung der ehrenw\"ortlichen Erkl\"arung
%
%
\seEhrenwoertlicheErklaerung{}


\end{document}












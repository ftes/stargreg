%
% Einlesen der .sty-Dateien
%
\input{sa-input-styles}

%
% Individuelle Konfiguration des Dokumentes
%
\input{sa-konfiguration}

%
% Definition von Abk\"urzungen, Symbolen und eventuell Glossareintr\"agen
%
\input{sa-abkuerzungen} 

\begin{document}

% Erzeugung des Titelblatts
%
%
%
\seTitelblattErsteProjektarbeit[
%hilfslinien=ja,
%dhbwlogoSkalierung=0.5,
%dhbwlogoDeltaX=2.4,
%dhbwlogoDeltaY=-10,
firmenlogo=firmenlogo,
firmenlogoSkalierung=0.5,
firmenlogoDeltaX=0,
firmenlogoDeltaY=0,
thema=\LaTeX-Layoutvorlage zur Anfertigung der ersten Projektarbeit\\Version 0.9,
verfasser=Jörg Baumgart,
%verfasserin=,
matrikelnummer=9999999,
kurs=WWI\,10\,SWM\,A,
firma=Ausbildungsfirma,
abteilung=Wirtschafsinformatik/Softwaremethodik,
%studiengangsleiterin=,
studiengangsleiter=Prof. Dr.-Ing. Jörg Baumgart,
wissenschaftlicheBetreuerinName=Dr. Melanie Mustermann,
wissenschaftlicheBetreuerinEmail=melanie.mustermann@musterfirma.de,
wissenschaftlicheBetreuerinTelefon=0621/999999,
%wissenschaftlicherBetreuerName=,
%wissenschaftlicherBetreuerEmail=,
%wissenschaftlicherBetreuerTelefon=,
firmenbetreuerinName=Dipl.-Ing. Ariane Meistermann,
firmenbetreuerinEmail=a.meistermann@andere-musterfirma.de,
firmenbetreuerinTelefon=06151/88888,
%firmenbetreuerName=,
%firmenbetreuerEmail=,
%firmenbetreuerTelefon=,
bearbeitungszeitraumVon=22. Juli 2011,
bearbeitungszeitraumBis=4. August 2011,
sperrvermerk=ja
]



% Erzeugung der Kurzfassung; Verfasser, Firma und Thema werden automatisch \"ubernommen
\seKurzfassung


% Beispiel f\"ur ein Kapitel, dass vor dem Einleitungskapitel kommt, z. B. ein Vorwort oder eine Danksagung
\seKapitelVorEinleitung{Vorwort}



% Ausgabe der verschiedenen Verzeichnisse
% abk: Abk\"urzungsverzeichnis
% sym: Symbolverzeichnis
% abb: Abbildungsverzeichnis
% tab: Tabellenverzeichnis
% prg: Listingverzeichnis
%
%
% Achtung: Abk\"urzungs- und Symbolverzeichnis werden nur ausgegeben, wenn mindest ein Symbol bzw. 
%                mindestens eine Abk\"urzung in der Arbeit verwendet wurden
%
%
% gliederungsebene:
% -- section: die Verzeichnisse werden einem Kapitel "Verzeichnisse" untergliedert
% -- chapter: die Verzeichnisse sind jeweils eigene Kapitel
% imInhaltsverzeichnis: ja/nein -- Sollen die Verzeichnisse im Inhaltsverzeichnis enthalten sein?
\seVerzeichnisse[gliederungsebene=section,imInhaltsverzeichnis=ja]{abk}{sym}{abb}{tab}{prg}


% Ausgabe des Inhaltsverzeichnisses
%
%
\seInhaltsverzeichnis[%
einrueckung=ja,
gliederungsebenen=4
]




% Erstes eigentliches Kapitel der Arbeit; typischerweise das Einleitungskapitel;
% hier muss wieder auf die Nummierung mit arabischen Seitenzahlen umgestellt werden
%
\chapter{Einleitung}\pagenumbering{arabic}

öblöäß


% Erstes Hauptkapitel der Arbeit 
%
%
%
\chapter{Der formale Aufbau einer Projektarbeit}
% Mit markright kann eine verk\"urzte Version der \"Uberschrift f\"ur den Seitenkopf generiert werden
%
%
%\markright{Formaler Aufbau}




% Anhang der Arbeit
% 
%
\seAppendix{}
\chapter{Einige wichtige \LaTeX{}-Kommandos}



%\input{se-test-zitieren}



%
%  Erzeugung eines Glossars
%
% Achtung: Das Glossar wird nur ausgegeben, wenn mindestens ein Eintrag in der Arbeit 
%                definiert wurde
%
%
\newpage
\sePrintGlossary{}


%
% Literaturverzeichnisses
%
%\newpage
\sePrintBibliography{}

%\input{se-test-literaturverzeichnis}


%
% Festlegung des grundlegenden Formatierungsstils des Literaturverzeichnis
%
\bibliographystyle{jurabib}

% Eigentliche Ausgabe der in der Arbeit verwendeten Quellen
%
%
% Angabe der bib-Dateien, in denen die Quellen beschrieben sind;
% die Angabe geht davon aus, dass eine wa.bib-Datei in demselben 
% Verzeichnis liegt, wie se-pa1-vorlage.tex
%
\seBibliography{wa}


%
% Erzeugung der ehrenw\"ortlichen Erkl\"arung
%
%
\seEhrenwoertlicheErklaerung{}


\end{document}












%  J\"org Baumgart
%  Definition einiger Abk\"urzungen
%  
%Befehle f�r Abk\"urzungen
\newacronym{dhbw}{DHBW}{Duale Hochschule Baden-W\"urttemberg}
\newacronym{usb}{USB}{Universal Serial Bus}

%Befehle f�r Symbole
%
% Achtung: ohne sort wird nach Name sortiert
\newglossaryentry{pi}{
name=$\pi$,
description={Die Kreiszahl},
type=symbolslist,
sort=a
}

\newglossaryentry{ND}{
name=$\mbox{\textsl{ND}}$,
description={Nutzungsdauer einer Maschine},
type=symbolslist,%
sort=b
}


% Glossareintr\"age
\newglossaryentry{glos:AD}{
first=Active Directory\textsuperscript{GL},
name=Active Directory,
description={Active Directory ist in einem Windows 2000/Windows
Server 2003-Netzwerk der Verzeichnisdienst, der die zentrale
Organisation und Verwaltung aller Netzwerkressourcen erlaubt. Es
erm\"oglicht den Benutzern \"uber eine einzige zentrale Anmeldung den
Zugriff auf alle Ressourcen und den Administratoren die zentral
organisierte Verwaltung, transparent von der Netzwerktopologie und
den eingesetzten Netzwerkprotokollen. Das daf\"ur ben\"otigte
Betriebssystem ist entweder Windows 2000 Server oder
Windows Server 2003, welches auf dem zentralen
Dom\"anencontroller installiert wird. Dieser h\"alt alle Daten des
Active Directory vor, wie z.\,B. Benutzernamen und
Kennw\"orter.\seFootcite{Vgl.}{S. 200}{Dud09}}
}